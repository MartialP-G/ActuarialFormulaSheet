% !TeX root = ActuarialFormSheet_MBFA-es.tex
% !TeX spellcheck = es_ES
\begin{f}[Seguros de vida y no vida]
	
La distinción entre seguros de vida y no vida es fundamental. Una aseguradora no puede ofrecer ambos tipos de seguros sin tener dos compañías separadas: 
\begin{itemize}
	\item \textbf{Seguro de vida}, es decir, seguro personal que excluye la cobertura de lesiones corporales.
	\item \textbf{Seguro distinto del de vida}, que incluye seguros de propiedad y de responsabilidad civil, así como seguros de lesiones corporales.
\end{itemize}
\end{f}


\begin{f}[Los principios del seguro]
	
	
Se asume que el seguro :
	\begin{itemize}
		\item basarse en la máxima buena fe,
		\item se aplican únicamente si el asegurado tiene un interés asegurable en preservar el artículo (seguro de propiedad),
		\item operar bajo el principio de indemnidad : 
		\begin{itemize}
			\item no permitir el enriquecimiento a partir de una liquidación de siniestros,
			\item ni siquiera a través de la acumulación de seguros, 
			\item incluir la subrogación (en el seguro de responsabilidad civil, si el asegurador indemniza a la víctima asegurada, éste no puede luego reclamar al responsable del siniestro). 
		\end{itemize}
		\item no reducirá los esfuerzos del asegurado en materia de prevención y protección, como persona razonable, incluso si está protegido económicamente.
	\end{itemize}
	\item establecer la causa de un accidente en Responsabilidad Civil, quién no es responsable si no contribuye a la causa del accidente.
\end{f}


\begin{f}[La póliza de seguro]
	
	La \textbf{póliza de seguro} (o contrato) es el documento contractual que rige la relación entre la compañía de seguros (o mutua) y el asegurado (tomador del seguro). 
	El presente contrato define en particular :
	
	\begin{enumerate}
		\item la lista de eventos cubiertos, incluidas las exclusiones,
		\item la cobertura, es decir, la asistencia prestada al asegurado en caso de pérdida,
		\item las obligaciones del asegurado :
		\begin{itemize}
			\item cualquier medida preventiva necesaria para reducir el riesgo,
			\item plazos para comunicar un siniestro a la aseguradora,
			\item el importe y las condiciones de pago de la prima (deducible, límite),
			\item las condiciones de cancelación de la póliza (renovación automática),
		\end{itemize}
		\item  Las obligaciones de la compañía de seguros : plazos para el pago de indemnizaciones.
		
	\end{enumerate}
	
\end{f}
\begin{f}[La prima y las reclamaciones]
Clásicamente, el papel de la aseguradora es sustituir una constante \(C\), la \textbf{contribución} o la \textbf{prima}, por un reclamo aleatorio \(S\).
La \textbf{prima pura} o \textbf{técnica} tiene como objetivo 
compensar las reclamaciones sin excedente ni beneficio, en general \(C_t = \mathbb{E}[S]\)
	
	La \textbf{prima neta} es superior a la prima pura. Su objetivo es cubrir el coste de las reclamaciones y proporcionar un margen de seguridad.
	
	La \textbf{prima bruta} es la prima neta + gastos generales + comisiones + beneficio esperado + impuestos.
	
	Por razones comerciales, la prima efectivamente cobrada puede diferir significativamente de la prima técnica.

	\textbf{Prima escrita:} prima cobrada al asegurado para cubrir reclamos que puedan ocurrir durante el período de 
cobertura definido por el contrato (generalmente 1 año en seguros de propiedad y accidentes).

\textbf{Prima devengada:} proporción de la prima emitida utilizada para cubrir el riesgo durante el período de exposición de un 
año de póliza.
\tikzstyle{NoeudR}=[rectangle, shape border rotate=90, draw,minimum height=0.6cm,align=center, BleuProfondIRA]

\resizebox{\linewidth}{!}
{\begin{tikzpicture} %[node distance=5cm]
	%	
	\draw[thick,<->,>=latex,BleuProfondIRA] (-3,0) -- (5,0) ;
	\node (JanNN) at (-2,0.8) [NoeudR] {\footnotesize  \nth{1} enero\\ \footnotesize \NN};
	\node (JuilNN) at (0,0.8) [NoeudR] {\footnotesize  \nth{1} febrero\\ \footnotesize \NN};
	\node (JanN) at (2,0.8) [NoeudR] {\footnotesize  31 diciembre\\ \footnotesize  \NN};
	\node (JuilN) at (4,0.8) [NoeudR] {\footnotesize  \nth{1} julio\\ \footnotesize \N};
	\draw[ BleuProfondIRA] (JanNN.south) -- ++(0,-0.8);
	\draw[ BleuProfondIRA] (JuilNN.south) -- ++(0,-0.8);
	\draw[ BleuProfondIRA] (JanN.south) -- ++(0,-0.8);
	\draw[ BleuProfondIRA] (JuilN.south) -- ++(0,-0.8);
	\draw[thick,<->,>=latex,VertIRA] ($(JuilNN.south) +(0,-1)$) -- ($(JuilN.south) +(0,-1)$) node [below, midway] {\footnotesize Prima emitida 600€} ;
	\draw[thick,<->,>=latex,OrangeProfondIRA] ($(JuilNN.south) +(0,-1.6)$) -- ($(JanN.south) +(0,-1.6)$) node [below, midway] {\footnotesize Prima ganada 300€} ;
	\draw[thick,<->,>=latex,OrangeProfondIRA,densely dotted] ($(JanN.south) +(0,-1.6)$) -- ($(JuilN.south) +(0,-1.6)$) node [below, midway,align=right] 
	{\footnotesize No ganado 300€ \\\footnotesize  Reservado para el periodo \N } ;
	%	
\end{tikzpicture}
}
El \(S/P\) es el indicador clave. Para que la aseguradora obtenga beneficios, el \(S/P\ll 1\).
\end{f}
%\end{multicols}

\begin{f}[Triángulo de pérdida / pago]
	
	La contabilidad de seguros se desglosa según el \textbf{año del accidente} del reclamo. 
	Si una prima cubre varios años naturales, se asignará una parte proporcional a cada uno de ellos.
	Cada pago y cada reserva de siniestro se asigna al año del accidente.
	El seguimiento de los pagos o gastos se expresa a través de un triángulo (matriz triangular) :
	
	\[
	\left(\begin{array}{ccccc}
		\rowcolor{white}	C_{1,1} & C_{1,2} & \ldots & & C_{1, n} \\
		\rowcolor{white}	C_{1,1} & C_{1,2} & \ldots & C_{2, n-1} \\
		\rowcolor{white}	\vdots & \vdots & & \\
		\rowcolor{white}	C_{n-1,1} & C_{n-1,2} & & \\
		\rowcolor{white}	C_{n, 1} & & &
	\end{array}\right)
	\]
	donde \(C_{i, j}=\sum_{k=1}^{j} X_{i, k}\) representa el monto acumulado de reclamaciones pagadas para el año de origen \(i\) y el año de desarrollo \(j\).
	
\end{f}
%\begin{multicols}{2}


	
\begin{f}[Solvencia II y Gestión de Riesgos]
	
	Solvencia II es el marco regulatorio europeo aplicable a aseguradoras y reaseguradoras desde 2016. Se basa en tres pilares interdependientes :
	
	\begin{itemize}
		\item \textbf{Pilar 1 : Requisitos cuantitativos} \\
		Determina los requisitos de capital :
		\begin{itemize}
			\item \textbf{SCR} (Requisito de Capital de Solvencia) : capital para absorber un shock extremo (99.5\% en un año),
			\item \textbf{MCR} (Requisito de Capital Mínimo) : umbral mínimo absoluto,
			\item activos admisibles para cubrir reservas técnicas y requerimientos de capital.
		\end{itemize}
		
		\item \textbf{Pilar 2 : Gobernanza, control interno y gestión de riesgos} \\
		Gira en torno a \textbf{ERM} (\emph{Enterprise Risk Management}). Los requisitos cubren :
		\begin{itemize}
			\item gobernanza : juntas directivas responsables del marco de gestión de riesgos ;
			\item un sistema de \textbf{control interno} eficaz ;
			\item funciones clave independientes : \textbf{actuario}, \textbf{gestión de riesgos}, \textbf{cumplimiento}, \textbf{auditoría interna} ;
			\item \textbf{ORSA} (\emph{Own Risk and Solvency Assessment}) : Evaluación interna de riesgos y solvencia, una herramienta central que alinea la estrategia, el apetito de riesgo y el capital económico.
		\end{itemize}
		
		\item \textbf{Pilar 3 : Disciplina de mercado} \\
		Basado en la \textbf{transparencia} y la comunicación :
		\begin{itemize}
			\item \textbf{SFCR} (\emph{Solvency and Financial Condition Report}) : público, resume la solvencia y la posición financiera,
			\item \textbf{RSR} (\emph{Regular Supervisory Report}) : destinado al supervisor,
			\item Informes cuantitativos : declaraciones regulatorias (\textbf{QRTs}), Presentación periódica de datos financieros y prudenciales.
		\end{itemize}
	\end{itemize}
		
\end{f}
\begin{f}[Principales ramas de seguros de vida y no vida]
	
	El seguro de vida cubre compromisos a largo plazo, con o sin componente de ahorro :
	\begin{itemize}[nosep]
		\item \textbf{Seguro de vida} : suma global o anualidad pagada si el asegurado está vivo en una fecha determinada.
		\item \textbf{Seguro de muerte} : pago si el asegurado fallece durante el período cubierto.
		\textbf{Seguro de dotación} : combinación de cobertura de vida y muerte.
		\textbf{Renta vitalicia} : pagos periódicos hasta la muerte.
		\textbf{Ahorro/jubilación} : productos con capital diferido o renta diferida.
		\item Pólizas vinculadas a unidades : beneficios que dependen del valor de los activos financieros.
		\item \textbf{Contratos colectivos} : pensiones de jubilación, seguros colectivos de bienestar.
	\end{itemize}
	
	
	El seguro de no vida cubre los riesgos que se producen a corto o medio plazo. :
	\begin{itemize}[nosep]
		\item \textbf{Automóvil} : responsabilidad civil, daños al vehículo.
		\item \textbf{Vivienda} : incendio, robo, daños por agua, responsabilidad.
		\item \textbf{Responsabilidad civil general }: responsabilidad civil vida privada, responsabilidad civil empresarial.
		\item \textbf{Salud y bienestar} : reembolsos médicos, discapacidad, incapacidad.
		\item \textbf{Lesiones personales} : capital en caso de accidente, invalidez o muerte.
		\item \textbf{Interrupciones de negocios} : pérdidas financieras vinculadas a un desastre.
		\item \textbf{Interrupciones comerciales} : pérdidas financieras vinculadas a un desastre.
		\item \textbf{Transporte, aviación, seguro marítimo} : mercancías en tránsito, responsabilidades específicas.
	\end{itemize}
	
	
\begin{center}
	\resizebox{1.1\linewidth}{!}{	\begin{tikzpicture}[ every node/.style={font=\small,text width=3cm}, node distance=0.75cm and 1.cm]
			% Rectangle central "Responsabilit\'e"
			\node[draw, rectangle, minimum width=2cm, minimum height=1cm, align=center] (resp) at (0,0) {Responsabilidad};
			%		
			% Branches vers "Responsabilit\'e p\'enale" et "Responsabilit\'e civile"  +(-.25,0) |-
			\node[below right=of resp, xshift=-.5cm, yshift=.5cm] (penale) {Responsabilidad penal};
			\node[below=0cm of penale] (inass) {No asegurable};
			\draw (resp.east) -| (penale.north);
			%		
			\node[below left=of resp] (civile) {Responsabilidad civil};
			\draw (resp.west) -| (civile.north);
			%		
			%		% Vers RC contractuelle et d\'elictuelle
			\node[below=of civile, xshift=-1.5cm] (rccontr) {Seguro de responsabilidad contractual};
			\node[below=of civile, xshift=3cm] (rcdel) {Responsabilidad extracontractual};
			\draw (civile.south) |-  +(0,-0.25) -| (rccontr.north);
			\draw (civile.south) |-  +(0,-0.25) -| (rcdel.north);
			%		
			% Obligation de moyen / de r\'esultat
			\node[below= of rccontr, xshift=-1.5cm, rotate=45] (moyen) {Obligación\\ de medios};
			\node[below= of rccontr, xshift=0.5cm, rotate=45] (resultat) {Obligación\\ de resultado};
			\draw (rccontr.south) |-  +(0,-0.25) -| (moyen.north);
			\draw (rccontr.south) |-  +(0,-0.25) -| (resultat.north);
			%		
			% Sous RC d\'elictuelle : fait personnel / d'autrui / de choses
			\node[below=of rcdel, xshift=-2cm, rotate=45] (choses) {Acto de cosas \\ Presunción};
			\node[below= of rcdel, rotate=45] (autrui) {Actos ajenos \\ Presunción \\ Artículo 1384 del CC (Francia)};
			\node[below= of rcdel, xshift=2cm, rotate=45] (personnel) {Acto personal\\Culpa a probar};
			\draw (rcdel.south) |-  +(0,-0.25) -| (choses.north);
			\draw (rcdel.south) |-  +(0,-0.25) -| (autrui.north);
			\draw (rcdel.south) |-  +(0,-0.25) -| (personnel.north);
			%		
			% Faute \`a prouver et fait conscient sous fait personnel
			\node[right= of personnel, rotate=0, anchor=west] (fconscient) {Acto consciente};
			\draw (personnel.south) -|  +(0.75,0) |- (fconscient.west);
			%		
			% Imprudence / N\'eglience sous Fait conscient
			\node[below=of fconscient] (impru) {Descuido};
			\node[below= of impru] (negl) {Negligencia};
			\draw (fconscient.west)  |- +(-0.25,0) |- (impru.west);
			\draw (impru.west)  |- +(-0.25,0) |- (negl.west);
			% Source
			%		\node[below=2cm of moyen, anchor=west, text width=10cm] (source) {\small Source : Les Grands Principes de l'Assurance, Couilbault et Eliashberg, 10e \'edition};
		\end{tikzpicture}
	}
\end{center}

	
\end{f}


\begin{f}[Actuario]
	
	En la práctica, el actuario :
	\begin{itemize}[nosep]
		\item precios de productos de seguros y bienestar, 
		\item estima las reservas técnicas,
		\item proyecta flujos de efectivo y valora pasivos a largo plazo,
		\item Mide el capital económico (SCR, ORSA) y contribuye a la ERM,
		\item asesora a la gerencia sobre estrategia, solvencia y cumplimiento normativo..
	\end{itemize}
	
\end{f}