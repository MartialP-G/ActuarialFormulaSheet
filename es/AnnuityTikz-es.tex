\ \newline


	Annuité certaine \(a_{\lcroof{n}}\) (ou \(a_{\lcroof{n}\;i}\) si le taux d'intérêt \(i\) est utile à préciser) : c'est le cas par défaut en mathématiques financières. Ses paiements sont par exemple garantis par un contrat.
	
\begin{center}
	\begin{tikzpicture}[scale=0.75]
	% Draw the x-axis and y-axis.
	\def\w{11}
	\def\n{9}
	\node[left] at (-.5,0) {\(a_{\lcroof{n}}\)};
	\foreach \y in  {0,...,3} {
	\draw (\y,0) -- (\y,-0.1);
	\ifthenelse{\y>0 }{	\node[below] at (\y,-0.1) {\tiny \( \scriptstyle \y\)};
		\draw[ line width=1, color=OrangeProfondIRA, arrows={-Stealth[length=4, inset=0]}] (\y,0) -- (\y,1);}{
		\node[below] at (\y,-0.1) {\tiny 0};}
}
\foreach \y in  {-3,...,1} {
	\draw (\y+\n,0) -- (\y+\n,-0.1);
	\ifthenelse{\y<0 }{\node[below] at (\y+\n,-0.1) {\tiny \(\scriptstyle n \y\)};}{}
	\ifthenelse{\y=0 }{\node[below] at (\y+\n,-0.1) {\tiny \(\scriptstyle n \)};}{}
	\ifthenelse{\y>0}{	\node[below] at (\y+\n,-0.1) {\tiny \(\scriptstyle n+\y\)};}{}
	\ifthenelse{\y<1 }{	\draw[ line width=1, color=OrangeProfondIRA, arrows={-Stealth[length=4, inset=0]}] (\y+\n,0) -- (\y+\n,1);}
}
\draw[ line width=1] (-.25,0) -- (4,0);
\draw[ line width=1, dashed] (4,0) -- (5,0);
\draw[arrows={-Stealth[length=4, inset=0]}, line width=1] (5,0) -- (\w,0);
\end{tikzpicture}
\end{center}
\[
\ddot{a}_{\lcroof{n}}=1+v+\cdots+v^{n-1}=\frac{1-v^{n}}{1-v}=\frac{1-v^{n}}{d}
\]


Annuité contingente \(\ddot{a}_{x}\): ses paiements sont conditionnés à un événement  aléatoire, comme lors d'une rente viagère d'un individu d'âge \(x\). Dans cet exemple, ils courent jusqu'à ce que mort s'ensuive :

\begin{center}
	\begin{tikzpicture}[scale=0.75]
	% Draw the x-axis and y-axis.
	\def\w{11}
	\def\n{6}
	\node[left] at (-.5,0) {\(a_{x}\)};
	
	\begin{scope}[shift={(\n+.5+3,.25)}]
		\draw[color=OrangeProfondIRA,scale=0.2,fill=OrangeProfondIRA] \Cerceuil;
	\end{scope}
	\foreach \y in  {0,...,3} {
		\draw (\y,0) -- (\y,-0.1);
		\ifthenelse{\y>0 }{	\node[below] at (\y,-0.1) {\tiny \( \scriptstyle x+\y\)};
			\draw[ line width=1, color=OrangeProfondIRA, arrows={-Stealth[length=4, inset=0]}] (\y,0) -- (\y,1);}{
			\node[below] at (\y,-0.1) {\tiny \( \scriptstyle x\)};}
	}
	\foreach \y in  {0,...,4} {
		\draw (\y+\n,0) -- (\y+\n,-0.1);
		\ifthenelse{\y>0 }{\node[below] at (\y+\n,-0.1) {\tiny \(\scriptstyle x+n+\y\)};}{
			\node[below] at (\y+\n,-0.1) {\tiny \(\scriptstyle x+n\)};}
		\ifthenelse{\y<4 }{	\draw[ line width=1, color=OrangeProfondIRA, arrows={-Stealth[length=4, inset=0]}] (\y+\n,0) -- (\y+\n,1);}
	}
\draw[ line width=1] (-.25,0) -- (4,0);
\draw[ line width=1, dashed] (4,0) -- (5,0);
\draw[arrows={-Stealth[length=4, inset=0]}, line width=1] (5,0) -- (\w,0);
\end{tikzpicture}	

\end{center}
La date du décès est représenté ici par un petit cercueil. Cette forme d'annuité sera largement étudiée dans la partie actuariat vie.

Annuité à terme échu (\engl{immédiate}) \(a_{\lcroof{n}}\) : ses paiements périodiques sont effectués à la fin de chaque période de paiement, comme pour le salaire payé en fin de mois. C'est le cas par défaut, illustré précédemment pour l'annuité certaine.
\[
\ddot{a}_{\lcroof{n}}=1+v+\cdots+v^{n-1}=\frac{1-v^{n}}{1-v}=\frac{1-v^{n}}{d}
\]
%
\[
	\mathrm{PV}_{\lcroof{n}}^{\text {due }}=K \ddot{a}_{\lcroof{n}}=K \frac{1-v^{n}}{d} 
\]

Annuité à terme à échoir (\engl{due}) \(\ddot{a}_{\lcroof{n}}\):  ses paiements périodiques sont effectués au début de chaque période de paiement, comme pour un loyer par exemple.

\begin{center}
	\begin{tikzpicture}[scale=0.75]
		% Draw the x-axis and y-axis.
		\def\w{11}
		\def\n{9}
		\node[left] at (-.5,0) {\(\ddot{a}_{\lcroof{n}}\)};
		\foreach \y in  {0,...,3} {
			\draw (\y,0) -- (\y,-0.1);
				\node[below] at (\y,-0.1) {\tiny \( \scriptstyle \y\)};
				\draw[ line width=1, color=OrangeProfondIRA, arrows={-Stealth[length=4, inset=0]}] (\y,0) -- (\y,1);
		}
		\foreach \y in  {-3,...,1} {
			\draw (\y+\n,0) -- (\y+\n,-0.1);
			\ifthenelse{\y<0 }{\node[below] at (\y+\n,-0.1) {\tiny \(\scriptstyle n \y\)};}{}
			\ifthenelse{\y=0 }{\node[below] at (\y+\n,-0.1) {\tiny \(\scriptstyle n \)};}{}
			\ifthenelse{\y>0}{	\node[below] at (\y+\n,-0.1) {\tiny \(\scriptstyle n+\y\)};}{}
			\ifthenelse{\y<0 }{	\draw[ line width=1, color=OrangeProfondIRA, arrows={-Stealth[length=4, inset=0]}] (\y+\n,0) -- (\y+\n,1);}{}
		}
		\draw[ line width=1] (-.25,0) -- (4,0);
		\draw[ line width=1, dashed] (4,0) -- (5,0);
		\draw[arrows={-Stealth[length=4, inset=0]}, line width=1] (5,0) -- (\w,0);
	\end{tikzpicture}
\end{center}
aussi notée \(\mathrm{PV}^{\mathrm{im}}\) :
\[
a_{\lcroof{n}}=v+v^{2}+\cdots+v^{n}=\frac{1-v^{n}}{i}=v \frac{1-v^{n}}{1-v}
\]
%
\[
	\mathrm{PV}_{\lcroof{n}}^{\mathrm{im}}=K a_{\lcroof{n}}=K \frac{1-v^{n}}{i} 
\]
Annuité  perpétuelle \(a\) ou  \(a_{\lcroof{\infty}}\): 
\[
a=1/i
\]
Annuité différée \(_{m|}a_{\lcroof{n}}\): ses paiements ne commencent pas dans la première période mais après \(m\) périodes, \(m\) étant fixé à l'avance. 
\begin{center}
	\begin{tikzpicture}[scale=0.75]
		% Draw the x-axis and y-axis.
\def\w{10}
\def\n{9}
\def\m{2}
\node[left] at (-.5,0) {\(_{m|}a_{\lcroof{n}}\)};
\foreach \y in  {0,...,2} {
	\draw (\y+\m,0) -- (\y+\m,-0.1);
	\ifthenelse{\y<0 }{\node[below] at (\y+\m,-0.1) {\tiny \(\scriptstyle m \y\)};}{}
\ifthenelse{\y=0 }{\node[below] at (\y+\m,-0.1) {\tiny \(\scriptstyle m \)};}{}
\ifthenelse{\y>0}{	\node[below] at (\y+\m,-0.1) {\tiny \(\scriptstyle m+\y\)};
		\draw[ line width=1, color=OrangeProfondIRA, arrows={-Stealth[length=4, inset=0]}] (\y+\m,0) -- (\y+\m,1);
		}
}
\foreach \y in  {-3,...,0} {
	\draw (\y+\n,0) -- (\y+\n,-0.1);
	\ifthenelse{\y<0 }{\node[below] at (\y+\n,-0.1) {\tiny \(\scriptstyle m+n \y\)};}{}
	\ifthenelse{\y=0 }{\node[below] at (\y+\n,-0.1) {\tiny \(\scriptstyle m+n \)};}{}
	\ifthenelse{\y>0}{	\node[below] at (\y+\n,-0.1) {\tiny \(\scriptstyle m+n+\y\)};}{}
	\ifthenelse{\y<1 }{	\draw[ line width=1, color=OrangeProfondIRA, arrows={-Stealth[length=4, inset=0]}] (\y+\n,0) -- (\y+\n,1);}
}
\draw[ line width=1] (-.25,0) -- (0.5,0);
\draw[ line width=1, dashed] (0.5,0) -- (1.5,0);
\draw[ line width=1] (1.5,0) -- (4,0);
\draw[ line width=1, dashed] (4,0) -- (5,0);
		\draw[arrows={-Stealth[length=4, inset=0]}, line width=1] (5,0) -- (\w,0);
	\end{tikzpicture}	
	
\end{center}


Annuité périodique / mensuelle \(a^{(m)}\) : la périodicité par défaut est un an, mais le paiement  unitaire peut être aussi réparti sur \(m\) périodes dans l'année.

\begin{center}
	\begin{tikzpicture}[scale=0.75]
		% Draw the x-axis and y-axis.
		\def\w{10}
		\def\n{9}
		\def\m{6}
		\node[left] at (-.5,0) {\(a_{\lcroof{n}}^{(m)}\)};
		\foreach \y in  {0,...,3} {
			\draw (\y,0) -- (\y,-0.1);
			\ifthenelse{\y>0 }{	\node[below] at (\y,-0.1) {\tiny \( \scriptstyle \y\)};}{
				\node[below] at (\y,-0.1) {\tiny 0};}
		}
		\pgfmathparse{3.5*\m} 
		\foreach \y in  {1,...,\pgfmathresult} {
			\ifthenelse{\y>0 }{	
				\draw[ line width=1, color=OrangeProfondIRA, arrows={-Stealth[length=4, inset=0]}] (\y/\m,0) -- (\y/\m,2/\m);}{}
		}
		\foreach \y in  {-3,...,0} {
			\draw (\y+\n,0) -- (\y+\n,-0.1);
			\ifthenelse{\y<0 }{\node[below] at (\y+\n,-0.1) {\tiny \(\scriptstyle n \y\)};}{}
			\ifthenelse{\y=0 }{\node[below] at (\y+\n,-0.1) {\tiny \(\scriptstyle n \)};}{}
			\ifthenelse{\y>0}{	\node[below] at (\y+\n,-0.1) {\tiny \(\scriptstyle n+\y\)};}{}
		}
		\pgfmathparse{3.5*\m} 
		\foreach \y  in  {1,...,\pgfmathresult} {
		\draw[ line width=1, color=OrangeProfondIRA, arrows={-Stealth[length=4, inset=0]}] (-\y/\m+\n,0) -- (-\y/\m+\n,2/\m);
		}
		\draw[ line width=1] (-.25,0) -- (3.5,0);
		\draw[ line width=1, dashed] (3.5,0) -- (5.5,0);
		\draw[arrows={-Stealth[length=4, inset=0]}, line width=1] (5.5,0) -- (\w,0);
	\end{tikzpicture}
\end{center}

 Si    \(i^{(m)}\) représente le taux d'intérêt (annuel) nominal équivalent avec \(m\) période annuelle \(i^{(m)}= m\left((1+i)^{1 / m}-1\right)\) .

De même \(d^{(m)} \) taux d'escompte nominal conforme à \(d\) et \(m\) : \(d^{(m)}= m\left(1-(1-d)^{1 / m}\right)\).

\[
\ddot{a}_{\lcroof{n}}^{(m)}=\frac{1}{m} \sum_{k=0}^{m n-1} v^{\frac{k}{m}}=\frac{d}{d^{(m)}} \ddot{a}_{\lcroof{n}}=\frac{1-v^{n}}{d^{(m)}} \approx \ddot{a}_{\lcroof{n}}+\frac{m-1}{2 m}\left(1-v^{n}\right)
\]
\[
a_{\lcroof{n}}^{(m)}=\frac{1}{m} \sum_{k=1}^{m n} v^{\frac{k}{m}}=\frac{i}{i^{(m)}} a_{\lcroof{n}}=\frac{1-v^{n}}{i^{(m)}} \approx a_{\lcroof{n}}-\frac{m-1}{2 m}\left(1-v^{n}\right)
\]
Annuité unitaire \(a\) : elle est utilisée lors de la construction de formules d'annuité.
Pour une annuité constante, le montant total versé chaque année est de~1, quelque soit \(m\).  

Annuité dynamique, croissante/décroissante \(Ia\)/\(Da\) : dans sa forme la plus simple, elle verse un montant qui démarre à 1 (\(n\)) et qui croit (décroit) à chaque période de façon arithmétique ou géométrique. Dans l'exemple suivant, la progression est arithmétique.
On utilise le préfixe \(I\) (\engl{increasing}) pour indiquer les annuités croissantes et \(D\) (\engl{decreasing}) pour les annuités dégressives.
	
\begin{center}
	\begin{tikzpicture}[scale=0.75]
		% Draw the x-axis and y-axis.
		\def\w{11}
		\def\n{9}
		\node[left] at (-.5,0) {\(Ia_{\lcroof{n}}\)};
		\foreach \y in  {0,...,3} {
			\draw (\y,0) -- (\y,-0.1);
			\ifthenelse{\y>0 }{	\node[below] at (\y,-0.1) {\tiny \( \scriptstyle \y\)};
				\draw[ line width=1, color=OrangeProfondIRA, arrows={-Stealth[length=4, inset=0]}] (\y,0) -- (\y,\y/3);}{
				\node[below] at (\y,-0.1) {\tiny 0};}
		}
		\foreach \y in  {-3,...,1} {
			\draw (\y+\n,0) -- (\y+\n,-0.1);
			\ifthenelse{\y<0 }{\node[below] at (\y+\n,-0.1) {\tiny \(\scriptstyle n \y\)};}{}
			\ifthenelse{\y=0 }{\node[below] at (\y+\n,-0.1) {\tiny \(\scriptstyle n \)};}{}
			\ifthenelse{\y>0}{	\node[below] at (\y+\n,-0.1) {\tiny \(\scriptstyle n+\y\)};}{}
			\ifthenelse{\y<1 }{	\draw[ line width=1, color=OrangeProfondIRA, arrows={-Stealth[length=4, inset=0]}] (\y+\n,0) -- (\y+\n,\y/3+\n/3);}
		}
		\draw[ line width=1] (-.25,0) -- (4,0);
		\draw[ line width=1, dashed] (4,0) -- (5,0);
		\draw[arrows={-Stealth[length=4, inset=0]}, line width=1] (5,0) -- (\w,0);
	\end{tikzpicture}
	\begin{tikzpicture}[scale=0.75]
	% Draw the x-axis and y-axis.
	\def\w{11}
	\def\n{9}
	\node[left] at (-.5,0) {\(Da_{\lcroof{n}}\)};
	\foreach \y in  {0,...,3} {
		\draw (\y,0) -- (\y,-0.1);
		\ifthenelse{\y>0 }{	\node[below] at (\y,-0.1) {\tiny \( \scriptstyle \y\)};
			\draw[ line width=1, color=OrangeProfondIRA, arrows={-Stealth[length=4, inset=0]}] (\y,0) -- (\y,\n/3-\y/3);}{
			\node[below] at (\y,-0.1) {\tiny 0};}
	}
	\foreach \y in  {-3,...,1} {
		\draw (\y+\n,0) -- (\y+\n,-0.1);
		\ifthenelse{\y<0 }{\node[below] at (\y+\n,-0.1) {\tiny \(\scriptstyle n \y\)};}{}
		\ifthenelse{\y=0 }{\node[below] at (\y+\n,-0.1) {\tiny \(\scriptstyle n \)};}{}
		\ifthenelse{\y>0}{	\node[below] at (\y+\n,-0.1) {\tiny \(\scriptstyle n+\y\)};}{}
		\ifthenelse{\y<1 }{	\draw[ line width=1, color=OrangeProfondIRA, arrows={-Stealth[length=4, inset=0]}] (\y+\n,0) -- (\y+\n,1/3-\y/3);}
	}
	\draw[ line width=1] (-.25,0) -- (4,0);
	\draw[ line width=1, dashed] (4,0) -- (5,0);
	\draw[arrows={-Stealth[length=4, inset=0]}, line width=1] (5,0) -- (\w,0);
\end{tikzpicture}
\end{center}

\begin{equation}
(I \ddot{a})_{\lcroof{n}}=1+2 v+\cdots+n v^{n-1}=\frac{1}{d}\left(\ddot{a}_{\lcroof{n}}-n v^{n}\right)
\label{Ian}
\end{equation}
avec, on le rappelle,  \(d=i/(1+i)\) et à terme échu (\engl{immediate})
\[
(I a)_{\lcroof{n}}=v+2 v^{2}+\cdots+n v^{n}=\frac{1}{i}\left(\ddot{a}_{\lcroof{n}}-n v^{n}\right)
\]
\[
(D \ddot{a})_{\lcroof{n}}=n+(n-1) v+\cdots+v^{n-1}=\frac{1}{d}\left(n-a_{\lcroof{n}}\right)
\]
et à terme échu :
\[
(D a)_{\lcroof{n}}=n v+(n-1) v^{2}+\cdots+v^{n}=\frac{1}{i}\left(n-a_{\lcroof{n}}\right)
\]



