%!TeX spellcheck = es_ES
%!TEX root = ActuarialFormSheet_MBFA-es.tex

%%Created by jPicEdt 1.4.1_03: mixed JPIC-XML/LaTeX format
%%Tue Jul 03 17:58:32 CEST 2012
%%Begin JPIC-XML
%<?xml version="1.0" standalone="yes"?>
%<jpic x-min="5" x-max="100" y-min="2.5" y-max="37.5" auto-bounding="true">
%<multicurve points= "(40,30);(40,30);(40,20);(40,20)"
%	 fill-style= "none"
%	 />
%<multicurve points= "(20,30);(20,30);(60,30);(60,30)"
%	 fill-style= "none"
%	 />
%<multicurve points= "(60,35);(60,35);(60,30);(60,30)"
%	 fill-style= "none"
%	 />
%<multicurve points= "(20,30);(20,30);(20,35);(20,35)"
%	 fill-style= "none"
%	 />
%<multicurve points= "(20,20);(20,20);(20,15);(20,15)"
%	 fill-style= "none"
%	 />
%<multicurve points= "(20,20);(20,20);(70,20);(70,20)"
%	 fill-style= "none"
%	 />
%<multicurve points= "(70,20);(70,20);(70,15);(70,15)"
%	 fill-style= "none"
%	 />
%<multicurve points= "(5,15);(5,15);(30,15);(30,15)"
%	 fill-style= "none"
%	 />
%<multicurve points= "(30,10);(30,10);(30,15);(30,15)"
%	 fill-style= "none"
%	 />
%<multicurve points= "(5,15);(5,15);(5,10);(5,10)"
%	 fill-style= "none"
%	 />
%<multicurve points= "(55,15);(55,15);(55,10);(55,10)"
%	 fill-style= "none"
%	 />
%<multicurve points= "(55,15);(55,15);(100,15);(100,15)"
%	 fill-style= "none"
%	 />
%<multicurve points= "(75,15);(75,15);(75,10);(75,10)"
%	 fill-style= "none"
%	 />
%<multicurve points= "(100,15);(100,15);(100,10);(100,10)"
%	 fill-style= "none"
%	 />
%<text text-vert-align= "center-v"
%	 anchor-point= "(5,7.5)"
%	 fill-style= "none"
%	 text-frame= "noframe"
%	 text-hor-align= "center-h"
%	 >
%Quote-Part
%</text>
%<text text-vert-align= "center-v"
%	 anchor-point= "(30,7.5)"
%	 fill-style= "none"
%	 text-frame= "noframe"
%	 text-hor-align= "center-h"
%	 >
%Excédent
%</text>
%<text text-vert-align= "center-v"
%	 anchor-point= "(30,2.5)"
%	 fill-style= "none"
%	 text-frame= "noframe"
%	 text-hor-align= "center-h"
%	 >
%de plein
%</text>
%<text text-vert-align= "center-v"
%	 anchor-point= "(55,7.5)"
%	 fill-style= "none"
%	 text-frame= "noframe"
%	 text-hor-align= "center-h"
%	 >
%Excédent
%</text>
%<text text-vert-align= "center-v"
%	 anchor-point= "(75,7.5)"
%	 fill-style= "none"
%	 text-frame= "noframe"
%	 text-hor-align= "center-h"
%	 >
%Excédent
%</text>
%<text text-vert-align= "center-v"
%	 anchor-point= "(100,7.5)"
%	 fill-style= "none"
%	 text-frame= "noframe"
%	 text-hor-align= "center-h"
%	 >
%Excédent Annuel
%</text>
%<text text-vert-align= "center-v"
%	 anchor-point= "(70,22.5)"
%	 fill-style= "none"
%	 text-frame= "noframe"
%	 text-hor-align= "center-h"
%	 >
%Non proportionnelle
%</text>
%<text text-vert-align= "center-v"
%	 anchor-point= "(20,22.5)"
%	 fill-style= "none"
%	 text-frame= "noframe"
%	 text-hor-align= "center-h"
%	 >
%Proportionnelle
%</text>
%<text text-vert-align= "center-v"
%	 anchor-point= "(20,37.5)"
%	 fill-style= "none"
%	 text-frame= "noframe"
%	 text-hor-align= "center-h"
%	 >
%Facultative
%</text>
%<text text-vert-align= "center-v"
%	 anchor-point= "(60,37.5)"
%	 fill-style= "none"
%	 text-frame= "noframe"
%	 text-hor-align= "center-h"
%	 >
%Traité
%</text>
%<text text-vert-align= "center-v"
%	 anchor-point= "(55,2.5)"
%	 fill-style= "none"
%	 text-frame= "noframe"
%	 text-hor-align= "center-h"
%	 >
%par sinistre
%</text>
%<text text-vert-align= "center-v"
%	 anchor-point= "(75,2.5)"
%	 fill-style= "none"
%	 text-frame= "noframe"
%	 text-hor-align= "center-h"
%	 >
%par événement
%</text>
%<text text-vert-align= "center-v"
%	 anchor-point= "(100,2.5)"
%	 fill-style= "none"
%	 text-frame= "noframe"
%	 text-hor-align= "center-h"
%	 >
%Stop Loss
%</text>
%</jpic>
%%End JPIC-XML
%LaTeX-picture environment using emulated lines and arcs
%You can rescale the whole picture (to 80% for instance) by using the command \def\JPicScale{0.8}
\ifx\JPicScale\undefined\def\JPicScale{1}\fi
\unitlength \JPicScale mm
\begin{picture}(100,45)(5,0)
\linethickness{0.3mm}
\put(40,20){\line(0,1){10}}
\linethickness{0.3mm}
\put(20,30){\line(1,0){40}}
\linethickness{0.3mm}
\put(60,30){\line(0,1){5}}
\linethickness{0.3mm}
\put(20,30){\line(0,1){5}}
\linethickness{0.3mm}
\put(20,15){\line(0,1){5}}
\linethickness{0.3mm}
\put(20,20){\line(1,0){50}}
\linethickness{0.3mm}
\put(70,15){\line(0,1){5}}
\linethickness{0.3mm}
\put(10,15){\line(1,0){20}}
\linethickness{0.3mm}
\put(30,10){\line(0,1){5}}
\linethickness{0.3mm}
\put(10,10){\line(0,1){5}}
\linethickness{0.3mm}
\put(50,10){\line(0,1){5}}
\linethickness{0.3mm}
\put(50,15){\line(1,0){45}}
\linethickness{0.3mm}
\put(70,10){\line(0,1){5}}
\linethickness{0.3mm}
\put(95,10){\line(0,1){5}}
\put(10,7.5){\makebox(0,0)[cc]{Cuota}}

\put(30,7.5){\makebox(0,0)[cc]{Superávit}}

\put(30,2.5){\makebox(0,0)[cc]{total}}

\put(50,7.5){\makebox(0,0)[cc]{Superávit }}

\put(70,7.5){\makebox(0,0)[cc]{Excedente de }}

\put(95,7.5){\makebox(0,0)[cc]{Pérdidas excesivas}}

\put(70,22.5){\makebox(0,0)[cc]{Exceso}}

\put(20,22.5){\makebox(0,0)[cc]{Proporcional}}

\put(20,37.5){\makebox(0,0)[cc]{Facultativo}}

\put(60,37.5){\makebox(0,0)[cc]{Tratado}}

\put(50,2.5){\makebox(0,0)[cc]{por siniestro}}

\put(70,2.5){\makebox(0,0)[cc]{evento}}

\put(95,2.5){\makebox(0,0)[cc]{pérdidas}}

\end{picture}
