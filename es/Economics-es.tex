% !TeX root = ActuarialFormSheet_MBFA-es.tex
% !TeX spellcheck = es_ES
\begin{f}[Concept d'utilité]
	L'utilité modélise les préférences d'un individu entre deux paniers de biens \(x\) et \(y\) dans un ensemble \(S\), via la relation \(x \succcurlyeq y\) (préféré ou indifférent).
	
	Une fonction \(U: S \rightarrow \mathbb{R}\) représente les préférences si :
	\[
	x \succcurlyeq y \iff U(x) \geq U(y)
	\]
	
	\textbf{Axiomes nécessaires à l'existence d'une fonction d'utilité :}
	\begin{enumerate}
		\item \textbf{Complétude :} Pour tout \(x, y \in S\), soit \(x \succcurlyeq y\), soit \(y \succcurlyeq x\)
		\item \textbf{Transitivité :} Si \(x \succcurlyeq y\) et \(y \succcurlyeq z\), alors \(x \succcurlyeq z\)
		\item \textbf{Continuité :} Si \(x_n \to x\) et \(y_n \to y\), et \(x_n \succcurlyeq y_n\) pour tout \(n\), alors \(x \succcurlyeq y\)
	\end{enumerate}
	
\end{f}


\begin{f}[Fonction d'utilité]
	Une fonction \(u : \mathbb{R}_+ \rightarrow \mathbb{R}\) représente les préférences d'un agent face à l'incertitude.
	
	\textbf{Critère d'utilité espérée :} L’agent préfère \(X\) à \(Y\) si :
	\[
	\mathbb{E}[u(X)] > \mathbb{E}[u(Y)]
	\]
	Il choisit \(X\) tel que \(\mathbb{E}[u(X)]\) soit maximale.
	
	\textbf{Propriétés de \(u\) :}
	\begin{itemize}
		\item \(u'\! > 0\) : l’agent préfère plus de richesse (monotonie)
		\item \(u'' < 0\) : l’agent est averse au risque (concavité)
	\end{itemize}
	
	\textbf{Exemples classiques :}
	\begin{itemize}
		\item \textit{Linéaire} (neutre au risque) : \(u(x) = x\)
		\item \textit{Logarithmique} : \(u(x) = \ln(x)\)
		\item \textit{CRRA}  (aversion relative constante): \(u(x) = \frac{x^{1 - \gamma}}{1 - \gamma}\), \(\gamma \ne 1\)
		\item \textit{CARA}  (aversion absolue constante): \(u(x) = -e^{-a x}\)
	\end{itemize}
	
\end{f}

\hrule

\begin{f}[Aversion au risque]
	Un agent est dit \textbf{averse au risque} (ou risquophobe) si :
	\[
	u(\mathbb{E}[X]) > \mathbb{E}[u(X)]
	\]
	Ce qui est équivalent à \(u\) concave, c.-à-d. \(u''(x) < 0\)
\end{f}

\begin{f}[Mesure de l'aversion au risque]
	\textbf{Indice d'aversion absolue} :
	\[
	A_a(x) = -\frac{u''(x)}{u'(x)}
	\]
	
	\textbf{Indice d'aversion relative} :
	\[
	A_r(x) = -x \cdot \frac{u''(x)}{u'(x)}
	\]
	
	\textbf{Inégalité de Jensen (cas concave)} :
	\[
	u(\mathbb{E}[X]) \geq \mathbb{E}[u(X)]
	\]
Avec l'égalité si et seulement si \(X\) est constante.
\end{f}
\hrule


\begin{f}[Primes de risque]
	
La \textbf{prime de risque} \(\pi\) est le montant maximal qu'un individu est prêt à payer pour remplacer une loterie de gain aléatoire \(H\) par son espérance certaine \(\mathbb{E}[H]\). Elle vérifie :
\[
\mathbb{E}[u(w + H)] = u(w + \mathbb{E}[H] - \pi)
\]


\(\pi\) est aussi appelée \emph{mesure de Markowitz} : elle capture l'écart entre utilité espérée et utilité certaine.

Inversement, la \textbf{prime compensatoire} \(\tilde{\pi}\) est le montant que l'on doit offrir à un individu pour qu’il accepte la loterie \(H\) au lieu d’un gain certain. Elle vérifie :
\[
\mathbb{E}[u(w + H + \tilde{\pi})] = u(w + \mathbb{E}[H])
\]

\end{f}
\hrule
\begin{f}[Diversification et utilité]
	
	Soient deux actifs :
	\begin{itemize}
		\item \(A\) : risqué
		\item \(B\) : certain, avec \(\mathbb{E}[A] = B\)
	\end{itemize}
	
	Un agent averse au risque préfère une combinaison \(Z = \alpha A + (1 - \alpha) B\), avec \(0 < \alpha < 1\), à l’actif risqué seul.
	Si  \(u\) est concave, alors
	\[
	\mathbb{E}[u(Z)] > \mathbb{E}[u(A)]
	\]
	
	\textbf{Portefeuille optimal} : choix des pondérations \((w_i)\) maximisant l'utilité espérée :
	\[
	\max \mathbb{E}[u(X)], \quad \text{où } X = \sum_{i} w_i X_i, \quad \text{s.c. } \sum w_i = 1
	\]
	
	\textbf{Principe} : la diversification réduit le risque (variance) sans affecter l’espérance.
	
	
\end{f}
\hrule

\begin{f}[Méthode de Lagrange pour l'optimisation sous contrainte]
	
	La méthode des multiplicateurs de Lagrange permet de résoudre un problème d'optimisation sous contrainte.
	
	\textbf{Objectif} : maximiser/minimiser \(f(\boldsymbol{x})\) sous la contrainte \(g(\boldsymbol{x}) = c\), où \(\boldsymbol{x} \in \mathbb{R}^d\) est un vecteur de variables.
	
	\textbf{Étapes de la méthode :}
	\begin{enumerate}
		\item \textbf{Identification} : déterminer la fonction objectif \(f(\boldsymbol{x})\) et la contrainte \(g(\boldsymbol{x}) = c\)
		\item \textbf{Lagrangien} :
		\[
		\mathcal{L}(\boldsymbol{x}, \lambda) = f(\boldsymbol{x}) + \lambda (g(\boldsymbol{x}) - c)
		\]
		\item \textbf{Système d'équations} : résoudre
		\[
		\nabla_{\boldsymbol{x}} \mathcal{L} = \nabla f(\boldsymbol{x}) + \lambda \nabla g(\boldsymbol{x}) = \boldsymbol{0}, \quad
		\frac{\partial \mathcal{L}}{\partial \lambda} = g(\boldsymbol{x}) - c = 0
		\]
		\item \textbf{Résolution} du système pour obtenir \(\boldsymbol{x}^*, \lambda^*\)
		\item \textbf{Vérification} : s'assurer que les solutions satisfont bien la contrainte et le type d'optimum (max/min)
	\end{enumerate}
	
	\textbf{Exemple (dimension 2)} : maximiser \(f(x, y) = xy\) sous la contrainte \(x + y = 10\)
	
	\[
	\mathcal{L}(x, y, \lambda) = xy + \lambda (x + y - 10)
	\]
	
	On dérive :
	\[
	\frac{\partial \mathcal{L}}{\partial x} = y + \lambda = 0, \quad
	\frac{\partial \mathcal{L}}{\partial y} = x + \lambda = 0, \quad
	\frac{\partial \mathcal{L}}{\partial \lambda} = x + y - 10 = 0
	\]
	
	On résout le système :
\[
\begin{cases}
	y + \lambda = 0 \\
	x + \lambda = 0 \\
	x + y = 10
\end{cases}
\Rightarrow 
\begin{cases}
	\lambda = -y \\
	x = -\lambda = y \\
	x + y = 10 \Rightarrow 2x = 10 
\end{cases}
\Rightarrow 
	\begin{cases} 
		x^* =  y^*= 5,\\
		f(5,5) = 25
	\end{cases}
\]

		\textbf{Exemple (Choix optimal et contrainte budgétaire)}
	
	Un agent rationnel est face à un choix de consommation \((c_1, c_2)\) entre deux biens, sous la contrainte :
	\[
	p_1 c_1 + p_2 c_2 = R
	\]
	où \(p_1, p_2\) sont les prix et \(R\) le revenu total.
	
	\textbf{Problème} :
	\(
	\max_{c_1, c_2} u(c_1, c_2) \quad \text{s.c. } p_1 c_1 + p_2 c_2 = R\)

	
	\textbf{Méthode} : introduire le \textbf{lagrangien}
	\[
	\mathcal{L}(c_1, c_2, \lambda) = u(c_1, c_2) + \lambda (R - p_1 c_1 - p_2 c_2)
	\]
	
	\textbf{Conditions du premier ordre (FOC)} :
	\[
	\begin{cases}
		\frac{\partial u}{\partial c_1} = \lambda p_1 \\
		\frac{\partial u}{\partial c_2} = \lambda p_2 \\
		p_1 c_1 + p_2 c_2 = R
	\end{cases}
	\]
	
	En divisant les deux premières équations :
	\[
	\frac{\partial u / \partial c_1}{\partial u / \partial c_2} = \frac{p_1}{p_2}
	\]
	
	Ce rapport est appelé \textbf{taux marginal de substitution (TMS)} : il mesure la quantité de bien 2 à laquelle l’agent est prêt à renoncer pour obtenir une unité supplémentaire de bien 1, tout en maintenant son niveau d’utilité constant.
	
\end{f}
\hrule

\begin{f}[Demande d’assurance (Mosin)]
	
Un agent possède un patrimoine initial \(w\) et fait face à une perte aléatoire \(L\). Il existe une \textbf{demande d'assurance} pour l'assurance qui verse l'indemnité \(0<I(L)<L\) ssi \(u(w-\pi_I) \geq \mathbb{E}(u(w-L))\) et l'\textbf{assurance optimale} maximise \(u(w-\pi_I)\).

Dans Mosin (1968) ou  Borch (1961) ou Smith (1968), le modèle de perte \(L\) se définit simplement par  \(s\) compris entre 0 et  \(w\) :
	\[
	L=\left\{\begin{array}{l}
		0 \text { avec proba. } 1-p \\
		s \text { avec proba. } p
	\end{array}\right.
	\]
La prime devient \(\pi_I=(1+\lambda) \mathbb{E}(I(L))=(1+\lambda) p I(s)\) avec \(\lambda\) un chargement. On note \(\pi\) le cas où \(I(L)=L\) avec  \(\pi= p s\). Si \(\lambda=0\), alors on parle de prime pure ou actuariellement juste. 
\medskip

\textbf{Coassurance} (partage du risque) :
 \(I(l)=\alpha l\) sachant \(L=l\) pour \(\alpha \in[0,1]\), \(\pi_I(\alpha)=\alpha \pi\) et :
\[
w_{f}=w-L+I(L)-\pi(\alpha)=w-L+\alpha L-\alpha \pi=w-(1-\alpha) L-\alpha \pi
\]
\[
U(\alpha)%=\mathbb{E}(u(w-(1-\alpha) L-\alpha \pi))
=(1-p) u(w-\alpha \pi)+p u(w-(1-\alpha) s-\alpha \pi)
\]

L'assurance partielle (\(\alpha^{\star}<1\)) est optimale ssi \(\lambda>0\). L'assurance totale (\(\alpha^{\star}=1\) ) est optimale si le chargement est nul.
\medskip

\textbf{L'assurance avec franchise} (part d'autoassurance):
Avec franchise \(d\) l'assureur verse une indemnité \(I(l)=(l-d)_{+}\) sachant \(L=l\). 
\[
\pi(d)=(1+\lambda) E\left((L-d)_{+}\right)=(1+\lambda)(s-d) p
\]
\[
w_{f}=w-X+(L-d)_{+}-\pi(d)=w-\min (X, d)-(1+\lambda)(s-d) p
\]
\[
U(d)=(1-p) u(\underbrace{w+(1+\lambda)(d-s) p}_{w_{f}^{+}})+p u(\underbrace{w-d+(1+\lambda)(d-s) p}_{w_{f}^{-}}) .
\]
Dans le modèle avec franchise, l'assurance partielle (\(d^{\star}>0\)) est optimale ssi la prime n'est pas actuariellement juste. De même,  l'assurance totale ( \(d^{\star}=0\) ) est optimale si le chargement est nul.
\medskip

\textbf{Modèle généralisé :}
Le risque de perte \(L>0\) aléatoire est définit sur \(\Re\), avec fonction de répartition \(F_L\) ), 

\[
\pi_I=(1+\lambda) \mathbb{E}(I(L))=(1+\lambda) \int_{0}^{\infty} I(l) d F_{L}(l)
\]



\begin{enumerate}
	\item L'assurance totale \((d^\star = 0)\) ou  \((\alpha^\star = 1)\) est optimale si et seulement si la prime est actuariellement juste .
	
	\item Si \(A_a(u, x)\) est décroissante, alors le niveau de franchise \(d^\star\)  ou  le taux de couverture \(\alpha^\star\) augmente avec la richesse initiale.  
	Pour les préférences CARA, \(d^\star\) est indépendant de \(w\) ou \(\alpha^\star\) est constant .
	
	\item Le niveau de couverture  décroît avec le coefficient de chargement \(\lambda\) lorsque \(A_a(u, x)\) est croissante ou constante.
	
	\item Un agent plus averse au risque choisit  une couverture plus élevée.
\end{enumerate}
\end{f}

\hrule

\begin{f}[Information et assurance]
	
	\textbf{Mosin avec hétérogénéïté :} Deux types d'individus : \(H\) pour haut risque et \(Lo\) pour faible risque. \(\theta \in[0,1]\) la proportion d'individu \(H\). Les individus de type \(i \in\{Lo, H\}\) font face à un risque de même montant \(s\)  survenant avec une probabilité différente \(p_{i}\) telle que \(1>p_{H}>p_{Lo}>0\).
	
	\[
	L_{i}=\left\{\begin{array}{l}
		0 \text { avec probabilité } 1-p_{i}, \\
		s \text { avec probabilité } p_{i} .
	\end{array}\right.
	\]
La probabilité du marché :
	\[
	p_{m}=\theta p_{H}+(1-\theta) p_{Lo} .
	\]
\medskip

\textbf{Absence d'antisélection :}  Dans ce modèle, en présence d'information totale, l'assureur préfère une assurance individuelle \(I_{i}=s\) et \(\pi_{i}=s p_{i}\), \(\forall i\).

\textbf{Le problème d'anti-sélection : }
L'assureur propose un  contrat non individualisé du marché \(M=\left(\pi_{m}=p_{m} I, I_{m}(s)=I(s)\right)\), qui ne dépend pas de \(i\). La fortune finale d'un individu de type \(i\) est \(W_{i}^{m}=w-\pi_{m}-X_{i}+I_{m}\).

En présence d'un contrat unique, les individus de type \(H\) préfèrent un contrat d'assurance tel que \(I_{H}(s)=s\)  et \(\pi_{H}=s p_{m}\), tandis que les individus de type \(Lo\) préfèrent une couverture partielle avec \(I_{L}^{\star}<s\) et \(\pi_{Lo}=I_{L}^{\star} p_{m}\).
\medskip

\textbf{Aléa moral :}
L'assurer réduit ou interrompt ses efforts maintenant qu'il est assuré. Les efforts de
\begin{itemize}
	\item prévention réduisent la probabilité de sinistre,
	\item protection réduisent le montant de sinistre.
\end{itemize}

 En l'absence d'effort \(e\) de prévention ou de réduction du risque, la fortune finale \(w_{f}\) est simplement définie par

\[
\begin{cases}w_{f}^{-}=w-s-\pi(I)+I & \text { avec probabilité } p \\ w_{f}^{+}=w-\pi(I) & \text { avec probabilité } 1-p\end{cases}
\]

S'il y a un effort \(e\) de prévention des risques, on considère

\[
\begin{cases}w_{f}^{-}=w-s-\pi(I)+I-e & \text { avec probabilité } p(e) \\ w_{f}^{+}=w-\pi(I)-e & \text { avec probabilité } 1-p(e)\end{cases}
\]

S'il y a un effort \(e\) de protection des risques, on considère

\[
\begin{cases}w_{f}^{-}=w-s(e)-\pi(I)+I-e & \text { avec probabilité } p \\ w_{f}^{+}=w-\pi(I)-e & \text { avec probabilité } 1-p\end{cases}
\]

avec
\begin{itemize}
	\item \(e \mapsto p(e)\) est strictement décroissante et strictement convexe.
	\item \(e \mapsto s(e)\) est strictement décroissante et strictement convexe.\\
	\item \(I \leq s\) implique \(w_{f}^{-} \leq w_{f}^{+}\)
\end{itemize}

\end{f}
\hrule


