% !TeX root = ActuarialFormSheet_MBFA-es.tex
% !TeX spellcheck = es_ES
\def\scaleBS{.95}

\begin{f}[Funcionamiento del mercado]
	
La \textbf{Bolsa} (\engl{Exchange}) - un lugar de intercambio – permite, de hecho, el encuentro físico entre demandantes y oferentes de capital.
Las principales cotizaciones se refieren a \textbf{acciones}, (\engl{equities}), los \textbf{bonos} (\engl{Fixed Income}) y las \textbf{materias primas} (\engl{commodities}).
Se enumeran \textbf{valores} como acciones o bonos, \textbf{fondos} ((\engl{Exhange Trade Funds} que replican índices de acciones, ETC o ETN que replican índices o materias primas más específicos, SICAV o FCP, bonos de suscripción, \engl{warrant}), \textbf{contratos de futuros}, \textbf{opciones}, \textbf{swaps} y \textbf{productos estructurados}.

L'\textbf{Comisión Nacional del Mercado de Valores} (\href{https://www.cnmv.es/Portal/home}{CNMV}) vela por :
\begin{itemize}
	\item la protección de los ahorros invertidos;
	\item la información de los inversores;
	\item el buen funcionamiento de los mercados.
\end{itemize}

\textbf{\href{https://www.euronext.com/fr}{Euronext}} (incluyendo
\href{https://www.euronext.com/en/markets/amsterdam}{Amsterdam}, 
\href{https://www.euronext.com/en/markets/brussels}{Brussels}, 
\href{https://www.euronext.com/en/markets/lisbon}{Lisbon}, 
y \href{https://www.euronext.com/en/markets/paris}{Paris}) es la principal bolsa de valores de Francia. 
%
Sus competidores incluyen \textbf{\href{https://www.deutsche-boerse.com/dbg-en/}{Deutsche Börse}} 
(que incluye \href{https://www.eurex.com/}{Eurex}, 
\href{https://www.eex.com/en/}{EEX}) en Europa, 
%
or \textbf{\href{https://www.ice.com/}{ICE}} (que incluye
\href{https://www.nyse.com/index}{NYSE (2012)}, 
\href{https://www.ice.com/about/history}{NYBOT (2005)}, 
\href{https://www.ice.com/futures-europe}{IPE (2001), LIFFE}) 
%
y \textbf{\href{https://www.cmegroup.com/}{CME Group}} (incluyendo
\href{https://www.cmegroup.com/company/cbot.html}{CBOT}, 
\href{https://www.cmegroup.com/company/nymex.html}{NYMEX}, 
\href{https://www.cmegroup.com/company/comex.html}{COMEX}) en los Estados Unidos.

El \textbf{mercado extrabursátil} (\textbf{OTC}) representa una parte importante de los volúmenes negociados fuera de los mercados organizados.
Depuis le G20 de Pittsburgh (2009), certains dérivés OTC standardisés doivent être compensés via une entité centrale.
Estas \textbf{CCP} (entidades de contrapartida central) desempeñan así el papel de \textbf{cámaras de compensación}: sustituyen el contrato bilateral por dos contratos entre cada parte y la CCP.

\begin{tikzpicture}[scale=.47]
	\node[businessman,minimum size=1.75cm,monogramtext=MPG,tie=OrangeProfondIRA, shirt=BleuProfondIRA,skin=white,hair=OrangeProfondIRA] at (-8,0) (MPG){Posición corta};
	\node[businessman,mirrored,minimum size=1.75cm,monogramtext=SPG,shirt=FushiaIRA, tie=black,skin=FushiaIRA, hair=black] at (8,0) (SPG){Posición larga};
	%  	
	\begin{scope}[yscale=-0.03, xscale=0.03, shift={(-150,-200)}]
		%  		
		%Usine
		\draw [OrangeMoyenIRA!40,fill] (65,132) rectangle (195,225)  node [pos=0.5,text width=1.8cm, align=center] {\color{black}\bfseries Cámara \-de comp\-ensación} ;
		%Straight Lines [id:da13081779728777443] 
		%	\draw (65,272) -- (474,272) -- (474,320);
		%Shape: Polygon [id:ds025531371089408395] 
		\draw   [GrisLogoIRA] (195,272) -- (65,272) -- (65,129) -- (85,129) -- (85,20) -- (117,20)  -- (117,52)-- (175,52) -- (175,129) -- (195,129) -- cycle ;
		%Shape: Grid [id:dp09099790819019526] 
		%	\draw  [draw opacity=0] (117,129) -- (195,129) -- (195,169) -- (117,169) -- cycle ; 
		\draw  [GrisLogoIRA] (95,100) -- (95,60)(105,100) -- (105,60)(115,100) -- (115,60)(125,100) -- (125,60)(135,100) -- (135,60)(145,100) -- (145,60)(155,100) -- (155,60)(165,100) -- (165,60) ; 
		\draw  [GrisLogoIRA]  (85,100) -- (175,100)(85,90) -- (175,90)(85,80) -- (175,80)(85,70) -- (175,70)(85,60) -- (175,60) ; 
		%
		\draw  (65,175) node (a) {} ;
		\draw  (195,175) node (b) {} ;
		%Shape: Rectangle [id:dp30583722611654374] 
	\end{scope}
	% 	
	\draw[->,>=latex, thick] (MPG) to [bend left]  node[pos=0.4, above]{Corto} (a);
	\draw[<-,>=latex, thick,OrangeProfondIRA] (MPG) to [bend right]  node[pos=0.4, below=.2]{Pago} (a);
	\draw[->,>=latex, thick,FushiaIRA] (SPG) to [bend left] node[pos=0.4, below=.2]{No pago}  (b);
	\draw[->,>=latex, thick] (SPG) to [bend right] node[pos=0.4, above]{Largo} (b);
	%  	
\end{tikzpicture}
\medskip

\end{f}
\hrule

\begin{f}[El mercado monetario]
Los valores a corto plazo que devengan intereses y que se negocian en los mercados monetarios generalmente tienen un \textbf{interés descontado}.
Las tasas nominales son entonces anuales y los cálculos utilizan \textbf{tasas proporcionales} para ajustarse a duraciones inferiores a un año.
Estos valores se cotizan o valoran según el principio de descuento y con un calendario Euro-30/360.

En el mercado estadounidense los títulos de deuda pública se denominan :
Letras del Tesoro (T-bills) : ZC < 1 año, Notas del Tesoro (T-notes): ZC < 10 años,
Bonos del Tesoro (T-bonds) : bonos con cupón y vencimiento > 10 años.


Ellas son principalmente :
\begin{itemize}
	\item \textbf{BTF (letras del Tesoro a tipo fijo, Francia) :} emitidas a 13, 26 y 52 semanas, tipo de interés descontado, subasta semanal, nominal 1~€, liquidación en T+2.
	
	\item \textbf{Letras del Tesoro > 1 año :} mismas reglas que los bonos (ver siguiente sección).
	
	\item \textbf{Certificados de depósito (CDN) :} valores emitidos por bancos a tasa fija/descontada (corto plazo) o variable/post-descontada (largo plazo), también llamados BMTN.
	
	\textbf{Eurodólares :} depósitos en dólares estadounidenses fuera de los EE. UU., anteriormente indexados a la LIBOR, ahora están disminuyendo.
	
	\item \textbf{Papel comercial :} valores a corto plazo no garantizados emitidos por grandes empresas para financiar su flujo de caja.
\end{itemize}

\textbf{Cálculos de precios de una letra del Tesoro a tipo fijo con interés descontado}

En el caso de un título con interés descontado según la convención Euro-30/360, el descuento \(D\) se expresa como :
\[
D=F \cdot d \cdot \frac{k}{360}
\]
donde \(F\) denota el valor nominal, \(d\) la tasa de descuento anual utilizada para valorar el título descontado, y \(k\) el vencimiento en días.

Si se conoce la tasa de descuento \(d\), entonces el precio \(P\) se expresa como :
\[
P=F-D=F\left(1-d \cdot \frac{k}{360}\right)
\]
De manera similar, si se conoce el precio \(P\), entonces la tasa de descuento \(d\) se deriva como :
\[
d=\frac{F-P}{F} \cdot \frac{360}{k}
\]


Principales contratos de futuros : Futuros de Fondos Federales (EE. UU.)
Futuros SOFR a tres meses (EE. UU.),
Futuros ESTR (UE),
Futuros SONIA (UK),
Futuros Euribor (UE).

\end{f}
\hrule



\begin{f}[Mercado de bonos]

Los \textbf{bonos} son títulos de deuda a largo plazo en los que el emisor (gobierno central o local, banco, empresa prestataria) promete al tenedor del bono (el prestamista) pagar intereses (\textbf{cupones}) periódicamente y reembolsar el \textbf{valor nominal} (o valor facial, o capital) al vencimiento.
Como se mencionó en la sección anterior, las \textbf{letras del Tesoro con vencimiento mayor a un año} serán tratadas como bonos con vencimiento menor a 5 años porque su funcionamiento es similar.

%Les \textbf{coupons} : sont payés régulièrement à la fin des périodes de coupon (annuelles ou semestrielles) jusqu'à la date d'échéance.

\textbf{Bonos cupón cero} : pagan solo el valor nominal al vencimiento. Con \(E\) el precio de emisión y \(R\) su valor de rescate :
	
\begin{center}
\begin{tikzpicture}[scale=0.85]
    % Draw the x-axis and y-axis.
    \def\w{11}
    \def\n{9}
    \draw[ line width=1, color=OrangeProfondIRA, arrows={-Stealth[length=4, inset=0]}] (0,0) -- (0,-0.4909) node[above left] (A) {\(E\)};	
    \foreach \y in  {0,...,3} {
        \draw (\y,0) -- (\y,-0.1);
        \ifthenelse{\y>0 }{	\node[below] at (\y,-0.1) {\tiny \( \scriptstyle \y\)};
        }{
            \node[above] at (\y,-0.1) {\tiny 0};}
    }
    \foreach \y in  {-3,...,1} {
        \draw (\y+\n,0) -- (\y+\n,-0.1);
        \ifthenelse{\y<0 }{\node[below] at (\y+\n,-0.1) {\tiny \(\scriptstyle n \y\)};}{}
        \ifthenelse{\y=0 }{\node[below] at (\y+\n,-0.1) {\tiny \(\scriptstyle n \)};
            \draw[ line width=1, color=OrangeProfondIRA, arrows={-Stealth[length=4, inset=0]}] (\y+\n,0) -- (\y+\n,1) node[below right] (B) {\(R\)};}{}
        \ifthenelse{\y>0}{	\node[below] at (\y+\n,-0.1) {\tiny \(\scriptstyle n+\y\)};}{}
    }
    \draw[ line width=1] (-.25,0) -- (4,0);
    \draw[ line width=1, dashed] (4,0) -- (5,0);
    \draw[arrows={-Stealth[length=4, inset=0]}, line width=1] (5,0) -- (\w,0);
\end{tikzpicture}
\end{center}

\textbf{Bonos con cupón} :
Los \textbf{bonos a tipo fijo} tienen un tipo de interés cupón que se mantiene constante hasta el vencimiento. Suponiendo un \emph{reembolso} de tipo de interés variable, con \(E\) como precio de emisión, \(c\) como cupones y \(R\) como valor de rescate, se puede ilustrar de la siguiente manera. :
\begin{center}
\begin{tikzpicture}[scale=0.85]
    % Draw the x-axis and y-axis.
    \def\w{11}
    \def\n{9}
    \draw[ line width=1, color=OrangeProfondIRA, arrows={-Stealth[length=4, inset=0]}] (0,0) -- (0,-1.8) node[above left] (A) {\(E\)};	
    \foreach \y in  {0,...,3} {
        \draw (\y,0) -- (\y,-0.1);
        \ifthenelse{\y>0 }{	\node[below] at (\y,-0.1) {\tiny \( \scriptstyle \y\)};
        \draw[ line width=1, color=OrangeProfondIRA, arrows={-Stealth[length=4, inset=0]}] (\y,0) -- (\y,.25) node [above] {\(c\)};
        }{
            \node[above] at (\y,-0.1) {\tiny 0};}
        }
    \foreach \y in  {-3,...,1} {
        \draw (\y+\n,0) -- (\y+\n,-0.1);
        \ifthenelse{\y<0 }{\node[below] at (\y+\n,-0.1) {\tiny \(\scriptstyle n \y\)};
        \draw[ line width=1, color=OrangeProfondIRA, arrows={-Stealth[length=4, inset=0]}] (\y+\n,0) -- (\y+\n,.25) node [above] {\(c\)};}{}
        \ifthenelse{\y=0 }{\node[below] at (\y+\n,-0.1) {\tiny \(\scriptstyle n \)};
        \draw[ line width=1, color=OrangeProfondIRA, arrows={-Stealth[length=4, inset=0]}] (\y+\n,0) -- (\y+\n,2) node[below right] (B) {\(R\)};
        \draw[ line width=1, color=OrangeProfondIRA, arrows={-Stealth[length=4, inset=0]}] (\y+\n,2) -- (\y+\n,2.25) node[right] (B) {\(c\)};}{}
        \ifthenelse{\y>0}{	\node[below] at (\y+\n,-0.1) {\tiny \(\scriptstyle n+\y\)};}{}
    }
    \draw[ line width=1] (-.25,0) -- (4,0);
    \draw[ line width=1, dashed] (4,0) -- (5,0);
    \draw[arrows={-Stealth[length=4, inset=0]}, line width=1] (5,0) -- (\w,0);
\end{tikzpicture}
\end{center}

 Los \textbf{bonos indexados} (bonos vinculados a la inflación) tienen cupones y, a veces, también el valor nominal indexado a la inflación u otro indicador económico, como los bonos asimilables del Tesoro.
		indexado a la inflación (OATi). Los valores de \(c\) varían.
        
Bonos con \textbf{tipo de interés flotante}, \textbf{tipo de interés variable} o \textbf{tipo de interés reajustable} : tienen un tipo de cupón vinculado a un tipo de interés de referencia (por ejemplo, el euro short-
		term rate (\href{https://www.cmegroup.com/markets/interest-rates/stirs/euro-short-term-rate.quotes.html}{€STR})).

Los \textbf{bonos perpetuos} no tienen fecha de vencimiento; el capital nunca se devuelve.
 
A menudo se hace una distinción entre los bonos del Estado (bonos del Tesoro) y
 bonos corporativos emitidos por empresas privadas.



Un bono se define principalmente por un \textbf{valor nominal} \(F\) (valor nominal), la \textbf{tasa nominal} \(i\), su duración o \textbf{vencimiento} \(n\).
En el caso predeterminado, el tenedor del bono presta la cantidad \(E=F\) en la emisión en el momento 0, recibe cada año un cupón \(c = i \times F\) y en \(n\) se devuelve el principal o capital \(R=F\).
Cuando \(E=F\), se dice que la emisión está a la par, y cuando \(R=F\), se dice que la redención está a la par.


El precio de un bono se determina por el valor actual de los flujos de efectivo futuros esperados (cupones y reembolso de capital) descontados a la tasa de rendimiento del mercado \(r\).

El cálculo del precio de los bonos se basa simplemente en la fórmula del valor actual :
\[
VP = \sum_{k=1}^{n} \frac{c}{(1 + r)^k} + \frac{F}{(1 + r)^n}
 \]
donde :
\begin{itemize}
	\item \(VP\) : precio o valor actual \index{Present Value} del bono,
	\item \(r\) : tasa de interés de mercado para el vencimiento correspondiente.
\end{itemize}


Para los bonos con cupones periódicos, el cupón se divide por el número de períodos (\(m\)) por año y la fórmula se convierte en :
\[ 
VP = \sum_{k=1}^{mn} \frac{c/m}{(1 + r^{(m)})^k} + \frac{R}{(1 + r^{(m)})^{mn}}
 \]
donde \(c/m\) representa el pago periódico del cupón y \(r^{(m)}\) la tasa de interés periódica.

El rendimiento del bono es el valor \(r^{(m)}\), la tasa equivalente de \(r\) durante \(m\) períodos en el año, que iguala el valor actual \(VP_r\) con el precio actual o de mercado de este bono. 

La cotización de un bono se expresa como porcentaje. Por lo tanto, una cotización de 97.9 en Euronext indica un valor cotizado de \(97.9 / 100 \times F\). 
Se cotiza excluyendo \textbf{cupones acumulados}, la parte del siguiente cupón a la que tiene derecho el vendedor si el bono se vende antes del pago de ese cupón.
\end{f}
\hrule

\begin{f}[Duración \& convexidad]

La duración de Macaulay :
\[ 		
D = \sum_{t=1}^{n} t \cdot w_t, \quad \text{où} \quad w_t = \frac{PV(C_t)}{P}.
 \]	
Si la frecuencia de pago es \(k\) por año, la duración expresada en años se obtiene dividiendo entre \(k\).
La duración modificada \(D^*\) :
\[ 	
D^* = \frac{D}{1 + i}.
 \]
Lo que permite aproximar el cambio de cartera \(\Delta P\) en caso de cambios en la tasa de interés \(\Delta_i\)
\[ 
\Delta P \approx -P\ D^* \Delta_i 
 \]
De manera similar, la convexidad
\[ 	
C = \frac{1}{P(i)} \times \frac{d^2 P(i)}{di^2},
 \]
lo que permite refinar la aproximación de \(\Delta P\)
\[ 	
P(i + \Delta_i) \approx P(i) \left( 1 -D^*\Delta_i + \frac{1}{2} C (\Delta_i)^2 \right).
 \]
\end{f}
\hrule



\begin{f}[CAPM]
  El CAPM (\textit{Capital Asset Pricing Model})  :

\[
E(r_i) = r_f + \beta_i (E(r_m) - r_f)
\]

\begin{itemize}
    \item \( E(r_i) \) es el rendimiento esperado del activo \( i \),
    \item \( r_f \) es la tasa libre de riesgo,
    \item \( E(r_m) \) es el rendimiento esperado del mercado,
    \item \( \beta_i \) es el coeficiente de sensibilidad del activo \( i \) con respecto a las variaciones del mercado.
\end{itemize}

El coeficiente \( \beta_i \) mide la volatilidad del activo \( i \) en relación con el mercado general.

\end{f}
\hrule


\begin{f}[Mercado de derivados]

Un \textbf{contrato derivado} (o activo contingente) es un instrumento financiero cuyo valor depende de un activo subyacente o variable. Las opciones forman parte de los contratos derivados.


Una \textbf{opción} es un contrato que otorga el derecho (sin obligación) de comprar (call) o vender (put) un activo subyacente a un precio fijo (precio de ejercicio) en una fecha futura, a cambio del pago de una prima.
El comprador (posición larga) paga la prima; el vendedor (posición corta) la recibe. \textbf{Opción europea} (ejercicio posible solo al vencimiento) y   
 \textbf{Opción americana} (ejercicio posible en cualquier momento hasta el vencimiento).

Las opciones que cotizan en acciones se denominan \textit{stock options}.

\end{f}
\hrule


\begin{f}[Estrategias simples]

\ %

%\textbf{La position longue sur l'option d'achat}

Con \(T\) el vencimiento, \(K\) el precio de ejercicio, \(S\) o \(S_T\) el subyacente al vencimiento, la recompensa (\engl{payoff}) es \(\max (0, S_T-K)=( S_T-K)^{+}\).
Si \(C\) es la prima, la ganancia obtenida es \(\max (0, S_T-K)-C\), con una ganancia si (\(S_T<V_{PM}=K + C\)) (\(PM\) representa el \textbf{punto de equilibrio}).

		\begin{tikzpicture}[yscale=.75]
\def\riskfreeBS{0.05}
\def\xminBS{5}
\def\xmaxBS{15}
\def\PxExerciceBS{10}
\def\sigmaBS{0.2}
\def\TBS{0.75}
\def\PremiumBS{{BSCall(10,{\PxExerciceBS},{\riskfreeBS},{\TBS},{\sigmaBS})}}
\begin{axis}[ 	extra tick style={tick style=BleuProfondIRA},
clip=false,
axis on top,
axis lines=middle, axis line style={BleuProfondIRA,thick,->},
scale only axis, xmin={\xminBS},xmax={\xmaxBS},enlarge x limits=0.05,
enlarge y limits=0.08,
color=BleuProfondIRA,
%		ylabel near ticks,
ylabel={Profit},
x label style={at={(axis cs:\xmaxBS+.1,0)},anchor=north east},
xlabel={subyacente (\(T\))},
%		    x label style={at={(axis description cs:0.5,-0.1)},anchor=north},
%		y label style={at={(axis description cs:-0.1,.5)},rotate=90,anchor=south},
ytick=\empty,
xtick=\empty,
extra y ticks ={0},
extra y tick labels={{0}},
extra x ticks ={\PxExerciceBS},
extra x tick labels={{\(E\)}},
extra x tick style={color=BleuProfondIRA,
	tick label style={yshift=7mm}	},
title ={ \textbf{Long Call}},
%			title style={yshift=-5mm},
%	legend style={draw=none,
	%		legend columns=-1,
	%		at={(0.5,1)},
	%		anchor=south,
	%		outer sep=1em,
	%		node font=\small,
	%	},
]
%		
\addplot[name path=A,BleuProfondIRA,thick,domain={{\xminBS}:{\xmaxBS}}, samples=21,dashdotted] {Call(x,\PxExerciceBS,\PremiumBS)} 
node [pos=0.15,yshift=3mm,color=OrangeProfondIRA] {Pérdidas}
node [pos=.8,yshift=-15mm,color=BleuProfondIRA] {Ganancias};	
%\draw[BleuProfondIRA,thick] \OV{\PxExerciceBS}{\PremiumBS}{0.4}{\xminBS}{\xmaxBS} ;
%		\addplot[name path=Option,BleuProfondIRA,thick,domain={250:350}, 			 		samples=10,dashdotted,smooth] {BSPut(x,\PxExerciceBS,\riskfreeBS,\TBS,\sigmaBS)} ;	
\path [save path=\xaxis,name path=xaxis]
({\xminBS},0)		-- ({\xmaxBS},0)		;
\addplot [bottom color=OrangeProfondIRA!50, top color=OrangeProfondIRA!10] fill between [
of=A and xaxis,
split,
every segment no 1/.style=
{top color = BleuProfondIRA!50, bottom color=BleuProfondIRA!10}] ;
%\draw[use path=\xaxis, ->,OrangeProfondIRA,thick];
\draw [fill]  ({-\PremiumBS*(-1)+\PxExerciceBS},0) circle (1mm) node [above=5mm] {\(S_{PM}\)};
\draw[BleuProfondIRA, thin] ({\PxExerciceBS},0) -- ({\PxExerciceBS},{-\PremiumBS});
\node at ({\xminBS+0.25*(\xmaxBS-\xminBS)},{\pgfkeysvalueof{/pgfplots/ymin}})
{OUT};
\node at ({\PxExerciceBS},{\pgfkeysvalueof{/pgfplots/ymin}})
{AT};
\node at ({\xminBS+0.8*(\xmaxBS-\xminBS)},{\pgfkeysvalueof{/pgfplots/ymin}})
{IN};
\draw [<->, xshift=-5mm] ({\pgfkeysvalueof{/pgfplots/xmin}},0) -- ({\pgfkeysvalueof{/pgfplots/xmin}},-\PremiumBS) node [pos=0.5, xshift=-2.5mm, rotate=90] {Prima};
\end{axis}
%
\end{tikzpicture}


\medskip

%\textbf{La position courte sur l'option d'achat}


Al vencimiento, el pago es \(\min (0,K- S_T)=-\max(0, S_T-K)=-( S_T-K)^{+}\) y la ganancia obtenida es \(C-\max (0, S_T-K)\).

\medskip

		\input{Graph/Vcalldef-es.tex}
\medskip


%\textbf{La position longue sur l'option de vente}

Al vencimiento, el pago es \(\max (0,K- S_T)=(K- S_T)^{+}\).
Si \(P\) es la prima de la opción de venta, el beneficio obtenido es \(\max (0,K- S_T)-P\), positivo si \(V_{PM}=K -P<S_T\).

\medskip

	\input{Graph/APutdef-es.tex}    	     

%\textbf{La position courte sur l'option de vente}

\medskip

	\input{Graph/VPutdef-es.tex}    	     

Al vencimiento, el pago es \(\min (0, S_T-K)=-(K- S_T)^+\).


\end{f}
\hrule

\begin{f}[Estrategias de spread]
\textbf{Estrategia de spread} : utiliza dos o más opciones del mismo tipo (dos opciones de compra o dos opciones de venta).  
Si los precios de ejercicio varían, se trata de un \textbf{spread vertical}.  
Si los vencimientos cambian, se trata de un \textbf{spread horizontal}.

Una estrategia de spread vertical implica una posición larga y una posición corta en opciones de compra sobre el mismo activo subyacente, con el mismo vencimiento pero diferentes precios de ejercicio.
Distinguimos : \textbf{spread vertical alcista} (\textit{Bull spread}) y \textbf{spread vertical bajista} (\textit{Bear spread}).

\textbf{Spread vertical alcista} : anticipando un aumento moderado del activo subyacente, el inversor toma una posición larga en \(C_1\) y una posición corta en \(C_2\) bajo la condición \(E_1 < E_2\).  
Resultado neto al vencimiento :

	\begin{tikzpicture}[scale=.52]
		\def\riskfreeBS{0.05}
		\def\xminBS{5}
		\def\xmaxBS{15}
		\def\PxExerciceBSa{9}
		\def\PxExerciceBSb{12}
		\def\sigmaBS{0.2}
		\def\TBS{0.75}
		\def\PremiumBSa{{BSCall(11,{\PxExerciceBSa},{\riskfreeBS},{\TBS},{\sigmaBS})}}
		\def\PremiumBSb{{BSCall(11,{\PxExerciceBSb},{\riskfreeBS},{\TBS},{\sigmaBS})}}
		\begin{axis}[ 
			width=0.8\textwidth,
			height=0.5\textwidth, 
			extra tick style={tick style=BleuProfondIRA},
			clip=false,
			axis on top,
			axis lines=middle, axis line style={BleuProfondIRA,thick,->},
			scale only axis, xmin={\xminBS},xmax={\xmaxBS},enlarge x limits=0.05,
			enlarge y limits=0.125,
			color=BleuProfondIRA,
			%		ylabel near ticks,
			ylabel={Profit},
			x label style={={at={(current axis.right of origin)}}},
			%    x label style={at={(axis description cs:1,-0.1)},anchor=south},
			%		x label style={at={(1,0.5)}},
			xlabel={subyacente (\(T\))},
			%		    x label style={at={(axis description cs:0.5,-0.1)},anchor=north},
			%		y label style={at={(axis description cs:-0.1,.5)},rotate=90,anchor=south},
			ytick=\empty,
			xtick=\empty,
			extra y ticks ={0},
			extra y tick labels={{0}},
			extra x ticks ={{\PxExerciceBSa},{\PxExerciceBSb}},
			extra x tick labels={{\(E_1\ \ \ \ \ \)},{\(E_2\)}},
			extra x tick style={color=BleuProfondIRA,
				tick label style={yshift=-0mm}	},
			]
			%		
			\addplot[name path=A,BleuProfondIRA,thin,domain={{\xminBS}:{\xmaxBS-1.75}}, samples=21,dashed] {Call(x,\PxExerciceBSa,\PremiumBSa)} 
			node [pos=0.15, above] {\small Long \(C_1\)};	
			\addplot[name path=B,BleuProfondIRA,thin,domain={{\xminBS}:{\xmaxBS}}, samples=21,dashed] {-Call(x,\PxExerciceBSb,\PremiumBSb)} 
			node [pos=0.15, above] {\small  Short \(C_2\)};	
			\addplot[name path=EVH,OrangeProfondIRA,thick,domain={{\xminBS}:{\xmaxBS}}, samples=41] {Call(x,\PxExerciceBSa,\PremiumBSa)-Call(x,\PxExerciceBSb,\PremiumBSb)} node [pos=0.15, above] {Spread vertical alcista};	
			%\draw[BleuProfondIRA,thick] \OV{\PxExerciceBSa}{\PremiumBSa}{0.4}{\xminBS}{\xmaxBS} ;
			%		\addplot[name path=Option,BleuProfondIRA,thick,domain={250:350}, 			 		samples=10,dashdotted,smooth] {BSPut(x,\PxExerciceBSa,\riskfreeBS,\TBS,\sigmaBS)} ;	
			
			\draw[BleuProfondIRA, thin, dashed] ({\PxExerciceBSa},0) -- ({\PxExerciceBSa},{-\PremiumBSa})	;
			\draw[BleuProfondIRA, thin, dashed] ({\PxExerciceBSb},0) -- ({\PxExerciceBSb},{\PremiumBSb})	;
		\end{axis}
		%
	\end{tikzpicture}



\textbf{Spread vertical bajista} : anticipando una caída moderada del activo subyacente, el inversor vende la opción más cara y compra la más barata.


	\begin{tikzpicture}[scale=.52]
		\def\riskfreeBS{0.05}
		\def\xminBS{5}
		\def\xmaxBS{15}
		\def\PxExerciceBSa{9}
		\def\PxExerciceBSb{12}
		\def\sigmaBS{0.2}
		\def\TBS{0.75}
		\def\PremiumBSa{{BSCall(11,{\PxExerciceBSa},{\riskfreeBS},{\TBS},{\sigmaBS})}}
		\def\PremiumBSb{{BSCall(11,{\PxExerciceBSb},{\riskfreeBS},{\TBS},{\sigmaBS})}}
		\begin{axis}[ 
			width=0.8\textwidth,
			height=0.5\textwidth, 
			extra tick style={tick style=BleuProfondIRA},
			clip=false,
			axis on top,
			axis lines=middle, axis line style={BleuProfondIRA,thick,->},
			scale only axis, xmin={\xminBS},xmax={\xmaxBS},enlarge x limits=0.05,
			enlarge y limits=0.125,
			color=BleuProfondIRA,
			%		ylabel near ticks,
			ylabel={Profit},
			x label style={={at={(current axis.right of origin)}}},
			%    x label style={at={(axis description cs:1,-0.1)},anchor=south},
			%		x label style={at={(1,0.5)}},
			xlabel={subyacente (\(T\))},
			%		    x label style={at={(axis description cs:0.5,-0.1)},anchor=north},
			%		y label style={at={(axis description cs:-0.1,.5)},rotate=90,anchor=south},
			ytick=\empty,
			xtick=\empty,
			extra y ticks ={0},
			extra y tick labels={{0}},
			extra x ticks ={{\PxExerciceBSa},{\PxExerciceBSb}},
			extra x tick labels={{\(E_1\ \ \ \ \ \)},{\(E_2\)}},
			extra x tick style={color=BleuProfondIRA,
				tick label style={yshift=-0mm}	},
			]
			%		
			\addplot[name path=A,BleuProfondIRA,thin,domain={{\xminBS}:{\xmaxBS-1.75}}, samples=21,dashed] {-Call(x,\PxExerciceBSa,\PremiumBSa)} 
			node [pos=0.15, above] {\small Short \(C_1\)};	
			\addplot[name path=B,BleuProfondIRA,thin,domain={{\xminBS}:{\xmaxBS}}, samples=21,dashed] {Call(x,\PxExerciceBSb,\PremiumBSb)} 
			node [pos=0.15, above] {\small  Long \(C_2\)};		
			\addplot[name path=EVH,OrangeProfondIRA,thick,domain={{\xminBS}:{\xmaxBS}}, samples=41] {-Call(x,\PxExerciceBSa,\PremiumBSa)+Call(x,\PxExerciceBSb,\PremiumBSb)} node [pos=0.15, above] {Spread vertical bajista};	
			%\draw[BleuProfondIRA,thick] \OV{\PxExerciceBSa}{\PremiumBSa}{0.4}{\xminBS}{\xmaxBS} ;
			%		\addplot[name path=Option,BleuProfondIRA,thick,domain={250:350}, 			 		samples=10,dashdotted,smooth] {BSPut(x,\PxExerciceBSa,\riskfreeBS,\TBS,\sigmaBS)} ;	
			
			\draw[BleuProfondIRA, thin, dashed] ({\PxExerciceBSa},0) -- ({\PxExerciceBSa},{\PremiumBSa})	;
			\draw[BleuProfondIRA, thin, dashed] ({\PxExerciceBSb},0) -- ({\PxExerciceBSb},{-\PremiumBSb})	;
		\end{axis}
		%
	\end{tikzpicture}

% Code TikZ conservé ici (éventuellement inséré)

\textbf{Spread mariposa} (\textit{butterfly spread}) : anticipa un pequeño movimiento en el activo subyacente.  
Es una combinación de un spread vertical alcista y un spread vertical bajista.  
Esta estrategia es adecuada cuando se considera poco probable que se produzcan grandes movimientos.  
Requiere una baja inversión inicial.

	\begin{tikzpicture}[scale=.52]
		\def\riskfreeBS{0.05}
		\def\xminBS{5}
		\def\xmaxBS{15}
		\def\PxExerciceBSa{8}
		\def\PxExerciceBSb{10}
		\def\PxExerciceBSc{12}
		\def\sigmaBS{0.2}
		\def\TBS{0.75}
		\def\PremiumBSa{BSCall(11,{\PxExerciceBSa},{\riskfreeBS},{\TBS},{\sigmaBS})}
		\def\PremiumBSb{BSCall(11,{\PxExerciceBSb},{\riskfreeBS},{\TBS},{\sigmaBS})}
		\def\PremiumBSc{BSCall(11,{\PxExerciceBSc},{\riskfreeBS},{\TBS},{\sigmaBS})}
		\begin{axis}[
			width=0.8\textwidth,
			height=0.5\textwidth, 
			extra tick style={tick style=BleuProfondIRA},
			clip=false,
			axis on top,
			axis lines=middle, axis line style={BleuProfondIRA,thick,->},
			scale only axis, xmin={\xminBS},xmax={\xmaxBS},enlarge x limits=0.05,
			enlarge y limits=0.125,
			color=BleuProfondIRA,
			%		ylabel near ticks,
			ylabel={Profit},
			x label style={={at={(current axis.right of origin)}}},
			%    x label style={at={(axis description cs:1,-0.1)},anchor=south},
			%		x label style={at={(1,0.5)}},
			xlabel={subyacente (\(T\))},
			%		    x label style={at={(axis description cs:0.5,-0.1)},anchor=north},
			%		y label style={at={(axis description cs:-0.1,.5)},rotate=90,anchor=south},
			ytick=\empty,
			xtick=\empty,
			extra y ticks ={0},
			extra y tick labels={{0}},
			extra x ticks ={{\PxExerciceBSa},{\PxExerciceBSb},{\PxExerciceBSc}},
			extra x tick labels={{\(E_1\ \ \ \ \ \)},{\(E_2\)},{\(E_3\)}},
			extra x tick style={color=BleuProfondIRA,
				tick label style={yshift=-0mm}	},
			]
			%		
			\addplot[name path=A,BleuProfondIRA,thin,domain={{\xminBS}:{\xmaxBS}}, samples=21,dashed] {Call(x,\PxExerciceBSa,\PremiumBSa)} 
			node [pos=0.15, above] {\small Long \(C_1\)};	
			\addplot[name path=B,BleuProfondIRA,thin,domain={{\xminBS}:{\xmaxBS-1.75}}, samples=21,dashed] {-2*Call(x,\PxExerciceBSb,\PremiumBSb)} 
			node [pos=0.15, above] {\small Short \(2\ C_2\)};	
			\addplot[name path=C,BleuProfondIRA,thin,domain={{\xminBS}:{\xmaxBS}}, samples=21,dashed] {Call(x,\PxExerciceBSc,\PremiumBSb)} 
			node [pos=0.15, above] {\small Long \(C_3\)};	
			\addplot[name path=EP,OrangeProfondIRA,thick,domain={{\xminBS}:{\xmaxBS}}, samples=41] {
				Call(x,\PxExerciceBSa,\PremiumBSa) - 2*Call(x,\PxExerciceBSb,\PremiumBSb)+
				Call(x,\PxExerciceBSc,\PremiumBSc)}
			node [pos=0.5, below=30pt] {Spread mariposa};	
			\draw[BleuProfondIRA, thin, dashed] ({\PxExerciceBSa},0) -- ({\PxExerciceBSa},{-\PremiumBSa})	;
			\draw[BleuProfondIRA, thin, dashed] ({\PxExerciceBSb},0) -- ({\PxExerciceBSb},{2*\PremiumBSb})	;
			\draw[BleuProfondIRA, thin, dashed] ({\PxExerciceBSc},0) -- ({\PxExerciceBSc},{-\PremiumBSc})	;
		\end{axis}
		%
	\end{tikzpicture}
% Code TikZ conservé ici (éventuellement inséré)

\end{f}
\hrule

\begin{f}[Estrategias combinadas]
	
	Una \textbf{estrategia combinada} utiliza opciones de compra y venta. En particular, distinguimos entre \textbf{straddles} y \textbf{strangles}.
	
	Un \textbf{straddle} combina la compra de una opción de compra (call) y una opción de venta (put) con la misma fecha de vencimiento y precio de ejercicio. Esta estrategia apuesta por una fuerte fluctuación del precio, ya sea al alza o a la baja. La pérdida máxima se produce si el precio al vencimiento es igual al precio de ejercicio.
	
	Un \textbf{strangle} es la compra de una opción call y una opción put con la misma fecha de vencimiento, pero con diferentes precios de ejercicio. Supone una fluctuación muy pronunciada en el valor del activo subyacente.
	
		\begin{tikzpicture}[scale=.52]
			\def\xminBS{200}
			\def\xmaxBS{275}
			\def\PxExerciceBSa{230}
			\def\PxExerciceBSb{245}
			\def\PremiumBSa{20.69}
			\def\PremiumBSb{23.79}
			\begin{axis}[ 
				width=0.8\textwidth,
				height=0.5\textwidth, 
				extra tick style={tick style=BleuProfondIRA},
				clip=false,
				axis on top,
				axis lines=middle, axis line style={BleuProfondIRA,thick,->},
				scale only axis, xmin={\xminBS},xmax={\xmaxBS},enlarge x limits=0.05,
				enlarge y limits=0.125,
				color=BleuProfondIRA,
				ylabel={Profit},
				x label style={={at={(current axis.right of origin)}}},
				xlabel={subyacente (\(T\))},
				ytick=\empty,
				extra y ticks ={0},
				extra y tick labels={{0}},
				extra x ticks ={{\PxExerciceBSa},{\PxExerciceBSb}},
				extra x tick labels={{\(E_1\ \ \ \ \ \)},{\(E_2\)}},
				extra x tick style={color=BleuProfondIRA,
					tick label style={yshift=-10mm}	},
				]
				\addplot[name path=A,BleuProfondIRA,thin,domain={{\xminBS}:{\xmaxBS-1.75}}, samples=21,dashed] {Call(x,\PxExerciceBSa,\PremiumBSa)} 
				node [pos=0.15, below] {\small Long \(C_1\)};	
				\addplot[name path=B,BleuProfondIRA,thin,domain={{\xminBS}:{\xmaxBS}}, samples=21,dashed] {Put(x,\PxExerciceBSb,\PremiumBSb)} 
				node [pos=0.85, above] {\small  Long \(P_2\)};		
				\addplot[name path=EVH,OrangeProfondIRA,thick,domain={{\xminBS}:{\xmaxBS}}, samples=21] {Call(x,\PxExerciceBSa,\PremiumBSa)+Put(x,\PxExerciceBSb,\PremiumBSb)} node [pos=0.5, above] {\small Strangle};	
				\draw[BleuProfondIRA, thin, dashed] ({\PxExerciceBSa},0) -- ({\PxExerciceBSa},{-\PremiumBSa})	;
				\draw[BleuProfondIRA, thin, dashed] ({\PxExerciceBSb},0) -- ({\PxExerciceBSb},{-\PremiumBSb})	;
			\end{axis}
		\end{tikzpicture}


		\begin{tikzpicture}[scale=.52]
			\def\xminBS{200}
			\def\xmaxBS{275}
			\def\PxExerciceBSa{230}
			\def\PxExerciceBSb{245}
			\def\PremiumBSa{15.19}
			\def\PremiumBSb{14.29}
			\begin{axis}[ 
				width=0.8\textwidth,
				height=0.5\textwidth, 
				extra tick style={tick style=BleuProfondIRA},
				clip=false,
				axis on top,
				axis lines=middle, axis line style={BleuProfondIRA,thick,->},
				scale only axis, xmin={\xminBS},xmax={\xmaxBS},enlarge x limits=0.05,
				enlarge y limits=0.125,
				color=BleuProfondIRA,
				ylabel={Profit},
				x label style={={at={(current axis.right of origin)}}},
				xlabel={subyacente (\(T\))},
				ytick=\empty,
				extra y ticks ={0},
				extra y tick labels={{0}},
				extra x ticks ={{\PxExerciceBSa},{\PxExerciceBSb}},
				extra x tick labels={{\(E_1\ \ \ \ \ \)},{\(E_2\)}},
				extra x tick style={color=BleuProfondIRA,
					tick label style={yshift=-10mm}	},
				]
				\addplot[name path=A,BleuProfondIRA,thin,domain={{\xminBS}:{\xmaxBS}}, samples=21,dashed] {Put(x,\PxExerciceBSa,\PremiumBSa)} 
				node [pos=0.85, below] {\small Long \(P_1\)};	
				\addplot[name path=B,BleuProfondIRA,thin,domain={{\xminBS}:{\xmaxBS}}, samples=21,dashed] {Call(x,\PxExerciceBSb,\PremiumBSb)} 
				node [pos=0.15, above] {\small  Long \(C_2\)};		
				\addplot[name path=EVH,OrangeProfondIRA,thick,domain={{\xminBS}:{\xmaxBS}}, samples=21] {Put(x,\PxExerciceBSa,\PremiumBSa)+Call(x,\PxExerciceBSb,\PremiumBSb)} node [pos=0.5, above] {\small Strangle};	
				\draw[BleuProfondIRA, thin, dashed] ({\PxExerciceBSa},0) -- ({\PxExerciceBSa},{-\PremiumBSa})	;
				\draw[BleuProfondIRA, thin, dashed] ({\PxExerciceBSb},0) -- ({\PxExerciceBSb},{-\PremiumBSb})	;
			\end{axis}
		\end{tikzpicture}
	
	
	
\end{f}
\hrule

\begin{f}[Ausencia de oportunidad de arbitraje]
Es imposible obtener una ganancia sin riesgo con una inversión inicial nula. Por lo tanto, no es posible obtener ganancias sin riesgo aprovechando las diferencias de precios. 

\end{f}

\begin{f}[Relación de paridad]
	
AAO implica la siguiente relación entre el Call y el Put (acción) :

\[S_t- C_t + P_t = K e^{-i_{f}.\tau}\]	
\end{f}
\hrule

\begin{f}[El modelo de Cox-Ross-Rubinstein]
	
Se basa en un proceso de tiempo discreto con dos posibles movimientos de precio en cada período : un aumento (factor \(u\)) o una disminución (factor \(d\)), con \(u > 1 + i_{f}\) y \(d < 1 + i_{f}\). El precio en \(t = 1\) es entonces \( S_{1}^{u} = S_{0} u \) o \( S_{1}^{d} = S_{0} d \), según una probabilidad \(q\) o \(1-q\).

\begin{tikzpicture}
	[sibling distance=5em,
	every node/.style = {shape=rectangle, rounded corners, fill=OrangeProfondIRA!20,
		align=center,  draw=OrangeProfondIRA, text=BleuProfondIRA } ,grow=right,
	edge from parent/.style={draw=OrangeProfondIRA, thick}]
	\node (A){\(S_0\)}
	child {node  (B) {\(S_d\)}
		child {node {\(S_{dd}\)}}
		child} 
	child {node (C) {\(S_u\)} 
		child {node  {\(S_{du}\)}}
		child {node {\(S_{uu}\)}}    
	};
	\draw [draw=none] 
	($ (A.east) + (0,0.2) $) -- node[draw=none, fill=none, above left, BleuProfondIRA] {\(q\)} ($ (C.west) + (0,-0.2) $);
	\draw [draw=none]  
	($ (A.east) + (0,-0.2) $) -- node[draw=none, fill=none,below left, BleuProfondIRA] {\(1 - q\)} ($ (B.west) + (0,0.2) $);
\end{tikzpicture}
\quad
\begin{tikzpicture}
	[sibling distance=5em,
	every node/.style = {shape=rectangle, rounded corners, fill=OrangeProfondIRA!20,
		align=center,  draw=OrangeProfondIRA, text=BleuProfondIRA } ,grow=right,
	edge from parent/.style={draw=OrangeProfondIRA, thick}]
	\node {\(C_0\)}
	child {node  {\(C_d\)}
		child {node {\(C_{dd}=(S_{dd}-K)^{+}\)}}
		child}
	child {node {\(C_u\)} 
		child {node  {\(C_{du}=(S_{du}-K)^{+}\)}}
		child {node  {\(C_{uu}=(S_{uu}-K)^{+}\)}}  
	};
\end{tikzpicture}

Este modelo se extiende a \(n\) periodos con \(n+1\) precios posibles para \(S_T\). Al vencimiento, el valor de una opción de compra viene dado por \( C_{1}^{u} = (S_{1}^{u} - K)^+ \) y \( C_{1}^{d} = (S_{1}^{d} - K)^+ \).

\textbf{La ausencia de oportunidad de arbitraje} implica
\[
d < 1 + i_{f} < u
\]
y una probabilidad neutral al riesgo 
\[q = \frac{(1 + i_{f}) - d}{u - d}\]

\textbf{Precio de Call} (con \( S_{1}^{d} < K < S_{1}^{u} \)) :
\[
C_{0} = \frac{1}{1+i_f} \left[ q C_{1}^{u} + (1 - q) C_{1}^{d} \right]
\]

También podemos construir una cartera de replicación compuesta por acciones \(\Delta\) y bonos \(B\), de modo que :
\[
\begin{cases}
	\Delta = \frac{S_{1}^{u} - K}{S_{1}^{u} - S_{1}^{d}}, \\
	B = \frac{-S_{1}^{d}}{1+i_f} \cdot \Delta
\end{cases}
\quad \Rightarrow \quad \Pi_0 = \Delta S_0 + B
\]

\textbf{Precio de Put} :
\[
P_{0} = \frac{1}{1+i_f} \left[ q P_{1}^{u} + (1 - q) P_{1}^{d} \right]
\]

\textbf{Determinación de \(q\), \(u\), \(d\)} :  
Al calibrar el modelo para que coincida con los primeros momentos del retorno bajo la probabilidad neutral al riesgo (valor esperado \(i_f\), varianza \(\sigma^2 \delta t\)), obtenemos :
\[
e^{i_{f} \delta t} = q u + (1-q) d, \qquad q u^2 + (1-q) d^2 - [q u + (1-q) d]^2 = \sigma^2 \delta t
\]

Con la restricción \(u = \frac{1}{d}\), obtenemos :
\[
\begin{array}{l}
	q = \frac{e^{-i_f \delta_t} - d}{u - d} \\
	u = e^{\sigma \sqrt{\delta t}} \\
	d = e^{-\sigma \sqrt{\delta t}}
\end{array}
\]

\end{f}
\hrule

\begin{f}[El modelo de Black \& Scholes]
Supuestos del modelo 
\begin{itemize}
	\item La tasa libre de riesgo \(R\) es constante. Definimos \(i_f = \ln(1+R)\), lo que implica \((1+R)^t = e^{i_f t}\).
	\item El precio de las acciones \(S_t\) sigue un movimiento browniano geométrico :
	\[
	dS_t = \mu S_t dt + \sigma S_t dW_t 
	\]
	\[
	 S_t = S_0 \exp\left(\sigma W_t + \left( \mu - \frac{1}{2}\sigma^2 \right)t \right)
	\]
	\item No habrá dividendos durante la vigencia de la opción.
	\item La opción es "europea" (se ejerce sólo al vencimiento).
	\item Mercado sin fricciones : sin impuestos ni costes de transacción.
	\item Se permiten las ventas en corto.
\end{itemize}

La ecuación de Black-Scholes-Merton para valorar un contrato derivado \(f\) es :
\[
\frac{\partial f}{\partial t} + i_f S \frac{\partial f}{\partial S} + \frac{1}{2}\sigma^2 S^2 \frac{\partial^2 f}{\partial S^2} = i_f f
\]

Al vencimiento, el precio de una opción de compra es \(C(S,T) = \max(0, S_T - K)\), y el de una opción de venta es \(P(S,T) = \max(0, K - S_T)\).


\begin{center}
	\begin{tabular}{|c|c|c|}
		\hline
		Determinantes & \textbf{call}&\textbf{put}\\
		\hline
		Precio subyacente	      & +&	-\\
		Precio de ejercicio	              & -&	+\\
		Madurez (o tiempo)    & + (-)&	+ (-)\\
		Volatilidad	              & +&	+\\
		Tasas de interés a corto plazo  & +&	-\\
		Pago de dividendos	      & -&	+\\
		\hline
	\end{tabular}
\end{center}

Las soluciones analíticas son :
\begin{align*}
	C_t &= S_t \Phi(d_1) - Ke^{-i_f \tau} \Phi(d_2) \\
	P_t &= Ke^{-i_f \tau} \Phi(-d_2) - S_t \Phi(-d_1)
\end{align*}
ó :
\begin{align*}
	d_1 &= \frac{\ln(S_t/K) + (i_f + \frac{1}{2}\sigma^2)\tau}{\sigma \sqrt{\tau}}, \quad
	d_2 = d_1 - \sigma \sqrt{\tau}
\end{align*}

%La sensibilité peut être mesurée par cinq paramètres (lettres grecques) :

\begin{itemize}
	\item \textbf{Delta} \(\Delta\) : variación en el precio de la opción dependiendo del subyacente.
	\item \textbf{Gamma} \(\Gamma\) : sensibilidad delta.
	\item \textbf{Thêta} \(\Theta\) : sensibilidad al tiempo.
	\item \textbf{Véga} \(\mathcal{V}\) : sensibilidad a la volatilidad.
	\item \textbf{Rho} \(\rho\) : sensibilité au taux d’intérêt.
\end{itemize}


El \textbf{Delta} mide el impacto de un cambio en el activo subyacente :

\begin{align*}
	\Delta_C &= \frac{\partial C}{\partial S} = \Phi(d_1), \quad \Delta \in (0,1) \\
	\Delta_P &= \frac{\partial P}{\partial S} = \Phi(d_1) - 1, \quad \Delta \in (-1,0)
\end{align*}


El Delta global de una cartera \(\Pi\) con pesos \(\omega_i\) es :
\[
\frac{\partial \Pi}{\partial S_t} = \sum_{i=1}^{n} \omega_i \Delta_i
\]



% Graphique TikZ conservé tel quel :

\begin{center}
\begin{tikzpicture}[scale=.52]
\def\riskfreeBS{0.05}
\def\xminBS{7.5}
\def\xmaxBS{12.5}
\def\PxExerciceBS{10}
\def\sigmaBS{0.3}
\def\TBS{0.4}
\def\PremiumBS{BSCall(\PxExerciceBS*exp(-\riskfreeBS*\TBS),\PxExerciceBS,\riskfreeBS,\TBS,\sigmaBS)}
\def\PxExerciceAct{\PxExerciceBS*exp(-(\riskfreeBS+\sigmaBS*\sigmaBS/2)*\TBS)}
\begin{axis}[
	width=0.8\textwidth,
	height=0.5\textwidth, 
	extra tick style={tick style=BleuProfondIRA},
	clip=false,
	axis on top,
	axis lines=middle, axis line style={BleuProfondIRA,thick,->},
	scale only axis, xmin={\xminBS},xmax={\xmaxBS},enlarge x limits=0.05,
	enlarge y limits=0.1,
	color=BleuProfondIRA,
	ylabel={\(\Delta\)},
	x label style={at={(axis cs:\xmaxBS+.1,0)},anchor=north east},
	xlabel={subyacente (\(T\))},
	ytick=\empty,
	xtick=\empty,
	extra y ticks ={-.5,0,0.5},
	extra y tick labels={{\(-\frac{1}{2}\)},{0},{\(\frac{1}{2}\)}},
	extra x ticks ={\PxExerciceAct,\PxExerciceBS},
	extra x tick labels={{\color{BleuProfondIRA}\(K'\)\ \ \ \ \ \ \ \ },{\color{BleuProfondIRA}\ \ \(K\)}},
	extra x tick style={color=BleuProfondIRA,
		tick label style={yshift=-0mm}	},
	title ={ \textbf{Delta de la opción}},
	title style={yshift=-10mm}
	]
	\addplot[name path=optionT,OrangeProfondIRA,thin,domain={{\xminBS}:{\xmaxBS}}, samples=21]
	{normcdf(-ddd(x,\PxExerciceBS,\riskfreeBS,\TBS,\sigmaBS),0,1)} node [above] {Call};
	\addplot[name path=optionT,OrangeProfondIRA,thin,domain={{\xminBS}:{\xmaxBS}}, samples=21]
	{normcdf(-ddd(x,\PxExerciceBS,\riskfreeBS,\TBS,\sigmaBS),0,1)-1} node [above] {Put};
	\draw[dashed,OrangeProfondIRA] 
	(axis cs:{\pgfkeysvalueof{/pgfplots/xmin}},-0.5) --
	(axis cs:{\PxExerciceAct},{-.5}) --
	(axis cs:{\PxExerciceAct},{0.5}) --
	(axis cs:{\pgfkeysvalueof{/pgfplots/xmin}},0.5);
\end{axis}
\end{tikzpicture}
\end{center}

\end{f}
\hrule


\begin{f}[La curva de rendimiento]
La \textbf{curva de rendimiento}, o curva de retornos, o \(r_f(\tau)\), proporciona una representación gráfica de las tasas de interés libres de riesgo en función del vencimiento (o plazo).
También se denomina curva de rendimiento \textbf{cupón cero} y hace referencia a un tipo de bono libre de riesgo y sin cupones (una deuda compuesta únicamente por dos flujos de caja opuestos, uno en \(t_0\) y el otro en \(T\)).
Esta curva también proporciona información sobre las expectativas del mercado con respecto a las tasas de interés futuras (tasas \engl{forward}).
\end{f}
\hrule


\begin{f}[Los modelos Nelson-Siegel y Svensson]


Las funciones \textbf{Nelson-Siegel} toman la forma

{\small\begin{align*}
y( m ) =& \beta _0 + \beta _1\frac{{\left[ {1 - \exp \left( { - m/\tau} \right)} \right]}}{m/\tau} + \\
		&\beta _2 {\left(\frac{{\left[ {1 - \exp \left( { - m/\tau} \right)} \right]}}{m/\tau} - \exp \left( { - m/\tau}\right)\right)}
\label{MTNSeq}
\end{align*}}
%
donde \(y\left( m \right)\) y \(m\) son como arriba, y \(\beta_0\), \(\beta_1\), \(\beta_2\) y \(\tau\) son parámetros :


\begin{itemize}

\item   \(\beta_0\) se interpreta como el nivel de tipos de interés a largo plazo (el coeficiente es 1, es una constante que no disminuye),

\item   \(\beta_1\) es el componente de corto plazo, teniendo en cuenta que :
\begin{equation*}
	\lim_{m \rightarrow 0} \frac{{\left[ {1 - \exp \left( { - m/\tau} \right)} \right]}}{m/\tau}=1
\end{equation*}
De ello se deduce que el tipo de interés a un día, como €str\index{Taux d'intérêts! Estr}, será igual a \(\beta_0 + \beta_1\) en este modelo.
\item   \(\beta_2\) es el componente de mediano plazo (comienza en 0, aumenta y luego disminuye nuevamente hacia cero, es decir, tiene forma de campana),
\item \(\tau\) es el factor de escala al vencimiento; determina dónde el término ponderado por \(\beta_2\) alcanza su máximo.
\end{itemize}

Svensson (1995) añade un segundo término en forma de campana; este es el modelo de Nelson-Siegel-Svensson. El término adicional es :
%
\begin{equation*}
+\beta _3 {\left(\frac{{\left[ {1 - \exp \left( { - m/\tau_2} \right)} \right]}}{m/\tau_2} - \exp \left( { - m/\tau_2}\right)\right)}
\label{MTSveq}
\end{equation*}
y la interpretación es la misma que para \(\beta_2\) y \(\tau\) anteriores; permite dos puntos de inflexión en la curva de rendimiento.

\newcommand{\traintunnel}{	        
\draw[thick, OrangeProfondIRA] svg "M 55.448002 56.380001L 40 39L 28 39L 12.552 56.380001M 12 34C 11.729672 21.575853 21.576109 11.281852 34 11C 46.423893 11.281852 56.270329 21.575853 56.000004 34L 56 55C 56 56.104568 55.104568 57 54 57L 14 57C 12.895431 57 12 56.104568 12 55ZM 28 39L 28 34C 28 30.132 30.302 27 34 27C 37.697998 27 40 30.132 40 34L 40 39M 34 51L 34 57M 34 43L 34 45";
}
\newcommand{\archibuilding}{
\draw[OrangeProfondIRA,yscale=-1] svg "M 12.296 28.886L 12.296 54.453999C 12.451618 55.715763 13.58469 56.623463 14.85 56.500004L 53.150002 56.5C 54.415314 56.623463 55.548386 55.715763 55.704002 54.454002L 55.703999 28.886M 12.296 28.886L 34 11.5L 55.703999 28.886M 34 46.296001L 42.212002 46.296001L 42.214001 50.386002L 49.32 50.386002L 49.32 56.5M 12.296 40.285999L 34 40.285999M 34 35.186001L 55.368 35.186001M 34 11.5L 34 56.236M 12.296 32.106003L 34 32.106003M 19.456001 32.106003L 19.456001 40.285999M 26.84 32.106003L 26.84 40.285999";	        
}
\newcommand{\familialcar}{	%
\draw[thick, OrangeProfondIRA] svg "M 21 45L 21 48C 21 48.552284 20.552284 49 20 49L 16 49C 15.447716 49 15 48.552284 15 48L 15 45M 53 45L 53 48C 53 48.552284 52.552284 49 52 49L 48 49C 47.447716 49 47 48.552284 47 48L 47 45M 54 45C 54.552284 45 55 44.552284 55 44L 55 37.414001C 54.999943 37.149296 54.894939 36.895409 54.708 36.708L 49 31L 19 31L 13.292 36.708C 13.105062 36.895409 13.000056 37.149296 13 37.414001L 13 44C 13 44.552284 13.447716 45 14 45ZM 49 31L 45.228001 19.684C 45.092045 19.275806 44.710239 19.000328 44.279999 19L 23.720001 19C 23.289761 19.000328 22.907955 19.275806 22.771999 19.684L 19 31M 19 31L 14 31C 13.447716 31 13 30.552284 13 30L 13 28C 13 27.447716 13.447716 27 14 27L 20.334 27M 47.666 27L 54 27C 54.552284 27 55 27.447716 55 28L 55 30C 55 30.552284 54.552284 31 54 31L 49 31M 13.092 37L 20 37C 20.552284 37 21 37.447716 21 38L 21 40C 21 40.552284 20.552284 41 20 41L 13 41M 55 41L 48 41C 47.447716 41 47 40.552284 47 40L 47 38C 47 37.447716 47.447716 37 48 37L 54.908001 37";
}

\newcommand{\familialTV}{	        
\draw[thick, OrangeProfondIRA] svg "M 12 17.5L 56 17.5C 56 17.5 57 17.5 57 18.5L 57 43.5C 57 43.5 57 44.5 56 44.5L 12 44.5C 12 44.5 11 44.5 11 43.5L 11 18.5C 11 18.5 11 17.5 12 17.5M 34 44.5L 34 50.5M 24 50.5L 44 50.5";
}

\newcommand{\TresorerieMngt}{	
\draw[OrangeProfondIRA] svg "M 11.008 31C 11.008 32.104568 15.485153 33 21.007999 33C 26.530848 33 31.007999 32.104568 31.007999 31C 31.007999 29.89543 26.530848 29 21.007999 29C 15.485153 29 11.008 29.89543 11.008 31ZM 31 31L 31 37C 31 38.106003 26.524 39 21 39C 15.476 39 11 38.106003 11 37L 11 31M 31 37L 31 43C 31 44.105999 26.524 45 21 45C 15.476 45 11 44.105999 11 43L 11 37M 31 43L 31 49C 31 50.105999 26.524 51 21 51C 15.476 51 11 50.105999 11 49L 11 43M 31 49L 31 55C 31 56.105999 26.524 57 21 57C 15.476 57 11 56.105999 11 55L 11 49M 11 25L 11 13C 11 11.895431 11.895431 11 13 11L 55 11C 56.104568 11 57 11.895431 57 13L 57 37C 57 38.104568 56.104568 39 55 39L 36 39M 28 25C 28.000584 21.94886 30.290909 19.384022 33.32254 19.039516C 36.354168 18.695011 39.161583 20.680555 39.846756 23.65377C 40.531929 26.626984 38.876644 29.640953 36 30.658001M 20 19.5C 20.276142 19.5 20.5 19.723858 20.5 20C 20.5 20.276142 20.276142 20.5 20 20.5C 19.723858 20.5 19.5 20.276142 19.5 20C 19.5 19.723858 19.723858 19.5 20 19.5M 48 29.5C 48.276142 29.5 48.5 29.723858 48.5 30C 48.5 30.276142 48.276142 30.5 48 30.5C 47.723858 30.5 47.5 30.276142 47.5 30C 47.5 29.723858 47.723858 29.5 48 29.5M 15 25L 15 16C 15 15.447716 15.447716 15 16 15L 52 15C 52.552284 15 53 15.447716 53 16L 53 34C 53 34.552284 52.552284 35 52 35L 36 35";	
}

Estas funciones de Nelson-Siegel y Svensson tienen la ventaja de comportarse bien a largo plazo y ser fáciles de parametrizar.
Ellas se ilustran en la figura donde los pictogramas \begin{tikzpicture}[xscale=0.2, yscale=-0.2]
\TresorerieMngt\end{tikzpicture} \begin{tikzpicture}[xscale=0.2, yscale=-0.2] \familialTV\end{tikzpicture} \begin{tikzpicture}[xscale=0.2, yscale=-0.2] \familialcar\end{tikzpicture} \begin{tikzpicture}[xscale=0.2, yscale=0.2] \archibuilding\end{tikzpicture} \begin{tikzpicture}[xscale=0.2, yscale=-0.2] \traintunnel
\end{tikzpicture} representan los diferentes vencimientos habituales para este tipo de propiedad o inversión.
Ellas permiten el modelado de una amplia curva de rendimiento. 
Una vez ajustado, el usuario puede evaluar los activos o definir diversas medidas de sensibilidad.


\begin{center}
\begin{tikzpicture}[scale=0.55]
\def\MTbetaa{0.03}
\def\MTbetab{-0.02}
\def\MTbetac{0.01}
\def\MTbetad{-0.005}  % Svensson
\def\MTtaua{4.5}
\def\MTtaub{11}  % Svensson
\begin{axis}[
	width=0.8\textwidth,
	height=0.5\textwidth, 
	xlabel={Vencimiento (años)},ylabel={Tasa (\%)},
	xmin=0, xmax=30,
	ymin=0, ymax=100*(\MTbetaa+.005),
	enlarge y limits=0.125,
	thick,
	axis x line=bottom,
	axis y line=left,
	yticklabel=\pgfmathprintnumber{\tick}\% 
	]
	\Large
	% Courbe de Nelson-Siegel
	\addplot[OrangeProfondIRA, thick, domain=0.01:27, samples=27] 
	{100*(\MTbetaa + \MTbetab * ((1 - exp(-x/\MTtaua)) / (x/\MTtaua)) + \MTbetac * (((1 - exp(-x/\MTtaua)) / (x/\MTtaua)) - exp(-x/\MTtaua)))}
	node  [pos=0.005] (M) {}
	node  [pos=0.10] (N) {}
	node  [pos=0.30] (O) {}
	node  [pos=0.75] (P) {}
	node  [pos=1] (Q) {};
	\addplot[dashed, OrangeProfondIRA, thick, domain=0.01:30, samples=27] 
	{100*(\MTbetaa + \MTbetab * ((1 - exp(-x/\MTtaua)) / (x/\MTtaua)) + \MTbetac * (((1 - exp(-x/\MTtaua)) / (x/\MTtaua)) - exp(-x/\MTtaua)) + \MTbetad * (((1 - exp(-x/\MTtaub)) / (x/\MTtaub)) - exp(-x/\MTtaub)))};
	% Affichage des paramètres
	\node[anchor=south east, text=BleuProfondIRA] at (rel axis cs:1,0.15) { 
	\(\beta_0 = \MTbetaa\)\quad
	\(\beta_1 = \MTbetab\)\quad
	\(\beta_2 = \MTbetac\)\quad
	\(\tau_1 = \MTtaua\)\quad			
};	\node[anchor=south east, text=BleuProfondIRA] at (rel axis cs:1,0.05) { 
	\(\beta_3 = \MTbetad\)\quad
	\(\tau_2 = \MTtaub\)			
};
\node[xscale=0.3, yscale=-0.3, above=35pt] at (M) {\TresorerieMngt};
\node[xscale=0.3, yscale=-0.3, above=35pt] at (N) {\familialTV};
\node[xscale=0.3, yscale=-0.3, above=30pt] at (O) {\familialcar};
\node[xscale=0.3, yscale=-0.3, above=30pt] at (P) {\archibuilding};
\node[xscale=0.3, yscale=-0.3, above=30pt] at (Q) {\traintunnel};
\end{axis}
\end{tikzpicture}
\end{center}
\end{f}
\hrule


\begin{f}[Modelo de Vasicek]
	
	Bajo una probabilidad neutral al riesgo \(\mathbb{Q}\), la tasa a corto plazo \((r_t)\) sigue un proceso de Ornstein-Uhlenbeck con coeficientes constantes :
	\[
	dr_t = \kappa(\theta - r_t)\, dt + \sigma\, dW_t, \quad r_0 \in \mathbb{R}
	\]
	o :
	\begin{itemize}[nosep]
		\item \(\kappa > 0\) es la velocidad de reversión a la media,
		\item \(\theta\) es el nivel medio a largo plazo,
		\item \(\sigma > 0\) es la volatilidad,
		\item \(W_t\) es un movimiento browniano estándar bajo \(\mathbb{Q}\).
	\end{itemize}
	
La solución EDS (aplicación del lema de Itô a \(Y_{t}=r(t) e^{\kappa t}\)) :
	\[
	r_t = r_s e^{-\kappa(t-s)} + \theta(1 - e^{-\kappa(t-s)}) + \sigma \int_s^t e^{-\kappa(t-u)} dW_u
	\]
	
	\textbf{Entonces} :
	\[
	\begin{aligned}
		\mathbb{E}_\mathbb{Q}[r_t \mid \mathcal{F}_s] &= r_s e^{-\kappa(t-s)} + \theta(1 - e^{-\kappa(t-s)}) \\
		\operatorname{Var}_\mathbb{Q}[r_t \mid \mathcal{F}_s] &= \frac{\sigma^2}{2\kappa} \left(1 - e^{-2\kappa(t-s)}\right)
	\end{aligned}
	\]
El proceso \((r_t)\) es gaussiano; son posibles tasas negativas.
	
\end{f}

\begin{f}[Precio de un bono cupón cero (Vasicek)]

El precio en el momento \(t\) de un bono cupón cero que vence en \(T\) viene dado por :
\[
ZC(t, T) = A(t, T) \, e^{-B(t, T)\, r_t}
\]
où :
\[
\begin{aligned}
	B(t, T) &= \frac{1 - e^{-\kappa(T - t)}}{\kappa} \\
	A(t, T) &= \exp \left[ \left(\theta - \frac{\sigma^2}{2\kappa^2}\right) (B(t, T) - (T - t)) - \frac{\sigma^2}{4\kappa} B(t, T)^2 \right]
\end{aligned}
\]

Esta formulación es posible debido al hecho de que \(\int_t^T r_s ds\) es una variable aleatoria gaussiana condicional a \(\mathcal{F}_t\).

\[
ZC(t, T) = \mathbb{E}_\mathbb{Q} \left[ \exp\left( -\int_t^T r_s\, ds \right) \Big| \mathcal{F}_t \right]
\]

\end{f}
\hrule

\begin{f}[Modelo Cox-Ingersoll-Ross (CIR)]
	
Bajo la medida neutral al riesgo \(\mathbb{Q}\), la tasa a corto plazo \((r_t)\) sigue la dinámica :
\[
dr_t = \kappa(\theta - r_t)\,dt + \sigma \sqrt{r_t}\, dW_t, \quad r_0 \geq 0
\]
con :
\begin{itemize}[nosep]
	\item \(\kappa > 0\) : velocidad de reversión a la media,
	\item \(\theta > 0\) : nivel a largo plazo,
	\item \(\sigma > 0\) : volatilidad,
	\item \(W_t\) : movimiento Browniano bajo \(\mathbb{Q}\).
\end{itemize}

\textbf{Entonces} :
\begin{itemize}
	\item La raíz cuadrada \(\sqrt{r_t}\) garantiza \(r_t \geq 0\) si \(2\kappa\theta \geq \sigma^2\) (condición de Feller).
	\item El proceso \((r_t)\) es un proceso de difusión no gaussiano pero con trayectorias continuas.
	\item La tasa es de \textbf{reversión a la media} alrededor de \(\theta\).
\end{itemize}

Por tanto, el proceso \((r_t)\) es una difusión con distribuciones condicionales explícitas (bajo \(\mathbb{Q}\)) :

Para \(s < t\), la variable \(r_t\) sigue una distribución no central \(\chi^2\) :
\[
r_t \mid \mathcal{F}_s \sim c \cdot \chi^2_{d}(\lambda)
\]
con :
\begin{itemize}[nosep]
	\item \(\displaystyle c = \frac{\sigma^2 (1 - e^{-\kappa (t - s)})}{4\kappa}\)
	\item \(\displaystyle d = \frac{4\kappa\theta}{\sigma^2}\) : grados de libertad
	\item \(\displaystyle \lambda = \frac{4\kappa e^{-\kappa (t - s)} r_s}{\sigma^2 (1 - e^{-\kappa (t - s)})}\)
\end{itemize}

y
\[
\begin{aligned}
	\mathbb{E}_\mathbb{Q}[r_t \mid \mathcal{F}_s] =& r_s e^{-\kappa(t-s)} + \theta (1 - e^{-\kappa(t-s)}) \\
	\operatorname{Var}_\mathbb{Q}[r_t \mid \mathcal{F}_s] =& \frac{\sigma^2 r_s e^{-\kappa(t-s)} (1 - e^{-\kappa(t-s)})}{\kappa} \\
			&+ \frac{\theta \sigma^2}{2\kappa} (1 - e^{-\kappa(t-s)})^2
\end{aligned}
\]

\end{f}
\begin{f}[Precio de un bono cupón cero (CIR)]
En el modelo CIR, el precio de un bono cupón cero en el momento \(t\) con vencimiento \(T\) viene dado por :
\[
ZC(t, T) = A(t, T) \cdot e^{-B(t, T)\, r_t}
\]
con :
\[
\begin{aligned}
	B(t, T) &= \frac{2 (e^{\gamma (T - t)} - 1)}{(\gamma + \kappa)(e^{\gamma (T - t)} - 1) + 2\gamma} \\
	A(t, T) &= \left[ \frac{2\gamma e^{\frac{(\kappa + \gamma)}{2}(T - t)}}{(\gamma + \kappa)(e^{\gamma (T - t)} - 1) + 2\gamma} \right]^{\frac{2\kappa\theta}{\sigma^2}}
\end{aligned}
\]
o :
\[
\gamma = \sqrt{\kappa^2 + 2\sigma^2}
\]

\end{f}
\hrule

\begin{f}[Swaption, Black model]

Una \textbf{swaption} es una opción sobre un swap de tasas de interés. Otorga el derecho (pero no la obligación) de suscribir un swap en una fecha futura \(T\).

\begin{itemize}[nosep]
	\item \textbf{Swaption del pagador}: derecho a \emph{pagar la tasa fija} y \emph{recibir la tasa flotante}.
	\item \textbf{Swap del receptor}: derecho a \emph{recibir la tasa fija} y \emph{pagar la tasa flotante}.
\end{itemize}

\textbf{Notación} :
\begin{itemize}[nosep]
	\item \(T\) : fecha de ejercicio del swaption
	\item \(K\) : tipo de interés fijo (strike)
	\item \(S(T)\) : tipo de cambio swap en la fecha \(T\)
	\item \(A(T)\) : valor actual de los flujos fijos futuros.
	\item \(\sigma\) : volatilidad de la tasa swap
\end{itemize}

El modelo Black (1976) es una adaptación del modelo Black-Scholes para productos de tipos de interés. En este modelo, el tipo swap \(S(T)\) actúa como activo subyacente, con una rentabilidad similar a la de una opción europea.

\textbf{Black's formula for a payer swaption} :
\[
\text{SW}_{\text{pagador}} = A(T) \left[ S_0 N(d_1) - K N(d_2) \right]
\]
or :
\[
\begin{aligned}
	d_1 &= \frac{\ln(S_0 / K) + \frac{1}{2} \sigma^2 T}{\sigma \sqrt{T}} \\
	d_2 &= d_1 - \sigma \sqrt{T}
\end{aligned}
\]
y \(N(\cdot)\) es la función de distribución acumulativa de la distribución normal estándar.

\textbf{Fórmula para un swaption de receptor} :
\[
\text{SW}_{\text{receptor}} = A(T) \left[ K N(-d_2) - S_0 N(-d_1) \right]
\]

\end{f}

