% !TeX root = ActuarialFormSheet_MBFA-es.tex
% !TeX spellcheck = fr_FR

\begin{f}[Capitalisation Actualisation]

\begin{tikzpicture}[scale=0.85]
% Draw the x-axis and y-axis.
\def\w{11}
\def\n{9}
\draw[ line width=1,dotted, color=OrangeProfondIRA, arrows={-Stealth[length=4, inset=0]}] (0,0) -- (0,0.4909) node (A) {};	
\foreach \y in  {0,...,3} {
	\draw (\y,0) -- (\y,-0.1);
	\ifthenelse{\y>0 }{	\node[below] at (\y,-0.1) {\tiny \( \scriptstyle \y\)};
		}{
		\node[below] at (\y,-0.1) {\tiny 0};}
}
\foreach \y in  {-3,...,1} {
	\draw (\y+\n,0) -- (\y+\n,-0.1);
	\ifthenelse{\y<0 }{\node[below] at (\y+\n,-0.1) {\tiny \(\scriptstyle n \y\)};}{}
	\ifthenelse{\y=0 }{\node[below] at (\y+\n,-0.1) {\tiny \(\scriptstyle n \)};
		\draw[ line width=1, color=OrangeProfondIRA, arrows={-Stealth[length=4, inset=0]}] (\y+\n,0) -- (\y+\n,1) node (B) {};}{}
	\ifthenelse{\y>0}{	\node[below] at (\y+\n,-0.1) {\tiny \(\scriptstyle n+\y\)};}{}
}
\draw[ line width=1] (-.25,0) -- (4,0);
\draw[ line width=1, dashed] (4,0) -- (5,0);
\draw[arrows={-Stealth[length=4, inset=0]}, line width=1] (5,0) -- (\w,0);
\draw[arrows={-Stealth[length=4, inset=0]},color=OrangeProfondIRA] (B)  to [bend right]  node [pos=0.5, below=10pt] {Actualisation}(A) ;
\end{tikzpicture}

\begin{tikzpicture}[scale=0.85]
% Draw the x-axis and y-axis.
\def\w{11}
\def\n{9}
\draw[ line width=1, color=OrangeProfondIRA, arrows={-Stealth[length=4, inset=0]}] (0,0) -- (0,1) node (A) {};	
\foreach \y in  {0,...,3} {
	\draw (\y,0) -- (\y,-0.1);
	\ifthenelse{\y>0 }{	\node[below] at (\y,-0.1) {\tiny \( \scriptstyle \y\)};
	}{
		\node[below] at (\y,-0.1) {\tiny 0};}
}
\foreach \y in  {-3,...,1} {
	\draw (\y+\n,0) -- (\y+\n,-0.1);
	\ifthenelse{\y<0 }{\node[below] at (\y+\n,-0.1) {\tiny \(\scriptstyle n \y\)};}{}
	\ifthenelse{\y=0 }{\node[below] at (\y+\n,-0.1) {\tiny \(\scriptstyle n \)};
		\draw[ line width=1,dotted, color=OrangeProfondIRA, arrows={-Stealth[length=4, inset=0]}] (\y+\n,0) -- (\y+\n,2.0368) node (B) {};}{}
	\ifthenelse{\y>0}{	\node[below] at (\y+\n,-0.1) {\tiny \(\scriptstyle n+\y\)};}{}
}
\draw[ line width=1] (-.25,0) -- (4,0);
\draw[ line width=1, dashed] (4,0) -- (5,0);
\draw[arrows={-Stealth[length=4, inset=0]}, line width=1] (5,0) -- (\w,0);
\draw[arrows={-Stealth[length=4, inset=0]},color=OrangeProfondIRA] (A)  to [bend left]  node [pos=0.55, below=10pt] {Capitalisation}(B) ;
\end{tikzpicture}

\end{f}
\hrule

\begin{f}[Les Intérêts]

\emph{Escompte ou taux précompté \(d\)}
     \[d=i/(1+i)\]
\emph{Intérêt simple \(i\)}
\[
I_t=P i t=P i \frac{k}{365}
\]
\emph{Intérêt composés \(i\)}
\[
V_n=P(1+i)^n=P\left(1+\frac{p}{100}\right)^n
\]
\emph{Intérêt continu \(r\)}
\[V_t=V_0\ e^{rt}\]

\emph{Taux effectif \(i_e\)}
\[
i_e=\left( 1+\frac{i}{m}\right) ^{m}-1
\]
où \(i\) est le taux nominal et \(m\) le nombre de périodes sur un an.

\emph{Taux équivalent \(i^{(m)}\)}
\[
i^{(m)}=m(1+i)^{1 / m}-1
\]

\emph{Taux nominal \(i\) et taux périodique}

Le taux \textbf{nominal} ou taux \textbf{facial} permet de calculer les intérêts dus sur un an.
Le taux \textbf{périodique} correspond au taux nominal divisé par le nombre de périodes sur un an \(i/m\).
Si le taux périodique est hebdomadaire, le taux nominal sera divisé par 52.
\end{f}
\hrule

\begin{f}[Valeur actuelle et valeur future]

La valeur actuelle (VA)  ou valeur présente (VP)  représente le capital qui doit être investi aujourd'hui à un taux d'intérêt composé annuel \(i\) pour obtenir des flux de trésorerie futurs (\(F_k\)) aux moments \(t_k\):
\begin{equation}
	VP = \sum_{k=1}^{n} F_k \times \frac{1}{(1+i)^{t_k}}
\label{ValeurActuelle}
\end{equation}
Lorsque les \(F_k\) sont constants
\begin{equation}
	VP = K  \frac{1 - (1+i)^{-n}} {i}
\label{ValeurActuelleFluxCt}
\end{equation}

La valeur future (VF) représente la valeur du capital en \(T\) qui, avec un taux d'intérêt composé annuel \(i\),  capitalise les flux de trésorerie futurs (\(F_k\)) aux moments \(t_k\).
\begin{equation}
	VF=V_n = \sum_{k=1}^{n} F_k \times (1+i)^{n-t_k}
\end{equation}
Plus généralement \(VF= (1+i)^{n}VP\).
\end{f}
\hrule

\begin{f}[Annuités]
\ \newline


	Anualidad determinada \(a_{\lcroof{n}}\) (o \(a_{\lcroof{n}\;i}\) si es necesario especificar el tipo de interés \(i\) : este es el caso por defecto en matemáticas financieras. Sus pagos están, por ejemplo, garantizados por un contrato.
	
\begin{center}
	\begin{tikzpicture}[scale=0.75]
	% Draw the x-axis and y-axis.
	\def\w{11}
	\def\n{9}
	\node[left] at (-.5,0) {\(a_{\lcroof{n}}\)};
	\foreach \y in  {0,...,3} {
	\draw (\y,0) -- (\y,-0.1);
	\ifthenelse{\y>0 }{	\node[below] at (\y,-0.1) {\tiny \( \scriptstyle \y\)};
		\draw[ line width=1, color=OrangeProfondIRA, arrows={-Stealth[length=4, inset=0]}] (\y,0) -- (\y,1);}{
		\node[below] at (\y,-0.1) {\tiny 0};}
}
\foreach \y in  {-3,...,1} {
	\draw (\y+\n,0) -- (\y+\n,-0.1);
	\ifthenelse{\y<0 }{\node[below] at (\y+\n,-0.1) {\tiny \(\scriptstyle n \y\)};}{}
	\ifthenelse{\y=0 }{\node[below] at (\y+\n,-0.1) {\tiny \(\scriptstyle n \)};}{}
	\ifthenelse{\y>0}{	\node[below] at (\y+\n,-0.1) {\tiny \(\scriptstyle n+\y\)};}{}
	\ifthenelse{\y<1 }{	\draw[ line width=1, color=OrangeProfondIRA, arrows={-Stealth[length=4, inset=0]}] (\y+\n,0) -- (\y+\n,1);}
}
\draw[ line width=1] (-.25,0) -- (4,0);
\draw[ line width=1, dashed] (4,0) -- (5,0);
\draw[arrows={-Stealth[length=4, inset=0]}, line width=1] (5,0) -- (\w,0);
\end{tikzpicture}
\end{center}
\[
a_{\lcroof{n}}=v+v^2+\cdots+v^{n}=v\frac{1-v^{n}}{1-v}=\frac{1-v^{n}}{i}
\]


Renta vitalicia contingente \(\ddot{a}_{x}\): sus pagos están condicionados a un evento aleatorio, como una renta vitalicia de una persona de \(x\). En este ejemplo, los pagos continúan hasta el fallecimiento :

\begin{center}
	\begin{tikzpicture}[scale=0.75]
	% Draw the x-axis and y-axis.
	\def\w{11}
	\def\n{6}
	\node[left] at (-.5,0) {\(a_{x}\)};
	
	\begin{scope}[shift={(\n+.5+3,.25)}]
		\draw[color=OrangeProfondIRA,scale=0.2,fill=OrangeProfondIRA] \Cerceuil;
	\end{scope}
	\foreach \y in  {0,...,3} {
		\draw (\y,0) -- (\y,-0.1);
		\ifthenelse{\y>0 }{	\node[below] at (\y,-0.1) {\tiny \( \scriptstyle x+\y\)};
			\draw[ line width=1, color=OrangeProfondIRA, arrows={-Stealth[length=4, inset=0]}] (\y,0) -- (\y,1);}{
			\node[below] at (\y,-0.1) {\tiny \( \scriptstyle x\)};}
	}
	\foreach \y in  {0,...,4} {
		\draw (\y+\n,0) -- (\y+\n,-0.1);
		\ifthenelse{\y>0 }{\node[below] at (\y+\n,-0.1) {\tiny \(\scriptstyle x+n+\y\)};}{
			\node[below] at (\y+\n,-0.1) {\tiny \(\scriptstyle x+n\)};}
		\ifthenelse{\y<4 }{	\draw[ line width=1, color=OrangeProfondIRA, arrows={-Stealth[length=4, inset=0]}] (\y+\n,0) -- (\y+\n,1);}
	}
\draw[ line width=1] (-.25,0) -- (4,0);
\draw[ line width=1, dashed] (4,0) -- (5,0);
\draw[arrows={-Stealth[length=4, inset=0]}, line width=1] (5,0) -- (\w,0);
\end{tikzpicture}	

\end{center}
La fecha del fallecimiento se representa aquí mediante un pequeño ataúd. Este tipo de anualidad se estudiará extensamente en la sección sobre seguros de vida.

Anualidad vencida (inmediata) \(a_{\lcroof{n}}\): sus pagos periódicos se realizan al final de cada período de pago, como un salario a fin de mes. Este es el caso predeterminado, ilustrado previamente para la anualidad en cuestión.
\[
\ddot{a}_{\lcroof{n}}=1+v+\cdots+v^{n-1}=\frac{1-v^{n}}{1-v}=\frac{1-v^{n}}{d}
\]
%
\[
	\mathrm{PV}_{\lcroof{n}}^{\text {due }}=K \ddot{a}_{\lcroof{n}}=K \frac{1-v^{n}}{d} 
\]

Anualidad anticipada (vencida) \(\ddot{a}_{\lcroof{n}}\): sus pagos periódicos se efectúan al inicio de cada período de pago, como ocurre, por ejemplo, con el pago del alquiler.

\begin{center}
	\begin{tikzpicture}[scale=0.75]
		% Draw the x-axis and y-axis.
		\def\w{11}
		\def\n{9}
		\node[left] at (-.5,0) {\(\ddot{a}_{\lcroof{n}}\)};
		\foreach \y in  {0,...,3} {
			\draw (\y,0) -- (\y,-0.1);
				\node[below] at (\y,-0.1) {\tiny \( \scriptstyle \y\)};
				\draw[ line width=1, color=OrangeProfondIRA, arrows={-Stealth[length=4, inset=0]}] (\y,0) -- (\y,1);
		}
		\foreach \y in  {-3,...,1} {
			\draw (\y+\n,0) -- (\y+\n,-0.1);
			\ifthenelse{\y<0 }{\node[below] at (\y+\n,-0.1) {\tiny \(\scriptstyle n \y\)};}{}
			\ifthenelse{\y=0 }{\node[below] at (\y+\n,-0.1) {\tiny \(\scriptstyle n \)};}{}
			\ifthenelse{\y>0}{	\node[below] at (\y+\n,-0.1) {\tiny \(\scriptstyle n+\y\)};}{}
			\ifthenelse{\y<0 }{	\draw[ line width=1, color=OrangeProfondIRA, arrows={-Stealth[length=4, inset=0]}] (\y+\n,0) -- (\y+\n,1);}{}
		}
		\draw[ line width=1] (-.25,0) -- (4,0);
		\draw[ line width=1, dashed] (4,0) -- (5,0);
		\draw[arrows={-Stealth[length=4, inset=0]}, line width=1] (5,0) -- (\w,0);
	\end{tikzpicture}
\end{center}
También denotada \(\mathrm{PV}^{\mathrm{im}}\) :
\[
a_{\lcroof{n}}=v+v^{2}+\cdots+v^{n}=\frac{1-v^{n}}{i}=v \frac{1-v^{n}}{1-v}
\]
%
\[
	\mathrm{PV}_{\lcroof{n}}^{\mathrm{im}}=K a_{\lcroof{n}}=K \frac{1-v^{n}}{i} 
\]
Anualidad perpetua \(a\) ó  \(a_{\lcroof{\infty}}\): 
\[
a=1/i
\]
Anualidad diferida \(_{m|}a_{\lcroof{n}}\): sus pagos no comienzan en el primer periodo sino después de \(m\) periodos, con \(m\) fijos de antemano.
\begin{center}
	\begin{tikzpicture}[scale=0.75]
		% Draw the x-axis and y-axis.
\def\w{10}
\def\n{9}
\def\m{2}
\node[left] at (-.5,0) {\(_{m|}a_{\lcroof{n}}\)};
\foreach \y in  {0,...,2} {
	\draw (\y+\m,0) -- (\y+\m,-0.1);
	\ifthenelse{\y<0 }{\node[below] at (\y+\m,-0.1) {\tiny \(\scriptstyle m \y\)};}{}
\ifthenelse{\y=0 }{\node[below] at (\y+\m,-0.1) {\tiny \(\scriptstyle m \)};}{}
\ifthenelse{\y>0}{	\node[below] at (\y+\m,-0.1) {\tiny \(\scriptstyle m+\y\)};
		\draw[ line width=1, color=OrangeProfondIRA, arrows={-Stealth[length=4, inset=0]}] (\y+\m,0) -- (\y+\m,1);
		}
}
\foreach \y in  {-3,...,0} {
	\draw (\y+\n,0) -- (\y+\n,-0.1);
	\ifthenelse{\y<0 }{\node[below] at (\y+\n,-0.1) {\tiny \(\scriptstyle m+n \y\)};}{}
	\ifthenelse{\y=0 }{\node[below] at (\y+\n,-0.1) {\tiny \(\scriptstyle m+n \)};}{}
	\ifthenelse{\y>0}{	\node[below] at (\y+\n,-0.1) {\tiny \(\scriptstyle m+n+\y\)};}{}
	\ifthenelse{\y<1 }{	\draw[ line width=1, color=OrangeProfondIRA, arrows={-Stealth[length=4, inset=0]}] (\y+\n,0) -- (\y+\n,1);}
}
\draw[ line width=1] (-.25,0) -- (0.5,0);
\draw[ line width=1, dashed] (0.5,0) -- (1.5,0);
\draw[ line width=1] (1.5,0) -- (4,0);
\draw[ line width=1, dashed] (4,0) -- (5,0);
		\draw[arrows={-Stealth[length=4, inset=0]}, line width=1] (5,0) -- (\w,0);
	\end{tikzpicture}	
	
\end{center}


Anualidad periódica/mensual \(a^{(m)}\) : la periodicidad predeterminada es de un año, pero el pago unitario también puede distribuirse en \(m\) períodos dentro del año.

\begin{center}
	\begin{tikzpicture}[scale=0.75]
		% Draw the x-axis and y-axis.
		\def\w{10}
		\def\n{9}
		\def\m{6}
		\node[left] at (-.5,0) {\(a_{\lcroof{n}}^{(m)}\)};
		\foreach \y in  {0,...,3} {
			\draw (\y,0) -- (\y,-0.1);
			\ifthenelse{\y>0 }{	\node[below] at (\y,-0.1) {\tiny \( \scriptstyle \y\)};}{
				\node[below] at (\y,-0.1) {\tiny 0};}
		}
		\pgfmathparse{3.5*\m} 
		\foreach \y in  {1,...,\pgfmathresult} {
			\ifthenelse{\y>0 }{	
				\draw[ line width=1, color=OrangeProfondIRA, arrows={-Stealth[length=4, inset=0]}] (\y/\m,0) -- (\y/\m,2/\m);}{}
		}
		\foreach \y in  {-3,...,0} {
			\draw (\y+\n,0) -- (\y+\n,-0.1);
			\ifthenelse{\y<0 }{\node[below] at (\y+\n,-0.1) {\tiny \(\scriptstyle n \y\)};}{}
			\ifthenelse{\y=0 }{\node[below] at (\y+\n,-0.1) {\tiny \(\scriptstyle n \)};}{}
			\ifthenelse{\y>0}{	\node[below] at (\y+\n,-0.1) {\tiny \(\scriptstyle n+\y\)};}{}
		}
		\pgfmathparse{3.5*\m} 
		\foreach \y  in  {1,...,\pgfmathresult} {
		\draw[ line width=1, color=OrangeProfondIRA, arrows={-Stealth[length=4, inset=0]}] (-\y/\m+\n,0) -- (-\y/\m+\n,2/\m);
		}
		\draw[ line width=1] (-.25,0) -- (3.5,0);
		\draw[ line width=1, dashed] (3.5,0) -- (5.5,0);
		\draw[arrows={-Stealth[length=4, inset=0]}, line width=1] (5.5,0) -- (\w,0);
	\end{tikzpicture}
\end{center}

 Si \(i^{(m)}\) representa la tasa de interés nominal (anual) equivalente con \(m\) períodos por año, enontces \(i^{(m)}= m\left((1+i)^{1 / m}-1\right)\) .

De manera similar, \(d^{(m)} \) es la tasa de descuento nominal consistente con \(d\) y \(m\) : \(d^{(m)}= m\left(1-(1-d)^{1 / m}\right)\).

\[
\ddot{a}_{\lcroof{n}}^{(m)}=\frac{1}{m} \sum_{k=0}^{m n-1} v^{\frac{k}{m}}=\frac{d}{d^{(m)}} \ddot{a}_{\lcroof{n}}=\frac{1-v^{n}}{d^{(m)}} \approx \ddot{a}_{\lcroof{n}}+\frac{m-1}{2 m}\left(1-v^{n}\right)
\]
\[
a_{\lcroof{n}}^{(m)}=\frac{1}{m} \sum_{k=1}^{m n} v^{\frac{k}{m}}=\frac{i}{i^{(m)}} a_{\lcroof{n}}=\frac{1-v^{n}}{i^{(m)}} \approx a_{\lcroof{n}}-\frac{m-1}{2 m}\left(1-v^{n}\right)
\]
Anualidad unitaria \(a\) : se utiliza al construir fórmulas de anualidades.
Para una anualidad constante, el monto total pagado cada año es 1, independientemente de \(m\). 

Anualidad dinámica, creciente/decreciente \(Ia\)/\(Da\): en su forma más simple, paga una cantidad que comienza en 1 (\(n\)) y aumenta (decrece) cada período aritmética o geométricamente. En el siguiente ejemplo, la progresión es aritmética.
El prefijo \(I\) (\engl{increasing}) se utiliza para indicar anualidades crecientes y \(D\) (\engl{decreasing}) para anualidades decrecientes.
	
\begin{center}
	\begin{tikzpicture}[scale=0.75]
		% Draw the x-axis and y-axis.
		\def\w{11}
		\def\n{9}
		\node[left] at (-.5,0) {\(Ia_{\lcroof{n}}\)};
		\foreach \y in  {0,...,3} {
			\draw (\y,0) -- (\y,-0.1);
			\ifthenelse{\y>0 }{	\node[below] at (\y,-0.1) {\tiny \( \scriptstyle \y\)};
				\draw[ line width=1, color=OrangeProfondIRA, arrows={-Stealth[length=4, inset=0]}] (\y,0) -- (\y,\y/3);}{
				\node[below] at (\y,-0.1) {\tiny 0};}
		}
		\foreach \y in  {-3,...,1} {
			\draw (\y+\n,0) -- (\y+\n,-0.1);
			\ifthenelse{\y<0 }{\node[below] at (\y+\n,-0.1) {\tiny \(\scriptstyle n \y\)};}{}
			\ifthenelse{\y=0 }{\node[below] at (\y+\n,-0.1) {\tiny \(\scriptstyle n \)};}{}
			\ifthenelse{\y>0}{	\node[below] at (\y+\n,-0.1) {\tiny \(\scriptstyle n+\y\)};}{}
			\ifthenelse{\y<1 }{	\draw[ line width=1, color=OrangeProfondIRA, arrows={-Stealth[length=4, inset=0]}] (\y+\n,0) -- (\y+\n,\y/3+\n/3);}
		}
		\draw[ line width=1] (-.25,0) -- (4,0);
		\draw[ line width=1, dashed] (4,0) -- (5,0);
		\draw[arrows={-Stealth[length=4, inset=0]}, line width=1] (5,0) -- (\w,0);
	\end{tikzpicture}
	\begin{tikzpicture}[scale=0.75]
	% Draw the x-axis and y-axis.
	\def\w{11}
	\def\n{9}
	\node[left] at (-.5,0) {\(Da_{\lcroof{n}}\)};
	\foreach \y in  {0,...,3} {
		\draw (\y,0) -- (\y,-0.1);
		\ifthenelse{\y>0 }{	\node[below] at (\y,-0.1) {\tiny \( \scriptstyle \y\)};
			\draw[ line width=1, color=OrangeProfondIRA, arrows={-Stealth[length=4, inset=0]}] (\y,0) -- (\y,\n/3-\y/3);}{
			\node[below] at (\y,-0.1) {\tiny 0};}
	}
	\foreach \y in  {-3,...,1} {
		\draw (\y+\n,0) -- (\y+\n,-0.1);
		\ifthenelse{\y<0 }{\node[below] at (\y+\n,-0.1) {\tiny \(\scriptstyle n \y\)};}{}
		\ifthenelse{\y=0 }{\node[below] at (\y+\n,-0.1) {\tiny \(\scriptstyle n \)};}{}
		\ifthenelse{\y>0}{	\node[below] at (\y+\n,-0.1) {\tiny \(\scriptstyle n+\y\)};}{}
		\ifthenelse{\y<1 }{	\draw[ line width=1, color=OrangeProfondIRA, arrows={-Stealth[length=4, inset=0]}] (\y+\n,0) -- (\y+\n,1/3-\y/3);}
	}
	\draw[ line width=1] (-.25,0) -- (4,0);
	\draw[ line width=1, dashed] (4,0) -- (5,0);
	\draw[arrows={-Stealth[length=4, inset=0]}, line width=1] (5,0) -- (\w,0);
\end{tikzpicture}
\end{center}

\begin{equation}
(I \ddot{a})_{\lcroof{n}}=1+2 v+\cdots+n v^{n-1}=\frac{1}{d}\left(\ddot{a}_{\lcroof{n}}-n v^{n}\right)
\label{Ian}
\end{equation}
con, recordamos,  \(d=i/(1+i)\) y vencida (inmediata)
\[
(I a)_{\lcroof{n}}=v+2 v^{2}+\cdots+n v^{n}=\frac{1}{i}\left(\ddot{a}_{\lcroof{n}}-n v^{n}\right)
\]
\[
(D \ddot{a})_{\lcroof{n}}=n+(n-1) v+\cdots+v^{n-1}=\frac{1}{d}\left(n-a_{\lcroof{n}}\right)
\]
y vencida :
\[
(D a)_{\lcroof{n}}=n v+(n-1) v^{2}+\cdots+v^{n}=\frac{1}{i}\left(n-a_{\lcroof{n}}\right)
\]




\end{f}
\hrule

\begin{f}[L'emprunt (Indivis)]
\ \newline

La principale propriété de l'emprunt est de considéré séparément les intérêts du remboursement (ou de l'amortissement).

Par un remboursement constant ou par annuité constante: la somme de l'amortissement et de l'intérêt à chaque période est constante. 

\begin{center}
\begin{tikzpicture}[scale=0.85]
% Draw the x-axis and y-axis.
\def\w{11}
\def\n{9}
\def\cap{13} % Capital initial
\def\a{1.2334723} % annuité exacte

\def\v{1.031415^(-1)} 
\draw[ line width=10, color=BleuProfondIRA, opacity=0.2] (0,0) -- (0,-\n);
\draw[ line width=1, color=BleuProfondIRA, arrows={-Stealth[length=4, inset=0]}] (0,0) -- (0,-\n);
\foreach \y in  {0,...,10} {
	\ifthenelse{\y=0}{\def\caprd{9.000000}}{\def\caprd{0.000000}}
	\ifthenelse{\y=1}{\def\caprd{8.174923}}{}
	\ifthenelse{\y=2}{\def\caprd{7.323926}}{}
	\ifthenelse{\y=3}{\def\caprd{6.446194}}{}

	\ifthenelse{\y=6}{\def\caprd{4.569466}}{}
	\ifthenelse{\y=7}{\def\caprd{3.4795435}}{}
	\ifthenelse{\y=8}{\def\caprd{2.355381}}{}
	\ifthenelse{\y=9}{\def\caprd{0.000000}}{}
\ifthenelse{\y=0}{	
	\draw (\y,0) -- (\y,-0.1);
	\node[below left] at (\y,-0.1) {\tiny 0};}{}
\ifthenelse{\y<4 \AND \y>0}{
	\draw (\y,0) -- (\y,-0.1);
	\node[below] at (\y,-0.1) {\tiny \( \scriptstyle \y\)};
	\pgfmathparse{\a-\cap*0.031415*\v^(\y) }
	\let\amortissement\pgfmathresult
		\draw[ line width=10, color=BleuProfondIRA, opacity=0.2] (\y,0) -- (\y,-\caprd);
		\draw[ line width=1, color=BleuProfondIRA, arrows={-Stealth[length=4, inset=0]}] (\y,0) -- (\y,\amortissement);
		\draw[ line width=1, color=OrangeProfondIRA, arrows={-Stealth[length=4, inset=0]}] (\y,\amortissement) -- (\y,\a);
		}{}
\ifthenelse{\y>5 }{
	\draw (\y,0) -- (\y,-0.1);
	\pgfmathparse{\a-\cap*0.031415*\v^(\y+4) }
	\let\amortissement\pgfmathresult
	\pgfmathparse{\y-\n} 	
	\let\zdec\pgfmathresult
	\pgfmathtruncatemacro{\z}{\zdec}
	\ifthenelse{\y<\n }{\node[below] at (\y,-0.1) {\tiny \(\scriptstyle n \z\)};}{}
	\ifthenelse{\y=\n }{\node[below] at (\y,-0.1) {\tiny \(\scriptstyle n \)};}{}
	\ifthenelse{\y>\n}{	\node[below] at (\y,-0.1) {\tiny \(\scriptstyle n+\z\)};}{}
	\ifthenelse{\y<\n \OR \y=\n}{
		\draw[ line width=10, color=BleuProfondIRA, opacity=0.2] (\y,0) -- (\y,-\caprd);
		\draw[ line width=1, color=BleuProfondIRA, arrows={-Stealth[length=4, inset=0]}] (\y,0) -- (\y,\amortissement);
		\draw[ line width=1, color=OrangeProfondIRA, arrows={-Stealth[length=4, inset=0]}] (\y,\amortissement) -- (\y,\a);}{}
}{}
}
\draw[ line width=1] (-.25,0) -- (4,0);
\draw[ line width=1, dashed] (4,0) -- (5,0);
\draw[arrows={-Stealth[length=4, inset=0]}, line width=1] (5,0) -- (\w,0);
\end{tikzpicture}
\end{center}
	
    
    Par un amortissement constant.
\begin{center}
\begin{tikzpicture}[scale=0.85]
% Draw the x-axis and y-axis.
\def\w{11}
\def\n{9}
\def\duree{13}
\def\i{0.031415}
\draw[ line width=10, color=BleuProfondIRA, opacity=0.2] (0,0) -- (0,-\n);
\draw[ line width=1, color=BleuProfondIRA, arrows={-Stealth[length=4, inset=0]}] (0,0) -- (0,-\n);
\foreach \y in  {0,...,3} {
\draw (\y,0) -- (\y,-0.1);
\ifthenelse{\y>0 }{	\node[below] at (\y,-0.1) {\tiny \( \scriptstyle \y\)};
	\pgfmathparse{(\duree-\y)*\i} 	
	\let\Interet\pgfmathresult
	\pgfmathparse{(\n) *(1-\y/\n))} 	
	\let\Cap\pgfmathresult
	\draw[ line width=10, color=BleuProfondIRA, opacity=0.2] (\y,0) -- (\y,-\Cap);
	\draw[ line width=1, color=BleuProfondIRA, arrows={-Stealth[length=4, inset=0]}] (\y,0) -- (\y,1);
	\draw[ line width=1, color=OrangeProfondIRA, arrows={-Stealth[length=4, inset=0]}] (\y,1) -- (\y,1+\Interet);}{
	\node[below left] at (\y,-0.1) {\tiny 0};}
}
\foreach \y in  {-3,...,1} {
	\pgfmathparse{(-\y+1)*\i} 	
\let\Interet\pgfmathresult
\draw (\y+\n,0) -- (\y+\n,-0.1);
\ifthenelse{\y<0 }{\node[below] at (\y+\n,-0.1) {\tiny \(\scriptstyle n \y\)};}{}
\ifthenelse{\y=0 }{\node[below] at (\y+\n,-0.1) {\tiny \(\scriptstyle n \)};}{}
\ifthenelse{\y>0}{	\node[below] at (\y+\n,-0.1) {\tiny \(\scriptstyle n+\y\)};}{}
\ifthenelse{\y<1 }{	
	\draw[ line width=10, color=BleuProfondIRA, opacity=0.2] (\y+\n,0) -- (\y+\n,\y);
	\draw[ line width=1, color=BleuProfondIRA, arrows={-Stealth[length=4, inset=0]}] (\y+\n,0) -- (\y+\n,1);
	\draw[ line width=1, color=OrangeProfondIRA, arrows={-Stealth[length=4, inset=0]}] (\y+\n,1) -- (\y+\n,1+\Interet);}{}
}
\draw[ line width=1] (-.25,0) -- (4,0);
\draw[ line width=1, dashed] (4,0) -- (5,0);
\draw[arrows={-Stealth[length=4, inset=0]}, line width=1] (5,0) -- (\w,0);
\end{tikzpicture}
\end{center}

    
Par un remboursement in fine, où l'intérêt est constant. Seuls les intérêts sont versés périodiquement jusqu'au terme, moment où le remboursement total est effectué.
	
\begin{center}
\begin{tikzpicture}[scale=0.85]
% Draw the x-axis and y-axis.
\def\w{11}
\def\n{9}
\def\duree{23}
\def\i{3*0.031415}
\draw[ line width=10, color=BleuProfondIRA, opacity=0.2] (0,0) -- (0,-\n);
\draw[ line width=1, color=BleuProfondIRA, arrows={-Stealth[length=4, inset=0]}] (0,0) -- (0,-\n);
\foreach \y in  {0,...,3} {
	\draw (\y,0) -- (\y,-0.1);
	\ifthenelse{\y>0 }{	\node[below] at (\y,-0.1) {\tiny \( \scriptstyle \y\)};
%				\draw[ line width=1, color=BleuProfondIRA, arrows={-Stealth[length=4, inset=0]}] (\y,0) -- (\y,1);
		\draw[ line width=10, color=BleuProfondIRA, opacity=0.2] (\y,0) -- (\y,-\n);
		\draw[ line width=1, color=OrangeProfondIRA, arrows={-Stealth[length=4, inset=0]}] (\y,0) -- (\y,2*\i);}{
		\node[below left] at (\y,-0.1) {\tiny 0};}
}
\foreach \y in  {-3,...,1} {
	\draw (\y+\n,0) -- (\y+\n,-0.1);
	\ifthenelse{\y<0 }{\node[below] at (\y+\n,-0.1) {\tiny \(\scriptstyle n \y\)};
		\draw[ line width=10, color=BleuProfondIRA, opacity=0.2] (\y+\n,0) -- (\y+\n,-\n);
		\draw[ line width=1, color=OrangeProfondIRA, arrows={-Stealth[length=4, inset=0]}] (\y+\n,0) -- (\y+\n,2*\i);}{}
	\ifthenelse{\y=0 }{
		\draw[ line width=1, color=BleuProfondIRA, arrows={-Stealth[length=4, inset=0]}] (\y+\n,0) -- (\y+\n,2);
		\draw[ line width=1, color=OrangeProfondIRA, arrows={-Stealth[length=4, inset=0]}] (\y+\n,2) -- (\y+\n,2+2*\i);
		\node[below] at (\y+\n,-0.1) {\tiny \(\scriptstyle n \)};}{}
	\ifthenelse{\y>0}{	\node[below] at (\y+\n,-0.1) {\tiny \(\scriptstyle n+\y\)};}{}
}
\draw[ line width=1] (-.25,0) -- (4,0);
\draw[ line width=1, dashed] (4,0) -- (5,0);
\draw[arrows={-Stealth[length=4, inset=0]}, line width=1] (5,0) -- (\n,0);
\end{tikzpicture}
\end{center}

\end{f}
\hrule \

\begin{f} [Tableau d'amortissement de l'emprunt]
\ \newline

	\footnotesize
\renewcommand{\arraystretch}{2}
\begin{tabular}{|m{10ex}|m{19ex}|m{19ex}|m{17ex}|}
\rowcolor{BleuProfondIRA!40}   	\hline &\textbf{In fine}			&   	\textbf{Amortissements  constants} &  		\textbf{Annuités  constantes} \\
	\hline Capital restant dû \(S_k\) & \(T_k=S_0\), \(T_n=0\)& \(S_{0}\left(1-\frac{k}{n}\right)\) & \(S_{0} \frac{1-v^{n-k}}{1-v^{n}}\)\\
	\hline Intérêts \(U_k\) 		& \(i\times S_0\) & \(S_{0}\left(1-\frac{k-1}{n}\right) i\) & \(K\left(1-v^{n-k+1}\right)\) \\
	\hline Amortis\-sements \(T_k\) &	\(T_k=O\), \(T_n=S_0\) & \(\frac{S_{0}}{n}\) & \(K v^{n-k+1}\) \\
	\hline Annuité \(K_k\) & 	\(K_k=i S_0\), \(K_n=(1+i)S_0\) 	 & \(\frac{S_{0}}{n}(1-(n-k+1) i) \) & \(K=S_{0} \frac{i}{1-v^{n}} \) \\
%	\hline Coût de l'emprunt & \(1 \times K \times N\) & \(1 \times K \times \frac{N+1}{2}\) & \(K\left(\frac{N i}{1-(1+i)^{-N}}-1\right)\) \\
	\hline
\end{tabular}

\end{f}
