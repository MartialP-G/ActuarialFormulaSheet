% !TeX root = ActuarialFormSheet_MBFA-en.tex
% !TeX spellcheck = en_GB
\begin{f}[Concept of utility]
	Utility models an individual's preferences between two baskets of goods \(x\) and \(y\) in a set \(S\), via the relation \(x \succcurlyeq y\) (preferred or indifferent).
	
	A function \(U: S \rightarrow \mathbb{R}\) represents preferences if :
	\[
	x \succcurlyeq y \iff U(x) \geq U(y)
	\]
	
	\textbf{Axioms necessary for the existence of a utility function :}
	\begin{enumerate}
		\item \textbf{Completeness:} For all \(x, y \in S\), either \(x \succcurlyeq y\), or \(y \succcurlyeq x\)
		\item \textbf{Transitivity:} If \(x \succcurlyeq y\) and \(y \succcurlyeq z\), then \(x \succcurlyeq z\)
		\item \textbf{Continuity:} If \(x_n \to x\) and \(y_n \to y\), and \(x_n \succcurlyeq y_n\) for all \(n\), then \(x \succcurlyeq y\)
	\end{enumerate}
	
\end{f}


\begin{f}[Utility function]
	A function \(u : \mathbb{R}_+ \rightarrow \mathbb{R}\) represents an agent's preferences in the face of uncertainty.
	
	\textbf{Expected utility criterion:} The agent prefers \(X\) to \(Y\) if :
	\[
	\mathbb{E}[u(X)] > \mathbb{E}[u(Y)]
	\]
	He chooses \(X\) such that \(\mathbb{E}[u(X)]\) is maximal.
	
	\textbf{Properties of \(u\):}
	\begin{itemize}
		\item \(u'\! > 0\) : the agent prefers more wealth (monotonicity)
		\item \(u'' < 0\) : the agent is risk averse (concavity)
	\end{itemize}
	
	\textbf{Classic examples:}
	\begin{itemize}
		\item \textit{Linear} (risk neutral): \(u(x) = x\)
		\item \textit{Logarithmic} : \(u(x) = \ln(x)\)
		\item \textit{CRRA}  (constant relative risk aversion): \(u(x) = \frac{x^{1 - \gamma}}{1 - \gamma}\), \(\gamma \ne 1\)
		\item \textit{CARA}  (constant absolute risk aversion): \(u(x) = -e^{-a x}\)
	\end{itemize}
	
\end{f}

\hrule

\begin{f}[Aversion au risque]
	An agent is said to be \textbf{risk-averse} (or risk-phobic) if:
	\[
	u(\mathbb{E}[X]) > \mathbb{E}[u(X)]
	\]
	Which is equivalent to \(u\) concave, i.e. \(u''(x) < 0\)
\end{f}

\begin{f}[Measurement of risk aversion]
	\textbf{Absolute Aversion Index} :
	\[
	A_a(x) = -\frac{u''(x)}{u'(x)}
	\]
	
	\textbf{Relative Aversion Index} :
	\[
	A_r(x) = -x \cdot \frac{u''(x)}{u'(x)}
	\]
	
	\textbf{Jensen's inequality (concave case)} :
	\[
	u(\mathbb{E}[X]) \geq \mathbb{E}[u(X)]
	\]
With equality if and only if \(X\) is constant.
\end{f}
\hrule


\begin{f}[Risk premiums]
	
The \textbf{risk premium} \(\pi\) is the maximum amount an individual is willing to pay to replace a random lottery win \(H\) with its certain expectation \(\mathbb{E}[H]\). It verifies:
\[
\mathbb{E}[u(w + H)] = u(w + \mathbb{E}[H] - \pi)
\]


\(\pi\) is also called \emph{Markowitz measure}: it captures the gap between expected utility and certain utility.

Conversely, the \textbf{compensatory bonus} \(\tilde{\pi}\) is the amount that must be offered to an individual so that he accepts the lottery \(H\) instead of a certain gain. It checks:
\[
\mathbb{E}[u(w + H + \tilde{\pi})] = u(w + \mathbb{E}[H])
\]

\end{f}
\hrule
\begin{f}[Diversification and utility]
	
	Let's have two assets:
	\begin{itemize}
		\item \(A\) : risk
		\item \(B\) : certain, with \(\mathbb{E}[A] = B\)
	\end{itemize}
	
	A risk-averse agent prefers a combination \(Z = \alpha A + (1 - \alpha) B\), with \(0 < \alpha < 1\), to the risky asset alone.
	If $u$ is concave, then
	\[
	\mathbb{E}[u(Z)] > \mathbb{E}[u(A)]
	\]
	
	\textbf{Optimal portfolio}: choice of weights \((w_i)\) maximizing expected utility:
	\[
	\max \mathbb{E}[u(X)], \quad \text{où } X = \sum_{i} w_i X_i, \quad \text{s.c. } \sum w_i = 1
	\]
	
	\textbf{Principle}: diversification reduces risk (variance) without affecting expectation.
	
	
\end{f}
\hrule

\begin{f}[Lagrange method for constrained optimization]
	
	The Lagrange multiplier method is used to solve a constrained optimization problem.
	
	\textbf{Objective}: maximize/minimize \(f(\boldsymbol{x})\) under the constraint \(g(\boldsymbol{x}) = c\), where \(\boldsymbol{x} \in \mathbb{R}^d\) is a vector of variables.
	
	\textbf{Steps of the method:}
	\begin{enumerate}
		\item \textbf{Identification} : determine the objective function \(f(\boldsymbol{x})\) and the constraint \(g(\boldsymbol{x}) = c\)
		\item \textbf{Lagrangian} :
		\[
		\mathcal{L}(\boldsymbol{x}, \lambda) = f(\boldsymbol{x}) + \lambda (g(\boldsymbol{x}) - c)
		\]
		\item \textbf{System of equations} : solve
		\[
		\nabla_{\boldsymbol{x}} \mathcal{L} = \nabla f(\boldsymbol{x}) + \lambda \nabla g(\boldsymbol{x}) = \boldsymbol{0}, \quad
		\frac{\partial \mathcal{L}}{\partial \lambda} = g(\boldsymbol{x}) - c = 0
		\]
		\item \textbf{Resolution} of the system to obtain \(\boldsymbol{x}^*, \lambda^*\)
		\item \textbf{Verification}: ensure that the solutions satisfy the constraint and the type of optimum (max/min)
	\end{enumerate}
	
	\textbf{Example (dimension 2)} : maximize \(f(x, y) = xy\) under the constraint \(x + y = 10\)
	
	\[
	\mathcal{L}(x, y, \lambda) = xy + \lambda (x + y - 10)
	\]
	
	We derive:
	\[
	\frac{\partial \mathcal{L}}{\partial x} = y + \lambda = 0, \quad
	\frac{\partial \mathcal{L}}{\partial y} = x + \lambda = 0, \quad
	\frac{\partial \mathcal{L}}{\partial \lambda} = x + y - 10 = 0
	\]
	
	We solve the system:
\[
\begin{cases}
	y + \lambda = 0 \\
	x + \lambda = 0 \\
	x + y = 10
\end{cases}
\Rightarrow 
\begin{cases}
	\lambda = -y \\
	x = -\lambda = y \\
	x + y = 10 \Rightarrow 2x = 10 
\end{cases}
\Rightarrow 
	\begin{cases} 
		x^* =  y^*= 5,\\
		f(5,5) = 25
	\end{cases}
\]

		\textbf{Example (Optimal choice and budget constraint)}
	
	A rational agent is faced with a consumption choice \((c_1, c_2)\) between two goods, under the constraint :
	\[
	p_1 c_1 + p_2 c_2 = R
	\]
	where \(p_1, p_2\) are prices and \(R\) total revenue.
	
	\textbf{Issue} :
	\(
	\max_{c_1, c_2} u(c_1, c_2) \quad \text{s.c. } p_1 c_1 + p_2 c_2 = R\)

	
	\textbf{Method}: introduce the \textbf{Lagrangian}
	\[
	\mathcal{L}(c_1, c_2, \lambda) = u(c_1, c_2) + \lambda (R - p_1 c_1 - p_2 c_2)
	\]
	
	\textbf{First order conditions (FOC)} :
	\[
	\begin{cases}
		\frac{\partial u}{\partial c_1} = \lambda p_1 \\
		\frac{\partial u}{\partial c_2} = \lambda p_2 \\
		p_1 c_1 + p_2 c_2 = R
	\end{cases}
	\]
	
	By dividing the first two equations :
	\[
	\frac{\partial u / \partial c_1}{\partial u / \partial c_2} = \frac{p_1}{p_2}
	\]
	
	This ratio is called the \textbf{marginal rate of substitution (MRS)}: it measures the quantity of good 2 that the agent is willing to give up to obtain an additional unit of good 1, while maintaining his level of utility constant.
	
\end{f}
\hrule

\begin{f}[Insurance Application (Mosin)]
	
An agent has an initial wealth \(w\) and faces a random loss \(L\). There exists an \textbf{insurance demand} for the insurance that pays the indemnity $0<I(L)<L$ iff $u(w-\pi_I) \geq \mathbb{E}(u(w-L))$ and the \textbf{optimal insurance} maximizes $u(w-\pi_I)$.

In Mosin (1968) or Borch (1961) or Smith (1968), the loss model $L$ is simply defined by $s$ between 0 and \(w\) :
	$$
	L=\left\{\begin{array}{l}
		0 \text { with prob. } 1-p \\
		s \text { with prob. } p
	\end{array}\right.
	$$
The premium becomes $\pi_I=(1+\lambda) \mathbb{E}(I(L))=(1+\lambda) p I(s)$ with $\lambda$ a loading. We denote by $\pi$ the case where $I(L)=L$ with $\pi= p s$. If $\lambda=0$, then we speak of a pure or actuarially fair premium.
\medskip

\textbf{Co-insurance} (risk sharing) :
$I(l)=\alpha l$ knowing $L=l$ for $\alpha \in[0,1]$, $\pi_I(\alpha)=\alpha \pi$ and :
$$
w_{f}=w-L+I(L)-\pi(\alpha)=w-L+\alpha L-\alpha \pi=w-(1-\alpha) L-\alpha \pi
$$
$$
U(\alpha)%=\mathbb{E}(u(w-(1-\alpha) L-\alpha \pi))
=(1-p) u(w-\alpha \pi)+p u(w-(1-\alpha) s-\alpha \pi)
$$

Partial insurance ($\alpha^{\star}<1$) is optimal iff $\lambda>0$. Total insurance ($\alpha^{\star}=1$) is optimal if the loading is zero.
\medskip

\textbf{Insurance with deductible} (self-insurance portion):
With deductible $d$ the insurer pays an indemnity $I(l)=(l-d)_{+}$ knowing $L=l$.
$$
\pi(d)=(1+\lambda) E\left((L-d)_{+}\right)=(1+\lambda)(s-d) p
$$
$$
w_{f}=w-X+(L-d)_{+}-\pi(d)=w-\min (X, d)-(1+\lambda)(s-d) p
$$
$$
U(d)=(1-p) u(\underbrace{w+(1+\lambda)(d-s) p}_{w_{f}^{+}})+p u(\underbrace{w-d+(1+\lambda)(d-s) p}_{w_{f}^{-}}) .
$$
In the deductible model, partial insurance ($d^{\star}>0$) is optimal iff the premium is not actuarially fair. Similarly, full insurance ($d^{\star}=0$) is optimal if the loading is zero.
\medskip

\textbf{Generalized model:}
The random loss risk $L>0$ is defined on $\Re$, with distribution function $F_L$ ),

$$
\pi_I=(1+\lambda) \mathbb{E}(I(L))=(1+\lambda) \int_{0}^{\infty} I(l) d F_{L}(l)
$$



\begin{enumerate}
	\item Total insurance \((d^\star = 0)\) or \((\alpha^\star = 1)\) is optimal if and only if the premium is actuarially fair.
	
	\item If \(A_a(u, x)\) is decreasing, then the deductible level \(d^\star\) or the coverage rate \(\alpha^\star\) increases with initial wealth.
	For CARA preferences, \(d^\star\) is independent of \(w\) or \(\alpha^\star\) is constant.
	
	\item The coverage level decreases with the loading coefficient \(\lambda\) when \(A_a(u, x)\) is increasing or constant.
	
	\item A more risk-averse agent chooses higher coverage.
\end{enumerate}
\end{f}

\hrule

\begin{f}[Information and insurance]
	
	\textbf{Mosin with heterogeneity:} Two types of individuals: $H$ for high risk and $Lo$ for low risk. $\theta \in[0,1]$ the proportion of individuals $H$. Individuals of type $i \in\{Lo, H\}$ face a risk of the same amount $s$ occurring with a different probability $p_{i}$ such that $1>p_{H}>p_{Lo}>0$.
	
	$$
	L_{i}=\left\{\begin{array}{l}
		0 \text { with probability } 1-p_{i}, \\
		s \text { with probability } p_{i} .
	\end{array}\right.
	$$
Market probability :
	$$
	p_{m}=\theta p_{H}+(1-\theta) p_{Lo} .
	$$
\medskip

\textbf{Absence of adverse selection:} In this model, in the presence of total information, the insurer prefers individual insurance $I_{i}=s$ and $\pi_{i}=s p_{i}$, $\forall i$.

\textbf{The adverse selection problem : }
The insurer offers a non-individualized contract from the market $M=\left(\pi_{m}=p_{m} I, I_{m}(s)=I(s)\right)$, which does not depend on $i$. The final fortune of an individual of type $i$ is $W_{i}^{m}=w-\pi_{m}-X_{i}+I_{m}$.

In the presence of a single contract, individuals of type $H$ prefer an insurance contract such that $I_{H}(s)=s$ and $\pi_{H}=s p_{m}$, while individuals of type $Lo$ prefer partial coverage with $I_{L}^{\star}<s$ and $\pi_{Lo}=I_{L}^{\star} p_{m}$.
\medskip

\textbf{Moral hazard :}
Insuring him reduces or interrupts his efforts now that he is insured. The efforts of
\begin{itemize}
	\item prevention reduces the probability of disaster,
	\item protection reduces the amount of loss.
\end{itemize}

 In the absence of any effort $e$ to prevent or reduce risk, the final fortune $w_{f}$ is simply defined by

$$
\begin{cases}w_{f}^{-}=w-s-\pi(I)+I & \text { with probability } p \\ w_{f}^{+}=w-\pi(I) & \text { with probability } 1-p\end{cases}
$$

If there is an effort $e$ to prevent risks, we consider

$$
\begin{cases}w_{f}^{-}=w-s-\pi(I)+I-e & \text { with probability } p(e) \\ w_{f}^{+}=w-\pi(I)-e & \text { with probability } 1-p(e)\end{cases}
$$

If there is an effort $e$ to protect against risks, we consider

$$
\begin{cases}w_{f}^{-}=w-s(e)-\pi(I)+I-e & \text { with probability } p \\ w_{f}^{+}=w-\pi(I)-e & \text { with probability } 1-p\end{cases}
$$

with
\begin{itemize}
	\item $e \mapsto p(e)$ is strictly decreasing and strictly convex.
	\item $e \mapsto s(e)$ is strictly decreasing and strictly convex.\\
	\item $I \leq s$ implies $w_{f}^{-} \leq w_{f}^{+}$
\end{itemize}

\end{f}
\hrule


