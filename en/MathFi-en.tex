% !TeX root = ActuarialFormSheet_MBFA-en.tex
% !TeX spellcheck = en_US

\begin{f}[Capitalization Discounting]

\begin{tikzpicture}[scale=0.85]
% Draw the x-axis and y-axis.
\def\w{11}
\def\n{9}
\draw[ line width=1,dotted, color=OrangeProfondIRA, arrows={-Stealth[length=4, inset=0]}] (0,0) -- (0,0.4909) node (A) {};	
\foreach \y in  {0,...,3} {
	\draw (\y,0) -- (\y,-0.1);
	\ifthenelse{\y>0 }{	\node[below] at (\y,-0.1) {\tiny $ \scriptstyle \y$};
		}{
		\node[below] at (\y,-0.1) {\tiny 0};}
}
\foreach \y in  {-3,...,1} {
	\draw (\y+\n,0) -- (\y+\n,-0.1);
	\ifthenelse{\y<0 }{\node[below] at (\y+\n,-0.1) {\tiny $\scriptstyle n \y$};}{}
	\ifthenelse{\y=0 }{\node[below] at (\y+\n,-0.1) {\tiny $\scriptstyle n $};
		\draw[ line width=1, color=OrangeProfondIRA, arrows={-Stealth[length=4, inset=0]}] (\y+\n,0) -- (\y+\n,1) node (B) {};}{}
	\ifthenelse{\y>0}{	\node[below] at (\y+\n,-0.1) {\tiny $\scriptstyle n+\y$};}{}
}
\draw[ line width=1] (-.25,0) -- (4,0);
\draw[ line width=1, dashed] (4,0) -- (5,0);
\draw[arrows={-Stealth[length=4, inset=0]}, line width=1] (5,0) -- (\w,0);
\draw[arrows={-Stealth[length=4, inset=0]},color=OrangeProfondIRA] (B)  to [bend right]  node [pos=0.5, below=10pt] {Discounting}(A) ;
\end{tikzpicture}

\begin{tikzpicture}[scale=0.85]
% Draw the x-axis and y-axis.
\def\w{11}
\def\n{9}
\draw[ line width=1, color=OrangeProfondIRA, arrows={-Stealth[length=4, inset=0]}] (0,0) -- (0,1) node (A) {};	
\foreach \y in  {0,...,3} {
	\draw (\y,0) -- (\y,-0.1);
	\ifthenelse{\y>0 }{	\node[below] at (\y,-0.1) {\tiny $ \scriptstyle \y$};
	}{
		\node[below] at (\y,-0.1) {\tiny 0};}
}
\foreach \y in  {-3,...,1} {
	\draw (\y+\n,0) -- (\y+\n,-0.1);
	\ifthenelse{\y<0 }{\node[below] at (\y+\n,-0.1) {\tiny $\scriptstyle n \y$};}{}
	\ifthenelse{\y=0 }{\node[below] at (\y+\n,-0.1) {\tiny $\scriptstyle n $};
		\draw[ line width=1,dotted, color=OrangeProfondIRA, arrows={-Stealth[length=4, inset=0]}] (\y+\n,0) -- (\y+\n,2.0368) node (B) {};}{}
	\ifthenelse{\y>0}{	\node[below] at (\y+\n,-0.1) {\tiny $\scriptstyle n+\y$};}{}
}
\draw[ line width=1] (-.25,0) -- (4,0);
\draw[ line width=1, dashed] (4,0) -- (5,0);
\draw[arrows={-Stealth[length=4, inset=0]}, line width=1] (5,0) -- (\w,0);
\draw[arrows={-Stealth[length=4, inset=0]},color=OrangeProfondIRA] (A)  to [bend left]  node [pos=0.55, below=10pt] {Capitalization}(B) ;
\end{tikzpicture}

\end{f}
\hrule

\begin{f}[Interests]

\emph{Discount rate $d$}
     $$d=i/(1+i)$$
\emph{Simple interest $i$}
$$
I_t=P i t=P i \frac{k}{365}
$$
\emph{Compound interest $i$}
$$
V_n=P(1+i)^n=P\left(1+\frac{p}{100}\right)^n
$$
\emph{Continuous interest $r$}
$$V_t=V_0\ e^{rt}$$

\emph{Effective rate $i_e$}
$$
i_e=\left( 1+\frac{i}{m}\right) ^{m}-1
$$
where $i$ is the nominal rate and $m$ the number of periods in a year.

\emph{Equivalent rate $i^{(m)}$}
$$
i^{(m)}=m(1+i)^{1 / m}-1
$$

\emph{Nominal rate $i$ and periodic rate}

The \textbf{nominal} or \textbf{face} rate allows calculating the interest due over one year.
The \textbf{periodic} rate corresponds to the nominal rate divided by the number of periods in a year $i/m$.
If the periodic rate is weekly, the nominal rate will be divided by 52.
\end{f}
\hrule

\begin{f}[Present Value and Future Value]

The present value (PV) represents the capital that must be invested today at an annual compound interest rate $i$ to obtain future cash flows ($F_k$) at times $t_k$:
\begin{equation}
	PV = \sum_{k=1}^{n} F_k \times \frac{1}{(1+i)^{t_k}}
\label{ValeurActuelle}
\end{equation}
When the $F_k$ are constant
\begin{equation}
	PV = K  \frac{1 - (1+i)^{-n}} {i}
\label{ValeurActuelleFluxCt}
\end{equation}

The future value (FV) represents the value of the capital at $T$ which, with an annual compound interest rate $i$, capitalizes the future cash flows ($F_k$) at times $t_k$.
\begin{equation}
	FV=V_n = \sum_{k=1}^{n} F_k \times (1+i)^{n-t_k}
\end{equation}
More generally $FV= (1+i)^{n}PV$.
\end{f}
\hrule

\begin{f}[Annuities]
\ \newline


	Certain annuity \(a_{\lcroof{n}}\) (or \(a_{\lcroof{n}\;i}\) if the interest rate \(i\) needs to be specified): this is the default case in financial mathematics. Its payments are, for example, guaranteed by a contract.
	
\begin{center}
	\begin{tikzpicture}[scale=0.75]
	% Draw the x-axis and y-axis.
	\def\w{11}
	\def\n{9}
	\node[left] at (-.5,0) {\(a_{\lcroof{n}}\)};
	\foreach \y in  {0,...,3} {
	\draw (\y,0) -- (\y,-0.1);
	\ifthenelse{\y>0 }{	\node[below] at (\y,-0.1) {\tiny \( \scriptstyle \y\)};
		\draw[ line width=1, color=OrangeProfondIRA, arrows={-Stealth[length=4, inset=0]}] (\y,0) -- (\y,1);}{
		\node[below] at (\y,-0.1) {\tiny 0};}
}
\foreach \y in  {-3,...,1} {
	\draw (\y+\n,0) -- (\y+\n,-0.1);
	\ifthenelse{\y<0 }{\node[below] at (\y+\n,-0.1) {\tiny \(\scriptstyle n \y\)};}{}
	\ifthenelse{\y=0 }{\node[below] at (\y+\n,-0.1) {\tiny \(\scriptstyle n \)};}{}
	\ifthenelse{\y>0}{	\node[below] at (\y+\n,-0.1) {\tiny \(\scriptstyle n+\y\)};}{}
	\ifthenelse{\y<1 }{	\draw[ line width=1, color=OrangeProfondIRA, arrows={-Stealth[length=4, inset=0]}] (\y+\n,0) -- (\y+\n,1);}
}
\draw[ line width=1] (-.25,0) -- (4,0);
\draw[ line width=1, dashed] (4,0) -- (5,0);
\draw[arrows={-Stealth[length=4, inset=0]}, line width=1] (5,0) -- (\w,0);
\end{tikzpicture}
\end{center}
\[
\ddot{a}_{\lcroof{n}}=1+v+\cdots+v^{n-1}=\frac{1-v^{n}}{1-v}=\frac{1-v^{n}}{d}
\]


Contingent annuity \(\ddot{a}_{x}\): its payments are conditional on a random event, such as a life annuity of an individual aged \(x\). In this example, payments continue until death occurs :

\begin{center}
	\begin{tikzpicture}[scale=0.75]
	% Draw the x-axis and y-axis.
	\def\w{11}
	\def\n{6}
	\node[left] at (-.5,0) {\(a_{x}\)};
	
	\begin{scope}[shift={(\n+.5+3,.25)}]
		\draw[color=OrangeProfondIRA,scale=0.2,fill=OrangeProfondIRA] \Cerceuil;
	\end{scope}
	\foreach \y in  {0,...,3} {
		\draw (\y,0) -- (\y,-0.1);
		\ifthenelse{\y>0 }{	\node[below] at (\y,-0.1) {\tiny \( \scriptstyle x+\y\)};
			\draw[ line width=1, color=OrangeProfondIRA, arrows={-Stealth[length=4, inset=0]}] (\y,0) -- (\y,1);}{
			\node[below] at (\y,-0.1) {\tiny \( \scriptstyle x\)};}
	}
	\foreach \y in  {0,...,4} {
		\draw (\y+\n,0) -- (\y+\n,-0.1);
		\ifthenelse{\y>0 }{\node[below] at (\y+\n,-0.1) {\tiny \(\scriptstyle x+n+\y\)};}{
			\node[below] at (\y+\n,-0.1) {\tiny \(\scriptstyle x+n\)};}
		\ifthenelse{\y<4 }{	\draw[ line width=1, color=OrangeProfondIRA, arrows={-Stealth[length=4, inset=0]}] (\y+\n,0) -- (\y+\n,1);}
	}
\draw[ line width=1] (-.25,0) -- (4,0);
\draw[ line width=1, dashed] (4,0) -- (5,0);
\draw[arrows={-Stealth[length=4, inset=0]}, line width=1] (5,0) -- (\w,0);
\end{tikzpicture}	

\end{center}
The date of death is represented here by a small coffin. This type of annuity will be extensively studied in the life actuarial section.

Annuity in arrears (immediate) \(a_{\lcroof{n}}\): its periodic payments are made at the end of each payment period, as with a salary paid at the end of the month. This is the default case, previously illustrated for the certain annuity.
\[
\ddot{a}_{\lcroof{n}}=1+v+\cdots+v^{n-1}=\frac{1-v^{n}}{1-v}=\frac{1-v^{n}}{d}
\]
%
\[
	\mathrm{PV}_{\lcroof{n}}^{\text {due }}=K \ddot{a}_{\lcroof{n}}=K \frac{1-v^{n}}{d} 
\]

Annuity in advance (due) \(\ddot{a}_{\lcroof{n}}\): its periodic payments are made at the beginning of each payment period, as with rent payments, for example.

\begin{center}
	\begin{tikzpicture}[scale=0.75]
		% Draw the x-axis and y-axis.
		\def\w{11}
		\def\n{9}
		\node[left] at (-.5,0) {\(\ddot{a}_{\lcroof{n}}\)};
		\foreach \y in  {0,...,3} {
			\draw (\y,0) -- (\y,-0.1);
				\node[below] at (\y,-0.1) {\tiny \( \scriptstyle \y\)};
				\draw[ line width=1, color=OrangeProfondIRA, arrows={-Stealth[length=4, inset=0]}] (\y,0) -- (\y,1);
		}
		\foreach \y in  {-3,...,1} {
			\draw (\y+\n,0) -- (\y+\n,-0.1);
			\ifthenelse{\y<0 }{\node[below] at (\y+\n,-0.1) {\tiny \(\scriptstyle n \y\)};}{}
			\ifthenelse{\y=0 }{\node[below] at (\y+\n,-0.1) {\tiny \(\scriptstyle n \)};}{}
			\ifthenelse{\y>0}{	\node[below] at (\y+\n,-0.1) {\tiny \(\scriptstyle n+\y\)};}{}
			\ifthenelse{\y<0 }{	\draw[ line width=1, color=OrangeProfondIRA, arrows={-Stealth[length=4, inset=0]}] (\y+\n,0) -- (\y+\n,1);}{}
		}
		\draw[ line width=1] (-.25,0) -- (4,0);
		\draw[ line width=1, dashed] (4,0) -- (5,0);
		\draw[arrows={-Stealth[length=4, inset=0]}, line width=1] (5,0) -- (\w,0);
	\end{tikzpicture}
\end{center}
Also denoted \(\mathrm{PV}^{\mathrm{im}}\) :
\[
a_{\lcroof{n}}=v+v^{2}+\cdots+v^{n}=\frac{1-v^{n}}{i}=v \frac{1-v^{n}}{1-v}
\]
%
\[
	\mathrm{PV}_{\lcroof{n}}^{\mathrm{im}}=K a_{\lcroof{n}}=K \frac{1-v^{n}}{i} 
\]
Perpetuity \(a\) or \(a_{\lcroof{\infty}}\):
\[
a=1/i
\]
Deferred annuity \(_{m|}a_{\lcroof{n}}\): its payments do not start in the first period but after \(m\) periods, with \(m\) fixed in advance.
\begin{center}
	\begin{tikzpicture}[scale=0.75]
		% Draw the x-axis and y-axis.
\def\w{10}
\def\n{9}
\def\m{2}
\node[left] at (-.5,0) {\(_{m|}a_{\lcroof{n}}\)};
\foreach \y in  {0,...,2} {
	\draw (\y+\m,0) -- (\y+\m,-0.1);
	\ifthenelse{\y<0 }{\node[below] at (\y+\m,-0.1) {\tiny \(\scriptstyle m \y\)};}{}
\ifthenelse{\y=0 }{\node[below] at (\y+\m,-0.1) {\tiny \(\scriptstyle m \)};}{}
\ifthenelse{\y>0}{	\node[below] at (\y+\m,-0.1) {\tiny \(\scriptstyle m+\y\)};
		\draw[ line width=1, color=OrangeProfondIRA, arrows={-Stealth[length=4, inset=0]}] (\y+\m,0) -- (\y+\m,1);
		}
}
\foreach \y in  {-3,...,0} {
	\draw (\y+\n,0) -- (\y+\n,-0.1);
	\ifthenelse{\y<0 }{\node[below] at (\y+\n,-0.1) {\tiny \(\scriptstyle m+n \y\)};}{}
	\ifthenelse{\y=0 }{\node[below] at (\y+\n,-0.1) {\tiny \(\scriptstyle m+n \)};}{}
	\ifthenelse{\y>0}{	\node[below] at (\y+\n,-0.1) {\tiny \(\scriptstyle m+n+\y\)};}{}
	\ifthenelse{\y<1 }{	\draw[ line width=1, color=OrangeProfondIRA, arrows={-Stealth[length=4, inset=0]}] (\y+\n,0) -- (\y+\n,1);}
}
\draw[ line width=1] (-.25,0) -- (0.5,0);
\draw[ line width=1, dashed] (0.5,0) -- (1.5,0);
\draw[ line width=1] (1.5,0) -- (4,0);
\draw[ line width=1, dashed] (4,0) -- (5,0);
		\draw[arrows={-Stealth[length=4, inset=0]}, line width=1] (5,0) -- (\w,0);
	\end{tikzpicture}	
	
\end{center}


Periodic / monthly annuity \(a^{(m)}\) : the default periodicity is one year, but the unit payment can also be spread over \(m\) periods within the year.

\begin{center}
	\begin{tikzpicture}[scale=0.75]
		% Draw the x-axis and y-axis.
		\def\w{10}
		\def\n{9}
		\def\m{6}
		\node[left] at (-.5,0) {\(a_{\lcroof{n}}^{(m)}\)};
		\foreach \y in  {0,...,3} {
			\draw (\y,0) -- (\y,-0.1);
			\ifthenelse{\y>0 }{	\node[below] at (\y,-0.1) {\tiny \( \scriptstyle \y\)};}{
				\node[below] at (\y,-0.1) {\tiny 0};}
		}
		\pgfmathparse{3.5*\m} 
		\foreach \y in  {1,...,\pgfmathresult} {
			\ifthenelse{\y>0 }{	
				\draw[ line width=1, color=OrangeProfondIRA, arrows={-Stealth[length=4, inset=0]}] (\y/\m,0) -- (\y/\m,2/\m);}{}
		}
		\foreach \y in  {-3,...,0} {
			\draw (\y+\n,0) -- (\y+\n,-0.1);
			\ifthenelse{\y<0 }{\node[below] at (\y+\n,-0.1) {\tiny \(\scriptstyle n \y\)};}{}
			\ifthenelse{\y=0 }{\node[below] at (\y+\n,-0.1) {\tiny \(\scriptstyle n \)};}{}
			\ifthenelse{\y>0}{	\node[below] at (\y+\n,-0.1) {\tiny \(\scriptstyle n+\y\)};}{}
		}
		\pgfmathparse{3.5*\m} 
		\foreach \y  in  {1,...,\pgfmathresult} {
		\draw[ line width=1, color=OrangeProfondIRA, arrows={-Stealth[length=4, inset=0]}] (-\y/\m+\n,0) -- (-\y/\m+\n,2/\m);
		}
		\draw[ line width=1] (-.25,0) -- (3.5,0);
		\draw[ line width=1, dashed] (3.5,0) -- (5.5,0);
		\draw[arrows={-Stealth[length=4, inset=0]}, line width=1] (5.5,0) -- (\w,0);
	\end{tikzpicture}
\end{center}

If \(i^{(m)}\) represents the equivalent nominal (annual) interest rate with \(m\) periods per year, then \(i^{(m)}= m\left((1+i)^{1 / m}-1\right)\) .

Similarly, \(d^{(m)} \) is the nominal discount rate consistent with \(d\) and \(m\) : \(d^{(m)}= m\left(1-(1-d)^{1 / m}\right)\).

\[
\ddot{a}_{\lcroof{n}}^{(m)}=\frac{1}{m} \sum_{k=0}^{m n-1} v^{\frac{k}{m}}=\frac{d}{d^{(m)}} \ddot{a}_{\lcroof{n}}=\frac{1-v^{n}}{d^{(m)}} \approx \ddot{a}_{\lcroof{n}}+\frac{m-1}{2 m}\left(1-v^{n}\right)
\]
\[
a_{\lcroof{n}}^{(m)}=\frac{1}{m} \sum_{k=1}^{m n} v^{\frac{k}{m}}=\frac{i}{i^{(m)}} a_{\lcroof{n}}=\frac{1-v^{n}}{i^{(m)}} \approx a_{\lcroof{n}}-\frac{m-1}{2 m}\left(1-v^{n}\right)
\]
Unit annuity \(a\): it is used when constructing annuity formulas.
For a constant annuity, the total amount paid each year is 1, regardless of \(m\). 

Dynamic annuity, increasing/decreasing \(Ia\)/\(Da\): in its simplest form, it pays an amount that starts at 1 (\(n\)) and increases (decreases) each period arithmetically or geometrically. In the following example, the progression is arithmetic.
The prefix \(I\) (increasing) is used to indicate increasing annuities and \(D\) (decreasing) for decreasing annuities.
	
\begin{center}
	\begin{tikzpicture}[scale=0.75]
		% Draw the x-axis and y-axis.
		\def\w{11}
		\def\n{9}
		\node[left] at (-.5,0) {\(Ia_{\lcroof{n}}\)};
		\foreach \y in  {0,...,3} {
			\draw (\y,0) -- (\y,-0.1);
			\ifthenelse{\y>0 }{	\node[below] at (\y,-0.1) {\tiny \( \scriptstyle \y\)};
				\draw[ line width=1, color=OrangeProfondIRA, arrows={-Stealth[length=4, inset=0]}] (\y,0) -- (\y,\y/3);}{
				\node[below] at (\y,-0.1) {\tiny 0};}
		}
		\foreach \y in  {-3,...,1} {
			\draw (\y+\n,0) -- (\y+\n,-0.1);
			\ifthenelse{\y<0 }{\node[below] at (\y+\n,-0.1) {\tiny \(\scriptstyle n \y\)};}{}
			\ifthenelse{\y=0 }{\node[below] at (\y+\n,-0.1) {\tiny \(\scriptstyle n \)};}{}
			\ifthenelse{\y>0}{	\node[below] at (\y+\n,-0.1) {\tiny \(\scriptstyle n+\y\)};}{}
			\ifthenelse{\y<1 }{	\draw[ line width=1, color=OrangeProfondIRA, arrows={-Stealth[length=4, inset=0]}] (\y+\n,0) -- (\y+\n,\y/3+\n/3);}
		}
		\draw[ line width=1] (-.25,0) -- (4,0);
		\draw[ line width=1, dashed] (4,0) -- (5,0);
		\draw[arrows={-Stealth[length=4, inset=0]}, line width=1] (5,0) -- (\w,0);
	\end{tikzpicture}
	\begin{tikzpicture}[scale=0.75]
	% Draw the x-axis and y-axis.
	\def\w{11}
	\def\n{9}
	\node[left] at (-.5,0) {\(Da_{\lcroof{n}}\)};
	\foreach \y in  {0,...,3} {
		\draw (\y,0) -- (\y,-0.1);
		\ifthenelse{\y>0 }{	\node[below] at (\y,-0.1) {\tiny \( \scriptstyle \y\)};
			\draw[ line width=1, color=OrangeProfondIRA, arrows={-Stealth[length=4, inset=0]}] (\y,0) -- (\y,\n/3-\y/3);}{
			\node[below] at (\y,-0.1) {\tiny 0};}
	}
	\foreach \y in  {-3,...,1} {
		\draw (\y+\n,0) -- (\y+\n,-0.1);
		\ifthenelse{\y<0 }{\node[below] at (\y+\n,-0.1) {\tiny \(\scriptstyle n \y\)};}{}
		\ifthenelse{\y=0 }{\node[below] at (\y+\n,-0.1) {\tiny \(\scriptstyle n \)};}{}
		\ifthenelse{\y>0}{	\node[below] at (\y+\n,-0.1) {\tiny \(\scriptstyle n+\y\)};}{}
		\ifthenelse{\y<1 }{	\draw[ line width=1, color=OrangeProfondIRA, arrows={-Stealth[length=4, inset=0]}] (\y+\n,0) -- (\y+\n,1/3-\y/3);}
	}
	\draw[ line width=1] (-.25,0) -- (4,0);
	\draw[ line width=1, dashed] (4,0) -- (5,0);
	\draw[arrows={-Stealth[length=4, inset=0]}, line width=1] (5,0) -- (\w,0);
\end{tikzpicture}
\end{center}

\begin{equation}
(I \ddot{a})_{\lcroof{n}}=1+2 v+\cdots+n v^{n-1}=\frac{1}{d}\left(\ddot{a}_{\lcroof{n}}-n v^{n}\right)
\label{Ian}
\end{equation}
with, we recall,  \(d=i/(1+i)\) and in arrears (immediate)
\[
(I a)_{\lcroof{n}}=v+2 v^{2}+\cdots+n v^{n}=\frac{1}{i}\left(\ddot{a}_{\lcroof{n}}-n v^{n}\right)
\]
\[
(D \ddot{a})_{\lcroof{n}}=n+(n-1) v+\cdots+v^{n-1}=\frac{1}{d}\left(n-a_{\lcroof{n}}\right)
\]
and in arrears :
\[
(D a)_{\lcroof{n}}=n v+(n-1) v^{2}+\cdots+v^{n}=\frac{1}{i}\left(n-a_{\lcroof{n}}\right)
\]




\end{f}
\hrule

\begin{f}[The Loan (Indivisible)]
\ \newline

The main property of the loan is to consider separately the interest from the repayment (or amortization).

By constant repayment or constant annuity: the sum of the amortization and the interest at each period is constant.

\begin{center}
\begin{tikzpicture}[scale=0.85]
% Draw the x-axis and y-axis.
\def\w{11}
\def\n{9}
\def\cap{13} % Capital initial
\def\a{1.2334723} % annuité exacte

\def\v{1.031415^(-1)} 
\draw[ line width=10, color=BleuProfondIRA, opacity=0.2] (0,0) -- (0,-\n);
\draw[ line width=1, color=BleuProfondIRA, arrows={-Stealth[length=4, inset=0]}] (0,0) -- (0,-\n);
\foreach \y in  {0,...,10} {
	\ifthenelse{\y=0}{\def\caprd{9.000000}}{\def\caprd{0.000000}}
	\ifthenelse{\y=1}{\def\caprd{8.174923}}{}
	\ifthenelse{\y=2}{\def\caprd{7.323926}}{}
	\ifthenelse{\y=3}{\def\caprd{6.446194}}{}

	\ifthenelse{\y=6}{\def\caprd{4.569466}}{}
	\ifthenelse{\y=7}{\def\caprd{3.4795435}}{}
	\ifthenelse{\y=8}{\def\caprd{2.355381}}{}
	\ifthenelse{\y=9}{\def\caprd{0.000000}}{}
\ifthenelse{\y=0}{	
	\draw (\y,0) -- (\y,-0.1);
	\node[below left] at (\y,-0.1) {\tiny 0};}{}
\ifthenelse{\y<4 \AND \y>0}{
	\draw (\y,0) -- (\y,-0.1);
	\node[below] at (\y,-0.1) {\tiny $ \scriptstyle \y$};
	\pgfmathparse{\a-\cap*0.031415*\v^(\y) }
	\let\amortissement\pgfmathresult
		\draw[ line width=10, color=BleuProfondIRA, opacity=0.2] (\y,0) -- (\y,-\caprd);
		\draw[ line width=1, color=BleuProfondIRA, arrows={-Stealth[length=4, inset=0]}] (\y,0) -- (\y,\amortissement);
		\draw[ line width=1, color=OrangeProfondIRA, arrows={-Stealth[length=4, inset=0]}] (\y,\amortissement) -- (\y,\a);
		}{}
\ifthenelse{\y>5 }{
	\draw (\y,0) -- (\y,-0.1);
	\pgfmathparse{\a-\cap*0.031415*\v^(\y+4) }
	\let\amortissement\pgfmathresult
	\pgfmathparse{\y-\n} 	
	\let\zdec\pgfmathresult
	\pgfmathtruncatemacro{\z}{\zdec}
	\ifthenelse{\y<\n }{\node[below] at (\y,-0.1) {\tiny $\scriptstyle n \z$};}{}
	\ifthenelse{\y=\n }{\node[below] at (\y,-0.1) {\tiny $\scriptstyle n $};}{}
	\ifthenelse{\y>\n}{	\node[below] at (\y,-0.1) {\tiny $\scriptstyle n+\z$};}{}
	\ifthenelse{\y<\n \OR \y=\n}{
		\draw[ line width=10, color=BleuProfondIRA, opacity=0.2] (\y,0) -- (\y,-\caprd);
		\draw[ line width=1, color=BleuProfondIRA, arrows={-Stealth[length=4, inset=0]}] (\y,0) -- (\y,\amortissement);
		\draw[ line width=1, color=OrangeProfondIRA, arrows={-Stealth[length=4, inset=0]}] (\y,\amortissement) -- (\y,\a);}{}
}{}
}
\draw[ line width=1] (-.25,0) -- (4,0);
\draw[ line width=1, dashed] (4,0) -- (5,0);
\draw[arrows={-Stealth[length=4, inset=0]}, line width=1] (5,0) -- (\w,0);
\end{tikzpicture}
\end{center}
	
    
    By constant amortization.
\begin{center}
\begin{tikzpicture}[scale=0.85]
% Draw the x-axis and y-axis.
\def\w{11}
\def\n{9}
\def\duree{13}
\def\i{0.031415}
\draw[ line width=10, color=BleuProfondIRA, opacity=0.2] (0,0) -- (0,-\n);
\draw[ line width=1, color=BleuProfondIRA, arrows={-Stealth[length=4, inset=0]}] (0,0) -- (0,-\n);
\foreach \y in  {0,...,3} {
\draw (\y,0) -- (\y,-0.1);
\ifthenelse{\y>0 }{	\node[below] at (\y,-0.1) {\tiny $ \scriptstyle \y$};
	\pgfmathparse{(\duree-\y)*\i} 	
	\let\Interet\pgfmathresult
	\pgfmathparse{(\n) *(1-\y/\n))} 	
	\let\Cap\pgfmathresult
	\draw[ line width=10, color=BleuProfondIRA, opacity=0.2] (\y,0) -- (\y,-\Cap);
	\draw[ line width=1, color=BleuProfondIRA, arrows={-Stealth[length=4, inset=0]}] (\y,0) -- (\y,1);
	\draw[ line width=1, color=OrangeProfondIRA, arrows={-Stealth[length=4, inset=0]}] (\y,1) -- (\y,1+\Interet);}{
	\node[below left] at (\y,-0.1) {\tiny 0};}
}
\foreach \y in  {-3,...,1} {
	\pgfmathparse{(-\y+1)*\i} 	
\let\Interet\pgfmathresult
\draw (\y+\n,0) -- (\y+\n,-0.1);
\ifthenelse{\y<0 }{\node[below] at (\y+\n,-0.1) {\tiny $\scriptstyle n \y$};}{}
\ifthenelse{\y=0 }{\node[below] at (\y+\n,-0.1) {\tiny $\scriptstyle n $};}{}
\ifthenelse{\y>0}{	\node[below] at (\y+\n,-0.1) {\tiny $\scriptstyle n+\y$};}{}
\ifthenelse{\y<1 }{	
	\draw[ line width=10, color=BleuProfondIRA, opacity=0.2] (\y+\n,0) -- (\y+\n,\y);
	\draw[ line width=1, color=BleuProfondIRA, arrows={-Stealth[length=4, inset=0]}] (\y+\n,0) -- (\y+\n,1);
	\draw[ line width=1, color=OrangeProfondIRA, arrows={-Stealth[length=4, inset=0]}] (\y+\n,1) -- (\y+\n,1+\Interet);}{}
}
\draw[ line width=1] (-.25,0) -- (4,0);
\draw[ line width=1, dashed] (4,0) -- (5,0);
\draw[arrows={-Stealth[length=4, inset=0]}, line width=1] (5,0) -- (\w,0);
\end{tikzpicture}
\end{center}

    
By a bullet repayment, where the interest is constant. Only the interests are paid periodically until maturity, when the total repayment is made.
	
\begin{center}
\begin{tikzpicture}[scale=0.85]
% Draw the x-axis and y-axis.
\def\w{11}
\def\n{9}
\def\duree{23}
\def\i{3*0.031415}
\draw[ line width=10, color=BleuProfondIRA, opacity=0.2] (0,0) -- (0,-\n);
\draw[ line width=1, color=BleuProfondIRA, arrows={-Stealth[length=4, inset=0]}] (0,0) -- (0,-\n);
\foreach \y in  {0,...,3} {
	\draw (\y,0) -- (\y,-0.1);
	\ifthenelse{\y>0 }{	\node[below] at (\y,-0.1) {\tiny $ \scriptstyle \y$};
%				\draw[ line width=1, color=BleuProfondIRA, arrows={-Stealth[length=4, inset=0]}] (\y,0) -- (\y,1);
		\draw[ line width=10, color=BleuProfondIRA, opacity=0.2] (\y,0) -- (\y,-\n);
		\draw[ line width=1, color=OrangeProfondIRA, arrows={-Stealth[length=4, inset=0]}] (\y,0) -- (\y,2*\i);}{
		\node[below left] at (\y,-0.1) {\tiny 0};}
}
\foreach \y in  {-3,...,1} {
	\draw (\y+\n,0) -- (\y+\n,-0.1);
	\ifthenelse{\y<0 }{\node[below] at (\y+\n,-0.1) {\tiny $\scriptstyle n \y$};
		\draw[ line width=10, color=BleuProfondIRA, opacity=0.2] (\y+\n,0) -- (\y+\n,-\n);
		\draw[ line width=1, color=OrangeProfondIRA, arrows={-Stealth[length=4, inset=0]}] (\y+\n,0) -- (\y+\n,2*\i);}{}
	\ifthenelse{\y=0 }{
		\draw[ line width=1, color=BleuProfondIRA, arrows={-Stealth[length=4, inset=0]}] (\y+\n,0) -- (\y+\n,2);
		\draw[ line width=1, color=OrangeProfondIRA, arrows={-Stealth[length=4, inset=0]}] (\y+\n,2) -- (\y+\n,2+2*\i);
		\node[below] at (\y+\n,-0.1) {\tiny $\scriptstyle n $};}{}
	\ifthenelse{\y>0}{	\node[below] at (\y+\n,-0.1) {\tiny $\scriptstyle n+\y$};}{}
}
\draw[ line width=1] (-.25,0) -- (4,0);
\draw[ line width=1, dashed] (4,0) -- (5,0);
\draw[arrows={-Stealth[length=4, inset=0]}, line width=1] (5,0) -- (\n,0);
\end{tikzpicture}
\end{center}

\end{f}
\hrule \

\begin{f} [Loan Amortization Schedule]
\ \newline

	\footnotesize
\renewcommand{\arraystretch}{2}
\begin{tabular}{|m{10ex}|m{19ex}|m{19ex}|m{17ex}|}
\rowcolor{BleuProfondIRA!40}   	\hline &\textbf{In fine}			&   	\textbf{Constant amortizations} &  		\textbf{Constant annuities} \\
	\hline Outstan\-ding principal $S_k$ & $T_k=S_0$, $T_n=0$& $S_{0}\left(1-\frac{k}{n}\right)$ & $S_{0} \frac{1-v^{n-k}}{1-v^{n}}$\\
	\hline Interest $U_k$ 		& $i\times S_0$ & $S_{0}\left(1-\frac{k-1}{n}\right) i$ & $K\left(1-v^{n-k+1}\right)$ \\
	\hline Amortiz\-ations $T_k$ &	$T_k=O$, $T_n=S_0$ & $\frac{S_{0}}{n}$ & $K v^{n-k+1}$ \\
	\hline Annuity $K_k$ & 	$K_k=i S_0$, $K_n=(1+i)S_0$ 	 & $\frac{S_{0}}{n}(1-(n-k+1) i) $ & $K=S_{0} \frac{i}{1-v^{n}} $ \\
%	\hline Coût de l'emprunt & $1 \times K \times N$ & $1 \times K \times \frac{N+1}{2}$ & $K\left(\frac{N i}{1-(1+i)^{-N}}-1\right)$ \\
	\hline
\end{tabular}

\end{f}
