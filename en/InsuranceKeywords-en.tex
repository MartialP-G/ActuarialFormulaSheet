% !TeX root = ActuarialFormSheet_MBFA-en.tex
% !TeX spellcheck = en_GB
\begin{f}[Life and Non-Life Insurance]
	
The distinction between life and non-life insurance is fundamental. An insurer cannot offer both types of insurance without holding two separate companies: 
\begin{itemize}
	\item \textbf{Life insurance}, i.e., personal insurance excluding coverage for bodily injuries.
	\item \textbf{Non-life insurance}, which includes property and liability insurance as well as insurance for bodily injuries.
\end{itemize}
\end{f}


\begin{f}[The Principles of Insurance]
	
	
Insurance is assumed to :
	\begin{itemize}
		\item be based on utmost good faith,
		\item apply only if the insured has an insurable interest in preserving the item (property insurance),
		\item operate under the indemnity principle :
		\begin{itemize}
			\item not allow enrichment from a claim settlement,
			\item not even through insurance accumulation,
			\item include subrogation (in Liability Insurance, if the insurer compensates the insured victim, the insured cannot then claim from the party responsible for the loss).
		\end{itemize}
		\item not reduce the insured’s efforts in prevention and protection, as a reasonable person, even if financially protected.
	\end{itemize}
	\item establish the cause of an accident in Civil Responsibility, who is not responsible if he or she does not contribute to the cause of the accident.
\end{f}


\begin{f}[The Insurance Policy]
	
	The \textbf{insurance policy} (or contract) is the contractual document that governs the relationship between the insurance company (or mutual insurance company) and the insured (policyholder). 
	This contract defines in particular :
	
	\begin{enumerate}
		\item the list of covered events, including any exclusions,
		\item the coverage, i.e., the assistance provided to the insured in case of a loss,
		\item the obligations of the insured :
		\begin{itemize}
			\item any preventive measures required to reduce risk,
			\item time limits for reporting a claim to the insurer,
			\item the amount and payment conditions of the premium (deductible, limit),
			\item the conditions for cancellation of the policy (automatic renewal),
		\end{itemize}
		\item the obligations of the insurance company : time limits for compensation payments.
		
	\end{enumerate}
	
\end{f}
\begin{f}[The Premium and Claims]
Classically, the role of the insurer is to substitute a constant \(C\), the \textbf{contribution} or the \textbf{premium}, for a random claim \(S\).
\textbf{Pure premium} or \textbf{technical premium} aims 
to compensate claims without surplus or profit, overall \(C_t = \mathbb{E}[S]\)
	
	The \textbf{net premium} is higher than the pure premium. It aims to cover the cost of claims and provide a safety margin.
	
	The \textbf{gross premium} is the net premium + overhead expenses + commissions + expected profit + taxes.
	
	For commercial reasons, the premium actually charged may differ significantly from the technical premium.

	\textbf{Written premium:} premium charged to the insured to cover claims that may occur during the 
coverage period defined by the contract (generally 1 year in Property and Casualty insurance).

\textbf{Earned premium:} proportion of the written premium used to cover the risk over the exposure period of 
one policy year.
\tikzstyle{NoeudR}=[rectangle, shape border rotate=90, draw,minimum height=0.6cm,align=center, BleuProfondIRA]

\resizebox{\linewidth}{!}
{\begin{tikzpicture} %[node distance=5cm]
	%	
	\draw[thick,<->,>=latex,BleuProfondIRA] (-3,0) -- (5,0) ;
	\node (JanNN) at (-2,0.8) [NoeudR] {\footnotesize  \nth{1} January\\ \footnotesize \NN};
	\node (JuilNN) at (0,0.8) [NoeudR] {\footnotesize  \nth{1} July\\ \footnotesize \NN};
	\node (JanN) at (2,0.8) [NoeudR] {\footnotesize  31 December\\ \footnotesize  \NN};
	\node (JuilN) at (4,0.8) [NoeudR] {\footnotesize  \nth{1} July\\ \footnotesize \N};
	\draw[ BleuProfondIRA] (JanNN.south) -- ++(0,-0.8);
	\draw[ BleuProfondIRA] (JuilNN.south) -- ++(0,-0.8);
	\draw[ BleuProfondIRA] (JanN.south) -- ++(0,-0.8);
	\draw[ BleuProfondIRA] (JuilN.south) -- ++(0,-0.8);
	\draw[thick,<->,>=latex,VertIRA] ($(JuilNN.south) +(0,-1)$) -- ($(JuilN.south) +(0,-1)$) node [below, midway] {\footnotesize Written premium 600€} ;
	\draw[thick,<->,>=latex,OrangeProfondIRA] ($(JuilNN.south) +(0,-1.6)$) -- ($(JanN.south) +(0,-1.6)$) node [below, midway] {\footnotesize Earned 300€} ;
	\draw[thick,<->,>=latex,OrangeProfondIRA,densely dotted] ($(JanN.south) +(0,-1.6)$) -- ($(JuilN.south) +(0,-1.6)$) node [below, midway,align=right] 
	{\footnotesize Unearned 300€ \\\footnotesize  Reserved for the period \N } ;
	%	
\end{tikzpicture}
}
The \(S/P\) is the key indicator. For the insurer to make a profit the \(S/P\ll 1\).
\end{f}
%\end{multicols}

\begin{f}[Loss / Payment Triangle]
	
	Insurance accounting is broken down by the \textbf{accident year} of the claim. 
	If a premium covers multiple calendar years, a proportional part will be allocated to each.
	Each payment and each claim reserving is assigned to the accident year. 
	The monitoring of payments or expenses is expressed through a triangle (triangular matrix) :
	
	\[
	\left(\begin{array}{ccccc}
		\rowcolor{white}	C_{1,1} & C_{1,2} & \ldots & & C_{1, n} \\
		\rowcolor{white}	C_{1,1} & C_{1,2} & \ldots & C_{2, n-1} \\
		\rowcolor{white}	\vdots & \vdots & & \\
		\rowcolor{white}	C_{n-1,1} & C_{n-1,2} & & \\
		\rowcolor{white}	C_{n, 1} & & &
	\end{array}\right)
	\]
	where \(C_{i, j}=\sum_{k=1}^{j} X_{i, k}\) represents the cumulative amount of claims paid for origin year \(i\) and development year \(j\).
	
\end{f}
%\begin{multicols}{2}


	
\begin{f}[Solvency II and Risk Management]
	
	\textbf{Solvency II} is the European regulatory framework applicable to insurers and reinsurers since 2016. It is based on three interdependent pillars :
	
	\begin{itemize}
		\item \textbf{Pillar 1 : Quantitative Requirements} \\
		Determines the capital requirements :
		\begin{itemize}
			\item \textbf{SCR} (Solvency Capital Requirement) : capital to absorb an extreme shock (99.5\% over 1 year),
			\item \textbf{MCR} (Minimum Capital Requirement) : absolute minimum threshold,
			\item admissible assets to cover technical reserves and capital requirements.
		\end{itemize}
		
		\item \textbf{Pillar 2 : Governance, Internal Control, and Risk Management} \\
		The core link with \textbf{ERM} (Enterprise Risk Management). The requirements cover :
		\begin{itemize}
			\item governance : boards of directors responsible for the risk management framework;
			\item an effective \textbf{internal control} system ;
			\item independent key functions : \textbf{actuarial}, \textbf{risk management}, \textbf{compliance}, \textbf{internal audit} ;
			\item \textbf{ORSA} (Own Risk and Solvency Assessment): internal assessment of risks and solvency, a central tool aligning strategy, risk appetite, and economic capital.
		\end{itemize}
		
		\item \textbf{Pillar 3 : Market Discipline} \\
		Based on \textbf{transparency} and communication :
		\begin{itemize}
			\item \textbf{SFCR} (Solvency and Financial Condition Report) : public, summarizes solvency and financial position,
			\item \textbf{RSR} (Regular Supervisory Report) : intended for the supervisor,
			\item quantitative reporting : regulatory statements (\textbf{QRTs}), regular submission of financial and prudential data.
		\end{itemize}
	\end{itemize}
		
\end{f}
\begin{f}[Main Branches of Life and Non-Life Insurance]
	
	Life insurance covers long-term commitments, with or without a savings component :
	\begin{itemize}[nosep]
		\item \textbf{Life insurance} : lump sum or annuity paid if the insured is alive at a given date.
		\item \textbf{Death insurance} : payment if the insured dies during the covered period.
		\item \textbf{Endowment insurance} : combination of life and death coverage.
		\item \textbf{Life annuity} : periodic payments until death.
		\item \textbf{Savings/retirement} : products with deferred capital or deferred annuity.
		\item \textbf{Unit-linked policies} : benefits dependent on the value of financial assets.
		\item \textbf{Group contracts} : occupational pensions, group welfare insurance.
	\end{itemize}
	
	
	Non-life insurance covers risks occurring in the short or medium term :
	\begin{itemize}[nosep]
		\item \textbf{Automobile} : third-party liability, vehicle damage.
		\item \textbf{Home} : fire, theft, water damage, liability.
		\item \textbf{General liability} : personal liability, business liability.
		\item \textbf{Health and welfare} : medical reimbursements, disability, incapacity.
		\item \textbf{Personal accident} : capital in case of accident, disability, or death.
		\item \textbf{Business interruption} : financial losses related to a claim.
		\item \textbf{Technical risks} : construction, machinery breakdown.
		\item \textbf{Transport, aviation, maritime insurance} : goods in transit, specific liabilities.
	\end{itemize}
	
	
\begin{center}
	\resizebox{1.1\linewidth}{!}{	\begin{tikzpicture}[ every node/.style={font=\small,text width=3cm}, node distance=0.75cm and 1.cm]
			% Rectangle central "Responsabilit\'e"
			\node[draw, rectangle, minimum width=2cm, minimum height=1cm, align=center] (resp) at (0,0) {Liability};
			%		
			% Branches vers "Responsabilit\'e p\'enale" et "Responsabilit\'e civile"  +(-.25,0) |-
			\node[below right=of resp, xshift=-.5cm, yshift=.5cm] (penale) {Criminal liability};
			\node[below=0cm of penale] (inass) {Uninsurable};
			\draw (resp.east) -| (penale.north);
			%		
			\node[below left=of resp] (civile) {Civil liability};
			\draw (resp.west) -| (civile.north);
			%		
			%		% Vers RC contractuelle et d\'elictuelle
			\node[below=of civile, xshift=-1.5cm] (rccontr) {Contractual liability (CL)};
			\node[below=of civile, xshift=3cm] (rcdel) {Tort liability (TL)};
			\draw (civile.south) |-  +(0,-0.25) -| (rccontr.north);
			\draw (civile.south) |-  +(0,-0.25) -| (rcdel.north);
			%		
			% Obligation de moyen / de r\'esultat
			\node[below= of rccontr, xshift=-1.5cm, rotate=45] (moyen) {Obligation\\ of means};
			\node[below= of rccontr, xshift=0.5cm, rotate=45] (resultat) {Obligation\\ of result};
			\draw (rccontr.south) |-  +(0,-0.25) -| (moyen.north);
			\draw (rccontr.south) |-  +(0,-0.25) -| (resultat.north);
			%		
			% Sous RC d\'elictuelle : fait personnel / d'autrui / de choses
			\node[below=of rcdel, xshift=-2cm, rotate=45] (choses) {Act of things \\ Presumption};
			\node[below= of rcdel, rotate=45] (autrui) {Acts of others \\ Presumption \\ Article 1384 of the CC (France)};
			\node[below= of rcdel, xshift=2cm, rotate=45] (personnel) {Personal act\\Fault to be proven};
			\draw (rcdel.south) |-  +(0,-0.25) -| (choses.north);
			\draw (rcdel.south) |-  +(0,-0.25) -| (autrui.north);
			\draw (rcdel.south) |-  +(0,-0.25) -| (personnel.north);
			%		
			% Faute \`a prouver et fait conscient sous fait personnel
			\node[right= of personnel, rotate=0, anchor=west] (fconscient) {Conscious act};
			\draw (personnel.south) -|  +(0.75,0) |- (fconscient.west);
			%		
			% Imprudence / N\'eglience sous Fait conscient
			\node[below=of fconscient] (impru) {Carelessness};
			\node[below= of impru] (negl) {Negligence};
			\draw (fconscient.west)  |- +(-0.25,0) |- (impru.west);
			\draw (impru.west)  |- +(-0.25,0) |- (negl.west);
			% Source
			%		\node[below=2cm of moyen, anchor=west, text width=10cm] (source) {\small Source : Les Grands Principes de l'Assurance, Couilbault et Eliashberg, 10e \'edition};
		\end{tikzpicture}
	}
\end{center}

	
\end{f}


\begin{f}[Actuary]
	
	In practice, the actuary :
	\begin{itemize}[nosep]
		\item prices insurance and welfare products,
		\item estimates technical reserves,
		\item projects cash flows and values long-term liabilities,
		\item measures economic capital (SCR, ORSA) and contributes to ERM,
		\item advises management on strategy, solvency, and regulatory compliance.
	\end{itemize}
	
\end{f}