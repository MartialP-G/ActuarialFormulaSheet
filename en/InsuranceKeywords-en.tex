% !TeX root = ActuarialFormSheet_MBFA-en.tex
% !TeX spellcheck = fr_FR
\begin{f}[Assurance Vie, Non-Vie]
	
La séparation entre vie et non vie est fondamentale, un assureur ne peut proposer les deux types d'assurance sans détenir deux entreprises distinctes: 	
\begin{itemize}
	\item Les \textbf{assurances vie}, c'est-à-dire les assurances de personnes à l'exception des assurances des dommages corporels,
	\item Les \textbf{assurances non vie} qui incluent les assurances de biens et de responsabilité ainsi que les assurances sur les dommages corporels. 
\end{itemize}
\end{f}


\begin{f}[Les principes de l'assurance]
	
	
L'assurance est supposée :
	\begin{itemize}
		\item être établie dans une bonne foi absolue,
		\item et seulement si l'assuré a un intérêt à la conservation d'une chose (assurance de biens),
		\item fonctionner en vertu du principe indemnitaire : 
		\begin{itemize}
			\item ne pas permettre l'enrichissement au règlement d'un sinistre,
			\item pas même à travers un cumul d'assurance, 
			\item la subrogation (en Responsabilité Civile, si l'assureur indemnise l'assuré victime, il ne peut plus faire de recours auprès du responsable du sinistre.) 
		\end{itemize}
		\item ne pas réduire les efforts de prévention et protection de l'assuré en la qualité de personne raisonnable et même s'il bénéficie de cette protection financière.
	\end{itemize}
	\item établir la cause pour un sinistre en Responsabilité Civile, je ne suis pas responsable si je ne contribue pas à la cause du sinistre.
\end{f}


\begin{f}[La police d'assurance]
	
	La \textbf{police d'assurance} (ou contrat) d'assurance est le document contractuel qui régit les relations entre la compagnie d'assurance (ou mutuelle d'assurance) et l'assuré (sociétaire). 
	Ce contrat fixe en particulier~:
	
	\begin{enumerate}
		\item  la liste des évènements garantis, avec les exclusions éventuelles,
		\item  la garantie, c'est-à-dire l'assistance apportée à l'assuré en cas de sinistre,
		\item  les obligations de l'assuré :
		\begin{itemize}
			\item les mesures de prévention éventuelles afin de diminuer le risque,
			\item les délais de déclaration à l'assureur en cas de sinistre,
			\item le montant et les conditions de paiement de la prime (franchise, limite),
			\item les possibilités de résiliation de la police (tacite reconduction),
		\end{itemize}
		\item  les obligations de la compagnie d'assurances : les délais de paiements pour l'indemnisation.
		
	\end{enumerate}
	
\end{f}
\begin{f}[La prime et les sinistres]
Classiquement, le rôle de l'assureur est de substituer une constante $C$, la \textbf{cotisation} ou la \textbf{prime}, à un sinistre aléatoire $S$.
La \textbf{prime pure} ou \textbf{prime technique} vise 
le dédommagement des sinistres sans excédents, ni bénéfice, globalement $C_t = \mathbb{E}[S]$
	
	La \textbf{prime nette} est supérieure à la prime pure. Elle vise à couvrir le coût des sinistres et à donner une marge de sécurité.
	
	La \textbf{prime commerciale} est la prime nette + frais généraux + commissions + bénéfice escomptés + impôts.
	
	Pour des raisons commerciales, la prime commerciale effectivement demandée peut être différente de la prime technique.

	\textbf{Prime émise:} prime demandée à l'assurée pour couvrir les sinistres qui peuvent survenir sur la
période de couverture définie par le contrat (généralement 1 an en IARD).

\textbf{Prime acquise:} prorata de prime émise servant à couvrir le risque sur la période d'exposition
d'une année d'exercice.
\tikzstyle{NoeudR}=[rectangle, shape border rotate=90, draw,minimum height=0.6cm,align=center, BleuProfondIRA]

\resizebox{\linewidth}{!}
{\begin{tikzpicture} %[node distance=5cm]
	%	
	\draw[thick,<->,>=latex,BleuProfondIRA] (-3,0) -- (5,0) ;
	\node (JanNN) at (-2,0.8) [NoeudR] {\footnotesize  1\ier{} Janvier\\ \footnotesize \NN};
	\node (JuilNN) at (0,0.8) [NoeudR] {\footnotesize  1\ier{} Juillet\\ \footnotesize \NN};
	\node (JanN) at (2,0.8) [NoeudR] {\footnotesize  31 décembre\\ \footnotesize  \NN};
	\node (JuilN) at (4,0.8) [NoeudR] {\footnotesize  1\ier{} Juillet\\ \footnotesize \N};
	\draw[ BleuProfondIRA] (JanNN.south) -- ++(0,-0.8);
	\draw[ BleuProfondIRA] (JuilNN.south) -- ++(0,-0.8);
	\draw[ BleuProfondIRA] (JanN.south) -- ++(0,-0.8);
	\draw[ BleuProfondIRA] (JuilN.south) -- ++(0,-0.8);
	\draw[thick,<->,>=latex,VertIRA] ($(JuilNN.south) +(0,-1)$) -- ($(JuilN.south) +(0,-1)$) node [below, midway] {\footnotesize Prime émise 600€} ;
	\draw[thick,<->,>=latex,OrangeProfondIRA] ($(JuilNN.south) +(0,-1.6)$) -- ($(JanN.south) +(0,-1.6)$) node [below, midway] {\footnotesize Prime acquise 300€} ;
	\draw[thick,<->,>=latex,OrangeProfondIRA,densely dotted] ($(JanN.south) +(0,-1.6)$) -- ($(JuilN.south) +(0,-1.6)$) node [below, midway,align=right] 
	{\footnotesize Prime NA 300€ \\\footnotesize  Provisionnée pour l'exercice \N } ;
	%	
\end{tikzpicture}
}
Le $S/P$ est l'indicateur majeur. Pour que l'assureur ait un profit le $S/P<<1$. 
\end{f}
%\end{multicols}

\begin{f}[Triangle de charge / de paiement]
	
	La comptabilité de l'assurance est déclinée par année de \textbf{survenance} du sinistre. 
	Si une prime couvre plusieurs années civiles, une quote part sera affectée pour chacune.
	Chaque règlement, chaque provisionnement de sinistre est affectée à l'année de survenance. 
	Le suivi des paiements, ou des charges, s'exprime à travers  un triangle (matrice triangulaire) :
	
	$$
	\left(\begin{array}{ccccc}
		\rowcolor{white}	C_{1,1} & C_{1,2} & \ldots & & C_{1, n} \\
		\rowcolor{white}	C_{1,1} & C_{1,2} & \ldots & C_{2, n-1} \\
		\rowcolor{white}	\vdots & \vdots & & \\
		\rowcolor{white}	C_{n-1,1} & C_{n-1,2} & & \\
		\rowcolor{white}	C_{n, 1} & & &
	\end{array}\right)
	$$
	où $C_{i, j}=\sum_{k=1}^{j} X_{i, k}$ représente le montant des sinistres cumulés  réglés pour l'année d'origine $i$ et l'année de développement $j$.
	
\end{f}
%\begin{multicols}{2}


	
\begin{f}[Solvabilité II et gestion des risques]
	
	\textbf{Solvabilité II} est le cadre réglementaire européen applicable aux assureurs et réassureurs depuis 2016. Il repose sur trois piliers interdépendants :
	
	\begin{itemize}
		\item \textbf{Pilier 1 : Exigences quantitatives} \\
		Détermine les exigences de capital :
		\begin{itemize}
			\item \textbf{SCR} (Solvency Capital Requirement) : capital pour absorber un choc extrême (99.5\% sur 1 an),
			\item \textbf{MCR} (Minimum Capital Requirement) : seuil minimum absolu,
			\item actifs admissibles pour couvrir les provisions techniques et les exigences de capital.
		\end{itemize}
		
		\item \textbf{Pilier 2 : Gouvernance, contrôle interne et gestion des risques} \\
		Cœur du lien avec l'\textbf{ERM} (\emph{Enterprise Risk Management}). Les exigences portent sur :
		\begin{itemize}
			\item la gouvernance : conseils d'administration responsables du dispositif de gestion des risques ;
			\item un système efficace de \textbf{contrôle interne} ;
			\item des fonctions clés indépendantes : \textbf{actuarielle}, \textbf{gestion des risques}, \textbf{conformité}, \textbf{audit interne} ;
			\item l'\textbf{ORSA} (\emph{Own Risk and Solvency Assessment}): évaluation interne des risques et de la solvabilité, outil central d'alignement entre stratégie, appétence au risque et capital économique.
		\end{itemize}
		
		\item \textbf{Pilier 3 : Discipline de marché} \\
		Repose sur la \textbf{transparence} et la communication :
		\begin{itemize}
			\item \textbf{SFCR} (\emph{Solvency and Financial Condition Report}) : public, résume la solvabilité et la situation financière,
			\item \textbf{RSR} (\emph{Regular Supervisory Report}) : destiné au superviseur,
			\item reporting quantitatif : états réglementaires (\textbf{QRTs}), transmission régulière des données financières et prudentielles.
		\end{itemize}
	\end{itemize}
		
\end{f}
\begin{f}[Les principales branches de l’assurance vie et non-vie]
	
	L’assurance vie couvre des engagements à long terme, avec ou sans composante épargne :
	\begin{itemize}[nosep]
		\item \textbf{Assurance en cas de vie} : capital ou rente versés si l’assuré est vivant à une date donnée.
		\item \textbf{Assurance en cas de décès} : versement si décès de l’assuré pendant la période couverte.
		\item \textbf{Assurance mixte} : combinaison vie et décès.
		\item \textbf{Rente viagère} : versements périodiques jusqu’au décès.
		\item \textbf{Épargne/retraite} : produits à capital différé ou à rente différée.
		\item \textbf{Unités de compte} : prestations dépendantes de la valeur de supports financiers.
		\item \textbf{Contrats collectifs} : retraites professionnelles, prévoyance collective.
	\end{itemize}
	
	
	L’assurance non-vie couvre des risques survenant à court ou moyen terme :
	\begin{itemize}[nosep]
		\item \textbf{Automobile} : responsabilité civile, dommages au véhicule.
		\item \textbf{Habitation} : incendie, vol, dégâts des eaux, responsabilité.
		\item \textbf{Responsabilité civile générale} : RC vie privée, RC entreprise.
		\item \textbf{Santé et prévoyance} : remboursements médicaux, invalidité, incapacité.
		\item \textbf{Accidents corporels} : capital en cas d’accident, invalidité ou décès.
		\item \textbf{Pertes d’exploitation} : pertes financières liées à un sinistre.
		\item \textbf{Risques techniques} : construction, bris de machine.
		\item \textbf{Assurance transport, aviation, maritime} : biens en transit, responsabilités spécifiques.
	\end{itemize}
	
	
\begin{center}
	\resizebox{1.1\linewidth}{!}{	\begin{tikzpicture}[ every node/.style={font=\small,text width=3cm}, node distance=0.75cm and 1.cm]
			% Rectangle central "Responsabilit\'e"
			\node[draw, rectangle, minimum width=2cm, minimum height=1cm, align=center] (resp) at (0,0) {Responsabilit\'e};
			%		
			% Branches vers "Responsabilit\'e p\'enale" et "Responsabilit\'e civile"  +(-.25,0) |-
			\node[below right=of resp, xshift=-.5cm, yshift=.5cm] (penale) {Responsabilit\'e p\'enale};
			\node[below=0cm of penale] (inass) {Inassurable};
			\draw (resp.east) -| (penale.north);
			%		
			\node[below left=of resp] (civile) {Responsabilit\'e civile};
			\draw (resp.west) -| (civile.north);
			%		
			%		% Vers RC contractuelle et d\'elictuelle
			\node[below=of civile, xshift=-1.5cm] (rccontr) {RC contractuelle};
			\node[below=of civile, xshift=3cm] (rcdel) {RC d\'elictuelle};
			\draw (civile.south) |-  +(0,-0.25) -| (rccontr.north);
			\draw (civile.south) |-  +(0,-0.25) -| (rcdel.north);
			%		
			% Obligation de moyen / de r\'esultat
			\node[below= of rccontr, xshift=-1.5cm, rotate=45] (moyen) {Obligation\\ de moyen};
			\node[below= of rccontr, xshift=0.5cm, rotate=45] (resultat) {Obligation\\ de r\'esultat};
			\draw (rccontr.south) |-  +(0,-0.25) -| (moyen.north);
			\draw (rccontr.south) |-  +(0,-0.25) -| (resultat.north);
			%		
			% Sous RC d\'elictuelle : fait personnel / d'autrui / de choses
			\node[below=of rcdel, xshift=-2cm, rotate=45] (choses) {Fait de choses \\Pr\'esomption};
			\node[below= of rcdel, rotate=45] (autrui) {Fait d'autrui\\ Pr\'esomption \\ 1384 C Civil};
			\node[below= of rcdel, xshift=2cm, rotate=45] (personnel) {Fait personnel\\Faute \`a prouver};
			\draw (rcdel.south) |-  +(0,-0.25) -| (choses.north);
			\draw (rcdel.south) |-  +(0,-0.25) -| (autrui.north);
			\draw (rcdel.south) |-  +(0,-0.25) -| (personnel.north);
			%		
			% Faute \`a prouver et fait conscient sous fait personnel
			\node[right= of personnel, rotate=0, anchor=west] (fconscient) {Fait conscient};
			\draw (personnel.south) -|  +(0.75,0) |- (fconscient.west);
			%		
			% Imprudence / N\'eglience sous Fait conscient
			\node[below=of fconscient] (impru) {Imprudence};
			\node[below= of impru] (negl) {N\'egligence};
			\draw (fconscient.west)  |- +(-0.25,0) |- (impru.west);
			\draw (impru.west)  |- +(-0.25,0) |- (negl.west);
			% Source
			%		\node[below=2cm of moyen, anchor=west, text width=10cm] (source) {\small Source : Les Grands Principes de l'Assurance, Couilbault et Eliashberg, 10e \'edition};
		\end{tikzpicture}
	}
\end{center}

	
\end{f}


\begin{f}[Actuaire]
	
	En pratique, l'actuaire :
	\begin{itemize}[nosep]
		\item tarifie les produits d'assurance et de prévoyance, 
		\item estime les provisions techniques,
		\item projette les flux de trésorerie et valorise les engagements long terme,
		\item mesure le capital économique (SCR, ORSA) et contribue à l'ERM,
		\item conseille la direction sur la stratégie, la solvabilité et la conformité réglementaire.
	\end{itemize}
	
\end{f}