%%%%%%%%%%%%%%%%%%%%%%%%%%%%%%%%%%%%%%%%%%%%%%%%%%%%%%%%%%%%%%%
%%%%%%%%%%%%%%%%%%%%%%%%%%%%%%%%%%%%%%%%%%%%%%%%%%%%%%%%%%%%%%%
\PassOptionsToPackage{svgnames,table}{xcolor}
\documentclass[a4paper]{article} 
\usepackage[T1]{fontenc}
\usepackage[utf8]{inputenc}
\RequirePackage{eurosym}
\DeclareUnicodeCharacter{20AC}{\EUR{}}{\tiny}
\usepackage{lmodern}

\usepackage[top=14pt, bottom=2.5cm, left=0.5cm, right=0.5cm]{geometry}
%\usepackage[a4paper]{geometry}
\setlength{\parindent}{0pt}%Cela supprime l'indentation pour tous les paragraphes.

\usepackage[english,french]{babel}
\usepackage{nth}
\usepackage[autostyle=true]{csquotes}
\usepackage{epsdice}
\usepackage{enumitem}
%
\usepackage[most]{tcolorbox}
\definecolor{mbfaulmbleu}{RGB}{15,91,164}
\definecolor{mbfaulmbleumarine}{RGB}{40,56,120}
\definecolor{mbfaulmorangesec}{RGB}{228,82,65}
\definecolor{mbfaulmorange}{RGB}{228,66,45}
\definecolor{mbfaulmgrissec}{RGB}{91,93,95}
\definecolor{mbfaulmgris}{RGB}{112,111,111}
\definecolor{mbfaulmcaption}{RGB}{228,66,45}
% Couleurs Droit
\definecolor{DEG}{RGB}{228,0,56}
% Couleurs IRA
\definecolor{IRAorange}{RGB}{229,68,46}
\definecolor{IRAbleu}{RGB}{40,57,121}

\definecolor{FushiaIRA}{HTML}{CC1C4F}
\definecolor{BleuIRA}{HTML}{0F398C}
\definecolor{VertIRA}{HTML}{5BA626}
\definecolor{VertDouxIRA}{HTML}{9BBF17}
\definecolor{BlancIRA}{HTML}{F2F2F2}
\definecolor{BleuProfondIRA}{HTML}{1F4E8C}
\definecolor{BleuMoyenIRA}{HTML}{2D6DA6}
\definecolor{OrangeProfondIRA}{HTML}{BF3F34}
\definecolor{OrangePastelIRA}{HTML}{E8AC89}
\definecolor{OrangeMoyenIRA}{HTML}{D9753B}
\definecolor{GrisLogoIRA}{HTML}{5A5A5C}
\usepackage[colorlinks=true,linkcolor=mbfaulmorange,urlcolor=mbfaulmorange,
citecolor=mbfaulmbleumarine,filecolor=mbfaulmorange,linktoc=all]{hyperref} 
\hypersetup{
	pdftitle    = {Actuarial Formula Cheat Sheet},
	pdfauthor   = {Martial Phelippe-Guinvarc h and Marcelo Moreno Porras},
	pdfsubject  = {Actuarial Science -- Formula Sheet},
	pdfkeywords = {actuarial science, formula sheet, probability, statistics, finance},
	pdfproducer = {LaTeX},
	pdfcreator  = {pdflatex},
	pdfstartview= FitH,
	pdfpagemode = UseOutlines,
	pdfduplex   = DuplexFlipLongEdge % <-- configuration d'impression recto-verso livret
}
\tcbset{
  beamerblock/.style={
    enhanced,
    colback=white,
    colframe=OrangeProfondIRA,
    coltitle=white,
    colbacktitle=OrangeProfondIRA,
    boxrule=1pt,
    arc=2pt,
    left=6pt,
    right=6pt,
    top=6pt,
    bottom=6pt,
    fonttitle=\bfseries
  }
}

\usepackage{multicol}
\usepackage{multirow}

\usepackage{tikz}
\usepackage{tikzpeople}
\usepackage{pgfplots}

\pgfplotsset{%
	compat=newest,
	colormap={whitetoblue}{color(0cm)=(white);
		color(1cm)=(BleuProfondIRA)},
  colormap={whitered}{color(0cm)=(white); color(1cm)=(orange!75!red)}
}

\usepackage{pgfmath}
\usepackage{pgf-pie}
\usetikzlibrary{animations} 
%\usepackage[final]{animate}
\usepackage{animate}
\usetikzlibrary{arrows}

\usetikzlibrary{intersections, pgfplots.fillbetween,pgfplots.dateplot}
\usetikzlibrary{fillbetween}
\usetikzlibrary{datavisualization}
\usetikzlibrary{calc}
\usetikzlibrary{math}
\usetikzlibrary{shapes}
\usetikzlibrary{decorations.pathmorphing}
\usetikzlibrary{shapes.geometric, arrows,arrows.meta,bending}
\usetikzlibrary{trees}
\usetikzlibrary{matrix}
\usetikzlibrary{positioning}
\usetikzlibrary{svg.path}
\usetikzlibrary {mindmap}

\newcommand{\Cerceuil}{(0,0) -- +(.5,0) -- +(1,4)  -- +(0.5,5) -- +(-0.5,5) -- +(-1,4) -- +(-0.5,0) -- cycle}
\newcommand{\engl}[1]{{\foreignlanguage{english}{\sffamily  #1}}}

%%%%%%%%%%%%%%%%%%%%%%%%%%%%%%%%%%%%%%%%%%%%%%


\usepackage{latexsym}  
\usepackage{textcomp}               
\usepackage{fontawesome}    
\usepackage{fancyhdr}%                 pour les en-têtes
\usepackage{amsfonts}
\usepackage{amsmath}
\usepackage{amssymb}
\usepackage{dsfont}
\usepackage{latexsym}
\usepackage{../_Common/lifecon}\let\comment\relax % cette commande est en conflit
%\usepackage{a4wide}
\usepackage{array}
\usepackage{graphicx}
%\usepackage{changes}
%%%%%%%%%%%%%%%%%%%%%%%%%%%%%%%%%%%%%%%%%%%%%%%%%
\newcommand{\dis}{\displaystyle}
\newcommand{\ve}{ \overrightarrow}
\mathchardef\times="2202
%%%%%%%%%%%%%%%%%%%%%%%%%%%%%%%%%%%%%%%%%%%%%%%%%%

\setlength{\columnseprule}{0.4pt}% the line between colomn paper
%\thispagestyle{empty}
%%%%%%%%%%%%%%%%%%%%%%%%%%%%%%%%%%%%%%%%%%%%%%%%%%%%%%%


%---EN-TETE-ET-PIED-DE-PAGE----------------------------------------------------
\pagestyle{fancy}
%\renewcommand{\headrulewidth}{1,5pt}
%\renewcommand{\footrulewidth}{1 pt}

\fancyhead{} % clear all header fields
%\fancyfoot{} % clear all footer fields
\fancyfoot[L]{\href{https://sites.google.com/site/martialphelippeguinvarch}{\color{OrangeProfondIRA}Martial Ph\'elipp\'e-Guinvarc'h\ } 
	\href{mailto:Martial.Phelippe-GuinvarcH@univ-lemans.fr?subject=ActuarialFormSheet}{\color{OrangeProfondIRA}\faAt\ \faEnvelope} 
	\href{https://www.linkedin.com/in/martial-phelippe-guinvarc-h}{\color{OrangeProfondIRA}\faLinkedinSquare}
	\href{https://orcid.org/0000-0002-9894-803X}{\tikz[scale=0.75, baseline=-1ex]{\fill[OrangeProfondIRA] (0, 0) circle (0.25); 
			\node[white,font=\sffamily\bfseries\scriptsize] at (0, 0) {ID};}}
	\href{https://github.com/MartialP-G}{\color{OrangeProfondIRA} \large \faGithub} 
}
\fancyfoot[R]{\href{https://servicios.urjc.es/pdi/ver/marcelo.moreno}{\color{OrangeProfondIRA}Marcelo Moreno Porras} 
	\href{mailto:marcelo.moreno@urjc.es?subject=ActuarialFormSheet}{\color{OrangeProfondIRA}\faAt\ \faEnvelope} 
	\href{https://www.linkedin.com/in/marcelomorenop/}{\color{OrangeProfondIRA}\faLinkedinSquare}
	\href{https://orcid.org/0009-0007-1714-6507}{\tikz[scale=0.75, baseline=-1ex]{\fill[OrangeProfondIRA] (0, 0) circle (0.25); 
			\node[white,font=\sffamily\bfseries\scriptsize] at (0, 0) {ID};}}
	\href{https://github.com/marcelomijas}{\color{OrangeProfondIRA} \large \faGithub} }
%\fancyfoot[LO,CE]{\textbf{TCS \& TCT}}
\fancyfoot[C]{\thepage}
\renewcommand{\headrulewidth}{0.4pt}
\renewcommand{\footrulewidth}{0.4pt}
\setlength{\headheight}{14pt}


%\lhead{fhgfgd}
%\cfoot{\thepage}
\def\Cset{\mathbb{C}} % complexes
\def\Hset{\mathbb{H}} % hilbert
\def\Nset{\mathbb{N}} % entiers naturels
\def\Qset{\mathbb{Q}} % rationnels
\def\Rset{\mathbb{R}} % réels
\def\Zset{\mathbb{Z}} % entiers relatifs
\def\Dset{\mathbb{D}} % disque unité ouvert
\def\Eset{\mathbb{E}} % espérance mathématique
\newcommand{\E}{\mathbb{E}}
\newcommand{\Var}{\mathrm{Var}}
\newcommand{\se}{\mathrm{se}}
\newcommand{\Cov}{\mathrm{Cov}}
\newcommand{\Corr}{\mathrm{Corr}}

%%%%%%%%%%%%%%%%%%%%%%%%%%%%%%%%%%%%%%%%%%%%%%%%

\tikzset{
	declare function={
		normcdf(\x,\m,\s)=1/(1 + exp(-0.07056*((\x-\m)/\s)^3 - 1.5976*(\x-\m)/\s));},
	%approximation http://dx.doi.org/10.3926/jiem.2009.v2n1.p114-127
	declare function={
		normal(\x,\m,\s)=1/(2*\s*sqrt(pi))*exp(-(\x-\m)^2/(2*\s^2));},
	declare function={
		lognormal(\x,\m,\s)=1/(2*\s*sqrt(pi))*exp(-(ln(\x)-\m)^2/(2*\s^2));},
	declare function={
		expdensity(\x,\s)=\s*exp(-(\s*\x));},
	declare function={
		Normale2(\x,\y,\ma,\sa,\mb,\sb,\corel)=
		1/(2*pi*\sa*\sb*(1-\corel^2)^0.5) * 
		exp(-((\x-\ma)^2/\sa^2 + (\y-\mb)^2/\sb^2) -2*\corel*((\x-\ma)/\sa*(\y-\mb)/\sb)/(2*(1-\corel^2)));},
	declare function={
		CopComonotone(\x,\y)=min(\x,\y);},
	declare function={
		CopAntimonotone(\x,\y)=max(\x+\y-1,0);},
	declare function={ % teta entre -1 et 0 ou entre 0 et infty
		CopClayton(\x,\y,\theta)=(\x^(-\theta)+\y^(-\theta)-1)^(-1/\theta);},
	declare function={
		dGumbel(\x,\mu,\sigma) = exp(-exp((\mu-\x)/\sigma))*exp((\mu-\x)/\sigma)/\sigma;},
	declare function={
		pGumbel(\x,\mu,\sigma) = exp(-exp((\mu-\x)/\sigma));},
	declare function={
		Put(\x,\Koption,\Premium)= -\Premium + (\x<\Koption) * (\Koption-\x);},
	declare function={
		Call(\x,\Koption,\Premium)= -\Premium + (\x>\Koption) * (\x-\Koption);},
	declare function={
		ddd(\x,\Koption ,\riskfree,\T,\sigma) = (ln(\x/\Koption)+(\riskfree+pow(\sigma,2)/2)*\T)/(\sigma*(sqrt(\T)));},
	declare function={
		dd(\x,\Koption,\riskfree,\T,\sigma) = ddd(\x, \Koption, \riskfree, \T, \sigma) - \sigma *( sqrt(\T));},
	declare function={
		BSCall(\S,\Koption,\riskfree,\T,\sigma)= \S*normcdf(ddd(\S, \Koption, \riskfree, \T, \sigma),0,1) - \Koption*exp(-\riskfree*\T)*normcdf(dd(\S, \Koption, \riskfree, \T, \sigma),0,1);},
	declare function={
		BSPut(\x,\Koption,\riskfree,\T,\sigma)= \Koption*exp(-\riskfree*\T) * normcdf(-dd(\x, \Koption, \riskfree, \T, \sigma),0,1) - \x*normcdf(-ddd(\x, \Koption, \riskfree, \T, \sigma),0,1);},
	declare function={
		Bi(\i)=1.75/(1 +(\i))^1+1.75/(1 +(\i))^2+1.75/(1 +(\i))^3+1.75/(1 +(\i))^4+101.75/(1 +(\i))^5 ;},
	declare function={
		BiDur(\i)=98.3132*(1 - 4.72945*(\i-.02109) );},
	declare function={
		BiConv(\i)=98.3132*(1 - 4.72945*(\i-.02109) + 22.96406/2*((\i-.02109))^2) ;}
		}

\newcommand{\N}{2026{}}
\newcommand{\NN}{2025{}}
\newcommand{\NNN}{2024{}}

\newcommand{\SASSVG}{ svg "M4.158.001c-.28-.006-.567.026-.862.1-1.364.34-2.89 2.096-1.224 4.12l.995 1.205.28.343c.19.224.497.22.724.033s.293-.496.105-.724l-.114-.138h.005l-.162-.2-.77-.933c-.724-.88-.4-1.964.58-2.58.865-.543 2.453-.35 3 .543C6.37.76 5.366.028 4.158.001z" ;}
\definecolor{SAS}{HTML}{007CC2}
\newcommand{\SAS}{\raisebox{-1pt}{%\tikzset{external/export next=false}
		\begin{tikzpicture}\fill [color=SAS, xscale=-1.1, yscale=-1.1] \SASSVG ;
		\fill [color=SAS,xscale=-1.1, yscale=-1.1, rotate=180, yshift=-8pt, xshift=-8pt] \SASSVG ;
	\end{tikzpicture}}SAS }
\makeatletter
% isvalinlist: retourne \foundmatch=true si valeur dans liste
\newcommand{\isvalinlist}[2]{%
	\def\valtolook{#1}%
	\def\foundmatch{false}%
	\@for\item:=#2\do{%
		\ifthenelse{\equal{\item}{\valtolook}}{%
			\def\foundmatch{true}%
		}{}%
	}%
}
\makeatother

% highlightnodeDualList : applique couleur selon la liste
\newcommand{\highlightnodeDualList}[5]{%
	\begingroup
	\edef\currentval{\number\value{higher}}%
	\isvalinlist{\currentval}{#1}%
	\ifthenelse{\equal{\foundmatch}{true}}{%
		\node[#3, fill=OrangeProfondIRA, text=white, draw=none] (#4) {#5};%
	}{%
		\isvalinlist{\currentval}{#2}%
		\ifthenelse{\equal{\foundmatch}{true}}{%
			\node[#3, fill=BleuProfondIRA, text=white, draw=none] (#4) {#5};%
		}{%
			\node[#3, draw=GrisLogoIRA, text=GrisLogoIRA] (#4) {#5};%
		}%
	}%
	\endgroup
}
%%%%%%%%%%%%%%%%%%%%%%%%%%%%%%%%%%%%%%%%%%%%%%%%



\usepackage{amsthm}
\makeatletter
\newtheoremstyle{sansparenthese}
  {6pt} % space above
  {6pt} % space below
  {\normalfont} % body font
  {0pt} % indent amount
  {\bfseries} % theorem head font
  {} % punctuation after theorem head
  {.5em} % space after theorem head
  {\thmname{#1}\thmnumber{ #2}\thmnote{ -- #3}\ \newline } % theorem head spec
\makeatother

\theoremstyle{sansparenthese}
\newtheorem{f}{}%[section]
%
\title{
\author{}
\emph{Actuarial Formula Cheat Sheet}\\
My Actuarial Revision Sheet\\\medskip\small
Master in Insurance Economics/Econometrics\\
Master in Actuarial Science}
\begin{document}
\maketitle
%
\newcounter{higher}\newcounter{rep} \setcounter{higher}{1}
\begin{center}
% !TeX root = ActuarialFormSheet_MBFA-en.tex
% !TeX spellcheck = en_US
\begin{animateinline}[autoplay,loop,poster=1]{1}%,timeline=Graph/ShemasTimesLines.txt
	\whiledo{\thehigher<12}{
		%		    \setcounter{rep}{0}
		%		\whiledo{\therep<3}{% répéter 3 fois la même frame pour simuler 3 secondes
			\begin{tikzpicture}[%
				node distance=1.25cm and 0.75cm,
				every node/.style={draw, align=center, rounded corners, font=\small, minimum height=1cm, minimum width=3cm},
				every path/.style={->, thick},
				bend/.style={bend left=15},
				bendrev/.style={bend right=15},
				]
				% Définir styles conditionnels
				% Centre
				\node[fill=BleuProfondIRA, text=white, scale=2] (actu) {Actuarial\\Sciences};
				% En bas
				% Non-Vie
				\highlightnodeDualList{}{2,5,8}{below=of actu, yshift=-7.5mm}{nonvie}{Non-Life}
				% Vie et Santé
				\highlightnodeDualList{}{1,4,7}{left=of nonvie}{vie}{Life}
				\highlightnodeDualList{}{3,6,9}{right=of nonvie}{sante}{Health}
				\node[right=of sante, draw=none] (rep1) {};
				
				% Primes, Provisions, Droit/Gestion
				\highlightnodeDualList{1,2,3,11}{4,5,6}{above=of rep1}{primes1}{Premiums}
				\highlightnodeDualList{1,2,3,11}{7,8,9}{above=of primes1}{provisions1}{Reserves}
				% Réassurance et Solvabilité
				\node[left=of vie, draw=none] (rep4) {};
				\highlightnodeDualList{1,2,3,4,5,6,12}{10}{above=of rep4}{solv}{Solvency}
				\highlightnodeDualList{1,2,3,10}{11}{above=of solv}{reass1}{Reinsurance}
				\highlightnodeDualList{10,11}{}{above=of solv, yshift=-11.5mm, xshift=10mm}{ERM}{ERM}
				
				\highlightnodeDualList{9,11}{}{above=of provisions1}{Info}{Programming\\Info}
				\highlightnodeDualList{9,11}{}{above=of Info, yshift=-10mm}{gestion}{Management}
				\highlightnodeDualList{9,11}{}{above=of gestion, yshift=-10mm}{droit}{Law}
				% Math Fi et ses sous-nœuds
				\highlightnodeDualList{1,4,7,10,11}{}{left=of Info}{mathfi}{Financial Math}
				\highlightnodeDualList{1,4,7,10,11}{}{above=of mathfi, yshift=-10mm}{taux}{Yield Curves}
				\highlightnodeDualList{10}{}{above=of taux, yshift=-10mm}{GP}{Portfolio\\Management}
				\highlightnodeDualList{7,10}{}{above=of GP, yshift=-10mm}{ALM}{Active Passive}
				\highlightnodeDualList{10,11}{}{above=of ALM, yshift=-10mm}{derive}{Derivatives}
				\highlightnodeDualList{10,11}{}{above=of derive, yshift=-10mm}{titrisation}{Securitization}
				
				% Proba & Stat et ses sous-nœuds
				\highlightnodeDualList{1,2,3,4,5,6,7,8,9,10,11}{}{left=of mathfi}{proba}{Proba \& Stat}
				\highlightnodeDualList{2,3,5,6,11}{}{above=of proba, yshift=-10mm}{glm}{GLM}
				\highlightnodeDualList{2,5,10,11}{}{above=of glm, yshift=-10mm}{extremes}{Theory\\Extremes}
				\highlightnodeDualList{2,5,10,11}{}{above=of extremes, yshift=-10mm}{copules}{Copulas}
				\highlightnodeDualList{2,3,5,6}{}{above=of copules, yshift=-10mm}{ml}{Machine Learning}
				\highlightnodeDualList{7,10}{}{above=of ml, yshift=-10mm}{stoch}{Stochastic}
				
				% Économie, économétrie et variantes
				\node[left=of proba, draw=none] (rep3) {};
				\highlightnodeDualList{2,3,5,6,11}{}{above=of rep3, yshift=-10mm}{econo}{Econometrics}
				\highlightnodeDualList{}{}{above=of econo, yshift=-10mm}{Temp}{Time Series}
				\highlightnodeDualList{4,5}{}{above=of Temp, yshift=-10mm}{Spatial}{Spatial}
				\highlightnodeDualList{10}{}{left=of rep3}{eco}{Economy}
				\highlightnodeDualList{10}{}{above=of eco, yshift=-10mm}{risk}{Risk Eco}
				\highlightnodeDualList{1,4,10,11}{}{above=of risk, yshift=-10mm}{demographie}{Demography}
				
				
				% Connexions principales
				\draw (actu.south)  |- +(0,-.750) -| (vie.north);
				\draw (actu) -- (nonvie);
				\draw (actu.south)  |- +(0,-.750) -| (sante.north);
				%		
				\draw (actu.east)  -| +(0.75,0) |- (primes1.west);
				\draw (actu.east)  -| +(0.75,0) |- (provisions1.west);
				\draw (actu.west)  -| +(-.750,0) |- (reass1.east);
				\draw (actu.west)  -| +(-.750,0) |- (solv.east);
				%		
				\draw (actu.north) |- +(0,.750) -| (proba.south);
				\draw (actu.north) |- +(0,.750) -| (eco.south);
				\draw (actu.north) |- +(0,.750) -| (mathfi.south);
				\draw (actu.north) |- +(0,.750) -| (Info.south);
				\draw (actu.north) |- +(0,.750) -|  +(+57.5mm,0.750) |-(gestion.west);
				\draw (actu.north) |- +(0,.750) -|  +(+57.5mm,0.750) |-(droit.west);
				\draw (proba) -| (econo);
				\draw (eco) -| (econo);		
				% Spécialisations Proba
				\draw (proba.west) |- +(-.25,0) |- (glm.west);
				\draw (proba.west) |- +(-.25,0) |-  (extremes.west);
				\draw (proba.west) |- +(-.25,0) |-  (copules.west);
				\draw (proba.west) |- +(-.25,0) |-  (ml.west);
				\draw (proba.west) |- +(-.25,0) |-  (stoch.west);		
				% Réassurance : liens
				\draw (econo.west) |- +(-.25,0) |- (Temp.west);
				\draw (econo.west) |- +(-.25,0) |- (Spatial.west);
				\draw (eco.west) |- +(-.25,0) |- (risk.west);
				\draw (eco.west) |- +(-.25,0) |- (demographie.west);
				\draw (mathfi.west) |- +(-.25,0) |- (taux.west);
				\draw (mathfi.west) |- +(-.25,0) |- (GP.west);
				\draw (mathfi.west) |- +(-.25,0) |- (ALM.west);
				\draw (mathfi.west) |- +(-.25,0) |- (derive.west);
				\draw (mathfi.west) |- +(-.25,0) |- (titrisation.west);
				
				%		\draw (reass1) -- (titri);
				%		\draw (reass1) -- (reass1);
				%		\draw (reass1) -- (primes1);
			\end{tikzpicture}
			%  \stepcounter{rep}
			\newframe %}
		\stepcounter{higher}
	}
\end{animateinline}

\end{center}
\newpage

\begin{center}
	\bigskip
	
\section*{Insurance Keywords}
    \medskip
\end{center}


\begin{multicols}{2}
	
% !TeX root = ActuarialFormSheet_MBFA-en.tex
% !TeX spellcheck = en_US
\begin{f}[Life and Non-Life Insurance]
	
The distinction between life and non-life insurance is fundamental. An insurer cannot offer both types of insurance without holding two separate companies: 	
\begin{itemize}
	\item \textbf{Life insurance}, i.e., personal insurance excluding coverage for bodily injuries,
	\item \textbf{Non-life insurance}, which includes property and liability insurance as well as insurance for bodily injuries.
\end{itemize}
\end{f}


\begin{f}[The Principles of Insurance]
	
	
Insurance is assumed to:
	\begin{itemize}
		\item be based on utmost good faith,
		\item apply only if the insured has an insurable interest in preserving the item (property insurance),
		\item operate under the indemnity principle:
		\begin{itemize}
			\item not allow enrichment from a claim settlement,
			\item not even through insurance accumulation,
			\item include subrogation (in Liability Insurance, if the insurer compensates the insured victim, the insured cannot then claim from the party responsible for the loss.)
		\end{itemize}
		\item not reduce the insured’s efforts in prevention and protection, as a reasonable person, even if financially protected.
	\end{itemize}
	\item establish causality in Liability Insurance — I am not liable if I did not contribute to the cause of the loss.
\end{f}


\begin{f}[The Insurance Policy]
	
	The \textbf{insurance policy} (or contract) is the contractual document that governs the relationship between the insurance company (or mutual insurance company) and the insured (policyholder). 
	This contract defines in particular:
	
	\begin{enumerate}
		\item the list of covered events, including any exclusions,
		\item the coverage, i.e., the assistance provided to the insured in case of a loss,
		\item the obligations of the insured:
		\begin{itemize}
			\item any preventive measures required to reduce risk,
			\item time limits for reporting a claim to the insurer,
			\item the amount and payment conditions of the premium (deductible, limit),
			\item the conditions for cancellation of the policy (automatic renewal),
		\end{itemize}
		\item the obligations of the insurance company: time limits for compensation payments.
		
	\end{enumerate}
	
\end{f}
\begin{f}[The Premium and Claims]
Classically, the role of the insurer is to substitute a constant $C$, the \textbf{contribution} or the \textbf{premium}, for a random claim $S$.
La \textbf{pure premium} or \textbf{technical premium} aims 
to compensate claims without surplus or profit, overall $C_t = \mathbb{E}[S]$
	
	The \textbf{net premium} is higher than the pure premium. It aims to cover the cost of claims and provide a safety margin.
	
	The \textbf{gross premium} is the net premium + overhead expenses + commissions + expected profit + taxes.
	
	For commercial reasons, the gross premium actually charged may differ from the technical premium.

	\textbf{Written premium:} premium charged to the insured to cover claims that may occur during the 
coverage period defined by the contract (generally 1 year in Property and Casualty insurance).

\textbf{Earned premium:} proportion of the written premium used to cover the risk over the exposure period of 
one policy year.
\tikzstyle{NoeudR}=[rectangle, shape border rotate=90, draw,minimum height=0.6cm,align=center, BleuProfondIRA]

\resizebox{\linewidth}{!}
{\begin{tikzpicture} %[node distance=5cm]
	%	
	\draw[thick,<->,>=latex,BleuProfondIRA] (-3,0) -- (5,0) ;
	\node (JanNN) at (-2,0.8) [NoeudR] {\footnotesize  1\ier{} January\\ \footnotesize \NN};
	\node (JuilNN) at (0,0.8) [NoeudR] {\footnotesize  1\ier{} July\\ \footnotesize \NN};
	\node (JanN) at (2,0.8) [NoeudR] {\footnotesize  31 December\\ \footnotesize  \NN};
	\node (JuilN) at (4,0.8) [NoeudR] {\footnotesize  1\ier{} July\\ \footnotesize \N};
	\draw[ BleuProfondIRA] (JanNN.south) -- ++(0,-0.8);
	\draw[ BleuProfondIRA] (JuilNN.south) -- ++(0,-0.8);
	\draw[ BleuProfondIRA] (JanN.south) -- ++(0,-0.8);
	\draw[ BleuProfondIRA] (JuilN.south) -- ++(0,-0.8);
	\draw[thick,<->,>=latex,VertIRA] ($(JuilNN.south) +(0,-1)$) -- ($(JuilN.south) +(0,-1)$) node [below, midway] {\footnotesize Written premium 600€} ;
	\draw[thick,<->,>=latex,OrangeProfondIRA] ($(JuilNN.south) +(0,-1.6)$) -- ($(JanN.south) +(0,-1.6)$) node [below, midway] {\footnotesize Earned 300€} ;
	\draw[thick,<->,>=latex,OrangeProfondIRA,densely dotted] ($(JanN.south) +(0,-1.6)$) -- ($(JuilN.south) +(0,-1.6)$) node [below, midway,align=right] 
	{\footnotesize Unearned 300€ \\\footnotesize  Provisioned for the period \N } ;
	%	
\end{tikzpicture}
}
The $S/P$ is the key indicator. For the insurer to make a profit the $S/P\ll 1$.
\end{f}
%\end{multicols}

\begin{f}[Loss / Payment Triangle]
	
	Insurance accounting is broken down by the \textbf{accident year} of the claim. 
	If a premium covers multiple calendar years, a proportional part will be allocated to each.
	Each payment and each claim provisioning is assigned to the accident year. 
	The monitoring of payments or expenses is expressed through a triangle (triangular matrix):
	
	$$
	\left(\begin{array}{ccccc}
		\rowcolor{white}	C_{1,1} & C_{1,2} & \ldots & & C_{1, n} \\
		\rowcolor{white}	C_{1,1} & C_{1,2} & \ldots & C_{2, n-1} \\
		\rowcolor{white}	\vdots & \vdots & & \\
		\rowcolor{white}	C_{n-1,1} & C_{n-1,2} & & \\
		\rowcolor{white}	C_{n, 1} & & &
	\end{array}\right)
	$$
	where $C_{i, j}=\sum_{k=1}^{j} X_{i, k}$ represents the cumulative amount of claims paid for origin year $i$ and development year $j$.
	
\end{f}
%\begin{multicols}{2}


	
\begin{f}[Solvency II and Risk Management]
	
	\textbf{Solvency II} is the European regulatory framework applicable to insurers and reinsurers since 2016. It is based on three interdependent pillars:
	
	\begin{itemize}
		\item \textbf{Pillar 1: Quantitative Requirements} \\
		Determines the capital requirements:
		\begin{itemize}
			\item \textbf{SCR} (Solvency Capital Requirement) : capital to absorb an extreme shock (99.5\% over 1 year),
			\item \textbf{MCR} (Minimum Capital Requirement) : absolute minimum threshold,
			\item admissible assets to cover technical provisions and capital requirements.
		\end{itemize}
		
		\item \textbf{Pillar 2: Governance, Internal Control, and Risk Management} \\
		The core link with \textbf{ERM} (\emph{Enterprise Risk Management}). The requirements cover:
		\begin{itemize}
			\item governance: boards of directors responsible for the risk management framework;
			\item an effective \textbf{internal control} system;
			\item independent key functions: \textbf{actuarial}, \textbf{risk management}, \textbf{compliance}, \textbf{internal audit};
			\item \textbf{ORSA} (\emph{Own Risk and Solvency Assessment}): internal assessment of risks and solvency, a central tool aligning strategy, risk appetite, and economic capital.
		\end{itemize}
		
		\item \textbf{Pillar 3: Market Discipline} \\
		Based on \textbf{transparency} and communication :
		\begin{itemize}
			\item \textbf{SFCR} (\emph{Solvency and Financial Condition Report}) : public, summarizes solvency and financial position,
			\item \textbf{RSR} (\emph{Regular Supervisory Report}): intended for the supervisor,
			\item quantitative reporting: regulatory statements (\textbf{QRTs}), regular submission of financial and prudential data.
		\end{itemize}
	\end{itemize}
		
\end{f}
\begin{f}[Main Branches of Life and Non-Life Insurance]
	
	Life insurance covers long-term commitments, with or without a savings component:
	\begin{itemize}[nosep]
		\item \textbf{Life insurance} : lump sum or annuity paid if the insured is alive at a given date.
		\item \textbf{Death insurance} : payment if the insured dies during the covered period.
		\item \textbf{Endowment insurance} : combination of life and death coverage.
		\item \textbf{Life annuity} : periodic payments until death.
		\item \textbf{Savings/retirement} : products with deferred capital or deferred annuity.
		\item \textbf{Unit-linked policies} : benefits dependent on the value of financial assets.
		\item \textbf{Group contracts} : occupational pensions, group welfare insurance.
	\end{itemize}
	
	
	Non-life insurance covers risks occurring in the short or medium term:
	\begin{itemize}[nosep]
		\item \textbf{Automobile} : third-party liability, vehicle damage.
		\item \textbf{Home} : fire, theft, water damage, liability.
		\item \textbf{General liability} : personal liability, business liability.
		\item \textbf{Health and welfare} : medical reimbursements, disability, incapacity.
		\item \textbf{Personal accident} : capital in case of accident, disability, or death.
		\item \textbf{Business interruption} : financial losses related to a claim.
		\item \textbf{Technical risks} : construction, machinery breakdown.
		\item \textbf{Transport, aviation, maritime insurance} : goods in transit, specific liabilities.
	\end{itemize}
	
	
\begin{center}
	\resizebox{1.1\linewidth}{!}{	\begin{tikzpicture}[ every node/.style={font=\small,text width=3cm}, node distance=0.75cm and 1.cm]
			% Rectangle central "Responsabilit\'e"
			\node[draw, rectangle, minimum width=2cm, minimum height=1cm, align=center] (resp) at (0,0) {Liability};
			%		
			% Branches vers "Responsabilit\'e p\'enale" et "Responsabilit\'e civile"  +(-.25,0) |-
			\node[below right=of resp, xshift=-.5cm, yshift=.5cm] (penale) {Criminal liability};
			\node[below=0cm of penale] (inass) {Uninsurable};
			\draw (resp.east) -| (penale.north);
			%		
			\node[below left=of resp] (civile) {Civil liability};
			\draw (resp.west) -| (civile.north);
			%		
			%		% Vers RC contractuelle et d\'elictuelle
			\node[below=of civile, xshift=-1.5cm] (rccontr) {Contractual liability (CL)};
			\node[below=of civile, xshift=3cm] (rcdel) {Tort liability (TL)};
			\draw (civile.south) |-  +(0,-0.25) -| (rccontr.north);
			\draw (civile.south) |-  +(0,-0.25) -| (rcdel.north);
			%		
			% Obligation de moyen / de r\'esultat
			\node[below= of rccontr, xshift=-1.5cm, rotate=45] (moyen) {Obligation\\ of means};
			\node[below= of rccontr, xshift=0.5cm, rotate=45] (resultat) {Obligation\\ of result};
			\draw (rccontr.south) |-  +(0,-0.25) -| (moyen.north);
			\draw (rccontr.south) |-  +(0,-0.25) -| (resultat.north);
			%		
			% Sous RC d\'elictuelle : fait personnel / d'autrui / de choses
			\node[below=of rcdel, xshift=-2cm, rotate=45] (choses) {Act of things \\ Presumption};
			\node[below= of rcdel, rotate=45] (autrui) {Acts of others \\ Presumption \\ Article 1384 of the CC};
			\node[below= of rcdel, xshift=2cm, rotate=45] (personnel) {Personal act\\Fault to be proven};
			\draw (rcdel.south) |-  +(0,-0.25) -| (choses.north);
			\draw (rcdel.south) |-  +(0,-0.25) -| (autrui.north);
			\draw (rcdel.south) |-  +(0,-0.25) -| (personnel.north);
			%		
			% Faute \`a prouver et fait conscient sous fait personnel
			\node[right= of personnel, rotate=0, anchor=west] (fconscient) {Conscious act};
			\draw (personnel.south) -|  +(0.75,0) |- (fconscient.west);
			%		
			% Imprudence / N\'eglience sous Fait conscient
			\node[below=of fconscient] (impru) {Carelessness};
			\node[below= of impru] (negl) {Negligence};
			\draw (fconscient.west)  |- +(-0.25,0) |- (impru.west);
			\draw (impru.west)  |- +(-0.25,0) |- (negl.west);
			% Source
			%		\node[below=2cm of moyen, anchor=west, text width=10cm] (source) {\small Source : Les Grands Principes de l'Assurance, Couilbault et Eliashberg, 10e \'edition};
		\end{tikzpicture}
	}
\end{center}

	
\end{f}


\begin{f}[Actuary]
	
	In practice, the actuary:
	\begin{itemize}[nosep]
		\item prices insurance and welfare products,
		\item estimates technical provisions,
		\item projects cash flows and values long-term liabilities,
		\item measures economic capital (SCR, ORSA) and contributes to ERM,
		\item advises management on strategy, solvency, and regulatory compliance.
	\end{itemize}
	
\end{f}
\end{multicols}

\newpage
\begin{center}
	\section*{Financial Mathematics}
	\medskip
\end{center}


\begin{multicols}{2}
	
	% !TeX root = ActuarialFormSheet_MBFA-en.tex
% !TeX spellcheck = fr_FR

\begin{f}[Capitalisation Actualisation]

\begin{tikzpicture}[scale=0.85]
% Draw the x-axis and y-axis.
\def\w{11}
\def\n{9}
\draw[ line width=1,dotted, color=OrangeProfondIRA, arrows={-Stealth[length=4, inset=0]}] (0,0) -- (0,0.4909) node (A) {};	
\foreach \y in  {0,...,3} {
	\draw (\y,0) -- (\y,-0.1);
	\ifthenelse{\y>0 }{	\node[below] at (\y,-0.1) {\tiny $ \scriptstyle \y$};
		}{
		\node[below] at (\y,-0.1) {\tiny 0};}
}
\foreach \y in  {-3,...,1} {
	\draw (\y+\n,0) -- (\y+\n,-0.1);
	\ifthenelse{\y<0 }{\node[below] at (\y+\n,-0.1) {\tiny $\scriptstyle n \y$};}{}
	\ifthenelse{\y=0 }{\node[below] at (\y+\n,-0.1) {\tiny $\scriptstyle n $};
		\draw[ line width=1, color=OrangeProfondIRA, arrows={-Stealth[length=4, inset=0]}] (\y+\n,0) -- (\y+\n,1) node (B) {};}{}
	\ifthenelse{\y>0}{	\node[below] at (\y+\n,-0.1) {\tiny $\scriptstyle n+\y$};}{}
}
\draw[ line width=1] (-.25,0) -- (4,0);
\draw[ line width=1, dashed] (4,0) -- (5,0);
\draw[arrows={-Stealth[length=4, inset=0]}, line width=1] (5,0) -- (\w,0);
\draw[arrows={-Stealth[length=4, inset=0]},color=OrangeProfondIRA] (B)  to [bend right]  node [pos=0.5, below=10pt] {Actualisation}(A) ;
\end{tikzpicture}

\begin{tikzpicture}[scale=0.85]
% Draw the x-axis and y-axis.
\def\w{11}
\def\n{9}
\draw[ line width=1, color=OrangeProfondIRA, arrows={-Stealth[length=4, inset=0]}] (0,0) -- (0,1) node (A) {};	
\foreach \y in  {0,...,3} {
	\draw (\y,0) -- (\y,-0.1);
	\ifthenelse{\y>0 }{	\node[below] at (\y,-0.1) {\tiny $ \scriptstyle \y$};
	}{
		\node[below] at (\y,-0.1) {\tiny 0};}
}
\foreach \y in  {-3,...,1} {
	\draw (\y+\n,0) -- (\y+\n,-0.1);
	\ifthenelse{\y<0 }{\node[below] at (\y+\n,-0.1) {\tiny $\scriptstyle n \y$};}{}
	\ifthenelse{\y=0 }{\node[below] at (\y+\n,-0.1) {\tiny $\scriptstyle n $};
		\draw[ line width=1,dotted, color=OrangeProfondIRA, arrows={-Stealth[length=4, inset=0]}] (\y+\n,0) -- (\y+\n,2.0368) node (B) {};}{}
	\ifthenelse{\y>0}{	\node[below] at (\y+\n,-0.1) {\tiny $\scriptstyle n+\y$};}{}
}
\draw[ line width=1] (-.25,0) -- (4,0);
\draw[ line width=1, dashed] (4,0) -- (5,0);
\draw[arrows={-Stealth[length=4, inset=0]}, line width=1] (5,0) -- (\w,0);
\draw[arrows={-Stealth[length=4, inset=0]},color=OrangeProfondIRA] (A)  to [bend left]  node [pos=0.55, below=10pt] {Capitalisation}(B) ;
\end{tikzpicture}

\end{f}
\hrule

\begin{f}[Les Intérêts]

\emph{Escompte ou taux précompté $d$}
     $$d=i/(1+i)$$
\emph{Intérêt simple $i$}
$$
I_t=P i t=P i \frac{k}{365}
$$
\emph{Intérêt composés $i$}
$$
V_n=P(1+i)^n=P\left(1+\frac{p}{100}\right)^n
$$
\emph{Intérêt continu $r$}
$$V_t=V_0\ e^{rt}$$

\emph{Taux effectif $i_e$}
$$
i_e=\left( 1+\frac{i}{m}\right) ^{m}-1
$$
où $i$ est le taux nominal et $m$ le nombre de périodes sur un an.

\emph{Taux équivalent $i^{(m)}$}
$$
i^{(m)}=m(1+i)^{1 / m}-1
$$

\emph{Taux nominal $i$ et taux périodique}

Le taux \textbf{nominal} ou taux \textbf{facial} permet de calculer les intérêts dus sur un an.
Le taux \textbf{périodique} correspond au taux nominal divisé par le nombre de périodes sur un an $i/m$.
Si le taux périodique est hebdomadaire, le taux nominal sera divisé par 52.
\end{f}
\hrule

\begin{f}[Valeur actuelle et valeur future]

La valeur actuelle (VA)  ou valeur présente (VP)  représente le capital qui doit être investi aujourd'hui à un taux d'intérêt composé annuel $i$ pour obtenir des flux de trésorerie futurs ($F_k$) aux moments $t_k$:
\begin{equation}
	VP = \sum_{k=1}^{n} F_k \times \frac{1}{(1+i)^{t_k}}
\label{ValeurActuelle}
\end{equation}
Lorsque les $F_k$ sont constants
\begin{equation}
	VP = K  \frac{1 - (1+i)^{-n}} {i}
\label{ValeurActuelleFluxCt}
\end{equation}

La valeur future (VF) représente la valeur du capital en $T$ qui, avec un taux d'intérêt composé annuel $i$,  capitalise les flux de trésorerie futurs ($F_k$) aux moments $t_k$.
\begin{equation}
	VF=V_n = \sum_{k=1}^{n} F_k \times (1+i)^{n-t_k}
\end{equation}
Plus généralement $VF= (1+i)^{n}VP$.
\end{f}
\hrule

\begin{f}[Annuités]
\ \newline


	Certain annuity \(a_{\lcroof{n}}\) (or \(a_{\lcroof{n}\;i}\) if the interest rate \(i\) needs to be specified): this is the default case in financial mathematics. Its payments are, for example, guaranteed by a contract.
	
\begin{center}
	\begin{tikzpicture}[scale=0.75]
	% Draw the x-axis and y-axis.
	\def\w{11}
	\def\n{9}
	\node[left] at (-.5,0) {\(a_{\lcroof{n}}\)};
	\foreach \y in  {0,...,3} {
	\draw (\y,0) -- (\y,-0.1);
	\ifthenelse{\y>0 }{	\node[below] at (\y,-0.1) {\tiny \( \scriptstyle \y\)};
		\draw[ line width=1, color=OrangeProfondIRA, arrows={-Stealth[length=4, inset=0]}] (\y,0) -- (\y,1);}{
		\node[below] at (\y,-0.1) {\tiny 0};}
}
\foreach \y in  {-3,...,1} {
	\draw (\y+\n,0) -- (\y+\n,-0.1);
	\ifthenelse{\y<0 }{\node[below] at (\y+\n,-0.1) {\tiny \(\scriptstyle n \y\)};}{}
	\ifthenelse{\y=0 }{\node[below] at (\y+\n,-0.1) {\tiny \(\scriptstyle n \)};}{}
	\ifthenelse{\y>0}{	\node[below] at (\y+\n,-0.1) {\tiny \(\scriptstyle n+\y\)};}{}
	\ifthenelse{\y<1 }{	\draw[ line width=1, color=OrangeProfondIRA, arrows={-Stealth[length=4, inset=0]}] (\y+\n,0) -- (\y+\n,1);}
}
\draw[ line width=1] (-.25,0) -- (4,0);
\draw[ line width=1, dashed] (4,0) -- (5,0);
\draw[arrows={-Stealth[length=4, inset=0]}, line width=1] (5,0) -- (\w,0);
\end{tikzpicture}
\end{center}
\[
\ddot{a}_{\lcroof{n}}=1+v+\cdots+v^{n-1}=\frac{1-v^{n}}{1-v}=\frac{1-v^{n}}{d}
\]


Contingent annuity \(\ddot{a}_{x}\): its payments are conditional on a random event, such as a life annuity of an individual aged \(x\). In this example, payments continue until death occurs :

\begin{center}
	\begin{tikzpicture}[scale=0.75]
	% Draw the x-axis and y-axis.
	\def\w{11}
	\def\n{6}
	\node[left] at (-.5,0) {\(a_{x}\)};
	
	\begin{scope}[shift={(\n+.5+3,.25)}]
		\draw[color=OrangeProfondIRA,scale=0.2,fill=OrangeProfondIRA] \Cerceuil;
	\end{scope}
	\foreach \y in  {0,...,3} {
		\draw (\y,0) -- (\y,-0.1);
		\ifthenelse{\y>0 }{	\node[below] at (\y,-0.1) {\tiny \( \scriptstyle x+\y\)};
			\draw[ line width=1, color=OrangeProfondIRA, arrows={-Stealth[length=4, inset=0]}] (\y,0) -- (\y,1);}{
			\node[below] at (\y,-0.1) {\tiny \( \scriptstyle x\)};}
	}
	\foreach \y in  {0,...,4} {
		\draw (\y+\n,0) -- (\y+\n,-0.1);
		\ifthenelse{\y>0 }{\node[below] at (\y+\n,-0.1) {\tiny \(\scriptstyle x+n+\y\)};}{
			\node[below] at (\y+\n,-0.1) {\tiny \(\scriptstyle x+n\)};}
		\ifthenelse{\y<4 }{	\draw[ line width=1, color=OrangeProfondIRA, arrows={-Stealth[length=4, inset=0]}] (\y+\n,0) -- (\y+\n,1);}
	}
\draw[ line width=1] (-.25,0) -- (4,0);
\draw[ line width=1, dashed] (4,0) -- (5,0);
\draw[arrows={-Stealth[length=4, inset=0]}, line width=1] (5,0) -- (\w,0);
\end{tikzpicture}	

\end{center}
The date of death is represented here by a small coffin. This type of annuity will be extensively studied in the life actuarial section.

Annuity in arrears (immediate) \(a_{\lcroof{n}}\): its periodic payments are made at the end of each payment period, as with a salary paid at the end of the month. This is the default case, previously illustrated for the certain annuity.
\[
\ddot{a}_{\lcroof{n}}=1+v+\cdots+v^{n-1}=\frac{1-v^{n}}{1-v}=\frac{1-v^{n}}{d}
\]
%
\[
	\mathrm{PV}_{\lcroof{n}}^{\text {due }}=K \ddot{a}_{\lcroof{n}}=K \frac{1-v^{n}}{d} 
\]

Annuity in advance (due) \(\ddot{a}_{\lcroof{n}}\): its periodic payments are made at the beginning of each payment period, as with rent payments, for example.

\begin{center}
	\begin{tikzpicture}[scale=0.75]
		% Draw the x-axis and y-axis.
		\def\w{11}
		\def\n{9}
		\node[left] at (-.5,0) {\(\ddot{a}_{\lcroof{n}}\)};
		\foreach \y in  {0,...,3} {
			\draw (\y,0) -- (\y,-0.1);
				\node[below] at (\y,-0.1) {\tiny \( \scriptstyle \y\)};
				\draw[ line width=1, color=OrangeProfondIRA, arrows={-Stealth[length=4, inset=0]}] (\y,0) -- (\y,1);
		}
		\foreach \y in  {-3,...,1} {
			\draw (\y+\n,0) -- (\y+\n,-0.1);
			\ifthenelse{\y<0 }{\node[below] at (\y+\n,-0.1) {\tiny \(\scriptstyle n \y\)};}{}
			\ifthenelse{\y=0 }{\node[below] at (\y+\n,-0.1) {\tiny \(\scriptstyle n \)};}{}
			\ifthenelse{\y>0}{	\node[below] at (\y+\n,-0.1) {\tiny \(\scriptstyle n+\y\)};}{}
			\ifthenelse{\y<0 }{	\draw[ line width=1, color=OrangeProfondIRA, arrows={-Stealth[length=4, inset=0]}] (\y+\n,0) -- (\y+\n,1);}{}
		}
		\draw[ line width=1] (-.25,0) -- (4,0);
		\draw[ line width=1, dashed] (4,0) -- (5,0);
		\draw[arrows={-Stealth[length=4, inset=0]}, line width=1] (5,0) -- (\w,0);
	\end{tikzpicture}
\end{center}
Also denoted \(\mathrm{PV}^{\mathrm{im}}\) :
\[
a_{\lcroof{n}}=v+v^{2}+\cdots+v^{n}=\frac{1-v^{n}}{i}=v \frac{1-v^{n}}{1-v}
\]
%
\[
	\mathrm{PV}_{\lcroof{n}}^{\mathrm{im}}=K a_{\lcroof{n}}=K \frac{1-v^{n}}{i} 
\]
Perpetuity \(a\) or \(a_{\lcroof{\infty}}\):
\[
a=1/i
\]
Deferred annuity \(_{m|}a_{\lcroof{n}}\): its payments do not start in the first period but after \(m\) periods, with \(m\) fixed in advance.
\begin{center}
	\begin{tikzpicture}[scale=0.75]
		% Draw the x-axis and y-axis.
\def\w{10}
\def\n{9}
\def\m{2}
\node[left] at (-.5,0) {\(_{m|}a_{\lcroof{n}}\)};
\foreach \y in  {0,...,2} {
	\draw (\y+\m,0) -- (\y+\m,-0.1);
	\ifthenelse{\y<0 }{\node[below] at (\y+\m,-0.1) {\tiny \(\scriptstyle m \y\)};}{}
\ifthenelse{\y=0 }{\node[below] at (\y+\m,-0.1) {\tiny \(\scriptstyle m \)};}{}
\ifthenelse{\y>0}{	\node[below] at (\y+\m,-0.1) {\tiny \(\scriptstyle m+\y\)};
		\draw[ line width=1, color=OrangeProfondIRA, arrows={-Stealth[length=4, inset=0]}] (\y+\m,0) -- (\y+\m,1);
		}
}
\foreach \y in  {-3,...,0} {
	\draw (\y+\n,0) -- (\y+\n,-0.1);
	\ifthenelse{\y<0 }{\node[below] at (\y+\n,-0.1) {\tiny \(\scriptstyle m+n \y\)};}{}
	\ifthenelse{\y=0 }{\node[below] at (\y+\n,-0.1) {\tiny \(\scriptstyle m+n \)};}{}
	\ifthenelse{\y>0}{	\node[below] at (\y+\n,-0.1) {\tiny \(\scriptstyle m+n+\y\)};}{}
	\ifthenelse{\y<1 }{	\draw[ line width=1, color=OrangeProfondIRA, arrows={-Stealth[length=4, inset=0]}] (\y+\n,0) -- (\y+\n,1);}
}
\draw[ line width=1] (-.25,0) -- (0.5,0);
\draw[ line width=1, dashed] (0.5,0) -- (1.5,0);
\draw[ line width=1] (1.5,0) -- (4,0);
\draw[ line width=1, dashed] (4,0) -- (5,0);
		\draw[arrows={-Stealth[length=4, inset=0]}, line width=1] (5,0) -- (\w,0);
	\end{tikzpicture}	
	
\end{center}


Periodic / monthly annuity \(a^{(m)}\) : the default periodicity is one year, but the unit payment can also be spread over \(m\) periods within the year.

\begin{center}
	\begin{tikzpicture}[scale=0.75]
		% Draw the x-axis and y-axis.
		\def\w{10}
		\def\n{9}
		\def\m{6}
		\node[left] at (-.5,0) {\(a_{\lcroof{n}}^{(m)}\)};
		\foreach \y in  {0,...,3} {
			\draw (\y,0) -- (\y,-0.1);
			\ifthenelse{\y>0 }{	\node[below] at (\y,-0.1) {\tiny \( \scriptstyle \y\)};}{
				\node[below] at (\y,-0.1) {\tiny 0};}
		}
		\pgfmathparse{3.5*\m} 
		\foreach \y in  {1,...,\pgfmathresult} {
			\ifthenelse{\y>0 }{	
				\draw[ line width=1, color=OrangeProfondIRA, arrows={-Stealth[length=4, inset=0]}] (\y/\m,0) -- (\y/\m,2/\m);}{}
		}
		\foreach \y in  {-3,...,0} {
			\draw (\y+\n,0) -- (\y+\n,-0.1);
			\ifthenelse{\y<0 }{\node[below] at (\y+\n,-0.1) {\tiny \(\scriptstyle n \y\)};}{}
			\ifthenelse{\y=0 }{\node[below] at (\y+\n,-0.1) {\tiny \(\scriptstyle n \)};}{}
			\ifthenelse{\y>0}{	\node[below] at (\y+\n,-0.1) {\tiny \(\scriptstyle n+\y\)};}{}
		}
		\pgfmathparse{3.5*\m} 
		\foreach \y  in  {1,...,\pgfmathresult} {
		\draw[ line width=1, color=OrangeProfondIRA, arrows={-Stealth[length=4, inset=0]}] (-\y/\m+\n,0) -- (-\y/\m+\n,2/\m);
		}
		\draw[ line width=1] (-.25,0) -- (3.5,0);
		\draw[ line width=1, dashed] (3.5,0) -- (5.5,0);
		\draw[arrows={-Stealth[length=4, inset=0]}, line width=1] (5.5,0) -- (\w,0);
	\end{tikzpicture}
\end{center}

If \(i^{(m)}\) represents the equivalent nominal (annual) interest rate with \(m\) periods per year, then \(i^{(m)}= m\left((1+i)^{1 / m}-1\right)\) .

Similarly, \(d^{(m)} \) is the nominal discount rate consistent with \(d\) and \(m\) : \(d^{(m)}= m\left(1-(1-d)^{1 / m}\right)\).

\[
\ddot{a}_{\lcroof{n}}^{(m)}=\frac{1}{m} \sum_{k=0}^{m n-1} v^{\frac{k}{m}}=\frac{d}{d^{(m)}} \ddot{a}_{\lcroof{n}}=\frac{1-v^{n}}{d^{(m)}} \approx \ddot{a}_{\lcroof{n}}+\frac{m-1}{2 m}\left(1-v^{n}\right)
\]
\[
a_{\lcroof{n}}^{(m)}=\frac{1}{m} \sum_{k=1}^{m n} v^{\frac{k}{m}}=\frac{i}{i^{(m)}} a_{\lcroof{n}}=\frac{1-v^{n}}{i^{(m)}} \approx a_{\lcroof{n}}-\frac{m-1}{2 m}\left(1-v^{n}\right)
\]
Unit annuity \(a\): it is used when constructing annuity formulas.
For a constant annuity, the total amount paid each year is 1, regardless of \(m\). 

Dynamic annuity, increasing/decreasing \(Ia\)/\(Da\): in its simplest form, it pays an amount that starts at 1 (\(n\)) and increases (decreases) each period arithmetically or geometrically. In the following example, the progression is arithmetic.
The prefix \(I\) (increasing) is used to indicate increasing annuities and \(D\) (decreasing) for decreasing annuities.
	
\begin{center}
	\begin{tikzpicture}[scale=0.75]
		% Draw the x-axis and y-axis.
		\def\w{11}
		\def\n{9}
		\node[left] at (-.5,0) {\(Ia_{\lcroof{n}}\)};
		\foreach \y in  {0,...,3} {
			\draw (\y,0) -- (\y,-0.1);
			\ifthenelse{\y>0 }{	\node[below] at (\y,-0.1) {\tiny \( \scriptstyle \y\)};
				\draw[ line width=1, color=OrangeProfondIRA, arrows={-Stealth[length=4, inset=0]}] (\y,0) -- (\y,\y/3);}{
				\node[below] at (\y,-0.1) {\tiny 0};}
		}
		\foreach \y in  {-3,...,1} {
			\draw (\y+\n,0) -- (\y+\n,-0.1);
			\ifthenelse{\y<0 }{\node[below] at (\y+\n,-0.1) {\tiny \(\scriptstyle n \y\)};}{}
			\ifthenelse{\y=0 }{\node[below] at (\y+\n,-0.1) {\tiny \(\scriptstyle n \)};}{}
			\ifthenelse{\y>0}{	\node[below] at (\y+\n,-0.1) {\tiny \(\scriptstyle n+\y\)};}{}
			\ifthenelse{\y<1 }{	\draw[ line width=1, color=OrangeProfondIRA, arrows={-Stealth[length=4, inset=0]}] (\y+\n,0) -- (\y+\n,\y/3+\n/3);}
		}
		\draw[ line width=1] (-.25,0) -- (4,0);
		\draw[ line width=1, dashed] (4,0) -- (5,0);
		\draw[arrows={-Stealth[length=4, inset=0]}, line width=1] (5,0) -- (\w,0);
	\end{tikzpicture}
	\begin{tikzpicture}[scale=0.75]
	% Draw the x-axis and y-axis.
	\def\w{11}
	\def\n{9}
	\node[left] at (-.5,0) {\(Da_{\lcroof{n}}\)};
	\foreach \y in  {0,...,3} {
		\draw (\y,0) -- (\y,-0.1);
		\ifthenelse{\y>0 }{	\node[below] at (\y,-0.1) {\tiny \( \scriptstyle \y\)};
			\draw[ line width=1, color=OrangeProfondIRA, arrows={-Stealth[length=4, inset=0]}] (\y,0) -- (\y,\n/3-\y/3);}{
			\node[below] at (\y,-0.1) {\tiny 0};}
	}
	\foreach \y in  {-3,...,1} {
		\draw (\y+\n,0) -- (\y+\n,-0.1);
		\ifthenelse{\y<0 }{\node[below] at (\y+\n,-0.1) {\tiny \(\scriptstyle n \y\)};}{}
		\ifthenelse{\y=0 }{\node[below] at (\y+\n,-0.1) {\tiny \(\scriptstyle n \)};}{}
		\ifthenelse{\y>0}{	\node[below] at (\y+\n,-0.1) {\tiny \(\scriptstyle n+\y\)};}{}
		\ifthenelse{\y<1 }{	\draw[ line width=1, color=OrangeProfondIRA, arrows={-Stealth[length=4, inset=0]}] (\y+\n,0) -- (\y+\n,1/3-\y/3);}
	}
	\draw[ line width=1] (-.25,0) -- (4,0);
	\draw[ line width=1, dashed] (4,0) -- (5,0);
	\draw[arrows={-Stealth[length=4, inset=0]}, line width=1] (5,0) -- (\w,0);
\end{tikzpicture}
\end{center}

\begin{equation}
(I \ddot{a})_{\lcroof{n}}=1+2 v+\cdots+n v^{n-1}=\frac{1}{d}\left(\ddot{a}_{\lcroof{n}}-n v^{n}\right)
\label{Ian}
\end{equation}
with, we recall,  \(d=i/(1+i)\) and in arrears (immediate)
\[
(I a)_{\lcroof{n}}=v+2 v^{2}+\cdots+n v^{n}=\frac{1}{i}\left(\ddot{a}_{\lcroof{n}}-n v^{n}\right)
\]
\[
(D \ddot{a})_{\lcroof{n}}=n+(n-1) v+\cdots+v^{n-1}=\frac{1}{d}\left(n-a_{\lcroof{n}}\right)
\]
and in arrears :
\[
(D a)_{\lcroof{n}}=n v+(n-1) v^{2}+\cdots+v^{n}=\frac{1}{i}\left(n-a_{\lcroof{n}}\right)
\]




\end{f}
\hrule

\begin{f}[L'emprunt (Indivis)]
\ \newline

La principale propriété de l'emprunt est de considéré séparément les intérêts du remboursement (ou de l'amortissement).

Par un remboursement constant ou par annuité constante: la somme de l'amortissement et de l'intérêt à chaque période est constante. 

\begin{center}
\begin{tikzpicture}[scale=0.85]
% Draw the x-axis and y-axis.
\def\w{11}
\def\n{9}
\def\duree{23}
\def\a{16.713332269}
\def\v{1.031415^(-1)} 
\draw[ line width=1, color=BleuProfondIRA, arrows={-Stealth[length=4, inset=0]}] (0,0) -- (0,-2);
\foreach \y in  {0,...,3} {
	\draw (\y,0) -- (\y,-0.1);
	\ifthenelse{\y>0 }{	\node[below] at (\y,-0.1) {\tiny $ \scriptstyle \y$};
	\pgfmathpow{\v}{\duree-\y} 	
	\let\Interet\pgfmathresult
		\draw[ line width=1, color=BleuProfondIRA, arrows={-Stealth[length=4, inset=0]}] (\y,0) -- (\y,\Interet);
		\draw[ line width=1, color=OrangeProfondIRA, arrows={-Stealth[length=4, inset=0]}] (\y,\Interet) -- (\y,1);}{
		\node[below] at (\y,-0.1) {\tiny 0};}
}
\foreach \y in  {-3,...,1} {
	\pgfmathpow{\v}{-\y+1} 	
	\let\Interet\pgfmathresult
	\draw (\y+\n,0) -- (\y+\n,-0.1);
	\ifthenelse{\y<0 }{\node[below] at (\y+\n,-0.1) {\tiny $\scriptstyle n \y$};}{}
	\ifthenelse{\y=0 }{\node[below] at (\y+\n,-0.1) {\tiny $\scriptstyle n $};}{}
	\ifthenelse{\y>0}{	\node[below] at (\y+\n,-0.1) {\tiny $\scriptstyle n+\y$};}{}
	\ifthenelse{\y<1 }{	
		\draw[ line width=1, color=BleuProfondIRA, arrows={-Stealth[length=4, inset=0]}] (\y+\n,0) -- (\y+\n,\Interet);
		\draw[ line width=1, color=OrangeProfondIRA, arrows={-Stealth[length=4, inset=0]}] (\y+\n,\Interet) -- (\y+\n,1);}
}
\draw[ line width=1] (-.25,0) -- (4,0);
\draw[ line width=1, dashed] (4,0) -- (5,0);
\draw[arrows={-Stealth[length=4, inset=0]}, line width=1] (5,0) -- (\w,0);
\end{tikzpicture}
\end{center}
	
    
    Par un amortissement constant.
\begin{center}
\begin{tikzpicture}[scale=0.85]
% Draw the x-axis and y-axis.
\def\w{11}
\def\n{9}
\def\duree{23}
\def\i{0.031415}
\draw[ line width=1, color=BleuProfondIRA, arrows={-Stealth[length=4, inset=0]}] (0,0) -- (0,-2);
\foreach \y in  {0,...,3} {
\draw (\y,0) -- (\y,-0.1);
\ifthenelse{\y>0 }{	\node[below] at (\y,-0.1) {\tiny $ \scriptstyle \y$};
	\pgfmathparse{(\duree-\y)*\i} 	
	\let\Interet\pgfmathresult
	\draw[ line width=1, color=BleuProfondIRA, arrows={-Stealth[length=4, inset=0]}] (\y,0) -- (\y,1);
	\draw[ line width=1, color=OrangeProfondIRA, arrows={-Stealth[length=4, inset=0]}] (\y,1) -- (\y,1+\Interet);}{
	\node[below] at (\y,-0.1) {\tiny 0};}
}
\foreach \y in  {-3,...,1} {
	\pgfmathparse{(-\y+1)*\i} 	
\let\Interet\pgfmathresult
\draw (\y+\n,0) -- (\y+\n,-0.1);
\ifthenelse{\y<0 }{\node[below] at (\y+\n,-0.1) {\tiny $\scriptstyle n \y$};}{}
\ifthenelse{\y=0 }{\node[below] at (\y+\n,-0.1) {\tiny $\scriptstyle n $};}{}
\ifthenelse{\y>0}{	\node[below] at (\y+\n,-0.1) {\tiny $\scriptstyle n+\y$};}{}
\ifthenelse{\y<1 }{	
	\draw[ line width=1, color=BleuProfondIRA, arrows={-Stealth[length=4, inset=0]}] (\y+\n,0) -- (\y+\n,1);
\draw[ line width=1, color=OrangeProfondIRA, arrows={-Stealth[length=4, inset=0]}] (\y+\n,1) -- (\y+\n,1+\Interet);}{}
}
\draw[ line width=1] (-.25,0) -- (4,0);
\draw[ line width=1, dashed] (4,0) -- (5,0);
\draw[arrows={-Stealth[length=4, inset=0]}, line width=1] (5,0) -- (\w,0);
\end{tikzpicture}
\end{center}

    
Par un remboursement in fine, où l'intérêt est constant. Seuls les intérêts sont versés périodiquement jusqu'au terme, moment où le remboursement total est effectué.
	
\begin{center}
\begin{tikzpicture}[scale=0.85]
% Draw the x-axis and y-axis.
\def\w{11}
\def\n{9}
\def\duree{23}
\def\i{3*0.031415}
\draw[ line width=1, color=BleuProfondIRA, arrows={-Stealth[length=4, inset=0]}] (0,0) -- (0,-2);
\foreach \y in  {0,...,3} {
	\draw (\y,0) -- (\y,-0.1);
	\ifthenelse{\y>0 }{	\node[below] at (\y,-0.1) {\tiny $ \scriptstyle \y$};
%				\draw[ line width=1, color=BleuProfondIRA, arrows={-Stealth[length=4, inset=0]}] (\y,0) -- (\y,1);
		\draw[ line width=1, color=OrangeProfondIRA, arrows={-Stealth[length=4, inset=0]}] (\y,0) -- (\y,2*\i);}{
		\node[below] at (\y,-0.1) {\tiny 0};}
}
\foreach \y in  {-3,...,1} {
	\draw (\y+\n,0) -- (\y+\n,-0.1);
	\ifthenelse{\y<0 }{\node[below] at (\y+\n,-0.1) {\tiny $\scriptstyle n \y$};
		\draw[ line width=1, color=OrangeProfondIRA, arrows={-Stealth[length=4, inset=0]}] (\y+\n,0) -- (\y+\n,2*\i);}{}
	\ifthenelse{\y=0 }{
		\draw[ line width=1, color=BleuProfondIRA, arrows={-Stealth[length=4, inset=0]}] (\y+\n,0) -- (\y+\n,2);
		\draw[ line width=1, color=OrangeProfondIRA, arrows={-Stealth[length=4, inset=0]}] (\y+\n,2) -- (\y+\n,2+2*\i);
		\node[below] at (\y+\n,-0.1) {\tiny $\scriptstyle n $};}{}
	\ifthenelse{\y>0}{	\node[below] at (\y+\n,-0.1) {\tiny $\scriptstyle n+\y$};}{}
}
\draw[ line width=1] (-.25,0) -- (4,0);
\draw[ line width=1, dashed] (4,0) -- (5,0);
\draw[arrows={-Stealth[length=4, inset=0]}, line width=1] (5,0) -- (\w,0);
\end{tikzpicture}
\end{center}

\end{f}
\hrule \

\begin{f} [Tableau d'amortissement de l'emprunt]
\ \newline

	\footnotesize
\renewcommand{\arraystretch}{2}
\begin{tabular}{|m{10ex}|m{19ex}|m{19ex}|m{17ex}|}
\rowcolor{BleuProfondIRA!40}   	\hline &\textbf{In fine}			&   	\textbf{Amortissements  constants} &  		\textbf{Annuités  constantes} \\
	\hline Capital restant dû $S_k$ & $T_k=S_0$, $T_n=0$& $S_{0}\left(1-\frac{k}{n}\right)$ & $S_{0} \frac{1-v^{n-k}}{1-v^{n}}$\\
	\hline Intérêts $U_k$ 		& $i\times S_0$ & $S_{0}\left(1-\frac{k-1}{n}\right) i$ & $K\left(1-v^{n-k+1}\right)$ \\
	\hline Amortis\-sements $T_k$ &	$T_k=O$, $T_n=S_0$ & $\frac{S_{0}}{n}$ & $K v^{n-k+1}$ \\
	\hline Annuité $K_k$ & 	$K_k=i S_0$, $K_n=(1+i)S_0$ 	 & $\frac{S_{0}}{n}(1-(n-k+1) i) $ & $K=S_{0} \frac{i}{1-v^{n}} $ \\
%	\hline Coût de l'emprunt & $1 \times K \times N$ & $1 \times K \times \frac{N+1}{2}$ & $K\left(\frac{N i}{1-(1+i)^{-N}}-1\right)$ \\
	\hline
\end{tabular}

\end{f}

\end{multicols}

\newpage
\begin{center}
\section*{Market Finance}
    \medskip
\end{center}

\begin{multicols}{2}

% !TeX root = ActuarialFormSheet_MBFA-en.tex
% !TeX spellcheck = en_GB
\def\scaleBS{.95}

\begin{f}[Market Functioning]
	
The \textbf{Exchange} – a place of exchange – enables, in fact, the physical meeting between capital demanders and suppliers.
The main listings concern \textbf{equities}, \textbf{bonds} (\engl{Fixed Income}), and \textbf{commodities}.
Listed are \textbf{securities} such as stocks or bonds, \textbf{funds} (\engl{Exchange Traded Funds} that replicate equity indices, ETC or ETN that replicate more specific indices or commodities, SICAV or FCP, subscription warrants), \textbf{futures contracts}, \textbf{options}, \textbf{swaps}, and \textbf{structured products}.

The \textbf{Financial Markets Authority} (\href{https://www.amf-france.org/fr}{AMF}) oversees:
\begin{itemize}
	\item the protection of invested savings; 
	\item the information of investors; 
	\item the proper functioning of the markets.
\end{itemize}

\textbf{\href{https://www.euronext.com/fr}{Euronext}} (including
\href{https://www.euronext.com/en/markets/amsterdam}{Amsterdam}, 
\href{https://www.euronext.com/en/markets/brussels}{Brussels}, 
\href{https://www.euronext.com/en/markets/lisbon}{Lisbon}, 
and \href{https://www.euronext.com/en/markets/paris}{Paris}) is the main stock exchange in France.
%
Its competitors include \textbf{\href{https://www.deutsche-boerse.com/dbg-en/}{Deutsche Börse}} 
(which includes \href{https://www.eurex.com/}{Eurex}, 
\href{https://www.eex.com/en/}{EEX}) in Europe,
%
or \textbf{\href{https://www.ice.com/}{ICE}} (which includes
\href{https://www.nyse.com/index}{NYSE (2012)}, 
\href{https://www.ice.com/about/history}{NYBOT (2005)}, 
\href{https://www.ice.com/futures-europe}{IPE (2001), LIFFE})
%
and \textbf{\href{https://www.cmegroup.com/}{CME Group}} (including
\href{https://www.cmegroup.com/company/cbot.html}{CBOT}, 
\href{https://www.cmegroup.com/company/nymex.html}{NYMEX}, 
\href{https://www.cmegroup.com/company/comex.html}{COMEX}) in the United States.

The \textbf{over-the-counter market} (\textbf{OTC}) represents a major share of volumes traded outside organized markets.
Since the Pittsburgh G20 (2009), certain standardized OTC derivatives must be cleared through a central entity.
These \textbf{CCPs} (\textit{Central Counterparties}) thus play the role of \textbf{clearinghouses}: they replace the bilateral contract with two contracts between each party and the CCP.

\begin{tikzpicture}[scale=.47]
	\node[businessman,minimum size=1.75cm,monogramtext=MPG,tie=OrangeProfondIRA, shirt=BleuProfondIRA,skin=white,hair=OrangeProfondIRA] at (-8,0) (MPG){Short position};
	\node[businessman,mirrored,minimum size=1.75cm,monogramtext=SPG,shirt=FushiaIRA, tie=black,skin=FushiaIRA, hair=black] at (8,0) (SPG){Long position};
	%  	
	\begin{scope}[yscale=-0.03, xscale=0.03, shift={(-150,-200)}]
		%  		
		%Usine
		\draw [OrangeMoyenIRA!40,fill] (65,132) rectangle (195,225)  node [pos=0.5,text width=1.8cm, align=center] {\color{black}\bfseries Clearinghouse} ;
		%Straight Lines [id:da13081779728777443] 
		%	\draw (65,272) -- (474,272) -- (474,320);
		%Shape: Polygon [id:ds025531371089408395] 
		\draw   [GrisLogoIRA] (195,272) -- (65,272) -- (65,129) -- (85,129) -- (85,20) -- (117,20)  -- (117,52)-- (175,52) -- (175,129) -- (195,129) -- cycle ;
		%Shape: Grid [id:dp09099790819019526] 
		%	\draw  [draw opacity=0] (117,129) -- (195,129) -- (195,169) -- (117,169) -- cycle ; 
		\draw  [GrisLogoIRA] (95,100) -- (95,60)(105,100) -- (105,60)(115,100) -- (115,60)(125,100) -- (125,60)(135,100) -- (135,60)(145,100) -- (145,60)(155,100) -- (155,60)(165,100) -- (165,60) ; 
		\draw  [GrisLogoIRA]  (85,100) -- (175,100)(85,90) -- (175,90)(85,80) -- (175,80)(85,70) -- (175,70)(85,60) -- (175,60) ; 
		%
		\draw  (65,175) node (a) {} ;
		\draw  (195,175) node (b) {} ;
		%Shape: Rectangle [id:dp30583722611654374] 
	\end{scope}
	% 	
	\draw[->,>=latex, thick] (MPG) to [bend left]  node[pos=0.4, above]{Short} (a);
	\draw[<-,>=latex, thick,OrangeProfondIRA] (MPG) to [bend right]  node[pos=0.4, below=.2]{payment} (a);
	\draw[->,>=latex, thick,FushiaIRA] (SPG) to [bend left] node[pos=0.4, below=.2]{No payment}  (b);
	\draw[->,>=latex, thick] (SPG) to [bend right] node[pos=0.4, above]{Long} (b);
	%  	
\end{tikzpicture}
\medskip

\end{f}
\hrule

\begin{f}[The Money Market]
Short-term interest-bearing securities, traded on money markets, are generally at \textbf{discounted interest}. 
Nominal rates are then annual and calculations use \textbf{proportional rates} to adjust for durations less than one year.
These securities are quoted or valued according to the discount principle and with a Euro-30/360 calendar convention.

In the American market, public debt securities are called:
Treasury Bills (T-bills): ZC < 1 year, Treasury Notes (T-notes): ZC < 10 years,
Treasury Bonds (T-bonds): coupon bonds with maturity > 10 years.


They are mainly:
\begin{itemize}
	\item \textbf{BTF (fixed-rate Treasury bills, France):} issued at 13, 26, 52 weeks, discounted rate, weekly auction, nominal 1~€, settlement at T+2.
	
	\item \textbf{Treasury bills > 1 year:} same rules as bonds (see next section).
	
	\item \textbf{Certificates of deposit (CDN):} securities issued by banks at fixed/discounted rate (short term) or variable/post-discounted rate (long term), also called BMTN.
	
	\item \textbf{Eurodollars:} USD deposits outside the USA, formerly indexed on LIBOR, now declining.
	
	\item \textbf{Commercial paper:} unsecured short-term securities issued by large companies to finance their cash flow.
\end{itemize}

\textbf{Price calculations of a fixed-rate Treasury bill with discounted interest}

In the case of a discounted interest security according to the Euro-30/360 convention, the discount \(D\) is expressed as :
\[
D=F \cdot d \cdot \frac{k}{360}
\]
where \(F\) denotes the nominal value, \(d\) the annual discount rate used to value the discounted security, and \(k\) the maturity in days.

If the discount rate \(d\) is known, then the price \(P\) is expressed as :
\[
P=F-D=F\left(1-d \cdot \frac{k}{360}\right)
\]
Similarly, if the price \(P\) is known, then the discount rate \(d\) is derived as :
\[
d=\frac{F-P}{F} \cdot \frac{360}{k}
\]


The main Futures Contracts: Federal Funds Futures (US),
Three-Month SOFR Futures (US),
ESTR Futures (EU),
SONIA Futures (UK),
Euribor Futures (EU).

\end{f}
\hrule



\begin{f}[Bond Market]

\textbf{Bonds} are long-term debt securities in which the issuer (central or local government, bank, borrowing company) promises the bondholder (the lender) to pay interest (\textbf{coupons}) periodically and to repay the \textbf{nominal value} (or face value, or principal) at maturity.
As mentioned in the previous section, \textbf{Treasury bills with a maturity greater than one year} will be treated as bonds with maturities under 5 years because their functioning is similar.

%Les \textbf{coupons} : sont payés régulièrement à la fin des périodes de coupon (annuelles ou semestrielles) jusqu'à la date d'échéance.

\textbf{Zero-coupon bonds}: pay only the nominal value at maturity. With \(E\) the issue price and \(R\) its redemption value:
	
\begin{center}
\begin{tikzpicture}[scale=0.85]
    % Draw the x-axis and y-axis.
    \def\w{11}
    \def\n{9}
    \draw[ line width=1, color=OrangeProfondIRA, arrows={-Stealth[length=4, inset=0]}] (0,0) -- (0,-0.4909) node[above left] (A) {\(E\)};	
    \foreach \y in  {0,...,3} {
        \draw (\y,0) -- (\y,-0.1);
        \ifthenelse{\y>0 }{	\node[below] at (\y,-0.1) {\tiny \( \scriptstyle \y\)};
        }{
            \node[above] at (\y,-0.1) {\tiny 0};}
    }
    \foreach \y in  {-3,...,1} {
        \draw (\y+\n,0) -- (\y+\n,-0.1);
        \ifthenelse{\y<0 }{\node[below] at (\y+\n,-0.1) {\tiny \(\scriptstyle n \y\)};}{}
        \ifthenelse{\y=0 }{\node[below] at (\y+\n,-0.1) {\tiny \(\scriptstyle n \)};
            \draw[ line width=1, color=OrangeProfondIRA, arrows={-Stealth[length=4, inset=0]}] (\y+\n,0) -- (\y+\n,1) node[below right] (B) {\(R\)};}{}
        \ifthenelse{\y>0}{	\node[below] at (\y+\n,-0.1) {\tiny \(\scriptstyle n+\y\)};}{}
    }
    \draw[ line width=1] (-.25,0) -- (4,0);
    \draw[ line width=1, dashed] (4,0) -- (5,0);
    \draw[arrows={-Stealth[length=4, inset=0]}, line width=1] (5,0) -- (\w,0);
\end{tikzpicture}
\end{center}

\textbf{Coupon bonds}:
\textbf{Fixed-rate bonds} have a coupon rate that remains constant until maturity. Assuming a \emph{bullet} repayment, with \(E\) the issue price, \(c\) the coupons, and \(R\) the redemption value, it can be illustrated as follows:
\begin{center}
\begin{tikzpicture}[scale=0.85]
    % Draw the x-axis and y-axis.
    \def\w{11}
    \def\n{9}
    \draw[ line width=1, color=OrangeProfondIRA, arrows={-Stealth[length=4, inset=0]}] (0,0) -- (0,-1.8) node[above left] (A) {\(E\)};	
    \foreach \y in  {0,...,3} {
        \draw (\y,0) -- (\y,-0.1);
        \ifthenelse{\y>0 }{	\node[below] at (\y,-0.1) {\tiny \( \scriptstyle \y\)};
        \draw[ line width=1, color=OrangeProfondIRA, arrows={-Stealth[length=4, inset=0]}] (\y,0) -- (\y,.25) node [above] {\(c\)};
        }{
            \node[above] at (\y,-0.1) {\tiny 0};}
        }
    \foreach \y in  {-3,...,1} {
        \draw (\y+\n,0) -- (\y+\n,-0.1);
        \ifthenelse{\y<0 }{\node[below] at (\y+\n,-0.1) {\tiny \(\scriptstyle n \y\)};
        \draw[ line width=1, color=OrangeProfondIRA, arrows={-Stealth[length=4, inset=0]}] (\y+\n,0) -- (\y+\n,.25) node [above] {\(c\)};}{}
        \ifthenelse{\y=0 }{\node[below] at (\y+\n,-0.1) {\tiny \(\scriptstyle n \)};
        \draw[ line width=1, color=OrangeProfondIRA, arrows={-Stealth[length=4, inset=0]}] (\y+\n,0) -- (\y+\n,2) node[below right] (B) {\(R\)};
        \draw[ line width=1, color=OrangeProfondIRA, arrows={-Stealth[length=4, inset=0]}] (\y+\n,2) -- (\y+\n,2.25) node[right] (B) {\(c\)};}{}
        \ifthenelse{\y>0}{	\node[below] at (\y+\n,-0.1) {\tiny \(\scriptstyle n+\y\)};}{}
    }
    \draw[ line width=1] (-.25,0) -- (4,0);
    \draw[ line width=1, dashed] (4,0) -- (5,0);
    \draw[arrows={-Stealth[length=4, inset=0]}, line width=1] (5,0) -- (\w,0);
\end{tikzpicture}
\end{center}

 \textbf{Indexed bonds} (inflation-linked bonds) have coupons and sometimes also the nominal value indexed to inflation or another economic indicator, such as Treasury Assimilable Bonds
		indexed to inflation (OATi). The values of \(c\) vary.
        
Bonds with \textbf{floating rate}, \textbf{variable rate}, or \textbf{resettable rate}: have a coupon rate linked to a reference interest rate (for example, the euro short-
		term rate (\href{https://www.cmegroup.com/markets/interest-rates/stirs/euro-short-term-rate.quotes.html}{€STR})).

\textbf{Perpetual bonds} have no maturity date; the principal is never repaid.
 
A distinction is often made between government bonds (Treasury bonds) and
 corporate bonds issued by private companies.



A bond is mainly defined by a \textbf{nominal value} \(F\) (Face Value), the \textbf{nominal rate} \(i\), its duration or \textbf{maturity} \(n\).
In the default case, the bondholder lends the amount \(E=F\) at issuance at time 0, receives each year a coupon \(c = i \times F\), and at \(n\), the principal or capital \(R=F\) is returned.
When \(E=F\), the issue is said to be at par, and when \(R=F\), the redemption is said to be at par.


The price of a bond is determined by the present value of the expected future cash flows (coupons and principal repayment) discounted at the market yield rate \(r\).
 
The price calculation of bonds simply relies on the present value formula:
\[
VP = \sum_{k=1}^{n} \frac{c}{(1 + r)^k} + \frac{F}{(1 + r)^n}
 \]
where:
\begin{itemize}
    \item \(PV\): price or present value\index{Present Value} of the bond,
	\item \(r\): market interest rate for the relevant maturity.
\end{itemize}


For bonds with periodic coupons, the coupon is divided by the number of periods (\(m\)) per year and the formula becomes:
\[ 
PV = \sum_{k=1}^{mn} \frac{c/m}{(1 + r^{(m)})^k} + \frac{R}{(1 + + r^{(m)})^{mn}}
 \]
where \(c/m\) represents the periodic coupon payment and \(r^{(m)}\) the periodic interest rate.

The bond yield is the value \(r^{(m)}\), the equivalent rate of \(r\) over \(m\) periods in the year, which equates the present value \(VP_r\) with the current or market price of this bond.

\textbf{The quotation of a bond} is given as a percentage. Thus, a quotation of 97.9 on Euronext indicates a quoted value of \(97.9 / 100 \times F\).  
It is quoted excluding \textbf{accrued coupons}, the portion of the next coupon to which the seller is entitled if the bond is sold before the payment of that coupon.
\end{f}
\hrule

\begin{f}[Duration \& Convexity]

The Macaulay duration:
\[ 		
D = \sum_{t=1}^{n} t \cdot w_t, \quad \text{où} \quad w_t = \frac{PV(C_t)}{P}.
 \]	
If the payment frequency is \(k\) per year, the duration expressed in years is obtained by dividing by \(k\).
The modified duration \(D^*\):
\[ 	
D^* = \frac{D}{1 + i}.
 \]
 Which allows approximating the portfolio change \(\Delta P\) in case of interest rate changes \(\Delta_i\)
\[ 
\Delta P \approx -P\ D^* \Delta_i 
 \]
Similarly, the convexity
\[ 	
C = \frac{1}{P(i)} \times \frac{d^2 P(i)}{di^2},
 \]
which allows refining the approximation of \(\Delta P\)
\[ 	
P(i + \Delta_i) \approx P(i) \left( 1 -D^*\Delta_i + \frac{1}{2} C (\Delta_i)^2 \right).
 \]
\end{f}
\hrule



\begin{f}[CAPM]
  Capital Asset Pricing Model :

\[
E(r_i) = r_f + \beta_i (E(r_m) - r_f)
\]

\begin{itemize}
    \item \( E(r_i) \) is the expected return of asset \( i \),
    \item \( r_f \) is the risk-free rate,
    \item \( E(r_m) \) is the expected market return,
    \item \( \beta_i \) is the sensitivity coefficient of asset \( i \) with respect to market variations.
\end{itemize}

The coefficient \( \beta_i \) measures the volatility of asset \( i \) relative to the overall market.

\end{f}
\hrule


\begin{f}[Derivatives Market]

A \textbf{derivative contract} (or contingent asset) is a financial instrument whose value depends on an underlying asset or variable. Options are part of derivative contracts.


An \textbf{option} is a contract that gives the right (without obligation) to buy (call) or sell (put) an underlying asset at a fixed price (strike price) at a future date, in exchange for the payment of a premium.
The buyer (long position) pays the premium; the seller (short position) receives it. \textbf{European option} (exercise possible only at maturity) and  
 \textbf{American option} (exercise possible at any time until maturity).

Options listed on stocks are called \textit{stock options}.

\end{f}
\hrule


\begin{f}[Simple Strategies]

\ %

%\textbf{La position longue sur l'option d'achat}

With \(T\) the maturity, \(K\) the strike price, \(S\) or \(S_T\) the underlying at maturity, the payoff is \(\max (0, S_T-K)=( S_T-K)^{+}\).
Letting \(C\) be the premium, the profit realized is \(\max (0, S_T-K)-C\), with a profit if (\(S_T<V_{PM}=K + C\)) (\(PM\) stands for \textbf{break-even point}).

		\begin{tikzpicture}[yscale=.75]
\def\riskfreeBS{0.05}
\def\xminBS{5}
\def\xmaxBS{15}
\def\PxExerciceBS{10}
\def\sigmaBS{0.2}
\def\TBS{0.75}
\def\PremiumBS{{BSCall(10,{\PxExerciceBS},{\riskfreeBS},{\TBS},{\sigmaBS})}}
\begin{axis}[ 	extra tick style={tick style=BleuProfondIRA},
clip=false,
axis on top,
axis lines=middle, axis line style={BleuProfondIRA,thick,->},
scale only axis, xmin={\xminBS},xmax={\xmaxBS},enlarge x limits=0.05,
enlarge y limits=0.08,
color=BleuProfondIRA,
%		ylabel near ticks,
ylabel={Profit},
x label style={at={(axis cs:\xmaxBS+.1,0)},anchor=north east},
xlabel={underlying (\(T\))},
%		    x label style={at={(axis description cs:0.5,-0.1)},anchor=north},
%		y label style={at={(axis description cs:-0.1,.5)},rotate=90,anchor=south},
ytick=\empty,
xtick=\empty,
extra y ticks ={0},
extra y tick labels={{0}},
extra x ticks ={\PxExerciceBS},
extra x tick labels={{\(E\)}},
extra x tick style={color=BleuProfondIRA,
	tick label style={yshift=7mm}	},
title ={ \textbf{Long Call}},
%			title style={yshift=-5mm},
%	legend style={draw=none,
	%		legend columns=-1,
	%		at={(0.5,1)},
	%		anchor=south,
	%		outer sep=1em,
	%		node font=\small,
	%	},
]
%		
\addplot[name path=A,BleuProfondIRA,thick,domain={{\xminBS}:{\xmaxBS}}, samples=21,dashdotted] {Call(x,\PxExerciceBS,\PremiumBS)} 
node [pos=0.15,yshift=3mm,color=OrangeProfondIRA] {Losses}
node [pos=.8,yshift=-15mm,color=BleuProfondIRA] {Earnings};	
%\draw[BleuProfondIRA,thick] \OV{\PxExerciceBS}{\PremiumBS}{0.4}{\xminBS}{\xmaxBS} ;
%		\addplot[name path=Option,BleuProfondIRA,thick,domain={250:350}, 			 		samples=10,dashdotted,smooth] {BSPut(x,\PxExerciceBS,\riskfreeBS,\TBS,\sigmaBS)} ;	
\path [save path=\xaxis,name path=xaxis]
({\xminBS},0)		-- ({\xmaxBS},0)		;
\addplot [bottom color=OrangeProfondIRA!50, top color=OrangeProfondIRA!10] fill between [
of=A and xaxis,
split,
every segment no 1/.style=
{top color = BleuProfondIRA!50, bottom color=BleuProfondIRA!10}] ;
%\draw[use path=\xaxis, ->,OrangeProfondIRA,thick];
\draw [fill]  ({-\PremiumBS*(-1)+\PxExerciceBS},0) circle (1mm) node [above=5mm] {\(S_{PM}\)};
\draw[BleuProfondIRA, thin] ({\PxExerciceBS},0) -- ({\PxExerciceBS},{-\PremiumBS});
\node at ({\xminBS+0.25*(\xmaxBS-\xminBS)},{\pgfkeysvalueof{/pgfplots/ymin}})
{OUT};
\node at ({\PxExerciceBS},{\pgfkeysvalueof{/pgfplots/ymin}})
{AT};
\node at ({\xminBS+0.8*(\xmaxBS-\xminBS)},{\pgfkeysvalueof{/pgfplots/ymin}})
{IN};
\draw [<->, xshift=-5mm] ({\pgfkeysvalueof{/pgfplots/xmin}},0) -- ({\pgfkeysvalueof{/pgfplots/xmin}},-\PremiumBS) node [pos=0.5, xshift=-2.5mm, rotate=90] {Premium};
\end{axis}
%
\end{tikzpicture}


\medskip

%\textbf{La position courte sur l'option d'achat}


At maturity, the payoff is \(\min (0,K- S_T)=-\max(0, S_T-K)=-( S_T-K)^{+}\) and the profit realized is \(C-\max (0, S_T-K)\).

\medskip

			\begin{tikzpicture}[yscale=.75]
		\def\riskfreeBS{0.05}
		\def\xminBS{5}
		\def\xmaxBS{15}
		\def\PxExerciceBS{10}
		\def\sigmaBS{0.2}
		\def\TBS{0.75}
		\def\PremiumBS{{BSCall(10,{\PxExerciceBS},{\riskfreeBS},{\TBS},{\sigmaBS})}}
\begin{axis}[ 
	clip=false,
	axis on top,
	axis lines=middle, axis line style={BleuProfondIRA,thick,->},
	scale only axis, xmin={\xminBS},xmax={\xmaxBS},enlarge x limits=0.05,
	enlarge y limits=0.125,
	color=BleuProfondIRA,
	%		ylabel near ticks,
	ylabel={Profit},
	x label style={={at={(current axis.right of origin)}}},
	%    x label style={at={(axis description cs:1,-0.1)},anchor=south},
	%		x label style={at={(1,0.5)}},
	xlabel={ss-jacent ($T$)},
	%		    x label style={at={(axis description cs:0.5,-0.1)},anchor=north},
	%		y label style={at={(axis description cs:-0.1,.5)},rotate=90,anchor=south},
	ytick=\empty,
	xtick=\empty,
	extra y ticks ={0},
	extra y tick labels={{0}},
	extra x ticks ={\PxExerciceBS},
	extra x tick labels={{$E$}},
	extra x tick style={color=BleuProfondIRA,
		tick label style={yshift=-0mm}	},
	title ={ \textbf{Short Call}},
	title style={yshift=10mm},
	%	legend style={draw=none,
		%		legend columns=-1,
		%		at={(0.5,1)},
		%		anchor=south,
		%		outer sep=1em,
		%		node font=\small,
		%	},
	]
	%		
\addplot[name path=A,BleuProfondIRA,thick,domain={{\xminBS}:{\xmaxBS}}, samples=21,dashdotted] {-Call(x,\PxExerciceBS,\PremiumBS)} 
node [pos=0.15,yshift=-4mm,color=BleuProfondIRA] {Gains}
node [pos=.8,yshift=10mm,color=OrangeProfondIRA] {Pertes};	
	%\draw[BleuProfondIRA,thick] \OV{\PxExerciceBS}{\PremiumBS}{0.4}{\xminBS}{\xmaxBS} ;
	%		\addplot[name path=Option,BleuProfondIRA,thick,domain={250:350}, 			 		samples=10,dashdotted,smooth] {BSPut(x,\PxExerciceBS,\riskfreeBS,\TBS,\sigmaBS)} ;	
	\path [save path=\xaxis,name path=xaxis]
	({\xminBS},0)		-- ({\xmaxBS},0)		;
	\addplot [bottom color=OrangeProfondIRA!50, top color=OrangeProfondIRA!10] fill between [
	of=A and xaxis,
	split,
	every segment no 0/.style=
	{top color = BleuProfondIRA!50, bottom color=BleuProfondIRA!10}] ;
	%\draw[use path=\xaxis, ->,OrangeProfondIRA,thick];
	\draw [fill]  ({-\PremiumBS*(-1)+\PxExerciceBS},0) circle (1mm) node [above=5mm] {$S_{PM}$};
	\draw[BleuProfondIRA, thin] ({\PxExerciceBS},0) -- ({\PxExerciceBS},{\PremiumBS});
	\node at ({\xminBS+0.25*(\xmaxBS-\xminBS)},{\pgfkeysvalueof{/pgfplots/ymin}+1})
	{OUT};
	\node at ({\PxExerciceBS},{\pgfkeysvalueof{/pgfplots/ymin}+1})
	{AT};
	\node at ({\xminBS+0.8*(\xmaxBS-\xminBS)},{\pgfkeysvalueof{/pgfplots/ymin}+01})
	{IN};
	\draw [<->, xshift=-5mm] ({\pgfkeysvalueof{/pgfplots/xmin}},0) -- ({\pgfkeysvalueof{/pgfplots/xmin}},\PremiumBS) node [pos=0.5, xshift=-2.5mm, rotate=90] {Prime};
\end{axis}
%
	\end{tikzpicture}

\medskip


%\textbf{La position longue sur l'option de vente}

At maturity, the payoff is \(\max (0,K- S_T)=(K- S_T)^{+}\).
Letting \(P\) be the put premium, the profit realized is \(\max (0,K- S_T)-P\), positive if \(V_{PM}=K -P<S_T\).

\medskip

		\begin{tikzpicture}[yscale=.75]
	\def\riskfreeBS{0.05}
	\def\xminBS{5}
	\def\xmaxBS{15}
	\def\PxExerciceBS{10}
	\def\sigmaBS{0.2}
	\def\TBS{0.75}
	\def\PremiumBS{{BSCall(10,{\PxExerciceBS},{\riskfreeBS},{\TBS},{\sigmaBS})}}
	\begin{axis}[ 
		clip=false,
		axis on top,
		axis lines=middle, axis line style={BleuProfondIRA,thick,->},
		scale only axis, xmin={\xminBS},xmax={\xmaxBS},enlarge x limits=0.05,
		enlarge y limits=0.1,
		color=BleuProfondIRA,
		%		ylabel near ticks,
		ylabel={Profit},
		x label style={={at={(current axis.right of origin)},below=5mm}},
		%    x label style={at={(axis description cs:1,-0.1)},anchor=south},
		%		x label style={at={(1,0.5)}},
		xlabel={underlying ($T$)},
		%		    x label style={at={(axis description cs:0.5,-0.1)},anchor=north},
		%		y label style={at={(axis description cs:-0.1,.5)},rotate=90,anchor=south},
		ytick=\empty,
		xtick=\empty,
		extra y ticks ={0},
		extra y tick labels={{0}},
		extra x ticks ={\PxExerciceBS},
		extra x tick labels={{$E$}},
		extra x tick style={color=BleuProfondIRA,
			tick label style={yshift=7mm}	},
		title ={ \textbf{Long Put}},
%		title style={yshift=0mm},
		%	legend style={draw=none,
			%		legend columns=-1,
			%		at={(0.5,1)},
			%		anchor=south,
			%		outer sep=1em,
			%		node font=\small,
			%	},
		]		%		
		\addplot[name path=A,BleuProfondIRA,thick,domain={{\xminBS}:{\xmaxBS}}, samples=21,dashdotted] {Put(x,\PxExerciceBS,\PremiumBS)} 
		node [pos=0.15,yshift=-15mm,color=BleuProfondIRA] {Earnings}
		node [pos=.75,yshift=4mm,color=OrangeProfondIRA] {Losses};	
		%\draw[BleuProfondIRA,thick] \OV{\PxExerciceBS}{\PremiumBS}{0.4}{\xminBS}{\xmaxBS} ;
		%	\addplot[BleuProfondIRA,thick,dashdotted] {Call(x,50,2)};
		%	\addplot[BleuProfondIRA,thick] {Put(x,50,3)+Call(x,50,2)};
		%\addplot[name path=Option,BleuProfondIRA,thick,domain={250:350}, 			 		samples=10,dashdotted,smooth] {BSPut(x,\PxExerciceBS,\riskfreeBS,\TBS,\sigmaBS)} ;	
		\path [save path=\xaxis,name path=xaxis]
		({\xminBS},0)		-- ({\xmaxBS},0)		;
		\addplot [bottom color=OrangeProfondIRA!50, top color=OrangeProfondIRA!10] fill between [
		of=A and xaxis,
		split,
		every segment no 0/.style=
		{top color = BleuProfondIRA!50, bottom color=BleuProfondIRA!10}] ;
		%\draw[use path=\xaxis, ->,OrangeProfondIRA,thick];
		\draw [fill]  ({-\PremiumBS+\PxExerciceBS},0) circle (1mm) node [above=5mm] {$S_{PM}$};
		\draw[BleuProfondIRA, thin] ({\PxExerciceBS},0) -- ({\PxExerciceBS},{-\PremiumBS});
		\node at ({\xminBS+0.25*(\xmaxBS-\xminBS)},{\pgfkeysvalueof{/pgfplots/ymin}})
		{IN};
		\node at ({\PxExerciceBS},{\pgfkeysvalueof{/pgfplots/ymin}})
		{AT};
		\node at ({\xminBS+0.8*(\xmaxBS-\xminBS)},{\pgfkeysvalueof{/pgfplots/ymin}})
		{OUT};
		\draw [<->, xshift=-5mm] ({\pgfkeysvalueof{/pgfplots/xmin}},0) -- ({\pgfkeysvalueof{/pgfplots/xmin}},-\PremiumBS) node [pos=0.5, xshift=-2.5mm, rotate=90] {Premium};
	\end{axis}
	%
\end{tikzpicture}

    	     

%\textbf{La position courte sur l'option de vente}

\medskip

		\begin{tikzpicture}[yscale=.75]
	\def\riskfreeBS{0.05}
	\def\xminBS{5}
	\def\xmaxBS{15}
	\def\PxExerciceBS{10}
	\def\sigmaBS{0.2}
	\def\TBS{0.75}
	\def\PremiumBS{{BSCall(10,{\PxExerciceBS},{\riskfreeBS},{\TBS},{\sigmaBS})}}
	\begin{axis}[ 	extra tick style={tick style=BleuProfondIRA},
		clip=false,
		axis on top,
		axis lines=middle, axis line style={BleuProfondIRA,thick,->},
		scale only axis, xmin={\xminBS},xmax={\xmaxBS},enlarge x limits=0.05,
		enlarge y limits=0.125,
		color=BleuProfondIRA,
		%		ylabel near ticks,
		ylabel={Profit},
x label style={at={(axis cs:\xmaxBS+.1,0)},anchor=north east},
		xlabel={ss-jacent ($T$)},
		%		    x label style={at={(axis description cs:0.5,-0.1)},anchor=north},
		%		y label style={at={(axis description cs:-0.1,.5)},rotate=90,anchor=south},
		ytick=\empty,
		xtick=\empty,
		extra y ticks ={0},
		extra y tick labels={{0}},
		extra x ticks ={\PxExerciceBS},
		extra x tick labels={{$E$}},
		extra x tick style={color=BleuProfondIRA,
			tick label style={yshift=0mm}	},
		title ={ \textbf{Short Put}},
		title style={yshift=10mm},
		%	legend style={draw=none,
			%		legend columns=-1,
			%		at={(0.5,1)},
			%		anchor=south,
			%		outer sep=1em,
			%		node font=\small,
			%	},
		]		%		
		\addplot[name path=A,BleuProfondIRA,thick,domain={{\xminBS}:{\xmaxBS}}, samples=21,dashdotted] {-Put(x,\PxExerciceBS,\PremiumBS)} 
		node [pos=0.15,yshift=10mm,color=OrangeProfondIRA] {Perte}
		node [pos=.75,yshift=-4mm,color=BleuProfondIRA] {Gains};	
		%\draw[BleuProfondIRA,thick] \OV{\PxExerciceBS}{\PremiumBS}{0.4}{\xminBS}{\xmaxBS} ;
		%	\addplot[BleuProfondIRA,thick,dashdotted] {Call(x,50,2)};
		%	\addplot[BleuProfondIRA,thick] {Put(x,50,3)+Call(x,50,2)};
		%\addplot[name path=Option,BleuProfondIRA,thick,domain={250:350}, 			 		samples=10,dashdotted,smooth] {BSPut(x,\PxExerciceBS,\riskfreeBS,\TBS,\sigmaBS)} ;	
		\path [save path=\xaxis,name path=xaxis]
		({\xminBS},0)		-- ({\xmaxBS},0)		;
		\addplot [bottom color=OrangeProfondIRA!50, top color=OrangeProfondIRA!10] fill between [
		of=A and xaxis,
		split,
		every segment no 1/.style=
		{top color = BleuProfondIRA!50, bottom color=BleuProfondIRA!10}] ;
		%\draw[use path=\xaxis, ->,OrangeProfondIRA,thick];
		\draw [fill]  ({-\PremiumBS+\PxExerciceBS},0) circle (1mm) node [above=5mm] {$S_{PM}$};
		\draw[BleuProfondIRA, thin] ({\PxExerciceBS},0) -- ({\PxExerciceBS},{\PremiumBS});
		\node at ({\xminBS+0.25*(\xmaxBS-\xminBS)},{\pgfkeysvalueof{/pgfplots/ymin}+1})
		{IN};
		\node at ({\PxExerciceBS},{\pgfkeysvalueof{/pgfplots/ymin}+1})
		{AT};
		\node at ({\xminBS+0.8*(\xmaxBS-\xminBS)},{\pgfkeysvalueof{/pgfplots/ymin}+1})
		{OUT};
		\draw [<->, xshift=-5mm] ({\pgfkeysvalueof{/pgfplots/xmin}},0) -- ({\pgfkeysvalueof{/pgfplots/xmin}},\PremiumBS) node [pos=0.5, xshift=-2.5mm, rotate=90] {Prime};
	\end{axis}
	%
\end{tikzpicture}
    	     

At maturity, the payoff is \(\min (0, S_T-K)=-(K- S_T)^+\).


\end{f}
\hrule

\begin{f}[Spread Strategies]
\textbf{Spread strategy}: uses two or more options of the same type (two call options or two put options).  
If the strike prices vary, it is a \textbf{vertical spread}.  
If the maturities change, it is a \textbf{horizontal spread}.

A vertical spread strategy involves a long position and a short position on call options on the same underlying asset, with the same maturity but different strike prices.  
We distinguish: \textbf{bull vertical spread}  and \textbf{bear vertical spread}.

\textbf{Bull vertical spread}: anticipating a moderate rise in the underlying asset, the investor takes a long position on \(C_1\) and a short position on \(C_2\) under the condition \(E_1 < E_2\).  
Net result at maturity :

	\begin{tikzpicture}[scale=.52]
		\def\riskfreeBS{0.05}
		\def\xminBS{5}
		\def\xmaxBS{15}
		\def\PxExerciceBSa{9}
		\def\PxExerciceBSb{12}
		\def\sigmaBS{0.2}
		\def\TBS{0.75}
		\def\PremiumBSa{{BSCall(11,{\PxExerciceBSa},{\riskfreeBS},{\TBS},{\sigmaBS})}}
		\def\PremiumBSb{{BSCall(11,{\PxExerciceBSb},{\riskfreeBS},{\TBS},{\sigmaBS})}}
		\begin{axis}[ 
			width=0.8\textwidth,
			height=0.5\textwidth, 
			extra tick style={tick style=BleuProfondIRA},
			clip=false,
			axis on top,
			axis lines=middle, axis line style={BleuProfondIRA,thick,->},
			scale only axis, xmin={\xminBS},xmax={\xmaxBS},enlarge x limits=0.05,
			enlarge y limits=0.125,
			color=BleuProfondIRA,
			%		ylabel near ticks,
			ylabel={Profit},
			x label style={={at={(current axis.right of origin)}}},
			%    x label style={at={(axis description cs:1,-0.1)},anchor=south},
			%		x label style={at={(1,0.5)}},
			xlabel={underlying (\(T\))},
			%		    x label style={at={(axis description cs:0.5,-0.1)},anchor=north},
			%		y label style={at={(axis description cs:-0.1,.5)},rotate=90,anchor=south},
			ytick=\empty,
			xtick=\empty,
			extra y ticks ={0},
			extra y tick labels={{0}},
			extra x ticks ={{\PxExerciceBSa},{\PxExerciceBSb}},
			extra x tick labels={{\(E_1\ \ \ \ \ \)},{\(E_2\)}},
			extra x tick style={color=BleuProfondIRA,
				tick label style={yshift=-0mm}	},
			]
			%		
			\addplot[name path=A,BleuProfondIRA,thin,domain={{\xminBS}:{\xmaxBS-1.75}}, samples=21,dashed] {Call(x,\PxExerciceBSa,\PremiumBSa)} 
			node [pos=0.15, above] {\small Long \(C_1\)};	
			\addplot[name path=B,BleuProfondIRA,thin,domain={{\xminBS}:{\xmaxBS}}, samples=21,dashed] {-Call(x,\PxExerciceBSb,\PremiumBSb)} 
			node [pos=0.15, above] {\small  Short \(C_2\)};	
			\addplot[name path=EVH,OrangeProfondIRA,thick,domain={{\xminBS}:{\xmaxBS}}, samples=41] {Call(x,\PxExerciceBSa,\PremiumBSa)-Call(x,\PxExerciceBSb,\PremiumBSb)} node [pos=0.15, above] {Bull vertical spread};	
			%\draw[BleuProfondIRA,thick] \OV{\PxExerciceBSa}{\PremiumBSa}{0.4}{\xminBS}{\xmaxBS} ;
			%		\addplot[name path=Option,BleuProfondIRA,thick,domain={250:350}, 			 		samples=10,dashdotted,smooth] {BSPut(x,\PxExerciceBSa,\riskfreeBS,\TBS,\sigmaBS)} ;	
			
			\draw[BleuProfondIRA, thin, dashed] ({\PxExerciceBSa},0) -- ({\PxExerciceBSa},{-\PremiumBSa})	;
			\draw[BleuProfondIRA, thin, dashed] ({\PxExerciceBSb},0) -- ({\PxExerciceBSb},{\PremiumBSb})	;
		\end{axis}
		%
	\end{tikzpicture}



\textbf{Bear vertical spread}: anticipating a moderate decline in the underlying asset, the investor sells the more expensive option and buys the cheaper one.


	\begin{tikzpicture}[scale=.52]
		\def\riskfreeBS{0.05}
		\def\xminBS{5}
		\def\xmaxBS{15}
		\def\PxExerciceBSa{9}
		\def\PxExerciceBSb{12}
		\def\sigmaBS{0.2}
		\def\TBS{0.75}
		\def\PremiumBSa{{BSCall(11,{\PxExerciceBSa},{\riskfreeBS},{\TBS},{\sigmaBS})}}
		\def\PremiumBSb{{BSCall(11,{\PxExerciceBSb},{\riskfreeBS},{\TBS},{\sigmaBS})}}
		\begin{axis}[ 
			width=0.8\textwidth,
			height=0.5\textwidth, 
			extra tick style={tick style=BleuProfondIRA},
			clip=false,
			axis on top,
			axis lines=middle, axis line style={BleuProfondIRA,thick,->},
			scale only axis, xmin={\xminBS},xmax={\xmaxBS},enlarge x limits=0.05,
			enlarge y limits=0.125,
			color=BleuProfondIRA,
			%		ylabel near ticks,
			ylabel={Profit},
			x label style={={at={(current axis.right of origin)}}},
			%    x label style={at={(axis description cs:1,-0.1)},anchor=south},
			%		x label style={at={(1,0.5)}},
			xlabel={underlying (\(T\))},
			%		    x label style={at={(axis description cs:0.5,-0.1)},anchor=north},
			%		y label style={at={(axis description cs:-0.1,.5)},rotate=90,anchor=south},
			ytick=\empty,
			xtick=\empty,
			extra y ticks ={0},
			extra y tick labels={{0}},
			extra x ticks ={{\PxExerciceBSa},{\PxExerciceBSb}},
			extra x tick labels={{\(E_1\ \ \ \ \ \)},{\(E_2\)}},
			extra x tick style={color=BleuProfondIRA,
				tick label style={yshift=-0mm}	},
			]
			%		
			\addplot[name path=A,BleuProfondIRA,thin,domain={{\xminBS}:{\xmaxBS-1.75}}, samples=21,dashed] {-Call(x,\PxExerciceBSa,\PremiumBSa)} 
			node [pos=0.15, above] {\small Short \(C_1\)};	
			\addplot[name path=B,BleuProfondIRA,thin,domain={{\xminBS}:{\xmaxBS}}, samples=21,dashed] {Call(x,\PxExerciceBSb,\PremiumBSb)} 
			node [pos=0.15, above] {\small  Long \(C_2\)};		
			\addplot[name path=EVH,OrangeProfondIRA,thick,domain={{\xminBS}:{\xmaxBS}}, samples=41] {-Call(x,\PxExerciceBSa,\PremiumBSa)+Call(x,\PxExerciceBSb,\PremiumBSb)} node [pos=0.15, above] {Bear vertical spread};	
			%\draw[BleuProfondIRA,thick] \OV{\PxExerciceBSa}{\PremiumBSa}{0.4}{\xminBS}{\xmaxBS} ;
			%		\addplot[name path=Option,BleuProfondIRA,thick,domain={250:350}, 			 		samples=10,dashdotted,smooth] {BSPut(x,\PxExerciceBSa,\riskfreeBS,\TBS,\sigmaBS)} ;	
			
			\draw[BleuProfondIRA, thin, dashed] ({\PxExerciceBSa},0) -- ({\PxExerciceBSa},{\PremiumBSa})	;
			\draw[BleuProfondIRA, thin, dashed] ({\PxExerciceBSb},0) -- ({\PxExerciceBSb},{-\PremiumBSb})	;
		\end{axis}
		%
	\end{tikzpicture}

% Code TikZ conservé ici (éventuellement inséré)

\textbf{Butterfly spread}: anticipates a small movement in the underlying asset.  
It is a combination of a bull vertical spread and a bear vertical spread.  
This strategy is suitable when large movements are considered unlikely.  
Requires a low initial investment.

	\begin{tikzpicture}[scale=.52]
		\def\riskfreeBS{0.05}
		\def\xminBS{5}
		\def\xmaxBS{15}
		\def\PxExerciceBSa{8}
		\def\PxExerciceBSb{10}
		\def\PxExerciceBSc{12}
		\def\sigmaBS{0.2}
		\def\TBS{0.75}
		\def\PremiumBSa{BSCall(11,{\PxExerciceBSa},{\riskfreeBS},{\TBS},{\sigmaBS})}
		\def\PremiumBSb{BSCall(11,{\PxExerciceBSb},{\riskfreeBS},{\TBS},{\sigmaBS})}
		\def\PremiumBSc{BSCall(11,{\PxExerciceBSc},{\riskfreeBS},{\TBS},{\sigmaBS})}
		\begin{axis}[
			width=0.8\textwidth,
			height=0.5\textwidth, 
			extra tick style={tick style=BleuProfondIRA},
			clip=false,
			axis on top,
			axis lines=middle, axis line style={BleuProfondIRA,thick,->},
			scale only axis, xmin={\xminBS},xmax={\xmaxBS},enlarge x limits=0.05,
			enlarge y limits=0.125,
			color=BleuProfondIRA,
			%		ylabel near ticks,
			ylabel={Profit},
			x label style={={at={(current axis.right of origin)}}},
			%    x label style={at={(axis description cs:1,-0.1)},anchor=south},
			%		x label style={at={(1,0.5)}},
			xlabel={underlying (\(T\))},
			%		    x label style={at={(axis description cs:0.5,-0.1)},anchor=north},
			%		y label style={at={(axis description cs:-0.1,.5)},rotate=90,anchor=south},
			ytick=\empty,
			xtick=\empty,
			extra y ticks ={0},
			extra y tick labels={{0}},
			extra x ticks ={{\PxExerciceBSa},{\PxExerciceBSb},{\PxExerciceBSc}},
			extra x tick labels={{\(E_1\ \ \ \ \ \)},{\(E_2\)},{\(E_3\)}},
			extra x tick style={color=BleuProfondIRA,
				tick label style={yshift=-0mm}	},
			]
			%		
			\addplot[name path=A,BleuProfondIRA,thin,domain={{\xminBS}:{\xmaxBS}}, samples=21,dashed] {Call(x,\PxExerciceBSa,\PremiumBSa)} 
			node [pos=0.15, above] {\small Long \(C_1\)};	
			\addplot[name path=B,BleuProfondIRA,thin,domain={{\xminBS}:{\xmaxBS-1.75}}, samples=21,dashed] {-2*Call(x,\PxExerciceBSb,\PremiumBSb)} 
			node [pos=0.15, above] {\small Short \(2\ C_2\)};	
			\addplot[name path=C,BleuProfondIRA,thin,domain={{\xminBS}:{\xmaxBS}}, samples=21,dashed] {Call(x,\PxExerciceBSc,\PremiumBSb)} 
			node [pos=0.15, above] {\small Long \(C_3\)};	
			\addplot[name path=EP,OrangeProfondIRA,thick,domain={{\xminBS}:{\xmaxBS}}, samples=41] {
				Call(x,\PxExerciceBSa,\PremiumBSa) - 2*Call(x,\PxExerciceBSb,\PremiumBSb)+
				Call(x,\PxExerciceBSc,\PremiumBSc)}
			node [pos=0.5, below=30pt] {Butterfly spread};	
			\draw[BleuProfondIRA, thin, dashed] ({\PxExerciceBSa},0) -- ({\PxExerciceBSa},{-\PremiumBSa})	;
			\draw[BleuProfondIRA, thin, dashed] ({\PxExerciceBSb},0) -- ({\PxExerciceBSb},{2*\PremiumBSb})	;
			\draw[BleuProfondIRA, thin, dashed] ({\PxExerciceBSc},0) -- ({\PxExerciceBSc},{-\PremiumBSc})	;
		\end{axis}
		%
	\end{tikzpicture}
% Code TikZ conservé ici (éventuellement inséré)

\end{f}
\hrule

\begin{f}[Combined strategies]
	
	A \textbf{combined strategy} uses both call and put options. Notably, we distinguish between \textbf{straddles} and \textbf{strangles}.
	
	A \textbf{straddle} combines the purchase of a call option and a put option with the same expiration date and strike price. This strategy bets on a large price movement, either upward or downward. The maximum loss occurs if the price at expiration is equal to the strike price.
	
	A \textbf{strangle} is the purchase of a call and a put with the same expiration date but different strike prices. It assumes a very large movement in the value of the underlying asset.
	
		\begin{tikzpicture}[scale=.52]
			\def\xminBS{200}
			\def\xmaxBS{275}
			\def\PxExerciceBSa{230}
			\def\PxExerciceBSb{245}
			\def\PremiumBSa{20.69}
			\def\PremiumBSb{23.79}
			\begin{axis}[ 
				width=0.8\textwidth,
				height=0.5\textwidth, 
				extra tick style={tick style=BleuProfondIRA},
				clip=false,
				axis on top,
				axis lines=middle, axis line style={BleuProfondIRA,thick,->},
				scale only axis, xmin={\xminBS},xmax={\xmaxBS},enlarge x limits=0.05,
				enlarge y limits=0.125,
				color=BleuProfondIRA,
				ylabel={Profit},
				x label style={={at={(current axis.right of origin)}}},
				xlabel={underlying (\(T\))},
				ytick=\empty,
				extra y ticks ={0},
				extra y tick labels={{0}},
				extra x ticks ={{\PxExerciceBSa},{\PxExerciceBSb}},
				extra x tick labels={{\(E_1\ \ \ \ \ \)},{\(E_2\)}},
				extra x tick style={color=BleuProfondIRA,
					tick label style={yshift=-10mm}	},
				]
				\addplot[name path=A,BleuProfondIRA,thin,domain={{\xminBS}:{\xmaxBS-1.75}}, samples=21,dashed] {Call(x,\PxExerciceBSa,\PremiumBSa)} 
				node [pos=0.15, below] {\small Long \(C_1\)};	
				\addplot[name path=B,BleuProfondIRA,thin,domain={{\xminBS}:{\xmaxBS}}, samples=21,dashed] {Put(x,\PxExerciceBSb,\PremiumBSb)} 
				node [pos=0.85, above] {\small  Long \(P_2\)};		
				\addplot[name path=EVH,OrangeProfondIRA,thick,domain={{\xminBS}:{\xmaxBS}}, samples=21] {Call(x,\PxExerciceBSa,\PremiumBSa)+Put(x,\PxExerciceBSb,\PremiumBSb)} node [pos=0.5, above] {\small Strangle};	
				\draw[BleuProfondIRA, thin, dashed] ({\PxExerciceBSa},0) -- ({\PxExerciceBSa},{-\PremiumBSa})	;
				\draw[BleuProfondIRA, thin, dashed] ({\PxExerciceBSb},0) -- ({\PxExerciceBSb},{-\PremiumBSb})	;
			\end{axis}
		\end{tikzpicture}


		\begin{tikzpicture}[scale=.52]
			\def\xminBS{200}
			\def\xmaxBS{275}
			\def\PxExerciceBSa{230}
			\def\PxExerciceBSb{245}
			\def\PremiumBSa{15.19}
			\def\PremiumBSb{14.29}
			\begin{axis}[ 
				width=0.8\textwidth,
				height=0.5\textwidth, 
				extra tick style={tick style=BleuProfondIRA},
				clip=false,
				axis on top,
				axis lines=middle, axis line style={BleuProfondIRA,thick,->},
				scale only axis, xmin={\xminBS},xmax={\xmaxBS},enlarge x limits=0.05,
				enlarge y limits=0.125,
				color=BleuProfondIRA,
				ylabel={Profit},
				x label style={={at={(current axis.right of origin)}}},
				xlabel={underlying (\(T\))},
				ytick=\empty,
				extra y ticks ={0},
				extra y tick labels={{0}},
				extra x ticks ={{\PxExerciceBSa},{\PxExerciceBSb}},
				extra x tick labels={{\(E_1\ \ \ \ \ \)},{\(E_2\)}},
				extra x tick style={color=BleuProfondIRA,
					tick label style={yshift=-10mm}	},
				]
				\addplot[name path=A,BleuProfondIRA,thin,domain={{\xminBS}:{\xmaxBS}}, samples=21,dashed] {Put(x,\PxExerciceBSa,\PremiumBSa)} 
				node [pos=0.85, below] {\small Long \(P_1\)};	
				\addplot[name path=B,BleuProfondIRA,thin,domain={{\xminBS}:{\xmaxBS}}, samples=21,dashed] {Call(x,\PxExerciceBSb,\PremiumBSb)} 
				node [pos=0.15, above] {\small  Long \(C_2\)};		
				\addplot[name path=EVH,OrangeProfondIRA,thick,domain={{\xminBS}:{\xmaxBS}}, samples=21] {Put(x,\PxExerciceBSa,\PremiumBSa)+Call(x,\PxExerciceBSb,\PremiumBSb)} node [pos=0.5, above] {\small Strangle};	
				\draw[BleuProfondIRA, thin, dashed] ({\PxExerciceBSa},0) -- ({\PxExerciceBSa},{-\PremiumBSa})	;
				\draw[BleuProfondIRA, thin, dashed] ({\PxExerciceBSb},0) -- ({\PxExerciceBSb},{-\PremiumBSb})	;
			\end{axis}
		\end{tikzpicture}
	
	
	
\end{f}
\hrule

\begin{f}[Absence of arbitrage opportunity]
It is impossible to realize a risk-free gain from a zero initial investment. Thus, no risk-free profit is possible by exploiting price differences. 

\end{f}

\begin{f}[Parity relation]
	
AAO implies the following relationship between the Call and the Put (stock):

\[S_t- C_t + P_t = K e^{-i_{f}.\tau}\]	
\end{f}
\hrule

\begin{f}[The Cox-Ross-Rubinstein model]
	
It is based on a discrete-time process with two possible price movements at each period: an increase (factor \(u\)) or a decrease (factor \(d\)), with \(u > 1 + i_{f}\) and \(d < 1 + i_{f}\). The price at \(t = 1\) is then \( S_{1}^{u} = S_{0} u \) or \( S_{1}^{d} = S_{0} d \), according to a probability \(q\) or \(1-q\).

\begin{tikzpicture}
	[sibling distance=5em,
	every node/.style = {shape=rectangle, rounded corners, fill=OrangeProfondIRA!20,
		align=center,  draw=OrangeProfondIRA, text=BleuProfondIRA } ,grow=right,
	edge from parent/.style={draw=OrangeProfondIRA, thick}]
	\node (A){\(S_0\)}
	child {node  (B) {\(S_d\)}
		child {node {\(S_{dd}\)}}
		child} 
	child {node (C) {\(S_u\)} 
		child {node  {\(S_{du}\)}}
		child {node {\(S_{uu}\)}}    
	};
	\draw [draw=none] 
	($ (A.east) + (0,0.2) $) -- node[draw=none, fill=none, above left, BleuProfondIRA] {\(q\)} ($ (C.west) + (0,-0.2) $);
	\draw [draw=none]  
	($ (A.east) + (0,-0.2) $) -- node[draw=none, fill=none,below left, BleuProfondIRA] {\(1 - q\)} ($ (B.west) + (0,0.2) $);
\end{tikzpicture}
\quad
\begin{tikzpicture}
	[sibling distance=5em,
	every node/.style = {shape=rectangle, rounded corners, fill=OrangeProfondIRA!20,
		align=center,  draw=OrangeProfondIRA, text=BleuProfondIRA } ,grow=right,
	edge from parent/.style={draw=OrangeProfondIRA, thick}]
	\node {\(C_0\)}
	child {node  {\(C_d\)}
		child {node {\(C_{dd}=(S_{dd}-K)^{+}\)}}
		child}
	child {node {\(C_u\)} 
		child {node  {\(C_{du}=(S_{du}-K)^{+}\)}}
		child {node  {\(C_{uu}=(S_{uu}-K)^{+}\)}}  
	};
\end{tikzpicture}

This model extends to \(n\) periods with \(n+1\) possible prices for \(S_T\). At expiration, the value of a call option is given by \( C_{1}^{u} = (S_{1}^{u} - K)^+ \) and \( C_{1}^{d} = (S_{1}^{d} - K)^+ \).

\textbf{Absence of arbitrage opportunity} implies
\[
d < 1 + i_{f} < u
\]
and a risk-neutral probability
\[q = \frac{(1 + i_{f}) - d}{u - d}\]

\textbf{Call price} (with \( S_{1}^{d} < K < S_{1}^{u} \)) :
\[
C_{0} = \frac{1}{1+i_f} \left[ q C_{1}^{u} + (1 - q) C_{1}^{d} \right]
\]

We can also construct a replication portfolio composed of \(\Delta\) shares and \(B\) bonds, such that :
\[
\begin{cases}
	\Delta = \frac{S_{1}^{u} - K}{S_{1}^{u} - S_{1}^{d}}, \\
	B = \frac{-S_{1}^{d}}{1+i_f} \cdot \Delta
\end{cases}
\quad \Rightarrow \quad \Pi_0 = \Delta S_0 + B
\]

\textbf{Put price} :
\[
P_{0} = \frac{1}{1+i_f} \left[ q P_{1}^{u} + (1 - q) P_{1}^{d} \right]
\]

\textbf{Determination of \(q\), \(u\), \(d\)}:  
By calibrating the model to match the first moments of the return under the risk-neutral probability (expected value \(i_f\), variance \(\sigma^2 \delta t\)), we obtain :
\[
e^{i_{f} \delta t} = q u + (1-q) d, \qquad q u^2 + (1-q) d^2 - [q u + (1-q) d]^2 = \sigma^2 \delta t
\]

With the constraint \(u = \frac{1}{d}\), we obtain:
\[
\begin{array}{l}
	q = \frac{e^{-i_f \delta_t} - d}{u - d} \\
	u = e^{\sigma \sqrt{\delta t}} \\
	d = e^{-\sigma \sqrt{\delta t}}
\end{array}
\]

\end{f}
\hrule

\begin{f}[The Black \& Scholes Model]
Assumptions of the model
\begin{itemize}
	\item The risk-free rate \(R\) is constant. We define \(i_f = \ln(1+R)\), which implies \((1+R)^t = e^{i_f t}\).
	\item The stock price \(S_t\) follows a geometric Brownian motion :
	\[
	dS_t = \mu S_t dt + \sigma S_t dW_t 
	\]
	\[
	 S_t = S_0 \exp\left(\sigma W_t + \left( \mu - \frac{1}{2}\sigma^2 \right)t \right)
	\]
	\item No dividend during the option's lifetime.  
	\item The option is "European" (exercised only at maturity).  
	\item Frictionless market : no taxes or transaction costs.  
	\item Short selling is allowed. 
\end{itemize}

The Black-Scholes-Merton equation for valuing a derivative contract \(f\) is:
\[
\frac{\partial f}{\partial t} + i_f S \frac{\partial f}{\partial S} + \frac{1}{2}\sigma^2 S^2 \frac{\partial^2 f}{\partial S^2} = i_f f
\]

At maturity, the price of a call option is \(C(S,T) = \max(0, S_T - K)\), and that of a put option is \(P(S,T) = \max(0, K - S_T)\).


\begin{center}
	\begin{tabular}{|c|c|c|}
		\hline
		Determinants & \textbf{call}&\textbf{put}\\
		\hline
		Underlying price	      & +&	-\\
		Strike price	              & -&	+\\
		Maturity (or time)    & + (-)&	+ (-)\\
		Volatility	              & +&	+\\
		Short-term interest rates  & +&	-\\
		Dividend payment	      & -&	+\\
		\hline
	\end{tabular}
\end{center}

The analytical solutions are :
\begin{align*}
	C_t &= S_t \Phi(d_1) - Ke^{-i_f \tau} \Phi(d_2) \\
	P_t &= Ke^{-i_f \tau} \Phi(-d_2) - S_t \Phi(-d_1)
\end{align*}
or :
\begin{align*}
	d_1 &= \frac{\ln(S_t/K) + (i_f + \frac{1}{2}\sigma^2)\tau}{\sigma \sqrt{\tau}}, \quad
	d_2 = d_1 - \sigma \sqrt{\tau}
\end{align*}

%La sensibilité peut être mesurée par cinq paramètres (lettres grecques) :

\begin{itemize}
	\item \textbf{Delta} \(\Delta\) : variation in the option price depending on the underlying.
	\item \textbf{Gamma} \(\Gamma\) : delta sensitivity.
	\item \textbf{Thêta} \(\Theta\) : sensitivity to time.
	\item \textbf{Véga} \(\mathcal{V}\) : sensitivity to volatility.
	\item \textbf{Rho} \(\rho\) : interest rate sensitivity.
\end{itemize}


The \textbf{Delta} measures the impact of a change in the underlying asset :

\begin{align*}
	\Delta_C &= \frac{\partial C}{\partial S} = \Phi(d_1), \quad \Delta \in (0,1) \\
	\Delta_P &= \frac{\partial P}{\partial S} = \Phi(d_1) - 1, \quad \Delta \in (-1,0)
\end{align*}


The global Delta of a portfolio \(\Pi\) with weights \(\omega_i\) is :
\[
\frac{\partial \Pi}{\partial S_t} = \sum_{i=1}^{n} \omega_i \Delta_i
\]



% Graphique TikZ conservé tel quel :

\begin{center}
\begin{tikzpicture}[scale=.52]
\def\riskfreeBS{0.05}
\def\xminBS{7.5}
\def\xmaxBS{12.5}
\def\PxExerciceBS{10}
\def\sigmaBS{0.3}
\def\TBS{0.4}
\def\PremiumBS{BSCall(\PxExerciceBS*exp(-\riskfreeBS*\TBS),\PxExerciceBS,\riskfreeBS,\TBS,\sigmaBS)}
\def\PxExerciceAct{\PxExerciceBS*exp(-(\riskfreeBS+\sigmaBS*\sigmaBS/2)*\TBS)}
\begin{axis}[
	width=0.8\textwidth,
	height=0.5\textwidth, 
	extra tick style={tick style=BleuProfondIRA},
	clip=false,
	axis on top,
	axis lines=middle, axis line style={BleuProfondIRA,thick,->},
	scale only axis, xmin={\xminBS},xmax={\xmaxBS},enlarge x limits=0.05,
	enlarge y limits=0.1,
	color=BleuProfondIRA,
	ylabel={\(\Delta\)},
	x label style={at={(axis cs:\xmaxBS+.1,0)},anchor=north east},
	xlabel={underlying (\(T\))},
	ytick=\empty,
	xtick=\empty,
	extra y ticks ={-.5,0,0.5},
	extra y tick labels={{\(-\frac{1}{2}\)},{0},{\(\frac{1}{2}\)}},
	extra x ticks ={\PxExerciceAct,\PxExerciceBS},
	extra x tick labels={{\color{BleuProfondIRA}\(K'\)\ \ \ \ \ \ \ \ },{\color{BleuProfondIRA}\ \ \(K\)}},
	extra x tick style={color=BleuProfondIRA,
		tick label style={yshift=-0mm}	},
	title ={ \textbf{Delta of the option}},
	title style={yshift=-10mm}
	]
	\addplot[name path=optionT,OrangeProfondIRA,thin,domain={{\xminBS}:{\xmaxBS}}, samples=21]
	{normcdf(-ddd(x,\PxExerciceBS,\riskfreeBS,\TBS,\sigmaBS),0,1)} node [above] {Call};
	\addplot[name path=optionT,OrangeProfondIRA,thin,domain={{\xminBS}:{\xmaxBS}}, samples=21]
	{normcdf(-ddd(x,\PxExerciceBS,\riskfreeBS,\TBS,\sigmaBS),0,1)-1} node [above] {Put};
	\draw[dashed,OrangeProfondIRA] 
	(axis cs:{\pgfkeysvalueof{/pgfplots/xmin}},-0.5) --
	(axis cs:{\PxExerciceAct},{-.5}) --
	(axis cs:{\PxExerciceAct},{0.5}) --
	(axis cs:{\pgfkeysvalueof{/pgfplots/xmin}},0.5);
\end{axis}
\end{tikzpicture}
\end{center}

\end{f}
\hrule


\begin{f}[The Yield Curve]
The \textbf{yield curve}, or the curve of returns, or \(r_f(\tau)\), provides a graphical representation of risk-free interest rates as a function of maturity (or term).
It is also called the \textbf{zero-coupon} yield curve, referring to a type of risk-free bond with no coupons (a debt composed only of two opposite cash flows, one at \(t_0\) and the other at \(T\)).
This curve also provides insight into market expectations regarding future interest rates (\engl{forward} rates).
\end{f}
\hrule


\begin{f}[The Nelson-Siegel and Svensson models]


The \textbf{Nelson-Siegel} functions take the form

{\small\begin{align*}
y( m ) =& \beta _0 + \beta _1\frac{{\left[ {1 - \exp \left( { - m/\tau} \right)} \right]}}{m/\tau} + \\
		&\beta _2 {\left(\frac{{\left[ {1 - \exp \left( { - m/\tau} \right)} \right]}}{m/\tau} - \exp \left( { - m/\tau}\right)\right)}
\label{MTNSeq}
\end{align*}}
%
where \(y\left( m \right)\) and \(m\) are as above, and \(\beta_0\), \(\beta_1\), \(\beta_2\), and \(\tau\) are parameters:


\begin{itemize}

\item   \(\beta_0\) is interpreted as the long-term level of interest rates (the coefficient is 1, it is a constant that does not decrease),

\item   \(\beta_1\) is the short-term component, noting that :
\begin{equation*}
	\lim_{m \rightarrow 0} \frac{{\left[ {1 - \exp \left( { - m/\tau} \right)} \right]}}{m/\tau}=1
\end{equation*}
It follows that the overnight rate such as €str\index{Taux d'intérêts! Estr} will equal \(\beta_0 + \beta_1\) in this model.
\item \(\beta_2\) is the medium-term component (it starts at 0, increases, then decreases back toward zero — i.e., bell-shaped),
\item \(\tau\) is the scale factor on maturity; it determines where the term weighted by \(\beta_2\) reaches its maximum.
\end{itemize}

Svensson (1995) adds a second bell-shaped term; this is the Nelson–Siegel–Svensson model. The additional term is :
%
\begin{equation*}
+\beta _3 {\left(\frac{{\left[ {1 - \exp \left( { - m/\tau_2} \right)} \right]}}{m/\tau_2} - \exp \left( { - m/\tau_2}\right)\right)}
\label{MTSveq}
\end{equation*}
and the interpretation is the same as for \(\beta_2\) and \(\tau\) above; it allows for two inflection points on the yield curve.

\newcommand{\traintunnel}{	        
\draw[thick, OrangeProfondIRA] svg "M 55.448002 56.380001L 40 39L 28 39L 12.552 56.380001M 12 34C 11.729672 21.575853 21.576109 11.281852 34 11C 46.423893 11.281852 56.270329 21.575853 56.000004 34L 56 55C 56 56.104568 55.104568 57 54 57L 14 57C 12.895431 57 12 56.104568 12 55ZM 28 39L 28 34C 28 30.132 30.302 27 34 27C 37.697998 27 40 30.132 40 34L 40 39M 34 51L 34 57M 34 43L 34 45";
}
\newcommand{\archibuilding}{
\draw[OrangeProfondIRA,yscale=-1] svg "M 12.296 28.886L 12.296 54.453999C 12.451618 55.715763 13.58469 56.623463 14.85 56.500004L 53.150002 56.5C 54.415314 56.623463 55.548386 55.715763 55.704002 54.454002L 55.703999 28.886M 12.296 28.886L 34 11.5L 55.703999 28.886M 34 46.296001L 42.212002 46.296001L 42.214001 50.386002L 49.32 50.386002L 49.32 56.5M 12.296 40.285999L 34 40.285999M 34 35.186001L 55.368 35.186001M 34 11.5L 34 56.236M 12.296 32.106003L 34 32.106003M 19.456001 32.106003L 19.456001 40.285999M 26.84 32.106003L 26.84 40.285999";	        
}
\newcommand{\familialcar}{	%
\draw[thick, OrangeProfondIRA] svg "M 21 45L 21 48C 21 48.552284 20.552284 49 20 49L 16 49C 15.447716 49 15 48.552284 15 48L 15 45M 53 45L 53 48C 53 48.552284 52.552284 49 52 49L 48 49C 47.447716 49 47 48.552284 47 48L 47 45M 54 45C 54.552284 45 55 44.552284 55 44L 55 37.414001C 54.999943 37.149296 54.894939 36.895409 54.708 36.708L 49 31L 19 31L 13.292 36.708C 13.105062 36.895409 13.000056 37.149296 13 37.414001L 13 44C 13 44.552284 13.447716 45 14 45ZM 49 31L 45.228001 19.684C 45.092045 19.275806 44.710239 19.000328 44.279999 19L 23.720001 19C 23.289761 19.000328 22.907955 19.275806 22.771999 19.684L 19 31M 19 31L 14 31C 13.447716 31 13 30.552284 13 30L 13 28C 13 27.447716 13.447716 27 14 27L 20.334 27M 47.666 27L 54 27C 54.552284 27 55 27.447716 55 28L 55 30C 55 30.552284 54.552284 31 54 31L 49 31M 13.092 37L 20 37C 20.552284 37 21 37.447716 21 38L 21 40C 21 40.552284 20.552284 41 20 41L 13 41M 55 41L 48 41C 47.447716 41 47 40.552284 47 40L 47 38C 47 37.447716 47.447716 37 48 37L 54.908001 37";
}

\newcommand{\familialTV}{	        
\draw[thick, OrangeProfondIRA] svg "M 12 17.5L 56 17.5C 56 17.5 57 17.5 57 18.5L 57 43.5C 57 43.5 57 44.5 56 44.5L 12 44.5C 12 44.5 11 44.5 11 43.5L 11 18.5C 11 18.5 11 17.5 12 17.5M 34 44.5L 34 50.5M 24 50.5L 44 50.5";
}

\newcommand{\TresorerieMngt}{	
\draw[OrangeProfondIRA] svg "M 11.008 31C 11.008 32.104568 15.485153 33 21.007999 33C 26.530848 33 31.007999 32.104568 31.007999 31C 31.007999 29.89543 26.530848 29 21.007999 29C 15.485153 29 11.008 29.89543 11.008 31ZM 31 31L 31 37C 31 38.106003 26.524 39 21 39C 15.476 39 11 38.106003 11 37L 11 31M 31 37L 31 43C 31 44.105999 26.524 45 21 45C 15.476 45 11 44.105999 11 43L 11 37M 31 43L 31 49C 31 50.105999 26.524 51 21 51C 15.476 51 11 50.105999 11 49L 11 43M 31 49L 31 55C 31 56.105999 26.524 57 21 57C 15.476 57 11 56.105999 11 55L 11 49M 11 25L 11 13C 11 11.895431 11.895431 11 13 11L 55 11C 56.104568 11 57 11.895431 57 13L 57 37C 57 38.104568 56.104568 39 55 39L 36 39M 28 25C 28.000584 21.94886 30.290909 19.384022 33.32254 19.039516C 36.354168 18.695011 39.161583 20.680555 39.846756 23.65377C 40.531929 26.626984 38.876644 29.640953 36 30.658001M 20 19.5C 20.276142 19.5 20.5 19.723858 20.5 20C 20.5 20.276142 20.276142 20.5 20 20.5C 19.723858 20.5 19.5 20.276142 19.5 20C 19.5 19.723858 19.723858 19.5 20 19.5M 48 29.5C 48.276142 29.5 48.5 29.723858 48.5 30C 48.5 30.276142 48.276142 30.5 48 30.5C 47.723858 30.5 47.5 30.276142 47.5 30C 47.5 29.723858 47.723858 29.5 48 29.5M 15 25L 15 16C 15 15.447716 15.447716 15 16 15L 52 15C 52.552284 15 53 15.447716 53 16L 53 34C 53 34.552284 52.552284 35 52 35L 36 35";	
}

These Nelson-Siegel and Svensson functions have the advantage of behaving well in the long term and being easy to parameterize.  
They are illustrated in the figure where the pictograms \begin{tikzpicture}[xscale=0.2, yscale=-0.2]
\TresorerieMngt\end{tikzpicture} \begin{tikzpicture}[xscale=0.2, yscale=-0.2] \familialTV\end{tikzpicture} \begin{tikzpicture}[xscale=0.2, yscale=-0.2] \familialcar\end{tikzpicture} \begin{tikzpicture}[xscale=0.2, yscale=0.2] \archibuilding\end{tikzpicture} \begin{tikzpicture}[xscale=0.2, yscale=-0.2] \traintunnel
\end{tikzpicture} represent the different usual maturities for this type of property or investment.
They allow for the modeling of a broad yield curve.
Once adjusted, the user can then evaluate assets or define various sensitivity measures.


\begin{center}
\begin{tikzpicture}[scale=0.55]
\def\MTbetaa{0.03}
\def\MTbetab{-0.02}
\def\MTbetac{0.01}
\def\MTbetad{-0.005}  % Svensson
\def\MTtaua{4.5}
\def\MTtaub{11}  % Svensson
\begin{axis}[
	width=0.8\textwidth,
	height=0.5\textwidth, 
	xlabel={Maturity (years)},ylabel={Rate (\%)},
	xmin=0, xmax=30,
	ymin=0, ymax=100*(\MTbetaa+.005),
	enlarge y limits=0.125,
	thick,
	axis x line=bottom,
	axis y line=left,
	yticklabel=\pgfmathprintnumber{\tick}\% 
	]
	\Large
	% Courbe de Nelson-Siegel
	\addplot[OrangeProfondIRA, thick, domain=0.01:27, samples=27] 
	{100*(\MTbetaa + \MTbetab * ((1 - exp(-x/\MTtaua)) / (x/\MTtaua)) + \MTbetac * (((1 - exp(-x/\MTtaua)) / (x/\MTtaua)) - exp(-x/\MTtaua)))}
	node  [pos=0.005] (M) {}
	node  [pos=0.10] (N) {}
	node  [pos=0.30] (O) {}
	node  [pos=0.75] (P) {}
	node  [pos=1] (Q) {};
	\addplot[dashed, OrangeProfondIRA, thick, domain=0.01:30, samples=27] 
	{100*(\MTbetaa + \MTbetab * ((1 - exp(-x/\MTtaua)) / (x/\MTtaua)) + \MTbetac * (((1 - exp(-x/\MTtaua)) / (x/\MTtaua)) - exp(-x/\MTtaua)) + \MTbetad * (((1 - exp(-x/\MTtaub)) / (x/\MTtaub)) - exp(-x/\MTtaub)))};
	% Affichage des paramètres
	\node[anchor=south east, text=BleuProfondIRA] at (rel axis cs:1,0.15) { 
	\(\beta_0 = \MTbetaa\)\quad
	\(\beta_1 = \MTbetab\)\quad
	\(\beta_2 = \MTbetac\)\quad
	\(\tau_1 = \MTtaua\)\quad			
};	\node[anchor=south east, text=BleuProfondIRA] at (rel axis cs:1,0.05) { 
	\(\beta_3 = \MTbetad\)\quad
	\(\tau_2 = \MTtaub\)			
};
\node[xscale=0.3, yscale=-0.3, above=35pt] at (M) {\TresorerieMngt};
\node[xscale=0.3, yscale=-0.3, above=35pt] at (N) {\familialTV};
\node[xscale=0.3, yscale=-0.3, above=30pt] at (O) {\familialcar};
\node[xscale=0.3, yscale=-0.3, above=30pt] at (P) {\archibuilding};
\node[xscale=0.3, yscale=-0.3, above=30pt] at (Q) {\traintunnel};
\end{axis}
\end{tikzpicture}
\end{center}
\end{f}
\hrule


\begin{f}[Vasicek model]
	
	Under a risk-neutral probability \(\mathbb{Q}\), the short rate \((r_t)\) follows an Ornstein–Uhlenbeck process with constant coefficients:
	\[
	dr_t = \kappa(\theta - r_t)\, dt + \sigma\, dW_t, \quad r_0 \in \mathbb{R}
	\]
	où :
	\begin{itemize}[nosep]
		\item \(\kappa > 0\) is the speed of mean reversion,
		\item \(\theta\) is the long-term mean level,
		\item \(\sigma > 0\) is the volatility,
		\item \(W_t\) is a standard Brownian motion under \(\mathbb{Q}\).
	\end{itemize}
	
The EDS solution (application of Itô’s lemma to \(Y_{t}=r(t) e^{\kappa t}\)):
	\[
	r_t = r_s e^{-\kappa(t-s)} + \theta(1 - e^{-\kappa(t-s)}) + \sigma \int_s^t e^{-\kappa(t-u)} dW_u
	\]
	
	\textbf{So} :
	\[
	\begin{aligned}
		\mathbb{E}_\mathbb{Q}[r_t \mid \mathcal{F}_s] &= r_s e^{-\kappa(t-s)} + \theta(1 - e^{-\kappa(t-s)}) \\
		\operatorname{Var}_\mathbb{Q}[r_t \mid \mathcal{F}_s] &= \frac{\sigma^2}{2\kappa} \left(1 - e^{-2\kappa(t-s)}\right)
	\end{aligned}
	\]
The process \((r_t)\) is Gaussian; negative rates are possible.
	
\end{f}

\begin{f}[Price of a zero-coupon bond (Vasicek)]

The price at time \(t\) of a zero-coupon bond maturing at \(T\) is given by :
\[
ZC(t, T) = A(t, T) \, e^{-B(t, T)\, r_t}
\]
où :
\[
\begin{aligned}
	B(t, T) &= \frac{1 - e^{-\kappa(T - t)}}{\kappa} \\
	A(t, T) &= \exp \left[ \left(\theta - \frac{\sigma^2}{2\kappa^2}\right) (B(t, T) - (T - t)) - \frac{\sigma^2}{4\kappa} B(t, T)^2 \right]
\end{aligned}
\]

This formulation is possible due to the fact that \(\int_t^T r_s ds\) is a Gaussian random variable conditional on \(\mathcal{F}_t\).

\[
ZC(t, T) = \mathbb{E}_\mathbb{Q} \left[ \exp\left( -\int_t^T r_s\, ds \right) \Big| \mathcal{F}_t \right]
\]

\end{f}
\hrule

\begin{f}[Cox–Ingersoll–Ross (CIR) model]
	
Under the risk-neutral measure \(\mathbb{Q}\), the short rate \((r_t)\) follows the dynamics :
\[
dr_t = \kappa(\theta - r_t)\,dt + \sigma \sqrt{r_t}\, dW_t, \quad r_0 \geq 0
\]
with :
\begin{itemize}[nosep]
	\item \(\kappa > 0\) : mean reversion speed,
	\item \(\theta > 0\) : long-term level,
	\item \(\sigma > 0\) : volatility,
	\item \(W_t\) : Brownian motion under \(\mathbb{Q}\).
\end{itemize}

\textbf{So} :
\begin{itemize}
	\item The square root \(\sqrt{r_t}\) guarantees \(r_t \geq 0\) if \(2\kappa\theta \geq \sigma^2\) (Feller condition).
	\item The process \((r_t)\) is a non-Gaussian diffusion process but with continuous trajectories.
	\item The rate is \textbf{mean-reverting} around \(\theta\).
\end{itemize}

Thus, the process \((r_t)\) is a diffusion with explicit conditional distributions (under \(\mathbb{Q}\)) :

For \(s < t\), the variable \(r_t\) follows a non-central \(\chi^2\) distribution:
\[
r_t \mid \mathcal{F}_s \sim c \cdot \chi^2_{d}(\lambda)
\]
with :
\begin{itemize}[nosep]
	\item \(\displaystyle c = \frac{\sigma^2 (1 - e^{-\kappa (t - s)})}{4\kappa}\)
	\item \(\displaystyle d = \frac{4\kappa\theta}{\sigma^2}\) : degrees of freedom
	\item \(\displaystyle \lambda = \frac{4\kappa e^{-\kappa (t - s)} r_s}{\sigma^2 (1 - e^{-\kappa (t - s)})}\)
\end{itemize}

and
\[
\begin{aligned}
	\mathbb{E}_\mathbb{Q}[r_t \mid \mathcal{F}_s] =& r_s e^{-\kappa(t-s)} + \theta (1 - e^{-\kappa(t-s)}) \\
	\operatorname{Var}_\mathbb{Q}[r_t \mid \mathcal{F}_s] =& \frac{\sigma^2 r_s e^{-\kappa(t-s)} (1 - e^{-\kappa(t-s)})}{\kappa} \\
			&+ \frac{\theta \sigma^2}{2\kappa} (1 - e^{-\kappa(t-s)})^2
\end{aligned}
\]

\end{f}
\begin{f}[Price of a zero-coupon bond (CIR)]
In the CIR model, the price of a zero-coupon bond at time \(t\) with maturity \(T\) is given by :
\[
ZC(t, T) = A(t, T) \cdot e^{-B(t, T)\, r_t}
\]
with :
\[
\begin{aligned}
	B(t, T) &= \frac{2 (e^{\gamma (T - t)} - 1)}{(\gamma + \kappa)(e^{\gamma (T - t)} - 1) + 2\gamma} \\
	A(t, T) &= \left[ \frac{2\gamma e^{\frac{(\kappa + \gamma)}{2}(T - t)}}{(\gamma + \kappa)(e^{\gamma (T - t)} - 1) + 2\gamma} \right]^{\frac{2\kappa\theta}{\sigma^2}}
\end{aligned}
\]
or :
\[
\gamma = \sqrt{\kappa^2 + 2\sigma^2}
\]

\end{f}
\hrule

\begin{f}[Swaption, Black model]

A \textbf{swaption} is an option on an interest rate swap. It gives the right (but not the obligation) to enter into a swap at a future date \(T\).

\begin{itemize}[nosep]
	\item \textbf{Payer swaption}: right to \emph{pay the fixed rate} and \emph{receive the floating rate}.
	\item \textbf{Receiver swaption}: right to \emph{receive the fixed rate} and \emph{pay the floating rate}.
\end{itemize}

\textbf{Notation} :
\begin{itemize}[nosep]
	\item \(T\) : swaption exercise date
	\item \(K\) : fixed rate (strike)
	\item \(S(T)\) : swap rate on the date \(T\)
	\item \(A(T)\) : present value of future fixed flows.
	\item \(\sigma\) : swap rate volatility
\end{itemize}

The Black (1976) model is an adaptation of the Black–Scholes model for interest rate products. Here, the swap rate \(S(T)\) plays the role of the underlying asset, with a European option-type payoff.

\textbf{Black's formula for a payer swaption} :
\[
\text{SW}_{\text{payer}} = A(T) \left[ S_0 N(d_1) - K N(d_2) \right]
\]
or :
\[
\begin{aligned}
	d_1 &= \frac{\ln(S_0 / K) + \frac{1}{2} \sigma^2 T}{\sigma \sqrt{T}} \\
	d_2 &= d_1 - \sigma \sqrt{T}
\end{aligned}
\]
and \(N(\cdot)\) is the cumulative distribution function of the standard normal distribution.

\textbf{Formula for a receiver swaption} :
\[
\text{SW}_{\text{receiver}} = A(T) \left[ K N(-d_2) - S_0 N(-d_1) \right]
\]

\end{f}


\end{multicols}

\newpage
\begin{center}
\section*{Life Actuarial Science}
    \medskip
\end{center}

\begin{multicols}{2}
	
% !TeX root = ActuarialFormSheet_MBFA-en.tex
% !TeX spellcheck = en_GB

\begin{f}[Life Table Notations]

Age $x$, $y$, $z$...    

$l_x$ is the number of people alive, relative to an initial cohort, at age $x$ (or $y$, $z$...)

$\omega$ is the age limit of mortality tables.

$d_x=l_x-l_{x+1}$ is the number of people who die between the age $x$ and age $x+1$.

$q_x$ is the probability of death between the ages of $x$ et age $x+1$.
$$
\,q_x = d_x / l_x 
$$

$p_x$ is the probability that the individual aged $x$ survives age $x+1$.
$$
\,p_x+q_x=1 
$$

Likewise,
$\,_nd_x = d_x + d_{x+1} + \cdots + d_{x+n-1} = l_x - l_{x+n}$ shows the number of people who die between the age $x$ and age $x+n$.

$\,_nq_x$ is the probability of death between the ages of $x$ and age $x+n$.

$$
\,_nq_x = {}_nd_x / l_x
$$
$\,_np_x$ is the probability of a person of age $x$ to survive the age $x+n$.
$$
\,_np_x = l_{x+n} / l_x 
$$


${}_{m|}q_{x}$, the probability that the individual of age $x$ dies in the ${m+1}^e$ year.
$${}_{m|}q_{x}=\frac{d_{x+m}}{l_x}=\frac{l_{x+m}-l_{x+m+1}}{l_x}$$

$\,e_x$ is the life expectancy for a person still alive at the age $x$. 
This is the number of birthdays you hope to live.
$$
\,e_x = \sum_{t=1}^{\infty} \ _tp_x 
$$
\end{f}
\hrule

\begin{f}[Coefficient or commutations]


These coefficients or commutations established by actuarial functions which depend on a mortality table and a rate $i$ ($v=1/(1+i)$) to establish the actuarial table.
$$
D_x=l_x .v^x
$$
can be seen "as" the actualized number of survivors. The sums

$$
N_x=\sum_{k\geq 0} D_{x+k}=\sum_{k= 0}^{\omega-x} D_{x+k}
$$

$$
S_x=\sum_{k\geq 0} N_{x+k}=\sum_{k\geq 0}(k+1). D_{x+k}
$$
will be used to simplify the calculations.
Likewise
$$
C_x = d_x v^{ x+1} 
$$
can be seen "as" the number of deaths discounted to age $x$. The sums

$$
M_x=\sum_{k= 0}^{\omega-x} C_{x+k}
$$
$$
R_x=\sum_{k= 0}^{\omega-x} M_{x+k}
$$
will be used to simplify the calculations.

The coefficients $D_x$ $N_x$ and $S_x$ will be used for calculations on operations in case of life and $C_x$ $M_x$ and $R_x$ for operations in case of death.

\end{f} 
\hrule

\begin{f}[Life annuities or annuities]


\medskip
	

\begin{tikzpicture}[scale=0.75]
    % Draw the x-axis and y-axis.
    \def\w{11}
    \def\n{6}
    \node[left] at (-.5,0) {${}_{}a_x$};
    
    \begin{scope}[shift={(3.75,.25)}]
        \draw[color=OrangeProfondIRA,scale=0.2,fill=OrangeProfondIRA] \Cerceuil;
    \end{scope}
    \foreach \y in  {0,...,3} {
        \draw (\y,0) -- (\y,-0.1);
        \ifthenelse{\y>0 }{	\node[below] at (\y,-0.1) {\tiny $ \scriptstyle x+\y$};
            \draw[ line width=1, color=OrangeProfondIRA, arrows={-Stealth[length=4, inset=0]}] (\y,0) -- (\y,1);}{
            \node[below] at (\y,-0.1) {\tiny $ \scriptstyle x$};}
    }
    \draw (\n,0) -- (\n,-0.1);
    \node[below] at (\n,-0.1) {\tiny $\scriptstyle  x+n$};
    \foreach \y in  {1,...,3} {
        \draw (\y+\n,0) -- (\y+\n,-0.1);
        \node[below] at (\y+\n,-0.1) {\tiny $\scriptstyle x+n+\y$};
        %		\draw[ line width=1, color=OrangeProfondIRA, arrows={-Stealth[length=4, inset=0]}] (\y+\n,0) -- (\y+\n,1);
    }
    \draw[arrows={-Stealth[length=4, inset=0]}, line width=1] (-.5,0) -- (\w,0);
\end{tikzpicture}

\begin{tikzpicture}[scale=0.75]
    % Draw the x-axis and y-axis.
    \def\w{11}
    \def\n{6}
    \node[left] at (-.5,0) {${}_{}a_x$};
    
    \begin{scope}[shift={(\n+.5+3,.25)}]
        \draw[color=OrangeProfondIRA,scale=0.2,fill=OrangeProfondIRA] \Cerceuil;
    \end{scope}
    \foreach \y in  {0,...,3} {
        \draw (\y,0) -- (\y,-0.1);
        \ifthenelse{\y>0 }{	\node[below] at (\y,-0.1) {\tiny $ \scriptstyle x+\y$};
            \draw[ line width=1, color=OrangeProfondIRA, arrows={-Stealth[length=4, inset=0]}] (\y,0) -- (\y,1);}{
            \node[below] at (\y,-0.1) {\tiny $ \scriptstyle x$};}
    }
    \foreach \y in  {0,...,4} {
        \draw (\y+\n,0) -- (\y+\n,-0.1);
        \ifthenelse{\y>0 }{\node[below] at (\y+\n,-0.1) {\tiny $\scriptstyle x+n+\y$};}{
            \node[below] at (\y+\n,-0.1) {\tiny $\scriptstyle x+n$};}
        \ifthenelse{\y<4 }{	\draw[ line width=1, color=OrangeProfondIRA, arrows={-Stealth[length=4, inset=0]}] (\y+\n,0) -- (\y+\n,1);}
    }
    \draw[arrows={-Stealth[length=4, inset=0]}, line width=1] (-.5,0) -- (\w,0);
\end{tikzpicture}
\begin{tikzpicture}[scale=0.75]
    % Draw the x-axis and y-axis.
    \def\w{11}
    \def\n{6}
    \node[left] at (-.5,0) {$\ddot{a}_x$};
    
    \begin{scope}[shift={(3.75,.25)}]
        \draw[color=OrangeProfondIRA,scale=0.2,fill=OrangeProfondIRA] \Cerceuil;
    \end{scope}
    \foreach \y in  {0,...,3} {
        \draw (\y,0) -- (\y,-0.1);
        \ifthenelse{\y>0 }{	\node[below] at (\y,-0.1) {\tiny $ \scriptstyle x+\y$};}{
            \node[below] at (\y,-0.1) {\tiny $ \scriptstyle x$};}
        \draw[ line width=1, color=OrangeProfondIRA, arrows={-Stealth[length=4, inset=0]}] (\y,0) -- (\y,1);
    }
    \draw (\n,0) -- (\n,-0.1);
    \node[below] at (\n,-0.1) {\tiny $\scriptstyle  x+n$};
    \foreach \y in  {1,...,4} {
        \draw (\y+\n,0) -- (\y+\n,-0.1);
        \node[below] at (\y+\n,-0.1) {\tiny $\scriptstyle x+n+\y$};
        %		\draw[ line width=1, color=OrangeProfondIRA, arrows={-Stealth[length=4, inset=0]}] (\y+\n,0) -- (\y+\n,1);
    }
    \draw[arrows={-Stealth[length=4, inset=0]}, line width=1] (-.5,0) -- (\w,0);
\end{tikzpicture}
\begin{tikzpicture}[scale=0.75]
    % Draw the x-axis and y-axis.
    \def\w{11}
    \def\n{6}
    \node[left] at (-.5,0) {$\ddot{a}_x$};
    
    \begin{scope}[shift={(\n+.5+3,.25)}]
        \draw[color=OrangeProfondIRA,scale=0.2,fill=OrangeProfondIRA] \Cerceuil;
    \end{scope}
    \foreach \y in  {0,...,3} {
        \draw (\y,0) -- (\y,-0.1);
        \ifthenelse{\y>0 }{	\node[below] at (\y,-0.1) {\tiny $ \scriptstyle x+\y$};}{
            \node[below] at (\y,-0.1) {\tiny $ \scriptstyle x$};}
        \draw[ line width=1, color=OrangeProfondIRA, arrows={-Stealth[length=4, inset=0]}] (\y,0) -- (\y,1);
    }
    \foreach \y in  {0,...,4} {
        \draw (\y+\n,0) -- (\y+\n,-0.1);
        \ifthenelse{\y>0 }{\node[below] at (\y+\n,-0.1) {\tiny $\scriptstyle x+n+\y$};}{
            \node[below] at (\y+\n,-0.1) {\tiny $\scriptstyle x+n$};}
        \ifthenelse{\y<4 }{	\draw[ line width=1, color=OrangeProfondIRA, arrows={-Stealth[length=4, inset=0]}] (\y+\n,0) -- (\y+\n,1);}
    }
    \draw[arrows={-Stealth[length=4, inset=0]}, line width=1] (-.5,0) -- (\w,0);
\end{tikzpicture}	
$$a_x %=\frac{N_{x+1}}{D_x} 
%=\sum_{k=1}^{\infty}{}_{k|}q_{x} \ddot{a}_{\lcroof{k+1}}
=\sum_{k=1}^{\infty}{}_{k}p_{x} v^{k}=\ddot{a}_x -1
=\frac{N_{x+1}}{D_{x}}
$$

$$\ddot{a}_x 
%\frac{N_x}{D_x}=
%=\sum_{k=0}^{\infty}{}_{k|}q_{x} \ddot{a}_{\lcroof{k+1}}
=\sum_{k=0}^{\infty}{}_{k}p_{x} v^{k}
=	\frac{N_{x}}{D_{x}} 
$$

If the periodicity corresponds to $m$ periods per year:
$$\ddot{a}_{x}^{(m)} 
=\sum_{k=0}^{\infty}\frac{1}{m}{}_{\frac{k}{m}}p_{x} v^{\frac{k}{m}}\approx\ddot{a}_x -\frac{m-1}{2m}
$$
Similarly, if he pays $1/m$ at the start of the $m$ periods
$$a_{x}^{(m)}\approx a_x +\frac{m-1}{2m}
$$


\textbf{Temporary life annuities}. Whole life annuity guaranteed for n years
$$
a_{x:\lcroof{n}} =
\sum_{k=1}^{n}{}_{k}p_{x} v^{k}
=\frac{N_{x+1}-N_{x+n+1}}{D_{x}}
$$
	
$$
\ddot{a}_{x:\lcroof{n}} =%\frac{N_x - N_{x+n}}{D_x}=
\sum_{k=0}^{n-1}{}_{k}p_{x} v^{k}
=\frac{N_{x}-N_{x+n}}{D_{x}}
$$


\textbf{Deferred life annuities}
${}_{m|}a_{x}$ represent the annuities on the individual of age $x$ deferred $m$ years. The first payment occurs in $m+1$ years in the case of life.

	%	\includegraphics[width=1\linewidth]{../../LifeActuarial/Graph/RenteViagereDifferee}
\begin{tikzpicture}[scale=0.75]
    % Draw the x-axis and y-axis.
    \def\w{11}
    \def\m{6}
    \node[left] at (-.5,0) {${}_{m|}a_x$};
    
    \begin{scope}[shift={(4.25,.25)}]
        \draw[color=OrangeProfondIRA,scale=0.2,fill=OrangeProfondIRA] \Cerceuil;
    \end{scope}
    \draw[dashed, color=BleuProfondIRA,arrows={Stealth[length=4, inset=0]-Stealth[length=4, inset=0]},  line width=1] (0,.3) -- (\m,.3) node [pos=0.5, above] {$m$};		
    \draw (0,0) -- (0,-0.1);
    \node[below] at (0,-0.1) {\tiny $x$};
    \foreach \y in  {1,...,3} {
        \draw (\y,0) -- (\y,-0.1);
        \node[below] at (\y,-0.1) {\tiny $ \scriptstyle x+\y$};
    }
    \draw (\m,0) -- (\m,-0.1);
    \node[below] at (\m,-0.1) {\tiny $\scriptstyle  x+m$};
    \foreach \y in  {1,...,3} {
        \draw (\y+\m,0) -- (\y+\m,-0.1);
        \node[below] at (\y+\m,-0.1) {\tiny $\scriptstyle x+m+\y$};
        %		\draw[ line width=1, color=OrangeProfondIRA, arrows={-Stealth[length=4, inset=0]}] (\y+\m,0) -- (\y+\m,1);
        \draw[arrows={-Stealth[length=4, inset=0]}, line width=1] (-.5,0) -- (\w,0);
    }
\end{tikzpicture}

\begin{tikzpicture}[scale=0.75]
    % Draw the x-axis and y-axis.
    \def\w{11}
    \def\m{6}
    \node[left] at (-.5,0) {${}_{m|}a_x$};
    
    \begin{scope}[shift={(\m+.5+3,.25)}]
        \draw[color=OrangeProfondIRA,scale=0.2,fill=OrangeProfondIRA] \Cerceuil;
    \end{scope}
    \draw[dashed, color=BleuProfondIRA,arrows={Stealth[length=4, inset=0]-Stealth[length=4, inset=0]},  line width=1] (0,.3) -- (\m,.3) node [pos=0.5, above] {$m$};		
    \draw (0,0) -- (0,-0.1);
    \node[below] at (0,-0.1) {\tiny $x$};
    \foreach \y in  {1,...,3} {
        \draw (\y,0) -- (\y,-0.1);
        \node[below] at (\y,-0.1) {\tiny $ \scriptstyle x+\y$};
    }
    \draw (\m,0) -- (\m,-0.1);
    \node[below] at (\m,-0.1) {\tiny $\scriptstyle  x+m$};
    \foreach \y in  {1,...,3} {
        \draw (\y+\m,0) -- (\y+\m,-0.1);
        \node[below] at (\y+\m,-0.1) {\tiny $\scriptstyle x+m+\y$};
        \draw[ line width=1, color=OrangeProfondIRA, arrows={-Stealth[length=4, inset=0]}] (\y+\m,0) -- (\y+\m,1);
    }
    \draw[arrows={-Stealth[length=4, inset=0]}, line width=1] (-.5,0) -- (\w,0);
\end{tikzpicture}


  
\end{f}
\hrule

\begin{f}[Death or survival benefits]

% !TeX spellcheck = en_US
\textbf{Death benefits} (Whole life insurance noted ${SP}_{x}$ or ${A}_{x}$)

$A_x$ indicates a death benefit at the end of the year of death (amount of 1), regardless of the date of occurrence, for an individual insured at age $x$ at the time of subscription.

$A_{x:\lcroof{n}}$ denotes a capital paid upon death if it occurs and at the latest in $n$ years (Endowment).

$\lcterm{A}{x}{n}$ % ou $\termins{x}{n}$ 
denotes a death benefit paid if $x$ dies within the next $n$ years (Term insurance).


$A_x^{(12)}$ indicates a benefit payable at the end of the month of death.

$\overline{A}_x$ indicates a benefit paid on the date of death.
\begin{tikzpicture}[scale=0.75]
% Draw the x-axis and y-axis.
\def\w{11}
\def\n{7}
\node[left] at (-.5,0) {$A_x$};

\begin{scope}[shift={(3.75,.25)}]
    \draw[color=OrangeProfondIRA,scale=0.2,fill=OrangeProfondIRA] \Cerceuil;
\end{scope}
\draw[ line width=1, color=OrangeProfondIRA, arrows={-Stealth[length=4, inset=0]}] (4,0) -- (4,1);
\foreach \y in  {0,...,4} {
    \draw (\y,0) -- (\y,-0.1);
    \ifthenelse{\y>0 }{	\node[below] at (\y,-0.1) {\tiny $ \scriptstyle x+\y$};}{
        \node[below] at (\y,-0.1) {\tiny $ \scriptstyle x$};}
}
\draw (\n,0) -- (\n,-0.1);
\node[below] at (\n,-0.1) {\tiny $\scriptstyle  x+n$};
\foreach \y in  {1,...,3} {
    \draw (\y+\n,0) -- (\y+\n,-0.1);
    \node[below] at (\y+\n,-0.1) {\tiny $\scriptstyle x+n+\y$};
    %		\draw[ line width=1, color=OrangeProfondIRA, arrows={-Stealth[length=4, inset=0]}] (\y+\n,0) -- (\y+\n,1);
}
\draw[arrows={-Stealth[length=4, inset=0]}, line width=1] (-.5,0) -- (\w,0);
\end{tikzpicture}
\begin{tikzpicture}[scale=0.75]
% Draw the x-axis and y-axis.
\def\w{11}
\def\n{7}
\node[left] at (-.5,0) {$A_x$};

\begin{scope}[shift={(\n+.5+2,.25)}]
    \draw[color=OrangeProfondIRA,scale=0.2,fill=OrangeProfondIRA] \Cerceuil;
\end{scope}
\draw[ line width=1, color=OrangeProfondIRA, arrows={-Stealth[length=4, inset=0]}] (\n+3,0) -- (\n+3,1);
\foreach \y in  {0,...,4} {
    \draw (\y,0) -- (\y,-0.1);
    \ifthenelse{\y>0 }{	\node[below] at (\y,-0.1) {\tiny $ \scriptstyle x+\y$};}{
        \node[below] at (\y,-0.1) {\tiny $ \scriptstyle x$};}
}
\draw (\n,0) -- (\n,-0.1);
\node[below] at (\n,-0.1) {\tiny $\scriptstyle  x+n$};
\foreach \y in  {1,...,3} {
    \draw (\y+\n,0) -- (\y+\n,-0.1);
    \node[below] at (\y+\n,-0.1) {\tiny $\scriptstyle x+n+\y$};
    %		\draw[ line width=1, color=OrangeProfondIRA, arrows={-Stealth[length=4, inset=0]}] (\y+\n,0) -- (\y+\n,1);
}
\draw[arrows={-Stealth[length=4, inset=0]}, line width=1] (-.5,0) -- (\w,0);
\end{tikzpicture}
\begin{tikzpicture}[scale=0.75]
% Draw the x-axis and y-axis.
\def\w{11}
\def\n{7}
\node[left] at (-.5,0) {$\overline{A}_x$};

\begin{scope}[shift={(3.75,.25)}]
    \draw[color=OrangeProfondIRA,scale=0.2,fill=OrangeProfondIRA] \Cerceuil;
\end{scope}
\draw[ line width=1, color=black, arrows={-Stealth[length=4, inset=0]}] (3.75,0) -- (3.75,1);
\foreach \y in  {0,...,4} {
    \draw (\y,0) -- (\y,-0.1);
    \ifthenelse{\y>0 }{	\node[below] at (\y,-0.1) {\tiny $ \scriptstyle x+\y$};}{
        \node[below] at (\y,-0.1) {\tiny $ \scriptstyle x$};}
}
\draw (\n,0) -- (\n,-0.1);
\node[below] at (\n,-0.1) {\tiny $\scriptstyle  x+n$};
\foreach \y in  {1,...,3} {
    \draw (\y+\n,0) -- (\y+\n,-0.1);
    \node[below] at (\y+\n,-0.1) {\tiny $\scriptstyle x+n+\y$};
    %		\draw[ line width=1, color=OrangeProfondIRA, arrows={-Stealth[length=4, inset=0]}] (\y+\n,0) -- (\y+\n,1);
}
\draw[arrows={-Stealth[length=4, inset=0]}, line width=1] (-.5,0) -- (\w,0);
\end{tikzpicture}
\begin{tikzpicture}[scale=0.75]
% Draw the x-axis and y-axis.
\def\w{11}
\def\n{7}
\node[left] at (-.5,0) {$\overline{A}_x$};

\begin{scope}[shift={(\n+.5+2,.25)}]
    \draw[color=OrangeProfondIRA,scale=0.2,fill=OrangeProfondIRA] \Cerceuil;
\end{scope}
\draw[ line width=1, color=black, arrows={-Stealth[length=4, inset=0]}] (\n+.5+2,0) -- (\n+.5+2,1);
\foreach \y in  {0,...,4} {
    \draw (\y,0) -- (\y,-0.1);
    \ifthenelse{\y>0 }{	\node[below] at (\y,-0.1) {\tiny $ \scriptstyle x+\y$};}{
        \node[below] at (\y,-0.1) {\tiny $ \scriptstyle x$};}
}
\draw (\n,0) -- (\n,-0.1);
\node[below] at (\n,-0.1) {\tiny $\scriptstyle  x+n$};
\foreach \y in  {1,...,3} {
    \draw (\y+\n,0) -- (\y+\n,-0.1);
    \node[below] at (\y+\n,-0.1) {\tiny $\scriptstyle x+n+\y$};
    %		\draw[ line width=1, color=OrangeProfondIRA, arrows={-Stealth[length=4, inset=0]}] (\y+\n,0) -- (\y+\n,1);
}
\draw[arrows={-Stealth[length=4, inset=0]}, line width=1] (-.5,0) -- (\w,0);
\end{tikzpicture}

\medskip

Whole life benefit
$$A_{x}=\sum_{k=0}^{\infty} {}_{k|}q_x\ \nu^{k+1}=\frac{M_x}{D_x}$$
	
$$\termins{x}{n}=\sum_{k=0}^{n-1} {}_{k|}q_x\ \nu^{k+1}=\frac{M_x-M_{x+n}}{D_x}
$$



\medskip    
\textbf{Deferred capital (Pure Endowment, unique capital in the event of survival)} noted $\lcend{A}{x}{n}$ %$\pureend{x}{n}$
or ${}_n E_x$.

\begin{tikzpicture}[scale=0.75]
% Draw the x-axis and y-axis.
\def\w{11}
\def\n{7}
\node[left] at (-.5,0) {${}_n E_x$};

\begin{scope}[shift={(3.75,.25)}]
    \draw[color=OrangeProfondIRA,scale=0.2,fill=OrangeProfondIRA] \Cerceuil;
\end{scope}
%		\draw[ line width=1, color=OrangeProfondIRA, arrows={-Stealth[length=4, inset=0]}] (4,0) -- (4,1);
\foreach \y in  {0,...,4} {
    \draw (\y,0) -- (\y,-0.1);
    \ifthenelse{\y>0 }{	\node[below] at (\y,-0.1) {\tiny $ \scriptstyle x+\y$};}{
        \node[below] at (\y,-0.1) {\tiny $ \scriptstyle x$};}
}
\draw (\n,0) -- (\n,-0.1);
\node[below] at (\n,-0.1) {\tiny $\scriptstyle  x+n$};
\foreach \y in  {1,...,3} {
    \draw (\y+\n,0) -- (\y+\n,-0.1);
    \node[below] at (\y+\n,-0.1) {\tiny $\scriptstyle x+n+\y$};
    %		\draw[ line width=1, color=OrangeProfondIRA, arrows={-Stealth[length=4, inset=0]}] (\y+\n,0) -- (\y+\n,1);
}
\draw[arrows={-Stealth[length=4, inset=0]}, line width=1] (-.5,0) -- (\w,0);
\end{tikzpicture}
\begin{tikzpicture}[scale=0.75]
% Draw the x-axis and y-axis.
\def\w{11}
\def\n{7}
\node[left] at (-.5,0) {${}_n E_x$};

\begin{scope}[shift={(\n+.5+2,.25)}]
    \draw[color=OrangeProfondIRA,scale=0.2,fill=OrangeProfondIRA] \Cerceuil;
\end{scope}
\draw[ line width=1, color=OrangeProfondIRA, arrows={-Stealth[length=4, inset=0]}] (\n,0) -- (\n,1);
\foreach \y in  {0,...,4} {
    \draw (\y,0) -- (\y,-0.1);
    \ifthenelse{\y>0 }{	\node[below] at (\y,-0.1) {\tiny $ \scriptstyle x+\y$};}{
        \node[below] at (\y,-0.1) {\tiny $ \scriptstyle x$};}
}
\draw (\n,0) -- (\n,-0.1);
\node[below] at (\n,-0.1) {\tiny $\scriptstyle  x+n$};
\foreach \y in  {1,...,3} {
    \draw (\y+\n,0) -- (\y+\n,-0.1);
    \node[below] at (\y+\n,-0.1) {\tiny $\scriptstyle x+n+\y$};
    %		\draw[ line width=1, color=OrangeProfondIRA, arrows={-Stealth[length=4, inset=0]}] (\y+\n,0) -- (\y+\n,1);
}
\draw[arrows={-Stealth[length=4, inset=0]}, line width=1] (-.5,0) -- (\w,0);
\end{tikzpicture}
	$$
	={}_n E_x={}_n p_x .v^n=\frac{l_{x+n}}{l_x} . v^n =\frac{D_{x+n}}{D_{x}}
	$$
Death benefit with payment of the capital in the event of survival (Endowment)
$$A_{x:\lcroof{n}}=\termins{x}{n}+\lcend{A}{x}{n}$$

\end{f}
\hrule

\begin{f}[Life insurance on several individuals] 

$a_{xyz}$ is an annual annuity, paid at the end of the first year and for as long as they live $(x)$, $(y)$ and $(z)$.

$a_{\overline{xyz}}$ is an annual annuity, paid at the end of the first year and for as long as they live $(x)$, $(y)$ or $(z)$.

$$
a_{\overline{xy}}=a_{y}+a_{x}-a_{xy}
$$

$A_{xyz}$ is an insurance that comes into effect at the end of the year of the first death of $(x)$, $(y)$ and $(z)$.

The vertical bar indicates conditionality :

$a_{x|y}$ is a survivor's annuity which benefits $(x)$ after the death of $(y)$.

$A_{x|yz}$ is a first-to-die insurance $(y)$ and $(z)$.	


\begin{tikzpicture}[scale=0.85]
    % Draw the x-axis and y-axis.
    \def\w{11}
    \def\n{6}
    \node[left] at (-.5,0) {$a_{{\color{OrangeProfondIRA}x}|{\color{BleuProfondIRA}y}}$};
    
    \begin{scope}[shift={(9.75,.25)}]
        \draw[color=OrangeProfondIRA,scale=0.2,fill=OrangeProfondIRA] \Cerceuil;
    \end{scope}
    \begin{scope}[shift={(7.55,.25)}]
        \draw[color=BleuProfondIRA,scale=0.2,fill=BleuProfondIRA] \Cerceuil;
    \end{scope}
    \draw[ line width=1, color=OrangeProfondIRA, arrows={-Stealth[length=4, inset=0]}] (8,0) -- (8,1);
    \draw[ line width=1, color=OrangeProfondIRA, arrows={-Stealth[length=4, inset=0]}] (9,0) -- (9,1);
    \foreach \y in  {0,...,3} {
        \draw (\y,0) -- (\y,-0.1);
        \ifthenelse{\y>0 }{	\node[below] at (\y,-0.1) {\tiny $ \scriptstyle x+\y$};}{
            \node[below] at (\y,-0.1) {\tiny $ \scriptstyle x$};}
    }
    \draw (\n,0) -- (\n,-0.1);
    \node[below] at (\n,-0.1) {\tiny $\scriptstyle  x+n$};
    \foreach \y in  {1,...,3} {
        \draw (\y+\n,0) -- (\y+\n,-0.1);
        \node[below] at (\y+\n,-0.1) {\tiny $\scriptstyle x+n+\y$};
    }
    \draw[arrows={-Stealth[length=4, inset=0]}, line width=1] (-.5,0) -- (\w,0);
\end{tikzpicture}
\begin{tikzpicture}[scale=0.85]
    % Draw the x-axis and y-axis.
    \def\w{11}
    \def\n{6}
    \node[left] at (-.5,0) {$a_{{\color{OrangeProfondIRA}x}|{\color{BleuProfondIRA}y}}$};
    
    \begin{scope}[shift={(8.4,.25)}]
        \draw[color=OrangeProfondIRA,scale=0.2,fill=OrangeProfondIRA] \Cerceuil;
    \end{scope}
    \begin{scope}[shift={(2.75,.25)}]
        \draw[color=BleuProfondIRA,scale=0.2,fill=BleuProfondIRA] \Cerceuil;
    \end{scope}
    \foreach \y in  {3,...,8} {
        \draw[ line width=1, color=OrangeProfondIRA, arrows={-Stealth[length=4, inset=0]}] (\y,0) -- (\y,1);
    }
    \foreach \y in  {0,...,3} {
        \draw (\y,0) -- (\y,-0.1);
        \ifthenelse{\y>0 }{	\node[below] at (\y,-0.1) {\tiny $ \scriptstyle x+\y$};}{
            \node[below] at (\y,-0.1) {\tiny $ \scriptstyle x$};}
    }
    \foreach \y in  {0,...,4} {
        \draw (\y+\n,0) -- (\y+\n,-0.1);
        \ifthenelse{\y>0 }{\node[below] at (\y+\n,-0.1) {\tiny $\scriptstyle x+n+\y$};}{
            \node[below] at (\y+\n,-0.1) {\tiny $\scriptstyle x+n$};}
    }
    \draw[arrows={-Stealth[length=4, inset=0]}, line width=1] (-.5,0) -- (\w,0);
\end{tikzpicture}
\begin{tikzpicture}[scale=0.85]
    % Draw the x-axis and y-axis.
    \def\w{11}
    \def\n{6}
    \node[left] at (-.5,0) {$a_{{\color{OrangeProfondIRA}x}|{\color{BleuProfondIRA}y}}$};
    
    \begin{scope}[shift={(3.75,.25)}]
        \draw[color=OrangeProfondIRA,scale=0.2,fill=OrangeProfondIRA] \Cerceuil;
    \end{scope}
    \begin{scope}[shift={(7.55,.25)}]
        \draw[color=BleuProfondIRA,scale=0.2,fill=BleuProfondIRA] \Cerceuil;
    \end{scope}
    \foreach \y in  {0,...,3} {
        \draw (\y,0) -- (\y,-0.1);
        \ifthenelse{\y>0 }{	\node[below] at (\y,-0.1) {\tiny $ \scriptstyle x+\y$};}{
            \node[below] at (\y,-0.1) {\tiny $ \scriptstyle x$};}
    }
    \draw (\n,0) -- (\n,-0.1);
    \node[below] at (\n,-0.1) {\tiny $\scriptstyle  x+n$};
    \foreach \y in  {1,...,4} {
        \draw (\y+\n,0) -- (\y+\n,-0.1);
        \node[below] at (\y+\n,-0.1) {\tiny $\scriptstyle x+n+\y$};
        %		\draw[ line width=1, color=OrangeProfondIRA, arrows={-Stealth[length=4, inset=0]}] (\y+\n,0) -- (\y+\n,1);
    }
    \draw[arrows={-Stealth[length=4, inset=0]}, line width=1] (-.5,0) -- (\w,0);
\end{tikzpicture}
\end{f}

\begin{f}[Simplified pricing schemes for periodic premiums and reserves]
	
\tikzstyle{startstop} = [rectangle, rounded corners, text width=4cm, minimum height=1cm,text centered, draw=black, fill=BleuProfondIRA!40]
\tikzstyle{process} = [rectangle, minimum width=4cm, minimum height=1cm, text centered, text width=3cm, draw=black, fill=OrangeProfondIRA!40]
\tikzstyle{arrow} = [thick,->,>=stealth]
\ \medskip

\centering
\begin{tikzpicture}[node distance=35mm and 100mm]
	\node (pu) [startstop] {Single Premium of the service\\ noted $PU(.)$ or $PV_0(.)$};
	\node  [below left of=pu, text width=55mm, node distance=18mm, xshift=12mm] {The point is $A$ or $a$ or $E$ or ...\\
	avec $x,k,n,m,K...$};
	\node (puu) [startstop, right of=pu, node distance=50mm] {Value of a unitary periodic premium\\ noted $PPU_0$ or $VP_P$};
	\node  [below of=puu, text width=3cm, node distance=12mm] { $PPU=_{m'|}\ddot{a}_{x:\lcroof{n'}}^{(k')} $};
	\node (ppp) [startstop, below of=pu] {Pure Periodic Premium\\ $PPP$};
	\node  [below right of=ppp, text width=3cm, node distance=25mm, yshift=6mm] {$\displaystyle PPP=\frac{PU(.)}{PPU_0}$};
	\node (ppc) [startstop, right of=ppp, node distance=50mm] {Periodic Premium Charged $PPC$};
	\node  [below of=ppc, node distance=12mm] {$\displaystyle PPC=\frac{PPP(1+\lambda)}{1-com}$};
	\node (Prov) [startstop, below of=ppp] {Mathematical reserves $t^e$ year\\ noted $PM_t$ or $V_t$};
	\node (provt) [below right of=Prov,text width=10cm, node distance=30mm] {$V_t=PV_t(.)-PPP\times PPU_t$ \\
	$PV_t(.)$ recalculated with  $x+t,k,n,m-t,K...\ if\ t\geq m$ \\
	or $x+t,k,n-(t-m),m=0,K...\ si\ t>m$\\
	likewise $PPU_t=_{(m'-t)^+|}\ddot{a}_{x+t:\lcroof{n'-(t-m)^+}}^{(k')} $};
%	\node (pro2b) [startstop, right of=dec1, xshift=2cm] {Process 2b};
%	\node (stop) [startstop, below of=pro2b] {Stop};
	
	\draw [arrow] (pu) -- (ppp);
	\draw [arrow] (puu) -- (ppp);
	\draw [arrow] (ppp) -- (ppc);
	\draw [arrow] (ppp) -- node[anchor=east] {$t$} (Prov);
%	\draw [arrow] (dec1) -- node[anchor=south] {no} (pro2b);
%	\draw [arrow] (pro2b) |- (ppp);
%	\draw [arrow] (ppp) -- (stop);
	
\end{tikzpicture}
\end{f}
\end{multicols}


\newpage
\begin{center}
    \section*{Probability \& Statistics}
    \medskip
\end{center}

\begin{multicols}{2}

% !TeX root = ActuarialFormSheet_MBFA-en.tex
% !TeX spellcheck = en_GB

\begin{f}[Axiomatic]{\ }
	
	A \textbf{universe} $\Omega$, is the set of all possible outcomes that can be obtained during a random experiment.
	
	The \textbf{random event} is an event $\omega\_i$ of the universe whose outcome (the result) is not certain.
	
	
	The \textbf{elementary event :}
	\begin{itemize}
		\item two distinct elementary events $\omega_i$ and $\omega_j$ are incompatible,
		\item the union of all the elementary events of the universe $\Omega $ corresponds to certainty.
	\end{itemize}
	
	The \textbf{sets} :
	\begin{itemize}
		\item $E=\lbrace \omega_{i1},\ldots , \omega_{ik}\rbrace$ a subset of $\Omega$ ($k$ elements).
		\item $\overline{E}$ the complement of $E$,
		\item $E\cap F$ the intersection of $E$ and $F$,
		\item $E\cup F$ the union of $E$ and $F$,
		\item $E\setminus F= E\cap\overline{F}$ $E$ minus $F$,
		\item $\varnothing$ the impossible or empty event.
	\end{itemize}
	
	Let $E$ be a set. We call \textbf{trib} or \textbf{$\sigma-$algebra} on $E$, a set $\mathcal{A}$ of parts of $E$ which satisfies :
	\begin{itemize}
		\item     $\mathcal{A} \not=\varnothing$,
		\item     $\forall A \in \mathcal{A} , \overline{A} \in\mathcal{A}$,
		\item     if $\forall n \in \mathbb{N}$, $A_n \in\mathcal{A}$ then $\cup_{n\in\mathbb{N} } A_n \in\mathcal{A}$.
	\end{itemize}
	
	We call \textbf{probability} $\mathbb{P}$ any application of the set of events $\mathcal{A}$ in the interval $[0,1]$, such that :      $$\mathbb{P} :      \mathcal{A}  \mapsto   [0,1]$$
	satisfying the following properties (or axioms)~:
	\begin{description}
		\item[(P1)] $A \subseteq \mathcal{A} $    then  $ \mathbb{P}(A) \geq 0$,
		\item[(P2)] $ \mathbb{P}(\Omega) = 1$,
		\item[(P3)] $A, B \subseteq \mathcal{A}$,  if  $A\cap B =\varnothing$    then   $\mathbb{P}(A\cup B)=\mathbb{P}(A) + \mathbb{P}(B)$.
	\end{description}
	
	The \textbf{probability space}\index{D\'efinition! espace de probabilité} is defined by
	\[ \lbrace \Omega, \mathcal{A}, \mathbb{P}(.) \rbrace \]
	
	The \textbf{Poincaré equality} is written~:
	$$\forall A \in F, \forall B \in F, \mathbb{P} (A \cup B) = \mathbb{P} (A) + \mathbb{P} (B) - \mathbb{P} (A \cap B)$$
	
\end{f}
\hrule

\begin{f}[Bayes]
	In probability theory, the \textbf{conditional probability} of an event $A$, given that another event $B$ of non-zero probability has occurred.
	$$
	\mathbb{P}(A|B) = \frac{\mathbb{P}(A \cap B)}{\mathbb{P}(B)}
	$$
	The real $\mathbb{P}(A|B)$ is read as 'probability of $A$, given $B$.
	Bayes' theorem allows us to write~:
	
	$$   \mathbb{P}(A|B) = \frac{\mathbb{P}(B|A)\mathbb{P}(A)}{\mathbb{P}(B)}. $$
\end{f}
\hrule

\begin{f}[Random variables]
	
	Let $ (\Omega, \mathcal{A}, \mathbb{P})$ be a probability space. We call \textbf{random variable} $X$ from $\Omega$ to $ \Re$ any measurable function $X:\Omega\mapsto \Re$.
	
	$$\lbrace X\leq x \rbrace\equiv \lbrace e\in \Omega \mid X(e)\leq x \rbrace \in   \mathcal{A}$$
	The set of events of $\Omega$ is often not explicit.
	
	The \textbf{distribution function} $(F_X)$ of a real random variable characterizes its probability distribution.
	$$
	F_X(x)=\mathbb {P}(X\leq x), x\in \Re
	$$
	where the right-hand side represents the probability that the real random variable $X$ takes a value less than or equal to $x$.
	The probability that $X$ is in the interval $]a, b]$ is therefore, if $a< b$,
	$
	\mathbb{P}(a< X\leq b)\ =\ F_X(b)-F_X(a)
	$
	
	A probability law has a \textbf{probability density} $f$, if $f$ is a function defined on $\mathbb{R}^{+}$, Lebesgue integral, such that the probability of the interval $[a, b]$ is given by
	$$
	\mathbb{P}(a< X\leq b)=\int_a^b f(x) \mathrm{dx} \mbox{ pour tous nombres tq }a<x<b.
	$$
\end{f}
\hrule
\begin{f}[Expectations]{\ }
	
	The mathematical expectation in the discrete case (discrete qualitative or quantitative variables)~:
	$$
	\mathbb{E}[X]=\sum_{j\in \mathbb{N}}x_j\mathbb{P}(x_j)
	$$
	where $\mathbb{P}(x_j)$ is the probability associated with each event $x_i$.
	
	The mathematical expectation in the continuous case~:
	$$
	\mathbb{E}[X]=\int_{-\infty}^{\infty} x. f(x)dx
	$$
	
	where $f$ denotes the density function of the random variable $x$, defined in our case on $\mathbb{R}$.
	When it comes to sum or integral, the expectation is linear, that is to say~:
	
	$$\mathbb{E}[c_0+c_1X_1+c_2X_2]=c_0+c_1\mathbb{E}[X_1]+c_2\mathbb{E}[X_2]$$
	
	$$
	\mathbb{E}[X]=\int x.f(x)dx =\int_0^1 F^{-1}(p)dp= \int \overline{F}(x)dx\\
	$$ 
\end{f}
\hrule
\begin{f}[Convolution or law of sum]{\ }
	
	The convolution of two functions \( f \) and \( g \), denoted \( (f * g)(x) \), is defined by :
	
	\[
	(f * g)(x) = \int f(t) g(x - t) \, dt
	\]
	
	Convolution measures how \( f(t) \) and \( g(t) \) interact at different points while taking into account the shift (or translation) between
	If \(X\) and \(Y\) are two independent random variables with respective densities \(f_X\) and \(f_Y\), then the density of the sum \(Z = X + Y\) is given by :
	\[
	f_Z(x) = (f_X * f_Y)(x) = \int_{-\infty}^{+\infty} f_X(t)\, f_Y(x - t)\, dt.
	\]
	
\end{f}
\hrule
\begin{f}
	[Compound law or frequency/gravity model]{\ }
	
	Let $N$ be a discrete random variable in $\mathbb{N}^+$, $\left(X_i\right)$ a sequence of $iid$ random variables with finite expectation and variance, then for $\displaystyle S=\sum_{i=1}^{N}X_i$ :
	$$
	\mathbb{E}(S) = \mathbb{E} (\mathbb{E} [S \mid N]) = \mathbb{E} (N.\mathbb{E}(X_1)) = \mathbb{E}(N).\mathbb{E}(X_1)
	$$
	$$
	Var (S) = \mathbb{E} (Var [S \mid N]) + Var (\mathbb{E} [S \mid N])
	$$
\end{f}

\hrule
\begin{f}[Fundamental theorems] {\ }
	
	Let $X$ be a real random variable defined on a probability space $\left(\Omega,\mathcal A,\mathbb P\right)$, and assumed to be almost surely positive or zero. The \textbf{Markov Inequality} gives~:
	$$
	\forall \alpha >0, \mathbb P(X\geq \alpha)\leqslant\frac{\mathbb{E}[X]}{\alpha}.
	$$
	
	The \textbf{Bienaymé-Tchebychev inequality}: 
	For any strictly positive real number $\alpha$, with $\mathbb{E}[X]=\mu$ and $\operatorname{Var}[X]=\sigma^2$
	$$
	\mathbb{P}\left(\left|X-\mu\right| \geq \alpha \right) \leq \frac{\sigma^2}{\alpha^2}.
	$$
	
	The \textbf{weak law of large numbers} considers a sequence $(X_i)_{i\geq n\in\mathbb{N}^*}$ of independent random variables defined on the same probability space, having the same finite expectation and variance denoted respectively $\mathbb{E}[X]$ and $\operatorname{Var}(X)$.
	
	$$
	\forall\varepsilon>0,\quad \lim_{n \to +\infty} \mathbb{P}\left(\left|\frac{X_1+X_2+\cdots+X_n}{n} - \mathbb{E}[X]\right| \geq \varepsilon\right) = 0
	$$
	
	Consider a sequence $(X_n)_{n\in \mathbb{N}}$ of independent random variables that follow the same probability law, integrable, i.e. $E(|X_0|)<+\infty$.
	
	Using the notations, the \textbf{strong law of large numbers} specifies that $(Y_n)_{n\in\mathbb{N}}$ converges to $E(X)$ \og{} almost surely\fg{}.
	%
	$$
	\mathbb{P}\left(\lim_{n \to +\infty} Y_n = E(X)\right)=1
	$$
	Consider the sum $S_n = X_1 + X_2 + \cdots + X_n$.
	$$   
	Z_n\ =\ \frac{S_n - n \mu}{\sigma \sqrt{n}}\ =\ \frac{\overline{X}_n-\mu}{\sigma/\sqrt{n}},
	$$
	
	the expectation and the standard deviation of $Z_n$ are respectively 0 and 1: the variable is thus said to be centered and reduced.
	
	The \textbf{central limit theorem} then states that the distribution of $Z_n$ converges in law to the reduced centered normal distribution $\mathcal{N} (0 , 1)$ as $n$ tends to infinity. This means that if $\Phi$ is the distribution function of $\mathcal{N} (0 , 1)$, then for any real number $z$ :
	$$
	\lim_{n \to \infty} \mbox{P}(Z_n \le z) = \Phi(z),
	$$
	or, equivalently :
	$$
	\lim_{n\to\infty}\mbox{P}\left(\frac{\overline{X}_n-\mu}{\sigma/\sqrt{n}}\leq z\right)=\Phi(z)
	$$
\end{f}
\hrule

\begin{f}[Multidimensional variables]
			
			
A probability law is said to be \textbf{multidimensional}, or $n-$dimensional, when the law describes several (random) values of a random phenomenon.
The multidimensional character thus appears during the transfer, by a random variable, of the probabilistic space $(\Omega,\mathcal{A})$ to a numerical space $E^n$ of dimension $n$.

Let $X$ be a random variable on the probability space $(\Omega, \mathcal A, \mathbb{P})$, with values in ${\mathbb{R}}^n$ equipped with the real Borel tribe product ${\mathcal {B}(\mathbb{R})}^{\otimes n}$.
The law of the random variable $X$ is the probability measure $\mathbb{P}_X$ defined by~:
$$
\mathbb{P}_X(B) = \mathbb{P}\big(X^{-1}(B)\big) = \mathbb{P}(X \in B).
$$
for everything $B \in {\mathcal B(\mathbb R)}^{\otimes n}$.

The Cramer-Wold theorem ensures that the ($n-$dimensional) law of this random vector is entirely determined by the (one-dimensional) laws of all linear combinations of these components:
$$\sum_{i = 1}^n a_i X_i\mbox{ for all }a_1, a_2, \dots, a_n$$


\end{f}
\hrule

\begin{f}[Marginal law]
The probability distribution of the $i^e$ coordinate of a random vector is called the $i^e$ marginal distribution. The \textbf{marginal distribution} $\mathbb{P}_i$ of $\mathbb{P}$ is obtained by the formula :
$$
\mathbb{P}_i(B) = \mathbb{P}_{X_i}(B) = \iint { \mathds{1}}_{\omega_i\in B} \mathbb{P}(\mathrm{d}(\omega_1,\dots,\omega_n)), \forall  B \in \mathcal B(\mathbb{R}).
$$
The marginal laws of an absolutely continuous law are expressed using their marginal densities.


The conditional density function $X_2$ given the value $x_1$ of $X_1$, can be written~:
$$
f_{X_2}(x_2 \mid X_1=x_1) = \frac{f_{X_1, X_2}(x_1,x_2)}{f_{X_1}(x_1)}, 
$$

$$
f_{X_2}(x_2 \mid X_1=x_1)f_{X_1}(x_1) = f_{X_1,X_2}(x_1, x_2) = f_{X_1}(x_1 \mid X_2=x_2)f_{X_2}(x_2). 
$$
\end{f}
\hrule

	\begin{f}[Independence]
$(X_1, X_2, \dots,X_n)$ is a family of \textbf{independent random variables} if one of the following two conditions is met :
$$
\forall (A_1,\dots,A_n)\in\mathcal{E}_1\times\dots\times\mathcal{E}_n
$$
$$
\mathbb{P}(X_1\in A_1\text{ and }X_2\in A_2\dots\text{ and }X_n\in A_n)\ =\ \prod_{i=1}^n\mathbb{P}(X_i\in A_i),
$$
we have equality
$$
\mathbb{E}\left[\prod_{i=1}^n\ \varphi_i(X_i)\right]\ =\ \prod_{i=1}^n\mathbb{E}\left[\varphi_i(X_i)\right],
$$
for any sequence of functions $\varphi_i$ defined on $(E_i,\mathcal{E}_i)$, with values in $\mathbb{R}$, as soon as the above expectations make sense.
$$
f_X(x)= \prod_{i=1}^{n}f_{X_i}(x_i)
$$
\end{f}
\hrule



\begin{f}[Perfect dependence in dimension 2]
Let $F_1,F_2$ be distribution functions $\mathbb{R}\rightarrow [0,1]$.

The \textbf{Fréchet classes} $\mathcal{F}_{(F_1,F_2)}$ group together the set of distribution functions $\mathbb{R}^2\rightarrow [0,1]$
whose marginal laws are precisely $F_1,F_2$.
	
For every $F \in \mathcal{F} (F_1,F_2)$, and for all $x$ in $\mathbb{R}^d$
$$
F^-(\boldsymbol{x})\leq F (\boldsymbol{x})\leq F^+(\boldsymbol{x})
$$
où $F^+(\boldsymbol{x}) = \min \{F_1(x_1),F_2(x_2)\}$, et 
$F^-(\boldsymbol{x}) = \max\{0,F_1(x_1) +F_2(x_2)-1\}$.



\begin{enumerate}
	\item The pair $\boldsymbol{X}=(X_1,X_2)$ is said to be comonotonic if and only if it admits $F^+$ as a distribution function.
	\item The pair $\boldsymbol{X}=(X_1,X_2)$ is said to be antimonotonic if and only if it admits $F^-$ as a distribution function.
\end{enumerate}

The pair $\boldsymbol{X}=(X_1,X_2)$ is said to be \textbf{comonotone} (\textbf{antimonotone}) if there exist non-decreasing (non-increasing) functions $g_1$ and $g_2$ of a random variable $Z$ such that
$$
\boldsymbol{X}=(g_1(Z),g_2(Z))
$$
\end{f}
\hrule

\begin{f}[The Gaussian vector]
A vector $X=(X_1,\cdots, X_n)$ is said to be a \textbf{Gaussian vector}, with law $\mathcal{N}(\boldsymbol{\mu}, \boldsymbol{\Sigma})$, when any linear combination $\sum_{j=1}^{n}\alpha_jX_j$ of its components is the univariate normal law. % (avec la convention de $\mathcal{N}(\mu, \sigma))$
In particular, each component $X_1,\cdots, X_n$ is of normal distribution.
\begin{tikzpicture}[scale=0.8]
	\def\mua{2}
	\def\mub{1.5}
	\def\sigmaa{1}
	\def\sigmab{1.4}
	\begin{axis}[
		colormap name  = whitetoblue,
		width          = 9cm,
		view           = {45}{65},
		enlargelimits  = false,
		grid           = major,
		domain         = -1:4,
		y domain       = -1:4,
		samples        = 26,
		xlabel         = {$x_1$},
		ylabel         = {$x_2$},
		zlabel         = {},
		colorbar,
		colorbar style = {
			at     = {(1.1,0)},
			anchor = south west,
			height = 0.25*\pgfkeysvalueof{/pgfplots/parent axis height},
			title  = {\ \ \ \ \ \ $f_{X_1,X_2}(x_1,x_2)$}
		}
		]
		\addplot3 [surf] {Normale2(\x,\y,\mua,\sigmaa,\mub,\sigmab,-0.6)};
		\addplot3 [domain=-1:4,samples=31, samples y=0, thick, smooth]
		(\x,4,{normal(\x,\mua,\sigmaa)});
		\addplot3 [domain=-1:4,samples=31, samples y=0, thick, smooth]
		(-1,\x,{normal(\x,\mub,\sigmab)});
		%		
		\draw [black!50] (axis cs:-1,0,0) -- (axis cs:4,0,0);
		\draw [black!50] (axis cs:0,-1,0) -- (axis cs:0,4,0);
		%
		\node at (axis cs:-1,4,-0.15) [pin=165:$f_{X_1}(x_1)$] {};
		\node at (axis cs:0,4,0.1) [pin=-15:$f_{X_2}(x_2)$] {};
	\end{axis}
\end{tikzpicture}


\begin{itemize}
	\item $\boldsymbol{\mu}$ of $\mathbb{R}^N$ its location,
	\item  $\boldsymbol{\Sigma}$ positive semi-definite of $\mathcal{M}_N(\mathbb{R})$, its variance-covariance.
\end{itemize} 

If $\boldsymbol{\Sigma}$ is well defined positive, therefore invertible, then
% $f_{\left(\boldsymbol{\mu},\boldsymbol{\Sigma}\right)} :\mathbb{R}^N \to \mathbb{R}$ :
$$
f_{\left(\boldsymbol{\mu},\boldsymbol{\Sigma}\right)}\left(\boldsymbol{x}\right)= \frac{1} {(2\pi)^{N/2} \left| \boldsymbol{\Sigma}\right|^{1/2}}e^{ -\frac{1}{2}\left(\boldsymbol{x}-\boldsymbol{\mu}\right)^\top\boldsymbol{\Sigma}^{-1}\left(\boldsymbol{x}-\boldsymbol{\mu}\right) }.
$$
where $\left| \boldsymbol{\Sigma}\right|$ is the determinant of $\boldsymbol{\Sigma}$.

\end{f}
\hrule

\begin{f}[Three measures of connection (correlations)]

The coefficient of \textbf{Pearson linear correlation} is called the value
$$
\rho_P = \frac{\sigma_{xy}}{\sigma_x \sigma_y}
$$
where $\sigma_{xy}$ denotes the covariance between the variables $x$ and $y$, and $\sigma_x$, $\sigma_y$ their standard deviation.
$\rho$ takes its values in $[-1,1]$ (application of the Cauchy-Schwartz theorem).

$X\bot Y \Rightarrow \rho_P=0$, Attention, $\rho_P=0 \nRightarrow X\bot Y $. 


 

\textbf{Kendall's tau} is defined by
$$
\tau_K=\mathbb{P}((X-X')(Y-Y')>0)-P((X-X')(Y-Y')<0)
$$
where $(X,Y)$  $(X',Y')$ are two independent pairs with the same joint density. This corresponds to the probability of the concordants reduced by that of the discordants :
\begin{align*}
\tau_K	=&\mathbb{P}\left(\operatorname{sgn}(X-X')=\mathbb{P}(\operatorname{sgn}(Y-Y')\right)-\\
		&\mathbb{P}\left(\operatorname{sgn}(X-X')\neq \operatorname{sgn}(Y-Y')\right)\\
=&\mathbb{E}\left[ \operatorname{sgn}(X-X')\operatorname{sgn}(Y-Y')\right]\\
=&\operatorname{Cov}(\operatorname{sgn}(X-X'),\operatorname{sgn}(Y-Y'))\\
=&4\mathbb{P}(X<X',Y<Y')-1 
\end{align*}

The correlation coefficient \textbf{Spearman's rho} of $(X,Y)$ is defined as the Pearson correlation coefficient of the ranks of the random variables $X$ and $Y$.
For a sample $n$, the $n$ values $X_i$, $Y_i$ are converted by their ranks $x_i$, $y_i$, and $\rho$ is calculated~:
$$
\rho_S = \frac{1/n\,\sum_i(x_i-\mathbb{E}[x])(y_i-\mathbb{E}[y])}{\sqrt{1/n\,\sum_i (x_i-\mathbb{E}[x])^2 \times 1/n\sum_i(y_i-\mathbb{E}[y])^2}}.
$$
Si on note $x_i= R(X_i)$ de 1 à $N$ et  $d_i = x_i - y_i$~:
$$    \rho_S = 1- {\frac {6 \sum_i d_i^2}{n(n^2 - 1)}}$$

\end{f}
\hrule

\begin{f}[Copula]
A \textbf{copula} is a distribution function, denoted $\mathcal{C}$, defined on $[0,1]^d$ whose margins are uniform on $[0,1]$. 
A characterization is then that $\mathcal{C}(u_1,...,u_d)=0$ if one of the components $u_i$ is zero, $\mathcal{C}(1,...,1,u_i,1,...,1)=u_i$, and $\mathcal{C}$ is $d-$increasing.
\medskip
		
Let $F^{(d)}$ be a distribution function in dimension $d$ where the $F_i$ are the marginal laws of $F$. 

\textbf{Sklar's theorem} states that $F^{(d)}$ has a copula representation~:
$$
F^{(d)} (x_1,...,x_d) = \mathcal{C} (F_1(x_1),...,F_d(x_d))
$$
If these marginal laws are all continuous, the copula $\mathcal{C}$ is then unique, and given by the relation 
$$
\mathcal{C}(u_1,...,u_d)=F^{(d)}(F_1^{-1} (u_1),...,F_d^{-1} (u_d))
$$
In this case, we can then speak of the copula associated with a random vector $(X_1,...,X_d)$.
This theorem is very important since we can separate the distribution margin part from the dependence part.
\medskip
	
The \textbf{Gaussian Copula} is a distribution on the unit cube of dimension $d$, $[0,1]^d$.
It is constructed on the basis of a normal law of dimension $d$ on $\mathbb{R}^d$.

Given the correlation matrix $\Sigma\in\mathbb{R}^{d\times d}$, the Gaussian copula with parameter $\Sigma$ can be written~:
$$
\mathcal{C}_\Sigma^{Gauss}(u) = \Phi_\Sigma\left(\Phi^{-1}(u_1),\dots, \Phi^{-1}(u_d) \right), 
$$
where $\Phi^{-1}$ is the inverse distribution function of the standard normal distribution and $\Phi_\Sigma$ is the joint distribution of a normal distribution of dimension $d$, with zero mean and covariance matrix equal to the correlation matrix $\Sigma$.
\medskip	

	A copula $\mathcal{C}$ is called \textbf{Archimedean} if it admits the following representation~:
$$
\mathcal{C}(u_1,\dots,u_d) = \psi^{-1}\left(\psi(u_1)+\dots+\psi(u_d)\right)\,
$$
where $\psi$ is then called \textbf{generator}.

Often, copulas admit an explicit formulation of $\mathcal{C}$. 
A single parameter allows to accentuate the dependence of the entire copula, whatever its dimension $d$.


This formula provides a copula if and only if $\psi\,$ is $d$-monotonic on $[0,\infty)$ \emph{i.e.} the $k^e$ derivative of $\psi\,$ satisfies
$$
(-1)^k\psi^{(k)}(x) \geq 0
$$
for all $x\geq 0$ and $k=0,1,\dots,d-2$ and $(-1)^{d-2}\psi^{d-2}(x)$ is non-increasing and convex.
\medskip

The following generators are all monotone, i.e. $d$-monotone for all $d\in\mathbb{N}$.

\footnotesize
\renewcommand\arraystretch{1.3}
\begin{tabular}{|m{10mm}|ccc|}	\rowcolor{BleuProfondIRA!40} 
	\hline
 Name 		& Generator $\psi^{-1}(t)$, 	& $\psi(t)$ &	Setting\\
	\hline
	Ali-Mikhail-Haq 	&$\frac{1-\theta}{\exp(t)-\theta} $	
	&$\log\left(\frac{1-\theta+\theta t}{t}\right)$ 	
	&$\theta\in[0,1)$\\
	Clayton		&$\left(1+\theta t\right)^{-1/\theta} 	$
	&$\frac1\theta\,(t^{-\theta}-1)\, 	$
	&$\theta\in(0,\infty)$\\
	Frank 		&$-\frac{1}{\theta}\exp(-t)$
	&$-\log\left(\frac{\exp(-\theta t)-1}{\exp(-\theta)-1}\right)$
	&$\theta\in(0,\infty)$\\
	&$\times\log(1-(1-\exp(-\theta)))$
	&
	&\\
	Gumbel 		&$\exp\left(-t^{1/\theta}\right) $	
	&$\left(-\log(t)\right)^\theta$	
	&$\theta\in[1,\infty)$\\
	$\perp$ 	&$\exp(-t)\,$
	&$-\log(t)\,$ 	
	& \\
	Joe		&$1-\left(1-\exp(-t)\right)^{1/\theta}$
	&$-\log\left(1-(1-t)^\theta\right)$
	&$\theta\in[1,\infty)$\\
	\hline
\end{tabular}
\renewcommand\arraystretch{1}
\end{f}
\hrule

\begin{f}[Brownian motion, filtration and martingales]
	
A \textbf{filtration} $(\mathcal{F}_t)_{t \geq 0}$ is an increasing family of $\sigma$-algebras or tribe representing the information available up to time $t$. A process $(X_t)$ is said to be \textbf{$\mathcal{F}_t$-adapted} if $X_t$ is measurable with respect to $\mathcal{F}_t$ for all $t$.
	
A process $(B_t)_{t \geq 0}$ is a \textbf{standard Brownian motion} (or Wiener process) if it verifies :
	\begin{itemize} 
		\item $B_0 = 0$ ;
		\item independent increments : $B_t - B_s$  independent of $\mathcal{F}_s$ ;
		\item stationary increments : $B_t - B_s \sim \mathcal{N}(0, t - s)$ ;
		\item trajectories continue almost surely.
	\end{itemize}
	
	A process $(M_t)$ is a \textbf{martingale} (with respect to $\mathcal{F}_t$) if :
	\[
	\mathbb{E}[|M_t|] < \infty \quad \text{et} \quad \mathbb{E}[M_t \mid \mathcal{F}_s] = M_s \quad \forall\, 0 \leq s < t
	\]
	Examples : Brownian motion, stochastic integrals of the form $\int_0^t \theta_s dB_s$ (under conditions) are martingales.
	\medskip
	
	\textbf{Quadratic variation} is denoted $
	\langle B \rangle_t = t, \quad \langle cB \rangle_t = c^2 t$
	
\textbf{Covariation} : for two Itô processes $X, Y$,
\[
\langle X, Y \rangle_t := \lim_{\|\Pi\| \to 0} \sum_{i} (X_{t_{i+1}} - X_{t_i})(Y_{t_{i+1}} - Y_{t_i})
\]
convergence in probability, where $\Pi = \{t_0 = 0 < t_1 < \dots < t_n = t\}$ is a partition of $[0,t]$.
\end{f}


\begin{f}[Itô's process and stochastic differential calculus]
	
A process $(X_t)$ is an \textbf{Itô process} if it can be written :
	\[
	X_t = X_0 + \int_0^t \phi_s\, ds + \int_0^t \theta_s\, dB_s
	\]
	or in differential
	\[dX_t = \phi_t dt + \theta_t dB_t\]
	with $\phi_t$, $\theta_t$ $\mathcal{F}_t$-adapted and $L^2$-integrable.
	
	\textbf{Itô's formula (1D)} : for $f \in C^2(\mathbb{R})$, we have :
	\[
	df(X_t) = f'(X_t) dX_t + \frac{1}{2} f''(X_t) d\langle X \rangle_t
	\]
	
	Example : if $dXt = \mu and + \sigma dBt$ then :
	\[
	dX_t^2 = 2X_t dX_t + d\langle X \rangle_t
	\]
	
	\textbf{Itô's formula (multi-dimensional)} :\\
	If $X = (X^1, \dots, X^d)$ is an Itô process, $f \in C^2(\mathbb{R}^d)$ :
	\[
	df(X_t) = \sum_i \frac{\partial f}{\partial x_i}(X_t) dX^i_t
	+ \frac{1}{2} \sum_{i,j} \frac{\partial^2 f}{\partial x_i \partial x_j}(X_t) d\langle X^i, X^j \rangle_t
	\]
	
	\textbf{Integration by parts (Itô)} :
	\[
	d(X_t Y_t) = X_t dY_t + Y_t dX_t + d\langle X, Y \rangle_t
	\]
	
\end{f}

\begin{f}[Stochastic Differential Equations (SDE)]
	
	An SDE is a stochastic equation of the form :
	\[
	dX_t = b(t, X_t) dt + a(t, X_t) dB_t, \quad X_0 = x
	\]
	or :
	\begin{itemize}
		\item $b(t,x)$ is the \textbf{drift} (\emph{drift}) : function $\mathbb{R}_+ \times \mathbb{R} \to \mathbb{R}$ ;
		\item $a(t,x)$ is the \textbf{diffusion} : function $\mathbb{R}_+ \times \mathbb{R} \to \mathbb{R}$ ;
		\item $B_t$ is a Brownian motion ;
		\item $X_t$ is the solution, adapted stochastic process.
	\end{itemize}
	
	\textbf{Integral form} :
	\[
	X_t = x + \int_0^t b(s, X_s) ds + \int_0^t a(s, X_s) dB_s
	\]
	
	\textbf{Conditions of existence and uniqueness} :
	\begin{itemize}
		\item \textbf{Lipschitz} : there exists $L > 0$ such that :
		\[
		|b(t,x) - b(t,y)| + |a(t,x) - a(t,y)| \leq L |x - y|
		\]
		\item \textbf{Linear growth} :
		\[
		|b(t,x)|^2 + |a(t,x)|^2 \leq C(1 + |x|^2)
		\]
	\end{itemize}
	
Classic examples :
	\begin{itemize}
		\item \textbf{Geometric Brownian} : $dS_t = \mu S_t dt + \sigma S_t dB_t$
		\item \textbf{Ornstein–Uhlenbeck} : $dX_t = \theta(\mu - X_t) dt + \sigma dB_t$
	\end{itemize}
	
	\textbf{Numerical methods} : Euler–Maruyama, Milstein.
	
\end{f}

\begin{f}[Risk-neutral probability]
	
	A probability $\mathbb{Q}$ is said to be \textbf{risk neutral} if, under $\mathbb{Q}$, 
	any asset $S_t$ has an updated price $ \frac{S_t}{B_t}$ which is a martingale
	where $(B_t)$ is the numerary (e.g. $B_t = e^{rt}$).
	
	The absence of arbitrage $\iff \exists\, \mathbb{Q} \sim \mathbb{P}$ such that the updated prices are martingales.
	This is the \textbf{fundamental theorem of asset pricing}.
	
	\textbf{Application} :\\
	Under $\mathbb{Q}$, the value at time $t$ of an asset giving a return $H$ at date $T$ is :
	\[
	S_t = B_t\, \mathbb{E}^\mathbb{Q} \left[ \left. \frac{H}{B_T} \right\|  \mathcal{F}_t \right]
	\]
	
	\textbf{Note} :\\
	The measure $\mathbb{Q}$ is equivalent to $\mathbb{P}$, but reflects a "risk-free" world, useful in valuation.
	
\end{f}
\newcolumn

%
%\begin{f}[Processus stochastiques en actuariat]
%	
%	\textbf{Brownien standard} $W_t$ :
%	\begin{itemize}
%		\item $W_0 = 0$ a.s.
%		\item $W_t$ est à accroissements indépendants et stationnaires
%		\item $W_t - W_s \sim \mathcal{N}(0, t-s)$ pour $t > s$
%		\item Trajectoires continues presque sûrement
%	\end{itemize}
%	
%	\textbf{Martingale} :
%	Un processus $(M_t)_{t \geq 0}$ adapté à une filtration $(\mathcal{F}_t)$ est une \textit{martingale} si :
%	\[
%	\mathbb{E}[|M_t|] < \infty \quad \text{et} \quad \mathbb{E}[M_t \mid \mathcal{F}_s] = M_s \quad \forall s < t
%	\]
%	
%	\textbf{Lemme d’Itô (dimension 1)} :\\
%	Soit $X_t$ une EDS : $\mathrm{d}X_t = \mu_t \,\mathrm{d}t + \sigma_t \,\mathrm{d}W_t$,\\
%	et $f : \mathbb{R} \to \mathbb{R}$ de classe $C^2$, alors :
%	\[
%	\mathrm{d}f(X_t) = f'(X_t) \,\mathrm{d}X_t + \frac{1}{2} f''(X_t) \,\sigma_t^2 \,\mathrm{d}t
%	\]
%	
%	\textbf{Équation différentielle stochastique (EDS)} :
%	\[
%	\mathrm{d}X_t = \mu(X_t, t)\,\mathrm{d}t + \sigma(X_t, t)\,\mathrm{d}W_t
%	\]
%	
%	\textbf{Exemple classique : processus de Black-Scholes} :
%	\[
%	\mathrm{d}S_t = \mu S_t\,\mathrm{d}t + \sigma S_t\,\mathrm{d}W_t
%	\quad \Rightarrow \quad
%	S_t = S_0 \exp\left((\mu - \frac{\sigma^2}{2})t + \sigma W_t\right)
%	\]
%	
%	\textbf{Mouvement brownien géométrique} :
%	\[
%	S_t = S_0 \exp(X_t), \quad \text{où } X_t = (\mu - \frac{\sigma^2}{2})t + \sigma W_t
%	\]
%	
%	\textbf{Lien avec Bachelier (1900)} :
%	Premier modèle de prix avec mouvement brownien additif :
%	\[
%	S_t = S_0 + \mu t + \sigma W_t
%	\quad \text{(modèle abandonné à cause de } S_t < 0 \text{ possible)}
%	\]
%	
%	\textbf{Lien avec Norbert Wiener} :\\
%	Formalisation rigoureuse du mouvement brownien comme processus stochastique à temps continu.
%	
%\end{f}
%\hrule

\begin{f}[Simulations]
	\tikz{
		\def\m{2};
		\def\s{1};
	}
	%
	
	The simulations allow in particular to approximate the expectation by the empirical average of the realizations $x_1,\ldots,x_n$:
	$$
	\frac{1}{n}(x_1+\ldots+x_n)\approx \int xdF(x)=\mathbb{E}[X]
	$$
	Then, under the TLC, we estimate the uncertainty or confidence interval based on the normal distribution:
	$$
	\left[\overline{x}-1,96\frac{S_n}{\sqrt{n}},\overline{x}+1,96\frac{S_n}{\sqrt{n}}\right]
	$$
	where $S_n$ unbiasedly estimates the variance of $X$~:
	$$
	S_n^2=\frac{1}{n-1}\sum_{i=1}^{n}(x_i-\overline{x})^2
	$$
	
	convergence is said to be in $\mathcal{O}(\frac{\sigma}{\sqrt{n}})$.
	This interval allows you to decide the number of simulations to be carried out.
\end{f}

\begin{f}[Pseudo-random generator on ${[}0,1{]}^d$]
The computer does not know how to roll the die ($\Omega=\{ \epsdice{1}, \epsdice{2}, \epsdice{3}, \epsdice{4}, \epsdice{5}, \epsdice{6} \}$).
It generates a pseudo-randomness, that is to say a deterministic algorithm which resembles a random event.
Generators usually produce a random number on $[0,1]^d$.
If the initial value (\emph{seed}) is defined or identified, the following draws are known and replicable.


The simplest algorithm is called the method of linear congruences.
$$
x_{n+1}=\Phi(x_n)= (a\times x_n+c) \mod m
$$ \small
each $x_n$ is an integer between 0 and $m-1$.
$a$ the multiplier, $c$ the increment, and $m$ the modulus of the form $2^p-1$, that is to say a Mersenne prime number ( $ p $ necessarily prime) :

Marsaglia generator: $a=69069, b=0, m=2^{32}$

Knuth\&Lewis generator: $a=1664525, b=1013904223, m=2^{32}$

Haynes Generator: $a=6364136223846793005, b=0, m=2^{64}$

%Le générateur de Tausworthe est une extension du générateur congruentiel linéaire qui consiste à ne  plus utiliser seulement $x_{╔n-1}$ pour fabriquer un nouvel élément mais plutôt un ensemble de valeurs précédentes
The Tausworth generator constitutes an 'autoregressive' extension:
$$
x_{n}=(a_{1}\times x_{n-1}+a_{2}\times x_{n-2}+\cdots +a_{k}\times x_{n-k}) \mod m \text{ with } n\geq k
$$
The generator period is $m^k-1$, with all $a_i$ relatively prime. If $m$ is of the form $2^p$ , machine computation times are reduced.


The default random generator is usually the Mersenne-Twister algorithm. It is based on a linear recurrence on a matrix $F_{2}$ (matrix whose values are in base 2, i.e. 0 or 1). 
Its name comes from the fact that the length of the period chosen is a Mersenne prime number.
\begin{enumerate}
	\item     its period is $2^{19937}-1 $
	\item     it is uniformly distributed over a large number of dimensions (623 for 32-bit numbers) ;
	\item     it is faster than most other generators ,
	\item     it is random regardless of the weight of the bit considered, and passes Diehard tests.
\end{enumerate}
\end{f}
%

\begin{f}[Simulate a random variable]
	
Simulating $X$ of any law $F_X$ often comes down to simulating $\left(p_i\right)_{i\in [1,n]}$ of law $\mathcal{U}ni(0,1)$.

\textbf{ If $F_X$ is invertible},  $x_i=F^{-1}_X(p_i)$ (or quantile function) delivers $\left(x_i\right)_{i\in [1,n]}$ a set of $n$ simulations of law $F_X$.
			
	\begin{center}
		%\tikzmath{
			%	\m = 2; 
			%	\s = 1;
			%}	
		
		\begin{tikzpicture}[xscale=.8,yscale=2]
			
			% define normal distribution function 'normaltwo'
			%	\def\normaltwo{\x,{4*1/exp(((\x-3)^2)/2)}}
			
			% input y parameter
			\def\z{3.5}
			\def\m{2};
			\def\s{1};
			
			% this line calculates f(y)
			\def\fz{normcdf(\z,\m, \s)}
			
			
			\draw[color=blue] plot [domain=-1:6] ({\x}, {normcdf(\x,\m, \s)})
			node[right] {\color{blue} $F_X(x)$};;
			
			% Add dashed line dropping down from normal.
			\draw[dashed] (0,{\fz}) node[left] {$p_i$};
			\draw[dashed,->]   (0,{\fz}) -- ({\z},{\fz}) -- ({\z},0) node[below] {$x_i=F^{-1}(p_i)$};
			
			% Optional: Add axis labels 
			\draw (6,0) node[below] {$x$};
			
			% Optional: Add axes
			\draw[->] (-1.2,0) -- (6.2,0) node[right] {};
			\draw[->] (0,0) -- (0,1.2) node[above] {};
			
		\end{tikzpicture}
	\end{center}	
	
If it is a discrete variable ($F^{-1}$ does not exist) $X_\ell=\min_{\ell} F(X_\ell)> p_i$, where $ \left(X_{\ell}\right)_{\ell}$ is the countable set of possible values, ordered in ascending order.

In the \textbf{change of variable} method, we assume that we know how to simulate a law $X$, and that there exists $\phi$ such that $Y=\varphi(X)$ follows a law $F_Y$. The natural example is that of $X\sim \mathcal{N}(0,1)$ and making the change $Y=\exp(X)$ to obtain Y which follows a lognormal law.

\textbf{The rejection method} is used in more complex cases, for example when $F^{-1}$ is not explicit or requires a lot of computation time.
Let $f$ be a probability density function. Assume that there exists a probability density $g$ such that :
$$
{\displaystyle \exists K>0\ ,\ \forall x\in \mathbb {R} \ ,\ f(x)\leq Kg(x)}
$$
We then simulate $Z$ according to the density law $g$, and ${\displaystyle Y\sim {\mathcal {U}}([0;Kg(Z)])} $.
Then the random variable ${\displaystyle X=\lbrace Z|Y\leq f(Z)\rbrace } $ follows the density law $f$.



\begin{tikzpicture}[domain=-4:10,scale=0.6]
	%   \draw[very thin,color=gray] (-0.1,-1.1) grid (3.9,3.9);
	\draw[->] (-4.2,0) -- (10.2,0) node[right] {$x$};
	\draw[->] (0,0) -- (0,6) node[above] {$f(x)$};
	\draw[BleuProfondIRA] (-4,0) -- (-4,6) ;
	\draw[BleuProfondIRA] (10,0) -- (10,6) ;
	\draw[name path=melange,color=OrangeProfondIRA,smooth] plot ({\x},{20*normal(\x,2,2)+10*normal(\x,6,1)}) node[below=.5,left] {$f(x)$ , normal law mixture};
	\draw[name path=g,color=BleuProfondIRA,smooth] plot  ({\x},{60*normal(\x,3,3)})   node[above=3.5,left] {$K\times g(x)$, normal distribution density };
	\tikzfillbetween[of=melange and g,on layer=main]{BleuProfondIRA, opacity=0.1};
\end{tikzpicture}

The performance of the algorithm depends on the number of rejections, represented by the blue area on the graph.
\end{f}
\hrule

%

\begin{f}[Monte Carlo methods]

Monte Carlo methods rely on the repeated simulation of random variables to approximate numerical quantities.

\textbf{Convergence} :
\begin{itemize}[nosep]
	\item By the \textbf{law of large numbers}, the estimator converges almost surely to the expected value.
	\item By the \textbf{central limit theorem}, the standard error is in $\mathcal{O}(N^{-1/2})$ :
	\[
	\sqrt{N}(\hat{\mu}_N - \mu) \xrightarrow{d} \mathcal{N}(0, \sigma^2)
	\]
	\item This slow convergence justifies the use of \textbf{convergence improvement} techniques.
\end{itemize}

\textbf{Variance reduction techniques} :
\begin{itemize}
	\item \textbf{Antithetical variables} : we simulate $X$ and $-X$ (or $1-U$ if $U \sim \mathcal{U}[0,1]$), then we average the results. Reduction is efficient if $f$ is monotone.
	\item \textbf{Control method} : if $\mathbb{E}[Y]$ is known, we simulate $(f(X), Y)$ and correct :
	\[
	\hat{\mu}_\text{corr} = \hat{\mu} - \beta(\bar{Y} - \mathbb{E}[Y])
	\]
	where $\beta$ optimal minimizes the variance.
	\item \textbf{Stratification} : we divide the simulation space into strata (subsets), and we simulate proportionally in each stratum.
	\item \textbf{Importance sampling} : we modify the simulation law to accentuate rare events, then we reweight :
	\[
	\mathbb{E}[f(X)] = \mathbb{E}^{Q}\left[f(X) \frac{\mathrm{d}P}{\mathrm{d}Q}(X)\right]
	\]
	used in particular to estimate the tails of the distribution (VaR, TVaR).
\end{itemize}

\end{f}
\begin{f}[The bootstrap]
	
	The \textbf{bootstrap} is a \textit{resampling} method for estimating the uncertainty of an estimator without assuming a parametric form for the underlying distribution.
	
	Let $\xi = (X_1, X_2, \ldots, X_n)$ be a sample of iid variables following an unknown distribution $F$. We seek to estimate a statistic $\theta = T(F)$ (e.g. mean, median, variance), via its empirical estimator $\hat{\theta} = T(\hat{F}_n)$.
	
	\begin{enumerate}
		\item We approximate $F$ by the empirical distribution function :
		\[
		\hat{F}_n(x) = \frac{1}{n} \sum_{k=1}^n \mathbf{1}_{\{X_k \le x\}}
		\]
		
		\item We generate $B$ bootstrap samples $\xi^{\ast(b)} = (X_1^{\ast(b)}, \ldots, X_n^{\ast(b)})$ by drawing \textbf{with replacement} from the initial sample.
		
		\item For each simulated sample, we calculate the estimate $T^{\ast(b)} = T(\hat{F}_n^{\ast(b)})$.
	\end{enumerate}
	
	The realizations $T^{\ast(1)}, \ldots, T^{\ast(B)}$ form an approximation of the distribution of the estimator $\hat{\theta}$.
	
	We can deduce from this:
	\begin{itemize}[nosep]
		\item an estimated bias : $\widehat{\text{bias}} = \overline{T^\ast} - \hat{\theta}$ ;
		\item a \textbf{confidence interval} at $(1-\alpha)$ : $[q_{\alpha/2},\ q_{1 - \alpha/2}]$ of the empirical quantiles of $T^{\ast(b)}$ ;
		\item an estimate of the \textbf{variance} : $\widehat{\mathrm{Var}}(T^\ast)$.
	\end{itemize}
	
	\textbf{Note}: Bootstrapping is particularly useful when the distribution of $T$ is unknown or difficult to estimate analytically.
	
\end{f}
\begin{f}[Parametric Bootstrap]
	
	The \textbf{parametric bootstrap} is based on the assumption that the data follow a parameterized family of laws $\{F_\theta\}$.
	
	Let $\xi = (X_1, \ldots, X_n)$ be an $iid$ sample according to an unknown $F_\theta$ distribution. We proceed as follows:
	\begin{enumerate}
		\item Estimate the parameter $\hat{\theta}$ from $\xi$ (e.g. by maximum likelihood).
		\item Generate $B$ samples $\xi^{\ast(b)}$ of size $n$, simulated according to the law $F_{\hat{\theta}}$.
		\item Calculate $T^{\ast(b)} = T(\xi^{\ast(b)})$ for each sample.
	\end{enumerate}
	
	This method approximates the distribution of the estimator $T(\xi)$ assuming the shape of $F$ is known. It is more efficient than the nonparametric bootstrap if the model assumption is well specified. The parametric bootstrap is faster, but inherits the biases of the model.
	
\end{f}

\begin{f}[Cross-validation]
	
	\textbf{Cross-validation} is a method for evaluating the predictive performance of a statistical model, used in particular in machine learning or pricing.
	
	\textbf{Principle} :
	\begin{itemize}
		\item Divide the data into $K$ blocks (or folds).
		\item For each $k = 1,\ldots,K$ :
		\begin{itemize}
			\item Train the model on the other $K-1$ blocks.
			\item Evaluate the performance (error, log-likelihood...) on the $k$-th block.
		\end{itemize}
		\item Aggregate the errors to obtain an overall estimate of out-of-sample performance.
	\end{itemize}
	
\end{f}

%
\begin{f}[Quasi-Monte Carlo methods]
	
	Quasi-Monte Carlo methods aim to accelerate the convergence of the expectation estimator without resorting to randomness. The typical error is of the order :
	\[
	\mathcal{O}\left( \frac{(\ln N)^s}{N} \right)
	\]
	where $N$ is the sample size and $s$ the dimension of the problem.
	
	These methods rely on the use of \textbf{low-discrepancy} sequences in $[0,1]^s$. The star discrepancy, denoted $D^*_N(P)$ for a set of points $P = \{x_1, \ldots, x_N\}$, measures the maximum difference between the proportion of points contained in rectangles \emph{anchored to the origin} and their volume. It is defined by :
	\[
	D^*_N(P) = \sup_{u \in [0,1]^s} \left| \frac{1}{N} \sum_{i=1}^N \mathbf{1}_{[0,u)}(x_i) - \lambda_s([0,u)) \right|
	\]
	with :
	\begin{itemize}
		\item $[0,u) = \prod_{j=1}^s [0, u_j)$ a rectangle anchored at the origin in $[0,1]^s$,
		\item $\mathbf{1}_{[0,u)}(x_i)$ the indicator of the membership of $x_i$ to this rectangle,
		\item $\lambda_s([0,u)) = \prod_{j=1}^s u_j$ the volume of this rectangle.
	\end{itemize}
	A low discrepancy means that the points are well distributed in space, which improves the convergence of the estimate.
	
	\medskip
	\textbf{Van der Corput sequence (dimension 1)} :
	Let $n$ be an integer. We write it in base $b$ :
	\[
	n = \sum_{k=0}^{L-1} d_k(n)\, b^k
	\]
	then we reverse the numbers around the decimal point to obtain :
	\[
	g_b(n) = \sum_{k=0}^{L-1} d_k(n)\, b^{-k-1}
	\]
	For example, for $b = 5$ and $n = 146$, we have $146 = (1\,0\,4\,1)_5$, so :
	\[
	g_5(146) = \frac{1}{5^4} + \frac{0}{5^3} + \frac{4}{5^2} + \frac{1}{5} = 0{,}3616
	\]
	
	\medskip
	\textbf{Halton sequence (dimension $s$)} :
	We generalize the van der Corput sequence using $s$ distinct prime integer bases $b_1, \dots, b_s$ :
	\[
	x(n) = \big( g_{b_1}(n), \dots, g_{b_s}(n) \big)
	\]
	This construction provides a sequence of points well distributed in $[0,1]^s$.
	
	\medskip
	\textbf{Koksma–Hlawka inequality}\\
	For a function $f$ of finite variation $V(f)$ (in the Hardy–Krause sense) on $[0,1]^s$ :
	\[
	\left| \int_{[0,1]^s} f(u)\, du - \frac{1}{N} \sum_{i=1}^N f(x_i) \right| \leq V(f)\, D_N
	\]
	where $D_N$ is the discrepancy of the sequence used.
	
This bound explains why quasi-Monte Carlo methods are often more efficient than Monte Carlo methods.
	
	
\end{f}

\end{multicols}

\newpage
\newpage
\renewcommand{\arraystretch}{2.6}
\begin{center}
    
\begin{tabular}{clll}
\hline\rowcolor{BleuProfondIRA!40}  Distribution & Density \& support & \begin{tabular}{c}
       Moments  \& \\ 
        distribution function
    \end{tabular} &     \begin{tabular}{c}
        Moment-generating \\ 
        function 
    \end{tabular}  \\
\hline 
\(\begin{gathered}
\mathcal{B}in(m,q)\\
(0<p<1, m \in \mathbb{N})
\end{gathered}\) & 
\(\begin{gathered}
\binom{m}{x} p^x(1-p)^{m-x} \\
x=0,1, \ldots, m
\end{gathered}\)
 & \(
\begin{aligned}
& E=m p, \operatorname{Var}=m p(1-p) \\
& \gamma=\frac{m p(1-p)(1-2 p)}{\sigma^3}
\end{aligned}
\) & \(\displaystyle
\left(1-p+p e^t\right)^m
\) \\
\hline \(\mathcal{B}er(q)\)  & \(\equiv \operatorname{Binomial}(1, p)\) & & \\
\hline
\(\begin{gathered}
   \mathcal{D}\mathcal{U}ni(n) \\
(n>0)
\end{gathered}\)  & \(\displaystyle
 \frac{1}{n}, x=0,1, \ldots n
\) & \(
\begin{aligned}
& \mathbb{E}=(n+1)/2 \\
&\operatorname{Var}=\left(n^{2}-1\right)/{12} 
\end{aligned}
\) & \(\displaystyle\frac{e^t (1 - e^{nt})}{n(1 - e^t)}\)
\\\hline
\(\begin{gathered}
    \mathcal{P}ois(\lambda)\\
(\lambda>0)
\end{gathered}\)  & \(\displaystyle
e^{-\lambda} \frac{\lambda^x}{x!}, x=0,1, \ldots
\) & \(
\begin{aligned}
& \mathbb{E}=\operatorname{Var}=\lambda \\
& \gamma=1 / \sqrt{\lambda} \\
& \kappa_j=\lambda, j=1,2, \ldots
\end{aligned}
\) & \(\exp \left[\lambda\left(e^t-1\right)\right]\) \\
\hline \(
\begin{aligned}
\mathcal{N}\mathcal{B}in(m,q)\\
(m>0,0<p<1)
\end{aligned}\)
 & \(
\begin{gathered}
\binom{m+x-1}{x} p^m(1-p)^x \\
x=0,1,2, \ldots
\end{gathered}
\) & \(
\begin{aligned}
& \mathbb{E}=m(1-p) / p \\
& \mathrm{Var}=\mathbb{E} / p \\
& \gamma=\frac{(2-p)}{p \sigma}
\end{aligned}
\) & \(\displaystyle
\left(\frac{p}{1-(1-p) e^t}\right)^m
\) \\
\hline
\(\mathcal{G}eo(q)\) & \(\equiv\mathcal{N}\mathcal{B}in(1,q)\)
 \\ \hline
\(
\begin{aligned}
& \mathcal{C}\mathcal{U}ni(a,b) \\
& (a<b)
\end{aligned}
\) & \(\displaystyle
\frac{1}{b-a} ; a<x<b
\) & \(
\begin{aligned}
&\mathbb{E}=(a+b) / 2, \\
&\operatorname{Var}=(b-a)^2 / 12,  \\
&\gamma=0 
\end{aligned}
\) & \(\displaystyle\frac{e^{b t}-e^{a t}}{(b-a) t}\)\\
\hline \(
\begin{aligned}
& \mathcal{N}\left(\mu, \sigma^2\right) \\
& (\sigma>0)
\end{aligned}
\) & \(\displaystyle
\frac{1}{\sigma \sqrt{2 \pi}} \exp \frac{-(x-\mu)^2}{2 \sigma^2}
\) & \(
\begin{aligned}
& \mathbb{E}=\mu, \mathrm{Var}=\sigma^2, \gamma=0 \\
& \left(\kappa_j=0, j \geq 3\right)
\end{aligned}
\)& \(\exp \left(\mu t+\frac{1}{2} \sigma^2 t^2\right)\) \\
\hline \(
\begin{aligned}
& \mathcal{G}am(k,\theta) \\
& (k, \theta>0)
\end{aligned}
\) & \(\displaystyle
\frac{\theta^k}{\Gamma(k)} x^{k-1} e^{-\theta x}, x>0
\) & \(
\begin{aligned}
& \mathbb{E}=k / \theta, \mathrm{Var}=k / \theta^2, \\
& \gamma=2 / \sqrt{k}
\end{aligned}
\) &\(\displaystyle \left(\frac{\theta}{\theta-t}\right)^k(t<\theta)\)\\
\hline  \(\mathcal{E}_{xp}(\lambda)\) & \(\equiv \mathcal{G}am(1,\lambda )\) & 
\(\begin{aligned}
& \mathbb{E}=1/\lambda\\
& \operatorname{Var}=1/\lambda^2
\end{aligned}
\)\\
\hline \(\chi^2(k)(k \in \mathbb{N})\) & \(\equiv \mathcal{G}am(  k / 2,1 / 2)\) & \\
\hline 
\(
\begin{gathered}
\mathcal{I}\mathcal{N}(\alpha, \beta) \\
(\alpha>0, \beta>0)
\end{gathered}
\) & \(\displaystyle
\frac{\alpha x^{-3 / 2}}{\sqrt{2 \pi \beta}} \exp \left(\frac{-(\alpha-\beta x)^2}{2 \beta x}\right) 
\) & \multicolumn{2}{l}{\(
\begin{array}{ll}
\mathbb{E}=\alpha / \beta, \mathrm{Var}=\alpha / \beta^2, & e^{\alpha(1-\sqrt{1-2 t / \beta})} \\
\gamma=3 / \sqrt{\alpha} & (t \leq \beta / 2) 
\end{array}
\) }\\
& \multicolumn{3}{l}{
\(F(x)=\Phi\left(\frac{-\alpha}{\sqrt{\beta x}}+\sqrt{\beta x}\right)
+e^{2 \alpha} \Phi\left(\frac{-\alpha}{\sqrt{\beta x}}-\sqrt{\beta x}\right),  x>0\)}\\
\hline \(
\begin{aligned}
& \mathcal{B}eta(\alpha, \beta) \\
& (\alpha>0,\beta>0)
\end{aligned}
\) & \(\displaystyle
\Gamma(\alpha+\beta)\frac{x^{(\alpha-1}(1-x)^{\beta-1}}{\Gamma(\alpha)\Gamma(\beta) }, 0<x<1
\) & \(\displaystyle
\mathbb{E}=\frac{\alpha}{\alpha+\beta}, \operatorname{Var}=\frac{\mathbb{E}(1-\mathbb{E})}{\alpha+\beta+1}
\) \\
\hline \(
\begin{aligned}
& \mathcal{L}\mathcal{N}orm\left(\mu, \sigma^2\right) \\
& (\sigma>0)
\end{aligned}
\) & \(\displaystyle
\frac{1}{x \sigma \sqrt{2 \pi}} \exp \frac{-(\log x-\mu)^2}{2 \sigma^2},x>0
\) & \multicolumn{2}{l}{\(
\begin{aligned}
& \mathbb{E}=e^{\mu+\sigma^2 / 2}, \operatorname{Var}=e^{2 \mu+2 \sigma^2}-e^{2 \mu+\sigma^2} \\
& \gamma=c^3+3 c \text { où } c^2=\operatorname{Var} / \mathbb{E}^2
\end{aligned}
\) }\\
\hline \(
\begin{aligned}
& \mathcal{P}areto\left(\alpha, x_\mathrm{m}\right) \\
& \left(\alpha, x_\mathrm{m}>0\right)
\end{aligned}
\) & \(\displaystyle
\frac{\alpha x_\mathrm{m}^\alpha}{x^{\alpha+1}}, x>x_\mathrm{m}
\) & \multicolumn{2}{l}{\(\displaystyle
\mathbb{E}=\frac{\alpha x_\mathrm{m}}{\alpha-1} \quad \alpha>1, \mathrm{Var}=\frac{\alpha x_\mathrm{m}^2}{(\alpha-1)^2(\alpha-2)} \quad \alpha>2
\)} \\\hline 
\(
\begin{aligned}
    \mathcal{W}eibull (\alpha, \beta)  \\
(\alpha, \beta>0)     
\end{aligned}\)
 & \(\alpha \beta(\beta y)^{\alpha-1} e^{-(\beta y)^\alpha}, x>0\) & \multicolumn{2}{l}{\(
\begin{aligned}
& \mathbb{E}=\Gamma(1+1 / \alpha) / \beta \\
& \mathrm{Var}=\Gamma(1+2 / \alpha) / \beta^2-\mathbb{E}^2 \\
& \mathbb{E}\left[Y^t\right]=\Gamma(1+t / \alpha) / \beta^t
\end{aligned}
\)} \\
\hline
\end{tabular}
\end{center}
\renewcommand{\arraystretch}{1}


\newpage

\begin{center}
    \section*{(Micro)-Economics of Insurance}
    \medskip
\end{center}


\begin{multicols}{2}
	% !TeX root = ActuarialFormSheet_MBFA-en.tex
% !TeX spellcheck = fr_FR
\begin{f}[Concept d'utilité]
	L'utilité modélise les préférences d'un individu entre deux paniers de biens \(x\) et \(y\) dans un ensemble \(S\), via la relation \(x \succcurlyeq y\) (préféré ou indifférent).
	
	Une fonction \(U: S \rightarrow \mathbb{R}\) représente les préférences si :
	\[
	x \succcurlyeq y \iff U(x) \geq U(y)
	\]
	
	\textbf{Axiomes nécessaires à l'existence d'une fonction d'utilité :}
	\begin{enumerate}
		\item \textbf{Complétude :} Pour tout \(x, y \in S\), soit \(x \succcurlyeq y\), soit \(y \succcurlyeq x\)
		\item \textbf{Transitivité :} Si \(x \succcurlyeq y\) et \(y \succcurlyeq z\), alors \(x \succcurlyeq z\)
		\item \textbf{Continuité :} Si \(x_n \to x\) et \(y_n \to y\), et \(x_n \succcurlyeq y_n\) pour tout \(n\), alors \(x \succcurlyeq y\)
	\end{enumerate}
	
\end{f}


\begin{f}[Fonction d'utilité]
	Une fonction \(u : \mathbb{R}_+ \rightarrow \mathbb{R}\) représente les préférences d'un agent face à l'incertitude.
	
	\textbf{Critère d'utilité espérée :} L’agent préfère \(X\) à \(Y\) si :
	\[
	\mathbb{E}[u(X)] > \mathbb{E}[u(Y)]
	\]
	Il choisit \(X\) tel que \(\mathbb{E}[u(X)]\) soit maximale.
	
	\textbf{Propriétés de \(u\) :}
	\begin{itemize}
		\item \(u'\! > 0\) : l’agent préfère plus de richesse (monotonie)
		\item \(u'' < 0\) : l’agent est averse au risque (concavité)
	\end{itemize}
	
	\textbf{Exemples classiques :}
	\begin{itemize}
		\item \textit{Linéaire} (neutre au risque) : \(u(x) = x\)
		\item \textit{Logarithmique} : \(u(x) = \ln(x)\)
		\item \textit{CRRA}  (aversion relative constante): \(u(x) = \frac{x^{1 - \gamma}}{1 - \gamma}\), \(\gamma \ne 1\)
		\item \textit{CARA}  (aversion absolue constante): \(u(x) = -e^{-a x}\)
	\end{itemize}
	
\end{f}

\hrule

\begin{f}[Aversion au risque]
	Un agent est dit \textbf{averse au risque} (ou risquophobe) si :
	\[
	u(\mathbb{E}[X]) > \mathbb{E}[u(X)]
	\]
	Ce qui est équivalent à \(u\) concave, c.-à-d. \(u''(x) < 0\)
\end{f}

\begin{f}[Mesure de l'aversion au risque]
	\textbf{Indice d'aversion absolue} :
	\[
	A_a(x) = -\frac{u''(x)}{u'(x)}
	\]
	
	\textbf{Indice d'aversion relative} :
	\[
	A_r(x) = -x \cdot \frac{u''(x)}{u'(x)}
	\]
	
	\textbf{Inégalité de Jensen (cas concave)} :
	\[
	u(\mathbb{E}[X]) \geq \mathbb{E}[u(X)]
	\]
Avec l'égalité si et seulement si \(X\) est constante.
\end{f}
\hrule


\begin{f}[Primes de risque]
	
La \textbf{prime de risque} \(\pi\) est le montant maximal qu'un individu est prêt à payer pour remplacer une loterie de gain aléatoire \(H\) par son espérance certaine \(\mathbb{E}[H]\). Elle vérifie :
\[
\mathbb{E}[u(w + H)] = u(w + \mathbb{E}[H] - \pi)
\]


\(\pi\) est aussi appelée \emph{mesure de Markowitz} : elle capture l'écart entre utilité espérée et utilité certaine.

Inversement, la \textbf{prime compensatoire} \(\tilde{\pi}\) est le montant que l'on doit offrir à un individu pour qu’il accepte la loterie \(H\) au lieu d’un gain certain. Elle vérifie :
\[
\mathbb{E}[u(w + H + \tilde{\pi})] = u(w + \mathbb{E}[H])
\]

\end{f}
\hrule
\begin{f}[Diversification et utilité]
	
	Soient deux actifs :
	\begin{itemize}
		\item \(A\) : risqué
		\item \(B\) : certain, avec \(\mathbb{E}[A] = B\)
	\end{itemize}
	
	Un agent averse au risque préfère une combinaison \(Z = \alpha A + (1 - \alpha) B\), avec \(0 < \alpha < 1\), à l’actif risqué seul.
	Si  $u$ est concave, alors
	\[
	\mathbb{E}[u(Z)] > \mathbb{E}[u(A)]
	\]
	
	\textbf{Portefeuille optimal} : choix des pondérations \((w_i)\) maximisant l'utilité espérée :
	\[
	\max \mathbb{E}[u(X)], \quad \text{où } X = \sum_{i} w_i X_i, \quad \text{s.c. } \sum w_i = 1
	\]
	
	\textbf{Principe} : la diversification réduit le risque (variance) sans affecter l’espérance.
	
	
\end{f}
\hrule

\begin{f}[Méthode de Lagrange pour l'optimisation sous contrainte]
	
	La méthode des multiplicateurs de Lagrange permet de résoudre un problème d'optimisation sous contrainte.
	
	\textbf{Objectif} : maximiser/minimiser \(f(\boldsymbol{x})\) sous la contrainte \(g(\boldsymbol{x}) = c\), où \(\boldsymbol{x} \in \mathbb{R}^d\) est un vecteur de variables.
	
	\textbf{Étapes de la méthode :}
	\begin{enumerate}
		\item \textbf{Identification} : déterminer la fonction objectif \(f(\boldsymbol{x})\) et la contrainte \(g(\boldsymbol{x}) = c\)
		\item \textbf{Lagrangien} :
		\[
		\mathcal{L}(\boldsymbol{x}, \lambda) = f(\boldsymbol{x}) + \lambda (g(\boldsymbol{x}) - c)
		\]
		\item \textbf{Système d'équations} : résoudre
		\[
		\nabla_{\boldsymbol{x}} \mathcal{L} = \nabla f(\boldsymbol{x}) + \lambda \nabla g(\boldsymbol{x}) = \boldsymbol{0}, \quad
		\frac{\partial \mathcal{L}}{\partial \lambda} = g(\boldsymbol{x}) - c = 0
		\]
		\item \textbf{Résolution} du système pour obtenir \(\boldsymbol{x}^*, \lambda^*\)
		\item \textbf{Vérification} : s'assurer que les solutions satisfont bien la contrainte et le type d'optimum (max/min)
	\end{enumerate}
	
	\textbf{Exemple (dimension 2)} : maximiser \(f(x, y) = xy\) sous la contrainte \(x + y = 10\)
	
	\[
	\mathcal{L}(x, y, \lambda) = xy + \lambda (x + y - 10)
	\]
	
	On dérive :
	\[
	\frac{\partial \mathcal{L}}{\partial x} = y + \lambda = 0, \quad
	\frac{\partial \mathcal{L}}{\partial y} = x + \lambda = 0, \quad
	\frac{\partial \mathcal{L}}{\partial \lambda} = x + y - 10 = 0
	\]
	
	On résout le système :
\[
\begin{cases}
	y + \lambda = 0 \\
	x + \lambda = 0 \\
	x + y = 10
\end{cases}
\Rightarrow 
\begin{cases}
	\lambda = -y \\
	x = -\lambda = y \\
	x + y = 10 \Rightarrow 2x = 10 
\end{cases}
\Rightarrow 
	\begin{cases} 
		x^* =  y^*= 5,\\
		f(5,5) = 25
	\end{cases}
\]

		\textbf{Exemple (Choix optimal et contrainte budgétaire)}
	
	Un agent rationnel est face à un choix de consommation \((c_1, c_2)\) entre deux biens, sous la contrainte :
	\[
	p_1 c_1 + p_2 c_2 = R
	\]
	où \(p_1, p_2\) sont les prix et \(R\) le revenu total.
	
	\textbf{Problème} :
	\(
	\max_{c_1, c_2} u(c_1, c_2) \quad \text{s.c. } p_1 c_1 + p_2 c_2 = R\)

	
	\textbf{Méthode} : introduire le \textbf{lagrangien}
	\[
	\mathcal{L}(c_1, c_2, \lambda) = u(c_1, c_2) + \lambda (R - p_1 c_1 - p_2 c_2)
	\]
	
	\textbf{Conditions du premier ordre (FOC)} :
	\[
	\begin{cases}
		\frac{\partial u}{\partial c_1} = \lambda p_1 \\
		\frac{\partial u}{\partial c_2} = \lambda p_2 \\
		p_1 c_1 + p_2 c_2 = R
	\end{cases}
	\]
	
	En divisant les deux premières équations :
	\[
	\frac{\partial u / \partial c_1}{\partial u / \partial c_2} = \frac{p_1}{p_2}
	\]
	
	Ce rapport est appelé \textbf{taux marginal de substitution (TMS)} : il mesure la quantité de bien 2 à laquelle l’agent est prêt à renoncer pour obtenir une unité supplémentaire de bien 1, tout en maintenant son niveau d’utilité constant.
	
\end{f}
\hrule

\begin{f}[Demande d’assurance (Mosin)]
	
Un agent possède un patrimoine initial \(w\) et fait face à une perte aléatoire \(L\). Il existe une \textbf{demande d'assurance} pour l'assurance qui verse l'indemnité $0<I(L)<L$ ssi $u(w-\pi_I) \geq \mathbb{E}(u(w-L))$ et l'\textbf{assurance optimale} maximise $u(w-\pi_I)$.

Dans Mosin (1968) ou  Borch (1961) ou Smith (1968), le modèle de perte $L$ se définit simplement par  $s$ compris entre 0 et  \(w\) :
	$$
	L=\left\{\begin{array}{l}
		0 \text { avec proba. } 1-p \\
		s \text { avec proba. } p
	\end{array}\right.
	$$
La prime devient $\pi_I=(1+\lambda) \mathbb{E}(I(L))=(1+\lambda) p I(s)$ avec $\lambda$ un chargement. On note $\pi$ le cas où $I(L)=L$ avec  $\pi= p s$. Si $\lambda=0$, alors on parle de prime pure ou actuariellement juste. 
\medskip

\textbf{Coassurance} (partage du risque) :
 $I(l)=\alpha l$ sachant $L=l$ pour $\alpha \in[0,1]$, $\pi_I(\alpha)=\alpha \pi$ et :
$$
w_{f}=w-L+I(L)-\pi(\alpha)=w-L+\alpha L-\alpha \pi=w-(1-\alpha) L-\alpha \pi
$$
$$
U(\alpha)%=\mathbb{E}(u(w-(1-\alpha) L-\alpha \pi))
=(1-p) u(w-\alpha \pi)+p u(w-(1-\alpha) s-\alpha \pi)
$$

L'assurance partielle ($\alpha^{\star}<1$) est optimale ssi $\lambda>0$. L'assurance totale ($\alpha^{\star}=1$ ) est optimale si le chargement est nul.
\medskip

\textbf{L'assurance avec franchise} (part d'autoassurance):
Avec franchise $d$ l'assureur verse une indemnité $I(l)=(l-d)_{+}$ sachant $L=l$. 
$$
\pi(d)=(1+\lambda) E\left((L-d)_{+}\right)=(1+\lambda)(s-d) p
$$
$$
w_{f}=w-X+(L-d)_{+}-\pi(d)=w-\min (X, d)-(1+\lambda)(s-d) p
$$
$$
U(d)=(1-p) u(\underbrace{w+(1+\lambda)(d-s) p}_{w_{f}^{+}})+p u(\underbrace{w-d+(1+\lambda)(d-s) p}_{w_{f}^{-}}) .
$$
Dans le modèle avec franchise, l'assurance partielle ($d^{\star}>0$) est optimale ssi la prime n'est pas actuariellement juste. De même,  l'assurance totale ( $d^{\star}=0$ ) est optimale si le chargement est nul.
\medskip

\textbf{Modèle généralisé :}
Le risque de perte $L>0$ aléatoire est définit sur $\Re$, avec fonction de répartition $F_L$ ), 

$$
\pi_I=(1+\lambda) \mathbb{E}(I(L))=(1+\lambda) \int_{0}^{\infty} I(l) d F_{L}(l)
$$



\begin{enumerate}
	\item L'assurance totale \((d^\star = 0)\) ou  \((\alpha^\star = 1)\) est optimale si et seulement si la prime est actuariellement juste .
	
	\item Si \(A_a(u, x)\) est décroissante, alors le niveau de franchise \(d^\star\)  ou  le taux de couverture \(\alpha^\star\) augmente avec la richesse initiale.  
	Pour les préférences CARA, \(d^\star\) est indépendant de \(w\) ou \(\alpha^\star\) est constant .
	
	\item Le niveau de couverture  décroît avec le coefficient de chargement \(\lambda\) lorsque \(A_a(u, x)\) est croissante ou constante.
	
	\item Un agent plus averse au risque choisit  une couverture plus élevée.
\end{enumerate}
\end{f}

\hrule

\begin{f}[Information et assurance]
	
	\textbf{Mosin avec hétérogénéïté :} Deux types d'individus : $H$ pour haut risque et $Lo$ pour faible risque. $\theta \in[0,1]$ la proportion d'individu $H$. Les individus de type $i \in\{Lo, H\}$ font face à un risque de même montant $s$  survenant avec une probabilité différente $p_{i}$ telle que $1>p_{H}>p_{Lo}>0$.
	
	$$
	L_{i}=\left\{\begin{array}{l}
		0 \text { avec probabilité } 1-p_{i}, \\
		s \text { avec probabilité } p_{i} .
	\end{array}\right.
	$$
La probabilité du marché :
	$$
	p_{m}=\theta p_{H}+(1-\theta) p_{Lo} .
	$$
\medskip

\textbf{Absence d'antisélection :}  Dans ce modèle, en présence d'information totale, l'assureur préfère une assurance individuelle $I_{i}=s$ et $\pi_{i}=s p_{i}$, $\forall i$.

\textbf{ Le problème d'anti-sélection : }
L'assureur propose un  contrat non individualisé du marché $M=\left(\pi_{m}=p_{m} I, I_{m}(s)=I(s)\right)$, qui ne dépend pas de $i$. La fortune finale d'un individu de type $i$ est $W_{i}^{m}=w-\pi_{m}-X_{i}+I_{m}$.

En présence d'un contrat unique, les individus de type $H$ préfèrent un contrat d'assurance tel que $I_{H}(s)=s$  et $\pi_{H}=s p_{m}$, tandis que les individus de type $Lo$ préfère une couverture partielle avec $I_{L}^{\star}<s$ et $\pi_{Lo}=I_{L}^{\star} p_{m}$.
\medskip

\textbf{Aléa moral :}
L'assurer réduit ou interrompt ses efforts maintenant qu'il est assuré. Les efforts de
\begin{itemize}
	\item prévention réduisent la probabilité de sinistre,
	\item protection réduisent le montant de sinistre.
\end{itemize}

 En l'absence d'effort $e$ de prévention ou de réduction du risque, la fortune finale $w_{f}$ est simplement définie par

$$
\begin{cases}w_{f}^{-}=w-s-\pi(I)+I & \text { avec probabilité } p \\ w_{f}^{+}=w-\pi(I) & \text { avec probabilité } 1-p\end{cases}
$$

S'il y a prévention de risque d'effort $e$, on considère

$$
\begin{cases}w_{f}^{-}=w-s-\pi(I)+I-e & \text { avec probabilité } p(e) \\ w_{f}^{+}=w-\pi(I)-e & \text { avec probabilité } 1-p(e)\end{cases}
$$

S'il y a réduction de risque d'effort $e$, on considère

$$
\begin{cases}w_{f}^{-}=w-s(e)-\pi(I)+I-e & \text { avec probabilité } p \\ w_{f}^{+}=w-\pi(I)-e & \text { avec probabilité } 1-p\end{cases}
$$

avec
\begin{itemize}
	\item $e \mapsto p(e)$ est strictement décroissante et strictement convexe.
	\item $e \mapsto s(e)$ est strictement décroissante et strictement convexe.\\
	\item $I \leq s$ implique $w_{f}^{-} \leq w_{f}^{+}$
\end{itemize}

\end{f}
\hrule



\end{multicols}

\newpage
\begin{center}
	\section*{Econometrics \& Time Series}
	\medskip
\end{center}


\begin{multicols}{2}
	% !TeX root = ActuarialFormSheet_MBFA-en.tex
% !TeX spellcheck = en_US
%\setlength{\parskip}{0cm} % paragraph spacing
%		\section*{Concepts de base}

	 \begin{f}[Definitions ]

\textbf{Time series} - is a succession of quantitative observations of a phenomena ordered in time.
	
There are some variations of time series :
		\begin{itemize}[leftmargin=*]	
		\item \textbf{Panel data} - consist of a time series for each observation of a cross section.
		\item \textbf{Pooled cross sections} - combines cross sections from different time periods.
	\end{itemize}	
%	
		 \textbf{Stochastic process} - sequence of random variables that are indexed in time.
	\end{f}    
	
	\begin{f}[Components of a time series]{\ }

	\begin{itemize}[leftmargin=*]
		\item \textbf{Trend} - is the long-term general movement of a series.
		\item \textbf{Seasonal variations} - are periodic oscillations that are produced in a period equal or inferior to a year, and can be easily identified on different years (usually are the result of climatology reasons).
		\item \textbf{Cyclical variations} - are periodic oscillations that are produced in a period greater than a year (are the result of the economic cycle).
		\item \textbf{Residual variations} - are movements that do not follow a recognizable periodic oscillation (are the result of eventual phenomena).
	\end{itemize}
	
	\end{f}    
	
	\begin{f}[Type of time series models]{\ }
		

	\begin{itemize}[leftmargin=*]
		\item \textbf{Static models} - the relation between $y$ and $x$ is contemporary. Conceptually :
\[y_{t} = \beta_{0} + \beta_{1} x_{t} + u_{t}\]
		
		\item \textbf{Distributed-lag models} - the relation between $y$ and $x$ is not contemporary. Conceptually :
\[y_{t} = \beta_{0} + \beta_{1} x_{t} + \beta_{2} x_{t - 1} + \cdots + \beta_{s} x_{t - (s - 1)} + u_{t}\]
		
		The long term cumulative effect in $y$ when $\Delta x$ is :
			$\beta_{1} + \beta_{2} + \cdots + \beta_{s}$
		
		\item \textbf{Dynamic models} - lags of the dependent variable (endogeneity). Conceptually :
\[y_{t} = \beta_{0} + \beta_{1} y_{t - 1} + \cdots + \beta_{s} y_{t - s} + u_{t}\]
		
		\item Combinations of the above, like the rational distributed-lag models (distributed-lag + dynamic).
	\end{itemize}	

\end{f} 
 \hrule 
 
 \begin{f}[OLS model assumptions under time series]

Under this assumptions, the OLS estimator will present good properties. \textbf{Gauss-Markov assumptions} extended applied to time series :

\begin{enumerate}[leftmargin=*, label=t\arabic{*}.]
	\item \textbf{Parameters linearity and weak dependence}.
	
	\begin{enumerate}[leftmargin=*, label=\alph{*}.]
		\item $y_{t}$ must be a linear function of the $\beta$'s.
		\item The stochastic $\lbrace( x_{t}, y_{t}) : t = 1, 2, \ldots, T \rbrace$ is stationary and weakly dependent.
	\end{enumerate}
	
	\item \textbf{No perfect collinearity}.
	
	\begin{itemize}[leftmargin=*]
		\item There are no independent variables that are constant : $\Var(x_{j}) \neq 0, \; \forall j = 1, \ldots, k$
		\item There is not an exact linear relation between independent variables.
	\end{itemize}
	
	\item \textbf{Conditional mean zero and correlation zero}.
	
	\begin{enumerate}[leftmargin=*, label=\alph{*}.]
		\item There are no systematic errors : $\E(u \mid x_{1}, \ldots, x_{k}) = \E(u) = 0 \rightarrow$ \textbf{strong exogeneity} (a implies b).
		\item There are no relevant variables left out of the model : $\Cov(x_{j} , u) = 0, \; \forall j = 1, \ldots, k \rightarrow$ \textbf{weak exogeneity}.
	\end{enumerate}
	
	\item \textbf{Homoscedasticity}. The variability of the residuals is the same for any $x$ : $\Var(u \mid x_{1}, \ldots, x_{k}) = \sigma^{2}_{u}$
	\item \textbf{No autocorrelation}. Residuals do not contain information about any other residuals : \\
	$\Corr(u_{t}, u_{s} \mid x_{1}, \ldots, x_{k}) = 0, \; \forall t \neq s$
	\item \textbf{Normality}. Residuals are independent and identically distributed (\textbf{i.i.d.}) : $u \sim \mathcal{N}(0, \sigma^{2}_{u})$
	\item \textbf{Data size}. The number of observations available must be greater than $(k + 1)$ parameters to estimate. (It is already satisfied under asymptotic situations)
\end{enumerate}

\end{f}    

\begin{f}[Asymptotic properties of OLS]
	
Under the econometric model assumptions and the Central Limit Theorem :

\begin{itemize}[leftmargin=*]
	\item Hold t1 to t3a : OLS is \textbf{unbiased}. $\E(\hat{\beta}_{j}) = \beta_{j}$
	\item Hold t1 to t3 : OLS is \textbf{consistent}. $\mathrm{plim}(\hat{\beta}_{j}) = \beta_{j}$ (to t3b left out t3a, weak exogeneity, biased but consistent)
	\item Hold t1 to t5: \textbf{asymptotic normality} of OLS (then, t6 is necessarily satisfied) : $u \underset{a}{\sim}\mathcal{N}(0, \sigma^{2}_{u})$
	\item Hold t1 to t5 : \textbf{unbiased estimate} of $\sigma^{2}_{u}$. $\E(\hat{\sigma}^{2}_{u}) = \sigma^{2}_{u}$
	\item Hold t1 to t5: OLS is \textcolor{blue}{BLUE} (Best Linear Unbiased Estimator) or \textbf{efficient}.
	\item Hold t1 to t6: hypothesis testing and confidence intervals can be done reliably.
\end{itemize}


\end{f}  \hrule

\begin{f}[Trends and seasonality]

\textbf{Spurious regression} - is when the relation between $y$ and $x$ is due to factors that affect $y$ and have correlation with $x$, $\Corr(x_{j}, u) \neq 0$. Is the \textbf {non-fulfillment of t3}.



Two time series can have the same (or contrary) trend, that should lend to a high level of correlation. This can provoke a false appearance of causality, the problem is \textbf{spurious regression}. Given the model :
%
\begin{center}
	$y_{t} = \beta_{0} + \beta_{1} x_{t} + u_{t}$
\end{center}
où :
\begin{center}
	$y_{t} = \alpha_{0} + \alpha_{1} \mathrm{Tendance} + v_{t}$
	
	$x_{t} = \gamma_{0} + \gamma_{1} \mathrm{Tendance} + v_{t}$
\end{center}
L'ajout d'une tendance au modèle peut résoudre le problème :
\begin{center}
	$y_{t} = \beta_{0} + \beta_{1} x_{t} + \beta_{2} \mathrm{Tendance} + u_{t}$
\end{center}
La tendance peut être linéaire ou non linéaire (quadratique, cubique, exponentielle, etc.).

Une autre méthode consiste à utiliser le \textbf{filtre de Hodrick-Prescott} pour extraire la tendance et la composante cyclique.

\end{f}    

\begin{f}[Saisonnalité]

	Une série temporelle peut présenter une saisonnalité. Cela signifie que la série est soumise à des variations ou à des schémas saisonniers, généralement liés aux conditions climatiques.
	
	Par exemple, le PIB (en noir) est généralement plus élevé en été et plus faible en hiver. Série corrigée des variations saisonnières (en orange {\color{OrangeProfondIRA}}) à titre de comparaison.
	
			\begin{tikzpicture}[xscale=0.4, yscale=.15]
	% \draw [step=1, gray, very thin] (0, 0) grid (20, 20);
	\draw [thick, <->] (0, 20) node [anchor=south] {$y$} -- (0, 0) -- (20, 0) node [anchor=south] {$t$};
	\draw [thick, black] 
	(0.0, 2.794) -- (0.5, 4.810) -- 
	(1.0, 2.500) -- (1.5, 7.619) -- 
	(2.0, 6.031) -- (2.5, 8.840) -- 
	(3.0, 5.420) -- (3.5, 10.855) -- 
	(4.0, 8.474) -- (4.5, 9.695) -- 
	(5.0, 5.481) -- (5.5, 9.512) -- 
	(6.0, 7.680) -- (6.5, 9.573) -- 
	(7.0, 5.787) -- (7.5, 10.366) -- 
	(8.0, 8.291) -- (8.5, 9.451) -- 
	(9.0, 5.604) -- (9.5, 10.099) -- 
	(10.0, 8.962) -- (10.5, 11.282) -- 
	(11.0, 7.130) -- (11.5, 11.709) -- 
	(12.0, 8.962) -- (12.5, 11.526) -- 
	(13.0, 8.168) -- (13.5, 13.358) -- 
	(14.0, 11.099) -- (14.5, 14.213) -- 
	(15.0, 10.916) -- (15.5, 16.290) -- 
	(16.0, 14.396) -- (16.5, 16.595) -- 
	(17.0, 13.419) -- (17.5, 18.000) -- 
	(18.0, 16.106) -- (18.5, 16.900); 
	\draw [thick, OrangeProfondIRA] 
	(0.0, 3.7939) -- (0.5, 3.9982) -- 
	(1.0, 3.9000) -- (1.5, 4.9183) -- 
	(2.0, 6.0905) -- (2.5, 6.9397) -- 
	(3.0, 6.9998) -- (3.5, 7.5450) -- 
	(4.0, 7.4733) -- (4.5, 7.6947) -- 
	(5.0, 7.4809) -- (5.5, 7.5115) -- 
	(6.0, 7.6794) -- (6.5, 7.5725) -- 
	(7.0, 7.7863) -- (7.5, 8.3336) -- 
	(8.0, 7.9901) -- (8.5, 8.1504) -- 
	(9.0, 8.6031) -- (9.5, 8.9008) -- 
	(10.0, 8.9618) -- (10.5, 8.7176) -- 
	(11.0, 8.9998) -- (11.5, 9.1901) -- 
	(12.0, 9.3618) -- (12.5, 9.3733) -- 
	(13.0, 10.6321) -- (13.5, 11.0588) --
	(14.0, 11.3992) -- (14.5, 12.2137) -- 
	(15.0, 12.5160) -- (15.5, 13.5901) -- 
	(16.0, 14.0969) -- (16.5, 15.2954) -- 
	(17.0, 15.3198) -- (17.5, 16.2000) -- 
	(18.0, 16.9069) -- (18.5, 17.2008);
\end{tikzpicture}

\begin{itemize}[leftmargin=*]
	\item Ce problème est une \textbf{régression parasite}. Un ajustement saisonnier peut le résoudre.
\end{itemize}

Un simple \textbf{ajustement saisonnier} pourrait consister à créer des variables binaires stationnaires et à les ajouter au modèle. Par exemple, pour les séries trimestrielles ($Q q_{t}$ sont des variables binaires) :

\begin{center}
	$y_{t} = \beta_{0} + \beta_{1} Q2_{t} + \beta_{2} Q3_{t} + \beta_{3} Q4_{t} + \beta_{4} x_{1t} + \cdots + \beta_{k} x_{kt} + u_{t}$
\end{center}

Une autre méthode consiste à ajuster les variables en fonction des variations saisonnières (sa), puis à effectuer la régression avec les variables ajustées :

\begin{center}
	$z_{t} = \beta_{0} + \beta_{1} Q2_{t} + \beta_{2} Q3_{t} + \beta_{3} Q4_{t} + v_{t} \rightarrow \hat{v}_{t} + \E(z_{t}) = \hat{z}_{t}^{sa}$
	
	$\hat{y}_{t}^{sa}= \beta_{0} + \beta_{1} \hat{x}_{1t}^{sa} + \cdots + \beta_{k} \hat{x}_{kt}^{sa} + u_{t}$
\end{center}

Il existe des méthodes bien plus efficaces et complexes pour ajuster saisonnièrement une série temporelle, comme la méthode \textbf{X-13ARIMA-SEATS}.

\end{f}  \hrule

\begin{f}[Autocorrélation]

Le résidu de toute observation, $u_{t}$, est corrélé avec le résidu de toute autre observation. Les observations ne sont pas indépendantes. Il s'agit d'un cas de \textbf{non-respect} de \textbf{t5}.

\begin{center}
	$\Corr(u_{t}, u_{s} \mid x_{1}, \ldots, x_{k}) = \Corr(u_{t}, u_{s}) \neq 0, \; \forall t \neq s$
\end{center}

\end{f}    
\begin{f}[Conséquences]

\begin{itemize}[leftmargin=*]
	\item Les estimateurs OLS restent non biaisés.
	\item Les estimateurs OLS restent cohérents.
	\item L'OLS n'est \textbf{plus efficace}, mais reste un LUE (estimateur linéaire non biaisé).
	\item Les \textbf{estimations de variance} des estimateurs sont \textbf{biaisées} : la construction des intervalles de confiance et les tests d'hypothèse ne sont pas fiables.
\end{itemize}

\end{f}    

\begin{f}[Détection]
{\ }

\begin{itemize}[leftmargin=*]
	\item \textbf{Diagrammes de dispersion} - recherchez des modèles de dispersion sur $u_{t - 1}$ par rapport à $u_{t}$.
	
	\setlength{\multicolsep}{0pt}
	\setlength{\columnsep}{6pt}
	\begin{multicols}{3}
		\begin{center}
			\begin{tikzpicture}[scale=0.11]
				\node at (16, 20) {\textbf{Ac.}}; 
				\draw [thick, ->] (0, 10) -- (20, 10) node [anchor=south] {$u_{t - 1}$}; 
				\draw [thick, -] (0, 0) -- (0, 20) node [anchor=west] {$u_{t}$}; 
				\draw plot [only marks, mark=*, mark size=6, domain=2:18, samples=50] (\x, {-0.2*(\x - 10)^2 + 13 + 6*rnd}); 
				\draw [thick, dashed, OrangeProfondIRA, -latex] plot [domain=2:18] (\x, {-0.2*(\x - 10)^2 + 16});
			\end{tikzpicture}
		\end{center}
		
		\columnbreak
		
		\begin{center}
			\begin{tikzpicture}[scale=0.11]
				\node at (16, 20) {\textbf{Ac. $+$}}; 
				\draw [thick, ->] (0, 10) -- (20, 10) node [anchor=north] {$u_{t - 1}$}; 
				\draw [thick, -] (0, 0) -- (0, 20) node [anchor=west] {$u_{t}$}; 
				\draw plot [only marks, mark=*, mark size=6, domain=2:18, samples=20] (\x, {5*rnd + 2.5 + 0.5*\x}); 
				\draw [thick, dashed, OrangeProfondIRA, -latex] plot [domain=2:18] (\x, {5 + 0.5*\x});
			\end{tikzpicture}
		\end{center}
		
		\columnbreak
		
		\begin{center}
			\begin{tikzpicture}[scale=0.11]
				\node at (16, 20) {\textbf{Ac. $-$}}; 
				\draw [thick, ->] (0, 10) -- (20, 10) node [anchor=south] {$u_{t - 1}$}; 
				\draw [thick, -] (0, 0) -- (0, 20) node [anchor=west] {$u_{t}$}; 
				\draw plot [only marks, mark=*, mark size=6, domain=2:18, samples=20] (\x, {5*rnd + 12.5 - 0.5*\x}); 
				\draw [thick, dashed, OrangeProfondIRA, -latex] plot [domain=2:18] (\x, {15 - 0.5*\x});
			\end{tikzpicture}
		\end{center}
	\end{multicols}
	
	\begin{multicols}{2}
		\item \textbf{Corrélogramme} - fonction d'autocorrélation (ACF) et fonction d'autocorrélation partielle (PACF).
		
		\columnbreak
		
		\begin{itemize}[leftmargin=*]
			\item Axe Y : corrélation.
			\item Axe X : nombre de décalages.
			\item Zone grise : $\pm 1,96/T^{0,5}$
	\end{itemize}
\end{multicols}

\begin{center}
\begin{tikzpicture}[scale=0.25]
% acf plot
\node at (-2.5, 14) {\small \rotatebox{90}{\textbf{ACF}}}; 
\node at (-1, 17.5) {\small 1};
\node at (-1, 14) {\small 0};
\node at (-1, 10.5) {\small -1}; 
\fill [lightgray] (0, 13) rectangle (30.5, 15); 
\draw [dashed, thin] (0, 14) -- (30.5, 14); 
\draw [thick, |->] (0, 18) -- (0, 10) -- (30.5, 10);
\fill [OrangeProfondIRA] (2, 14) rectangle (2.5, 17.95);
\fill [OrangeProfondIRA] (5, 14) rectangle (5.5, 16.96);
\fill [OrangeProfondIRA] (8, 14) rectangle (8.5, 16.22); 
\fill [OrangeProfondIRA] (11, 14) rectangle (11.5, 15.67);
\fill [OrangeProfondIRA] (14, 14) rectangle (14.5, 15.25);
\fill [OrangeProfondIRA] (17, 14) rectangle (17.5, 14.94);
\fill [OrangeProfondIRA] (20, 14) rectangle (20.5, 14.70);
\fill [OrangeProfondIRA] (23, 14) rectangle (23.5, 14.53);
\fill [OrangeProfondIRA] (26, 14) rectangle (26.5, 14.40);
\fill [OrangeProfondIRA] (29, 14) rectangle (29.5, 14.30);
% pacf plot
\node at (-2.5, 4) {\small \rotatebox{90}{\textbf{PACF}}};
\node at (-1, 7.5) {\small 1};
\node at (-1, 4) {\small 0};
\node at (-1, 0.5) {\small -1};
\fill [lightgray] (0, 3) rectangle (30.5, 5);
\draw [dashed, thin] (0, 4) -- (30.5, 4);
\draw [thick, |->] (0, 8) -- (0, 0) -- (30.5, 0);
\fill [OrangeProfondIRA] (2, 4) rectangle (2.5, 7.90);
\fill [OrangeProfondIRA] (5, 4) rectangle (5.5, 7.00);
\fill [OrangeProfondIRA] (8, 4) rectangle (8.5, 3.47);
\fill [OrangeProfondIRA]	(11, 4) rectangle (11.5, 4.24);
\fill [OrangeProfondIRA] (14, 4) rectangle (14.5, 4.43);
\fill [OrangeProfondIRA] (17, 4) rectangle (17.5, 4.89);
\fill [OrangeProfondIRA] (20, 4) rectangle (20.5, 3.09);
\fill [OrangeProfondIRA] (23, 4) rectangle (23.5, 3.58);
\fill [OrangeProfondIRA] (26, 4) rectangle (26.5, 4.46);
\fill [OrangeProfondIRA] (29, 4) rectangle (29.5, 4.86);
\end{tikzpicture}
\end{center}
	

	
	\textbf{Processus MA($q$)}. \underline{ACF} : seuls les premiers coefficients $q$ sont significatifs, les autres sont brusquement annulés. \underline{PACF} : décroissance exponentielle rapide atténuée ou ondes sinusoïdales.
	
	\textbf{Processus AR($p$)}. \underline{ACF} : décroissance exponentielle rapide atténuée ou ondes sinusoïdales. \underline{PACF} : seuls les premiers coefficients $p$ sont significatifs, les autres sont brusquement annulés.

	
	\textbf{Processus ARMA($p, q$)}. \underline{ACF} et \underline{PACF} : les coefficients ne sont pas brusquement annulés et présentent une décroissance rapide.
	
	Si les coefficients ACF ne décroissent pas rapidement, cela indique clairement un manque de stationnarité dans la moyenne.
	
	\item \textbf{Tests formels} - En général, $H_{0}$ : pas d'autocorrélation.
	
	En supposant que $u_{t}$ suit un processus AR(1) :
	
	\begin{center}
		$u_{t} = \rho_{1} u_{t - 1} + \varepsilon_{t}$
	\end{center}
	
	où $\varepsilon_{t}$ est un bruit blanc.
	
	\textbf{Test t AR(1)} (régresseurs exogènes) :
	
	\begin{center}
		$t = \dfrac{\hat{\rho}_{1}}{\se(\hat{\rho}_{1})} \sim t_{T - k - 1, \alpha/2}$
	\end{center}
	
	\begin{itemize}[leftmargin=*]
		\item $H_{1}$ : Autocorrélation d'ordre un, AR(1).
	\end{itemize}
	
	\textbf{Statistique de Durbin-Watson} (régresseurs exogènes et normalité des résidus) :
	
	\begin{center}
		$d = \dfrac{\sum_{t=2}^{n} (\hat{u}_{t} - \hat{u}_{t - 1})^{2}}{\sum_{t=1}^{n} \hat{u}_{t}^{2}} \approx 2 \cdot (1 - \hat{\rho}_{1})$
	\end{center}
	
	Où $0 \leq d \leq 4$
	
	\begin{itemize}[leftmargin=*]
		\item $H_{1}$ : Autocorrélation d'ordre un, AR(1).
	\end{itemize}
	
\begin{center}
			\begin{tabular}{ c | c | c | c }
			$d =$          & 0 & 2 & 4  \\ \hline
			$\rho \approx$ & 1 & 0 & -1
		\end{tabular}

\end{center}		

	\begin{tikzpicture}[scale=0.3]
		\fill [lightgray] (5, 0) rectangle (9, 6); 
		\draw (5, 0) -- (5, 6);
		\draw (9, 0) -- (9, 6);
		\fill [lightgray] (16, 0) rectangle (20, 6);
		\draw (16, 0) -- (16, 6);
		\draw (20, 0) -- (20, 6);
		\draw [thick] (0, 6) -- (0, 0) -- (25, 0);
		\draw [dashed] (12.5, 0) -- (12.5, 6);
		\node at (-0.5, 6.5) {\small $f(d)$};
		\node at (0, -0.6) {\small 0};
		\node at (5, -0.6) {\small $d_{L}$};
		\node at (9, -0.6) {\small $d_{U}$};
		\node at (12.5, -0.6) {\small 2};
		\node at (16.7, -0.6) {\tiny $(4 - d_{U})$};
		\node at (20.7, -0.6) {\tiny $(4 - d_{L})$};
		\node at (25, -0.6) {\small 4};
		\node at (2.5, 3.5) {\small Rej. $H_{0}$};
		\node at (2.5, 2.5) {\small AR $+$};
		\node [text=OrangeProfondIRA] at (7, 3) {\textbf{?}};
		\node at (12.5, 3.5) {\small Not rej. $H_{0}$};
		\node at (12.5, 2.5) {\small No AR};
		\node [text=OrangeProfondIRA] at (18, 3) {\textbf{?}};
		\node at (22.5, 3.5) {\small Rej. $H_{0}$};
		\node at (22.5, 2.5) {\small AR $-$};
	\end{tikzpicture}

			\textbf{h de Durbin} (régresseurs endogènes) :

\begin{center}
	$h = \hat{\rho} \cdot \sqrt{\dfrac{T}{1 - T \cdot \upsilon}}$
\end{center}

où $\upsilon$ est la variance estimée du coefficient associé à la variable endogène.

\begin{itemize}[leftmargin=*]
	\item $H_{1}$ : Autocorrélation d'ordre un, AR(1).
\end{itemize}

\textbf{Test de Breusch-Godfrey} (régresseurs endogènes) : il permet de détecter les processus MA($q$) et AR($p$) ($\varepsilon_{t}$ est w. bruit) :

\begin{itemize}[leftmargin=*]
	\item MA($q$) : $u_{t} = \varepsilon_{t} - m_{1} u_{t - 1} - \cdots - m_{q} u_{t - q}$
	\item AR($p$) : $u_{t} = \rho_{1} u_{t - 1} + \cdots + \rho_{p} u_{t - p}+ \varepsilon_{t}$
\end{itemize}

Sous $H_{0}$ : Pas d'autocorrélation :

\begin {center}
$\hfill T \cdot R^{2}_{\hat{u}_t}\underset{a}{\sim}\chi^{2}_{q} \hfill \textbf{ou} \hfill T \cdot R^{2}_{\hat{u}_t}\underset{a}{\sim}\chi^{2}_{p} \hfill$
\end{center}

\begin{itemize}[leftmargin=*]
\item $H_{1}$ : Autocorrélation d'ordre $q$ (ou $p$).
\end{itemize}

\textbf{Test Q de Ljung-Box} :

\begin{itemize}[leftmargin=*]
\item $H_{1}$ : Autocorrélation jusqu'au décalage $h$.
\end{itemize}

\end{itemize}



\end{f}  

\begin{f}[Correction]

\begin{itemize}[leftmargin=*]
\item Utiliser la méthode des moindres carrés ordinaires (OLS) avec un estimateur de matrice de variance-covariance \textbf{robuste à l'hétéroscédasticité et à l'autocorrélation} (HAC), par exemple celui proposé par \textbf{Newey-West}.
\item Utiliser les \textbf{moindres carrés généralisés} (GLS). Supposons que $y_{t} = \beta_{0} + \beta_{1} x_{t} + u_{t}$, avec $u_{t} = \rho u_{t - 1}+ \varepsilon_{t}$, où $\lvert \rho \rvert < 1$ et $\varepsilon_{t}$ est un \underline{bruit blanc}.

\begin{itemize}[leftmargin=*]
\item Si $\rho$ est \textbf{connu}, utilisez un \textbf{modèle quasi-différencié} :

\begin{center}
	$y_{t} - \rho y_{t - 1}= \beta_{0} (1 - \rho) + \beta_{1} (x_{t} - \rho x_{t - 1}) + u_{t} - \rho u_{t - 1}$
	
	$y_{t}^{*} = \beta_{0}^{*} + \beta_{1}' x_{t}^{*} + \varepsilon_{t}$
\end{center}

où $\beta_{1}' = \beta_{1}$ ; et estimez-le par OLS.

\item Si $\rho$ n'est \textbf{pas connu}, l'estimer par exemple par la \textbf{méthode itérative de Cochrane-Orcutt} (la méthode de Prais-Winsten est également valable) :

\begin{enumerate}[leftmargin=*]
	\item Obtenir $\hat{u}_{t}$ à partir du modèle original.
	\item Estimez $\hat{u}_{t} = \rho \hat{u}_{t-1} + \varepsilon_{t}$ et obtenez $\hat{\rho}$.
	\item Créez un modèle quasi-différencié :
	
	\begin{center}
		$y_{t} - \hat{\rho}y_{t - 1} = \beta_{0} (1 - \hat{\rho}) + \beta_{1} (x_{t} - \hat{\rho} x_{t - 1}) + u_{t} - \hat{\rho}u_{t - 1}$
		
		$y_{t}^{*} = \beta_{0}^{*} + \beta_{1}' x_{t}^{*} + \varepsilon_{t}$
	\end{center}
	
	où $\beta_{1}' = \beta_{1}$ ; et l'estimer par OLS.
	
	\item Obtenir $\hat{u}_{t}^{*} = y_{t} - (\hat{\beta}_{0}^{*} + \hat{\beta}_{1}' x_{t}) \neq y_ {t} - (\hat{\beta}_{0}^{*} + \hat{\beta}_{1}' x_{t}^{*})$.
	\item Répéter à partir de l'étape 2. L'algorithme se termine lorsque les paramètres estimés varient très peu entre les itérations.
\end{enumerate}
\end{itemize}

\item Si le problème n'est pas résolu, rechercher une \textbf{forte dépendance} dans la série.
\end{itemize}

\end{f}  
\columnbreak

\hrule

\begin{f}[Lissage exponentiel]

\begin{center}
	$f_{t} = \alpha y_{t} + (1 - \alpha) f_{t - 1}$
\end{center}

où $0 < \alpha < 1$ est le paramètre de lissage.

\end{f}  \hrule

\begin{f}[Prévisions]

Deux types de prévisions :

\begin{itemize}[leftmargin=*]
	\item De la valeur moyenne de $y$ pour une valeur spécifique de $x$.
	\item D'une valeur individuelle de $y$ pour une valeur spécifique de $x$.
\end{itemize}

\textbf{Statistique U de Theil} - compare les résultats prévus avec les résultats des prévisions réalisées à partir d'un minimum de données historiques.

\begin{center}
	$U = \sqrt{\frac{\sum_{t=1}^{T-1} \left( \frac{\hat{y}_{t+1} - y_{t+1}}{y_t} \right)^2}{\sum_{t=1}^{T-1} \left( \frac{y_ {t+1} - y_t}{y_t} \right)^2}}$
\end{center}

\begin{itemize}[leftmargin=*]
	\item $< 1$ : la prévision est meilleure qu'une simple estimation.
	\item $= 1$ : la prévision est à peu près aussi bonne qu'une simple estimation.
	\item $> 1$ : La prévision est moins bonne qu'une simple estimation.
\end{itemize}



\end{f}  \hrule

  \begin{f}[Stationnarité]

La stationnarité permet d'identifier correctement les relations entre les variables qui restent inchangées dans le temps.

\begin{itemize}[leftmargin=*]
	\item \textbf{Processus stationnaire} (stationnarité stricte) : si un ensemble de variables aléatoires est pris et décalé de $h$ périodes (changements de temps), la distribution de probabilité conjointe doit rester inchangée.
	\item \textbf{Processus non stationnaire} : par exemple, une série avec une tendance, où au moins la moyenne change avec le temps.
	\item \textbf{Processus stationnaire de covariance} - il s'agit d'une forme plus faible de stationnarité :
	
	\begin{itemize}[leftmargin=*]
		\begin{multicols}{2}
			\item $\E(x_{t})$ est constant.
			
			\columnbreak
			
			\item $\Var(x_{t})$ est constant.
		\end{multicols}
		
		\item Pour tout $t$, $h \geq 1$, $\Cov(x_{t}, x_{t + h})$ dépend uniquement de $h$, et non de $t$.
	\end{itemize}
\end{itemize}

\end{f}  \hrule

  \begin{f}[Faible dépendance]

La faible dépendance remplace l'hypothèse d'échantillonnage aléatoire pour les séries temporelles.

\begin{itemize}[leftmargin=*]
	\item Un processus stationnaire $\lbrace x_{t} \rbrace$ est \textbf{faiblement dépendant} lorsque $x_{t}$ et $x_{t + h}$ sont presque indépendants lorsque $h$ augmente sans limite.
	\item Un processus stationnaire de covariance est \textbf{faiblement dépendant} si la corrélation entre $x_{t}$ et $x_{t + h}$ tend vers $0$ suffisamment rapidement lorsque $h \rightarrow \infty$ (ils ne sont pas corrélés de manière asymptotique).
\end{itemize}

Les processus faiblement dépendants sont appelés \textbf{intégrés d'ordre zéro}, I(0). Quelques exemples :

\begin{itemize}[leftmargin=*]
\item \textbf{Moyenne mobile} - $\lbrace x_{t} \rbrace$ est une moyenne mobile d'ordre $q$, MA($q$) :

\begin{center}
	$x_{t} = e_{t} + m_{1} e_{t - 1} + \cdots + m_{q} e_{t - q}$
\end{center}

où $\lbrace e_{t} : t = 0, 1, \ldots, T \rbrace$ est une séquence \textsl{i.i.d.} avec une moyenne nulle et une variance $\sigma^{2}_{e}$.

\item \textbf{Processus autorégressif} - $\lbrace x_{t} \rbrace$ est un processus autorégressif d'ordre $p$, AR($p$) :

\begin{center}
	$x_{t} = \rho_{1} x_{t - 1} + \cdots + \rho_{p} x_{t - p} + e_{t}$
\end{center}

où $\lbrace e_{t} : t = 1, 2, \ldots, T \rbrace$ est une séquence \textsl{i.i.d.} avec une moyenne nulle et une variance $\sigma^{2}_{e}$.

\textbf{Condition de stabilité} : si $1 - \rho_{1} z - \cdots - \rho_{p} z^{p} = 0$ pour $\lvert z \rvert > 1$, alors $\lbrace x_{t} \rbrace$ est un processus AR($p$) stable qui est faiblement dépendant. Pour AR(1), la condition est : $\lvert \rho_{1} \rvert < 1$.

\item \textbf{Processus ARMA} - est une combinaison de AR($p$) et MA($q$) ; $\lbrace x_{t} \rbrace$ est un ARMA($p, q$) :

\begin{center}
	$x_{t} = e_{t} + m_{1} e_{t - 1} + \cdots + m_{q} e_{t - q} + \rho_{1} x_{t - 1} + \cdots + \rho_{p} x_{t - p}$
\end{center}
\end{itemize}


		\end{f}  \hrule

  \begin{f}[Racines unitaires]

Un processus est I($d$), c'est-à-dire intégré d'ordre $d$, si l'application de différences $d$ fois rend le processus stationnaire.

Lorsque $d \geq 1$, le processus est appelé \textbf{processus à racine unitaire} ou on dit qu'il a une racine unitaire.

Un processus a une racine unitaire lorsque la condition de stabilité n'est pas remplie (il existe des racines sur le cercle unitaire).

\end{f}  \hrule  

\begin{f}[Forte dépendance]

La plupart du temps, les séries économiques sont fortement dépendantes (ou très persistantes). Quelques exemples de \textbf{racine unitaire} I(1) :

\begin{itemize}[leftmargin=*]
	\item \textbf{Marche aléatoire} - un processus AR(1) avec $\rho_{1} = 1$.
	
	\begin{center}
		$y_{t} = y_{t - 1} + e_{t}$
	\end{center}
	
	où $\lbrace e_{t} : t = 1, 2, \ldots, T \rbrace$ est une séquence \textsl{i.i.d.} avec une moyenne nulle et une variance $\sigma^{2}_{e}$.
	
	\item \textbf{Marche aléatoire avec dérive} - un processus AR(1) avec $\rho_{1} = 1$ et une constante.
	
	\begin{center}
		$y_{t} = \beta_{0} + y_{t - 1} + e_{t}$
	\end{center}
	
	où $\lbrace e_{t} : t = 1, 2, \ldots, T \rbrace$ est une séquence \textsl{i.i.d.} avec une moyenne nulle et une variance $\sigma^{2}_{e}$.
\end{itemize}

\end{f} 

 \begin{f}[Tests de racine unitaire]{\ }

\begin{center}
	\begin{tabular}{ c | c | c }
		Test            & $H_{0}$    & Rejeter $H_{0}$                     \\ \hline
		ADF             & I(1)       & tau \textless \, Valeur critique    \\ \hline
		KPSS            & Niveau I(0) & mu \textgreater \, Valeur critique  \\
		& Tendance I(0) & tau \textgreater \, Valeur critique \\ \hline
		Phillips-Perron & I(1)       & Z-tau \textless \, Valeur critique  \\ \hline
		Zivot-Andrews   & I(1)       & tau \textless \, Valeur critique
	\end{tabular}
\end{center}
\medskip

\end{f}  

\begin{f}[De la racine unitaire à la faible dépendance]

Intégré d'ordre \textbf{un}, I(1), signifie que \textbf{la première différence} du processus est \textbf{faiblement dépendante} ou I(0) (et généralement stationnaire). Par exemple, soit $\lbrace y_{t} \rbrace$ une marche aléatoire :

\begin{multicols}{2}
	\begin{center}
		$\Delta y_{t} = y_{t} - y_{t - 1} = e_{t}$
	\end{center}
	
	où $\lbrace e_{t} \rbrace = \lbrace \Delta y_{t} \rbrace$ est \textsl{i.i.d.} \\
	
	Remarque :
	\begin{itemize}[leftmargin=*]
		\item La première différence d'une série supprime sa tendance.
		\item Les logarithmes d'une série stabilisent sa variance.
	\end{itemize}
	
	\columnbreak
	
			\begin{tikzpicture}[scale=0.18]
	% \draw [step=1, gray, very thin] (0, 0) grid (20, 20); 
	\draw [thick, <->] (0, 20) node [anchor=south west] {$y, {\color{OrangeProfondIRA} \Delta y}$} -- (0, 0) -- (20, 0) node [anchor=south] {$t$}; 
	\draw [thick, black] 
	(0.0, 2.000) -- (0.5, 2.459) -- 
	(1.0, 2.716) -- (1.5, 3.205) -- 
	(2.0, 3.571) -- (2.5, 3.952) -- 
	(3.0, 4.047) -- (3.5, 4.514) -- 
	(4.0, 4.719) -- (4.5, 5.160) -- 
	(5.0, 5.674) -- (5.5, 5.987) -- 
	(6.0, 6.242) -- (6.5, 6.471) -- 
	(7.0, 6.944) -- (7.5, 7.104) -- 
	(8.0, 7.584) -- (8.5, 8.087) -- 
	(9.0, 8.112) -- (9.5, 8.834) -- 
	(10.0, 9.470) -- (10.5, 9.718) -- 
	(11.0, 10.032) -- (11.5, 10.491) -- 
	(12.0, 10.748) -- (12.5, 10.805) -- 
	(13.0, 11.016) -- (13.5, 11.439) --
	(14.0, 11.810) -- (14.5, 12.247) -- 
	(15.0, 12.668) -- (15.5, 13.052) -- 
	(16.0, 13.586) -- (16.5, 14.322) -- 
	(17.0, 14.913) -- (17.5, 15.704) -- 
	(18.0, 16.081) -- (18.5, 16.431); 
	\draw [thick, OrangeProfondIRA] 
	(0.5, 11.283) -- (1.0, 7.201) -- 
	(1.5, 11.889) -- (2.0, 9.405) -- 
	(2.5, 9.701) -- (3.0, 3.926) -- 
	(3.5, 11.454) -- (4.0, 6.136) -- 
	(4.5, 10.926) -- (5.0, 12.393) -- 
	(5.5, 8.345) -- (6.0, 7.157) -- 
	(6.5, 6.627) -- (7.0, 11.572) -- 
	(7.5, 5.235) -- (8.0, 11.703) -- 
	(8.5, 12.186) -- (9.0, 2.513) -- 
	(9.5, 16.607) -- (10.0, 14.869) -- 
	(10.5, 7.015) -- (11.0, 8.368) -- 
	(11.5, 11.283) -- (12.0, 7.196) -- 
	(12.5, 3.153) -- (13.0, 6.277) -- 
	(13.5, 10.547) -- (14.0, 9.517) -- 
	(14.5, 10.834) -- (15.0, 10.526) -- 
	(15.5, 9.754) -- (16.0, 12.816) -- 
	(16.5, 16.875) -- (17.0, 13.961) -- 
	(17.5, 18.000) -- (18.0, 9.644) -- 
	(18.5, 9.075);
\end{tikzpicture}
\end{multicols}



\subsubsection*{De la racine unitaire au pourcentage de variation}

Lorsqu'une série I(1) est strictement positive, elle est généralement convertie en logarithmes avant de prendre la première différence pour obtenir le pourcentage de variation (approximatif) de la série :

\begin{center}
	$\Delta \log(y_{t}) = \log(y_{t}) - \log(y_{t - 1}) \approx \dfrac{y_t - y_{t - 1}} {y_{t - 1}}$
\end{center}

		\end{f}  \hrule

  \begin{f}[Coiintégration]

Lorsque \textbf{deux séries sont I(1), mais qu'une combinaison linéaire de celles-ci est I(0)}. Dans ce cas, la régression d'une série sur l'autre n'est pas fallacieuse, mais exprime quelque chose sur la relation à long terme. Les variables sont dites cointégrées si elles ont une tendance stochastique commune.

Par exemple, $\lbrace x_{t} \rbrace$ et $\lbrace y_{t} \rbrace$ sont I(1), mais $y_{t} - \beta x_{t} = u_{t}$ où $\lbrace u_{t} \rbrace$ est I(0). ($\beta$ est le paramètre de cointégration).

\end{f}  

\begin{f}[Test de cointégration]

En suivant l'exemple ci-dessus :

\begin{enumerate}[leftmargin=*]
	\item Estimer $y_{t} = \alpha + \beta x_{t} + \varepsilon_{t}$ et obtenir $\hat{\varepsilon}_{t}$.
	\item Effectuer un test ADF sur $\hat{\varepsilon}_{t}$ avec une distribution modifiée.
	
	Le résultat de ce test est équivalent à :
	
	\begin{itemize}[leftmargin=*]
		\item $H_{0}$ : $\beta = 0$ (pas de cointégration)
		\item $H_{1}$ : $\beta \neq 0$ (cointégration)
	\end{itemize}
	
	si la statistique du test $>$ valeur critique, rejeter $H_0$.
\end{enumerate}

\end{f}  \hrule

  \begin{f}[Hétéroscédasticité sur les séries temporelles]

L'\textbf{hypothèse} affectée est \textbf{t4}, ce qui conduit à \textbf{une inefficacité de l'OLS}.

Utilisez des tests tels que Breusch-Pagan ou White, où $H_{0}$ : pas d'hétéroscédasticité. Il est \textbf{important} pour que les tests fonctionnent qu'il n'y ait \textbf{pas d'autocorrélation}.

\end{f}  \hrule  
\begin{f}[ARCH]

Une hétéroscédasticité conditionnelle autorégressive (ARCH) est un modèle permettant d'analyser une forme d'hétéroscédasticité dynamique, où la variance de l'erreur suit un processus AR($p$).

Étant donné le modèle : $y_{t} = \beta_{0} + \beta_{1} z_{t} + u_{t}$ où il y a AR(1) et hétéroscédasticité :

\begin{center}
	$\E(u^{2}_{t} \mid u_{t - 1}) = \alpha_{0} + \alpha_{1} u^{2}_{t - 1}$
\end{center}

\end{f}  \hrule 
 \begin{f}[GARCH]

Un modèle général d'hétéroscédasticité conditionnelle autorégressive (GARCH) est similaire au modèle ARCH, mais dans ce cas, la variance de l'erreur suit un processus ARMA($p, q$).
\end{f}
\end{multicols}

\newpage


\begin{center}
	\section*{Non-Life Actuarial}
	\medskip
\end{center}

\begin{multicols}{2}	
	% !TeX root = ActuarialFormSheet_MBFA-en.tex
% !TeX spellcheck = fr_FR


\begin{f}[La tarification en assurance non-vie]
	
Une approche générale, mais non exhaustive, car les possibles sont nombreux : \vspace{4mm}

	
\resizebox{\linewidth}{!}
{	\begin{tikzpicture}[every node/.style={draw, align=center, rounded corners,fill=BleuProfondIRA!30, 
			font=\footnotesize, minimum height=1cm, text width=3cm},
		every path/.style={->, thick},
		node distance=.5cm and .5cm]
		% Noeuds
		\node[] (data) {Collecte de données \\ (sinistres, production, ...)};
		\node[ right=of data] (prep) {Nettoyage et traitement \\ des données};
		\node[ fill=OrangePastelIRA!30, right=of prep] (method) {Choix des méthodes};
		%
		\node[below =of method] (glm) {Modèles GLM / Logistic\\ (Poisson, Gamma,...)};
		\node[below =of glm] (Seg) {Segmenation \\ (CART, Lasso, ...)};
		\node[below =of Seg] (ml) {GAM, Machine learning\\ (Random Forest, Gradian Boosting,...) };
		%
		\node[left=of Seg] (fit) {Ajustement des modèles};
		\node[left=of fit] (eval) {Validation :\\ résidus, Gini, AIC, etc.};
		\node[below=of eval] (pure) {Estimation de la prime pure};
		\node[below=of pure] (loadings) {Chargements : frais, marge, taxes};
		\node[right=of loadings] (final) {Prime commerciale};
		%
		% Flèches
		\draw (data) -- (prep);
		\draw (prep) -- (method);
		\draw (method.east) |- +(.5,0) |-   (Seg.east);
		\draw (method.east) |- +(.5,0) |-  (glm.east);
		\draw (method.east) |- +(.5,0) |- (ml.east);
		%
		\draw (glm.west) |- +(-.25,0) |- (fit.east);
		\draw (Seg.west) -- (fit.east);
		\draw (ml.west) |- +(-.25,0) |- (fit.east);
		%
		\draw (fit) -- (eval);
		\draw (eval) -- (pure);
		\draw (pure) -- (loadings);
		\draw (loadings) -- (final);
	\end{tikzpicture}
}
\end{f}


\begin{f}[Structure générale des données en assurance]

	
Une structure classique des données en assurance. Là encore, les possibles sont nombreux :



	\tikzstyle{NoeudR}=[cylinder, shape border rotate=90, draw,minimum height=1.5cm,shape aspect=.25,align=center]
		\begin{tikzpicture} %[node distance=5cm]
	
	\node (db01) at (-2,2) [fill=BleuProfondIRA!30, minimum width=1cm,NoeudR] {\footnotesize  Assurés};
	\node (db02) at (0,2) [minimum width=1cm,fill=BleuProfondIRA!30,NoeudR] {\footnotesize  Entité \\ \footnotesize  Assurée};
	\node (db03) at (2,2) [fill=BleuProfondIRA!30,minimum width=1cm,NoeudR] {\footnotesize  Risques \\ \footnotesize  Assurés};
	
	\node (db1) at (0,0) [ fill=BleuProfondIRA!30,minimum width=2cm, NoeudR] { Contrat \\ Production};
	\node (db2) at (3,0) [minimum height=1cm,fill=OrangeProfondIRA!30,	minimum width=2cm, NoeudR] { Sinistre};
	\node (db3) at (6,0) [ fill=OrangePastelIRA!20,minimum width=2cm,NoeudR,densely dotted] {\small Données \\ \small externes};
	\node (db4) at (3,-2) [fill=VertIRA!30,minimum width=3cm,NoeudR] {\small Base de données \\ \small  Tarification};
	
	\node (TB) at (-1.2,-2) [rectangle,fill=FushiaIRA!30, minimum width=2cm,
	shape border rotate=90, draw,minimum height=1.5cm,	shape aspect=.25,align=center] {\small Tableaux \\ \small  de bord};
	\draw[->,>=latex] (db01.south) -- (db1);
	\draw[->,>=latex] (db02) -- (db1);
	\draw[->,>=latex] (db03.south) -- (db1);
	\draw[->,>=latex] (db1.south) -- (db4);
	\draw[->,>=latex] (db2.south) -- (db4);
	\draw[->,>=latex,densely dotted] (db3.south) -- (db4);
	\draw[<->,>=latex] (db4) -- (TB);
\end{tikzpicture}
\end{f}


%		\begin{tikzpicture} %[node distance=5cm]
%	%	
%	\draw[thick,<-,>=latex,BleuProfondIRA] (-3,0) -- (10,0) ;
%	\draw[thick,->,>=latex,BleuProfondIRA,densely dotted] (10,0) -- (15,0) ;
%	\node (A) at (-2,1) [NoeudR] {Incured \\date \\\footnotesize  15/10/\NN};
%	\node (B) at (0,1) [NoeudR] {Reported \\date\\\footnotesize  19/10/\NN};
%	\node (C) at (4,1.2) [NoeudR] {First\\ payment \\ date\\\footnotesize  25/11/\NN};
%	\node (D) at (10,1) [NoeudR] {Payment \\ date \\ \footnotesize 8/01/\N};
%	\node (E) at (12,1.2) [NoeudR,densely dotted] {New \\ information \\ date\\ \footnotesize 15/02/\N};
%	\node (F) at (14,1.2) [NoeudR,densely dotted] {Last \\ Payment \\ date\\ \footnotesize 31/04/\N};
%	\draw[ BleuProfondIRA] (A.south) -- ++(0,-1);
%	\draw[ BleuProfondIRA] (B.south) -- ++(0,-1);
%	\draw[ BleuProfondIRA] (C.south) -- ++(0,-1);
%	\draw[ BleuProfondIRA] (D.south) -- ++(0,-1);
%	\draw[ BleuProfondIRA] (E.south) -| ++(0,-1);
%	\draw[ BleuProfondIRA] (F.south) -- ++(0,-0.1) -| ++(-1,-0.9);
%	\draw[thick,<->,>=latex,OrangeProfondIRA] ($(A.south) +(0,-1)$) -- ($(B.south) +(0,-1)$) node [below, midway] {\footnotesize Not reported} ;
%	\draw[thick,<->,>=latex,BleuProfondIRA] ($(B.south) +(0,-1)$) -- ($(D.south) +(0,-1)$) node [below, midway] {\footnotesize Open} ;
%	\draw[thick,<->,>=latex,BleuProfondIRA] ($(B.south) +(0,-1)$) -- ($(D.south) +(0,-1)$) node [below, midway] {\footnotesize Open} ;
%	\draw[thick,<->,>=latex,GrisLogoIRA] ($(C.south) +(0,-1.6)$) -- ($(D.south) +(0,-1.6)$) node [below, midway] {\footnotesize Partially paid} ;
%	\draw[thick,<->,>=latex,GrisLogoIRA] ($(C.south) +(0,-1.6)$) -- ($(C.south) +(-1,-1.6)$) node [below, midway] {\footnotesize Not Paid} ;
%	\draw[thick,<->,>=latex,GrisLogoIRA] ($(B.south) +(0,-1.6)$) -- ($(C.south) +(-1,-1.6)$) node [below, midway] {\footnotesize Not Valued} ;
%	\draw[very thick,<->,>=latex,FushiaIRA] ($(D.south) +(0,-2.2)$) -- ($(C.south) +(-1,-2.2)$) node [below, midway] {\footnotesize Reserved};
%	\draw[thick,<->,>=latex,VertIRA] ($(D.south) +(0,-1)$) -- ($(E.south) +(0,-1)$) node [below, midway] {\footnotesize  Closed} ;
%	\draw[thick,<->,>=latex,OrangeProfondIRA,densely dotted] ($(E.south) +(0,-1)$) -- ($(F.south) +(-1,-1)$) node [below, midway,align=right] 
%	{\footnotesize Reopen } ;
%	\draw[thick,<->,>=latex,VertIRA] ($(F.south) +(-1,-1)$) -- ($(F.south) +(1,-1)$) node [below, midway] {\footnotesize  Closed} ;
%	%	
%\end{tikzpicture}

\hrule

\begin{f}[Provision]
	
L'actuaire non-vie évalue principalement les 
provisions suivantes:
\begin{itemize}
	\item Des provisions pour sinistres à payer (PSAP, \emph{Reserves for claims reported but not settel (RBNS)}) 
	\item Des provisions pour sinistres non encore manifestés (PSNEM, \emph{Reserve for claims incurred but not reported (IBNR)})
	\item Des provisions pour primes non acquises (PPNA, \emph{Reserves for unearned premiums})
	\item Des provisions pour risques en cours (PREC,  \emph{Reserves for outstanding risks (non-life)}
	)
\end{itemize}

\end{f}




\begin{f}[Chain Ladder déterministe]
	%
		Soit $C_{i k}$ le montant, cumulé jusqu'en l'année de développement $k$, des sinistres survenus en l'année d'accident $i$, pour $1 \leq i, k \leq n$. $C_{i k}$ peut représenter soit le montant payé, soit le cout total estimé (paiement déjà effectué plus réserve) du sinistre. Les montants $C_{i k}$ sont connus pour $i+k \leq n+1$ et on cherche à estimer les valeurs des $C_{i k}$ pour $i+k>n+1,$ et en particulier les valeurs ultimes $C_{i n}$ pour $2 \leq i \leq n$. Ces notations sont illustrées dans le triangle suivant:
		%	$$
		%	C=\left(\begin{array}{ccccc}
			%	\rowcolor{white}	C_{1,1} & C_{1,2} & \cdots & C_{1, n-1} & C_{1, n} \\
			%	\rowcolor{white}	C_{2,1} & C_{2,2} & \cdots & C_{2, n-1} & \\
			%	\rowcolor{white}	\vdots & \vdots & \ddots & & \\
			%	\rowcolor{white}	C_{n-1,1} & C_{n-1,2} & & &  \\
			%	\rowcolor{white}	C_{n, 1} & & & 
			%	\end{array}\right)
		%	$$
		La méthode de Chain Ladder estime les montants inconnus, $C_{i k}$ pour $i+k>n+1,$ par
		\begin{equation}\label{CL1}
			\hat{C}_{i k}=C_{i, n+1-i} \cdot \hat{f}_{n+1-i} \cdots \hat{f}_{k-1} \quad i+k>n+1
		\end{equation}	où
		\begin{equation}\label{CL2}
			\hat{f}_{k}=\frac{\color{OrangeProfondIRA}\sum_{i=1}^{n-k} C_{i, k+1}}{\color{BleuProfondIRA}\sum_{i=1}^{n-k} C_{i k}} \quad 1 \leq k \leq n-1 .
		\end{equation}
		La réserve de sinistre pour l'année d'accident ($R_{i}$, $2 \leq i \leq n$), est alors estimée par
\begin{align*}
			\hat{R}_{i}=&C_{in }-C_{i, n+1-i}\\
			&=C_{i, n+1-i} \cdot \hat{f}_{n+1-i} \cdots \hat{f}_{n-1}-C_{i, n+1-i} 
\end{align*}
		
\tikzset{BarreStyle/.style =   {opacity=.3,line width=15 mm,line cap=round,color=#1}}
\begin{tikzpicture}[baseline=(A.center)]
	%	
	\matrix (A) [matrix of math nodes,%
	left delimiter  = (,%
	right delimiter =)]%
	{%
		C_{1,1} & C_{1,2} & \cdots &C_{1, n+1-i} & \cdots & C_{1, n-1} & \node (A-1-7) {\color{BleuProfondIRA}C_{1, n}}; \\
		C_{2,1} & C_{2,2} & \cdots &C_{2, n+1-i} & \cdots & \node (A-2-6) {\color{BleuProfondIRA}C_{2, n-1} };& \\
		\vdots & \vdots & \cdots & \vdots & \ddots & & \\
		C_{i,1} & 	C_{i,2} & \cdots & \node (A-4-4){\color{BleuProfondIRA} C_{i,n+1-i}}; &  & & \\
		\vdots & \vdots & \ddots & & \\
		C_{n-1,1} &\node (A-6-2) {\color{BleuProfondIRA}C_{n-1,2}}; & & & & & \\
		\node (A-7-1) {\color{BleuProfondIRA}C_{n, 1}}; & & & & & &\\
	};
	\node [draw,above=10pt] at (A.north) 	{ Délais de réglement};
	\node [draw,left=20pt,rotate=90, align=right, xshift=-1cm] at (A.north west) 	{ Années d'origine $i$};
	\draw[BarreStyle=BleuProfondIRA] (A-7-1.south west) to  (A-1-7.north east) ;
	\draw  (A.south east) node [ left,color=BleuProfondIRA, align=right] {Réglements de l'année $n$\\ (où $i+j=n+1$)};
	\draw[color=BleuProfondIRA,thick] (A-7-1) to[bend right] (A-6-2) to[bend right=40] node[below right, pos=1] {$\sum$}  (A-4-4) to[bend right=40]   (A-2-6) to[bend right]   (A-1-7);
\end{tikzpicture}

\end{f}


\begin{f}[Méthode de Mack]

Les deux premières hypothèses sont les suivantes:
%
\begin{equation}\label{CL3}
E\left(C_{i, k+1} \mid C_{i 1}, \ldots, C_{i k}\right)=C_{i k} f_{k} \quad 1 \leq i \leq n, 1 \leq k \leq n-1
\end{equation}	
\begin{equation}\label{CL4}
\left\{C_{i 1}, \ldots, C_{i n}\right\},\left\{C_{j 1}, \ldots, C_{j n}\right\} \quad \forall i, j \quad \text{sont indépendants}
\end{equation}


Mack démontre que si on estime les paramètres du modèle (\ref{CL3}) par (\ref{CL2}) alors ce modèle stochastique (\ref{CL3}), combiné avec l'hypothèse (\ref{CL4}) fournit exactement les mêmes réserves que la méthode originale de Chain Ladder (\ref{CL1}).

Avec la notation $f_{i, k}= \frac{C_{i, k+1}}{C_{i, k}}$,  $\hat{f}_{k}$ est la moyenne des ${\color{OrangeProfondIRA}f_{i, k}}$ pondérée par les ${\color{BleuProfondIRA} C_{i, k}}$:
\begin{align*}
	\hat{f}_{k}&=\frac{\sum_{i=1}^{n-k}{\color{BleuProfondIRA} C_{i, k}}\times \color{OrangeProfondIRA}f_{i, k}}{\color{BleuProfondIRA}\sum_{i=1}^{n-k} C_{i k}}
\end{align*}	
La variance s'écrit :
\begin{align*}
	\hat{\sigma}_{k}^{2}&=\frac{1}{n-k-1} \sum_{i=1}^{n-k} C_{i k}\left(\frac{C_{i, k+1}}{C_{i k}}-\hat{f}_{k}\right)^{2} \\
	&=\frac{1}{n-k-1} \sum_{i=1}^{n-k}  \left(\frac{\color{BleuProfondIRA} C_{i, k+1}-{\color{BleuProfondIRA} C_{i, k}} \hat{f}_{k}}{\color{BleuProfondIRA} \sqrt{C_{i, k}}}\right)^{2} 
\end{align*}		

%\begin{center}
%	\begin{tikzpicture}
%		%\draw[Bracket-Bracket] (0,0) -- (2,0);
%		%\draw[{Bracket[reversed]-Bracket[reversed]}] (0,1) -- (2,1);
%		%\draw[{Parenthesis-Parenthesis[reversed]}] (0,2) -- (2,2);
%		\draw [color=BleuProfondIRA,decorate,decoration={brace,amplitude=10pt,raise=1pt},yshift=0pt] (0,0)	-- (0,1);
%		\draw[gray!40] (0,0) -- (0,1) -- (1,1) -- cycle node[below] {$C_{i, k+1}$};
%		\draw[color=BleuProfondIRA] (0.2,0.2) -- (0.2,1) -- (1,1) -- cycle node {};
%		\node at (1.2,0.3) {$-$};
%		\draw[gray!40]  (2.3,0.2) --  (2.3,1) -- (1.5,0.2) -- cycle node[left=6pt, above] {$0$};
%		\draw[gray!40] (1.5,0) -- (1.5,1) -- (2.5,1) -- cycle node[below] {$C_{i, k}$};
%		\draw[color=BleuProfondIRA] (1.5,0.2) -- (1.5,1) -- (2.3,1) -- cycle node {};
%		\node at (2.5,0.3) {$\times$};
%		\draw[color=BleuProfondIRA] (3,1) -- (3.2,1) -- (3.8,0.2) -- (3.6,0.2) -- cycle node[below] {$f_{ k}$};	
%		\draw [color=BleuProfondIRA,decorate,decoration={brace,amplitude=10pt,mirror,raise=0pt},yshift=0pt] (3.8,0)	-- (3.8,1);
%		\node at (4.3,0.3) {$\oslash$};
%		\draw[gray!40] (4.5,0) -- (4.5,1) -- (5.5,1) -- cycle node[below] {$\sqrt{C_{i, k}}$};
%		\draw[color=BleuProfondIRA] (4.5,0.2) -- (4.5,1) -- (5.3,1) -- cycle node {};
%		\draw [color=BleuProfondIRA!30,decorate,decoration={brace,amplitude=10pt,raise=1pt,mirror},yshift=-15] (0,0)	-- (5.5,0);
%		\draw[color=BleuProfondIRA,->] (2.75,-1) -- (2.75,-2) node[pos=0.5,right,align=left] {\tiny$\sum$ of square\\\tiny by col};	
%		\draw[BleuProfondIRA]  (2.25,-2) rectangle (3.25,-2.3)
%		node[below] {$\hat{\sigma}_{k}^{2}$}; 
%	\end{tikzpicture}
%\end{center}
%où $\oslash$ désigne la division de Hamadard désigné par $/$ en iml (ou division élément par élément de deux matrices).
%
%Il n'y a pas besoin d'enlever les valeurs de la diagonale de la première matrice car on ajoute des valeurs manquantes avec la deuxième matrice.

		

Le troisième hypothèse concerne la distribution de $R_{i}$ pour pouvoir construire facilement des intervalles de confiance sur les réserves estimées. Si  la distribution est normale, de moyenne la valeur estimée $\hat{R}_{i}$ et d'écart-type donné par l'erreur standard $\operatorname{se}\left(\hat{R}_{i}\right)$. Un intervalle de confiance à $95 \%$ est alors donné par $\left[\hat{R}_{i}-2 \operatorname{se}\left(\hat{R}_{i}\right), \hat{R}_{i}+2 \operatorname{se}\left(\hat{R}_{i}\right)\right]$.

Si la distribution est supposée lognormale, les bornes d'un intervalle de confiance à $95 \%$ seront alors données par
$$
\left[\hat{R}_{i} \exp \left(\frac{-\sigma_{i}^{2}}{2}-2 \sigma_{i}\right), \hat{R}_{i} \exp \left(\frac{-\sigma_{i}^{2}}{2}+2 \sigma_{i}\right)\right]
$$
		
\end{f}
\hrule

\begin{f}[Le modèle risque collectif]
Le modèle collectif est le modèle de base en actuariat non-vie?
	 $X_{i}$ désigne le montant du $i^e$, $N$ désigne le nombre de sinistres et $S$ le montant total au cours d'une année
	$$
	S=\sum_{i=1}^{N} X_{i}
	$$
	en sachant que $S=0$ lorsque $N=0 $ et que $\left\{X_{i}\right\}_{i=1}^{\infty}$ est une séquence $iid$ et  $N \perp \left\{X_{i}\right\}_{i=1}^{\infty}$. 
La difficulté est d'obtenir la distribution de $S$, alors même que $\E [N]$ n'est pas grand au sens du TCL.
\end{f}


\begin{f}[La distribution de $S$]
	
Soit $G(x)=\mathbb{P}(S \leq x)$, $F(x)=\mathbb{P}\left(X_{1} \leq x\right)$ , et $p_{n}=\mathbb{P}(N=n)$ de sorte que $\left\{p_{n}\right\}_{n=0}^{\infty}$ soit la fonction de probabilité pour le nombre de sinistres.
	
	$$
	\{S \leq x\}=\bigcup_{n=0}^{\infty}\{S \leq x \text { et } N=n\}
	$$
	$$
	\mathbb{P}(S \leq x \mid N=n)=\mathbb{P}\left(\sum_{i=1}^{n} X_{i} \leq x\right)=F^{n *}(x)
	$$
	Ainsi, pour $x \geq 0$
	
	\begin{equation*}\label{GxCollectif}
		G(x)=\sum_{n=0}^{\infty} p_{n} F^{n *}(x)
	\end{equation*}
	où $F^{n *}$ désigne la convolution $n^e$, malheureusement elle n'existe pas sous forme fermée pour de nombreuses distributions.	
	
Si $E[X]=m$
	$$
	E[S]=E\left[N m\right]=E[N] m
	$$
	Ce résultat est très intéressant, car il indique que le montant total attendu des sinistres est le produit du nombre attendu de sinistres et du montant attendu de chaque sinistre. De même, en utilisant le fait que $\left\{X_{i}\right\}_{i=1}^{\infty}$ sont des variables aléatoires indépendantes,
	$$
	V[S \mid N=n]=V\left[\sum_{i=1}^{n} X_{i}\right]=\sum_{i=1}^{n} V\left[X_{i}\right]
	$$
	$$
	\begin{aligned}
		V[S] &=E[V(S \mid N)]+V[E(S \mid N)] \\
		&=E[N] V\left[X_{i}\right]+V[N] m^{2}
	\end{aligned}
	$$    
\end{f}

\hrule


\begin{f}[La classe de distributions $(a, b, 0)$]
	
	Une distribution de comptage est dite  $(a, b, 0)$ si sa fonction de probabilité $\left\{p_{n}\right]_{n=0}^{\infty}$ peut être calculée de manière récursive à partir de la formule
	$$
	p_{n}=\left(a+\frac{b}{n}\right) p_{n-1}
	$$
	pour $n=1,2,3, \ldots,$ où $a$ et $b$ sont des constantes.
	
	Il existe exactement trois distributions non triviales dans la classe $(a, b, 0)$, à savoir Poisson, binomiale et binomiale négative. Voici les valeurs de $a$ et $b$ pour les principales distributions $(a, b, 0)$:
	\begin{center}
		\begin{tabular}{ccc} 
			& $a$ & $b$ \\
			\hline$\mathcal{P}_{ois}(\lambda)$ & 0 & $\lambda$ \\
			$\mathcal{B}_{in}(n, q)$ & $-q /(1-q)$ & $(n+1) q /(1-q)$ \\
			$\mathcal{N}\mathcal{B}_{in}(k, q)$ & $1-q$ & $(1-q)(k-1)$ \\
			$\mathcal{G}_{eo}( q)$ & $1-q$ & 0 \\
			\textit{    Distribution de Panjer} & $\frac{\lambda}{\alpha+\lambda}$ &    $\frac {(\alpha -1)\lambda }{\alpha +\lambda }$ \\
			\hline 
		\end{tabular}
		
	\end{center}
	
	{\footnotesize\color{OrangeProfondIRA} La loi géométrique est un cas particulier de la binomiale négative où k=1.}
\end{f}

\begin{f}[Algorithme d'agrégation de Panjer]
	
	L'\textbf{algorithme de Panjer} vise l'estimation de distribution d'une loi composée coût-fréquence dans des conditions particulières.
	\begin{itemize}
		\item $(X_i)_{i=1}^{N}$ $iid$ discrètes définies sur $\{0,h,2h,3h...\}$
		\item la loi du nombre dans la classe dite $(a,b,0)$
	\end{itemize}
	

Puisque nous supposons désormais que les montants individuels des demandes sont répartis sur les entiers non négatifs, il s'ensuit que $S$ est également réparti sur les entiers non négatifs. 
Comme $S=\sum_{i=1}^{N} X_{i}$, il s'ensuit que $S=0$ si $N=0$ ou si $N=n$ et $\sum_{i=1}^{n} X_{i}=0 . $ Comme $\sum_{i=1}^{n} X_{i}=0$ uniquement si chaque $X_{i}=0,$ il s'ensuit par indépendance que
	$$
	\mathbb{P}\left(\sum_{i=1}^{n} X_{i}=0\right)=f_{0}^{n}
	$$
	
\begin{equation*}
		\begin{cases}\label{Panjer0}\displaystyle
		g_{0}=p_{0}+\sum_{n=1}^{\infty} p_{n} f_{0}^{n}=P_{N}\left(f_{0}\right){\color{OrangeProfondIRA}\text{ si }a \ne 0},\\
	g_0=p_0\cdot \exp(f_0 b){\color{OrangeProfondIRA}\text{ si }a = 0,}\\
		g_{k}=\frac{1}{1-a f_{0}} \sum_{j=1}^{k}\left(a+\frac{b j}{k}\right) f_{j} g_{k-j}
		\end{cases}	
	\end{equation*}
$g_{x}$ est exprimé en fonction de $g_{0}, g_{1}, \ldots, g_{x-1},$ de sorte que le calcul de la fonction de probabilité est récursif. Dans toutes les applications pratiques de cette formule, un ordinateur est nécessaire pour effectuer les calculs. Cependant, l'avantage de la formule de récursivité de Panjer par rapport à la formule pour $g_{x}$ est qu'il n'est pas nécessaire de calculer les convolutions, ce qui est beaucoup plus efficace d'un point de vue computationnel.
\end{f}



%\begin{f}{Algorithme de Panjer}{Exemple}
%\begin{figure}
%	\includegraphics[width=0.90\textwidth]{../../Reinsurance_M2/Graph/Expba07.pdf}
%\end{figure}
%The following example shows the approximated density of 
%$\scriptstyle S = \sum_{i=1}^N X_i$ where $\scriptstyle N \sim \text{NegBin}(3.5,0.3)$ and $\scriptstyle X \sim \text{Frechet}(1.7,1)$ with lattice width $h = 0.04$.
%\end{f}

\begin{f}[Panjer et la loi de Poisson]
	Lorsque la fréquence suit une loi de Poisson, cela implique que {\color{OrangeProfondIRA} $a=0$ et $b=\lambda$.}
	$$
\begin{cases}
		g_0= e^{-\lambda (1-f_0)}
	\\
	g_k=\frac{\lambda}{k}\sum_{j=1}^k j . f_j  . g_{k-j}
\end{cases}	
$$
	
	L'algorithme de Panjer nécessite la discrétisation de la variable $Y_i$, l'excédent de sinistre entre la la franchise et la limite.
\end{f}

\begin{f}[Panjer et Pollaczeck-Khinchine-Beekman]
	
	Soit $\tau_1$ le premier instant où $R_t<\kappa(=\kappa_0)$. On pose alors  $L_1=\kappa -R_{\tau_1}$.
	On redémarre le processus avec $\kappa_1=\kappa_0-R_{\tau_1}$ pour trouver  $\tau_2$ et $L_2=\kappa_1 -R_{\tau_2}$.
	En continuant de la sorte, on constate que :
	$$
	M=\sup_{t\geq 0}\{ S_t -ct \} =\sum_{k=1}^{K}L_k
	$$
	où $K\sim \mathcal{G}eo(q)$ avec $q=1-\psi(0)$.  
	En remarquant que les variables $\left(  L_k\right)_{1\leq k\leq K}$ sont $iid$ ($F$), on a alors $\psi(k)=\mathbb{P}[M>\kappa]$ donnée par la formule de Pollaczeck-Khinchine-Beekman.
	
	
	La représentation 
	$$
	\psi(\kappa)=\mathbb{P}\left[\sum_{j = 1}^{K}L_j>\kappa \right] 
	$$
	permet d'évaluer la probabilité de ruine sur horizon infini à l'aide de l'algorithme de Panjer.
	
\end{f}






\end{multicols}

\newpage

\begin{center}
	\section*{Reinsurance}
	\medskip
\end{center}

\begin{multicols}{2}	
	% !TeX root = ActuarialFormSheet_MBFA-en.tex
% !TeX spellcheck = en_GB


\begin{f}[Transfer/retrocession schemes]

\tikzstyle{Sources} = [rectangle, rounded corners, minimum width=5cm, minimum height=1cm,text centered, text width=5cm, draw=black, fill=BleuProfondIRA!40]
\tikzstyle{projections} = [ellipse, trapezium left angle=70, trapezium right angle=110, minimum height=1cm, text centered, draw=black, fill=FushiaIRA!30]
\tikzstyle{Calculs} = [rectangle, minimum width=5cm, minimum height=1cm, text centered, text width=5cm, draw=black, fill=OrangeProfondIRA!30]
\tikzstyle{modalite} = [ellipse, minimum width=3cm, minimum height=1cm, text centered, draw=black, fill=BleuProfondIRA!40]
\tikzstyle{arrow} = [thick,->,>=stealth]


\resizebox{\linewidth}{!}{			%
\begin{tikzpicture}[node distance=1.5cm]
\node (Assure) [Sources] {\begin{tabular}{c} INSURED\\ \rowcolor{BleuProfondIRA!40}\tiny subscriber\end{tabular}};
\node (Ca) [projections, below left = of Assure, xshift=2.5cm, yshift=.5cm] {Insurance contract};
\node (AG) [modalite, below right = of Assure, xshift=-2.5cm, yshift=.5cm] {General Agent / Broker};
\node (Assureur) [Calculs, below of=Assure, node distance = 3cm] {\begin{tabular}{c}DIRECT INSURER\\
\rowcolor{OrangeProfondIRA!30}  \tiny  transferor\end{tabular} };
\node (ConvR) [projections, below left = of Assureur, xshift=2.5cm, yshift=.5cm] {Reinsurance agreement};
\node (Courtr) [modalite, below right = of Assureur, xshift=-2.5cm, yshift=.5cm] {Reinsurance broker};
\node (Reass) [Calculs, below of=Assureur, node distance = 3cm] {REINSURER(S)};
\node (ConvR2) [projections, below left = of Reass, xshift=2.5cm, yshift=.5cm] {Reinsurance agreement};
\node (Courtr2) [modalite, below right = of Reass, xshift=-2.5cm, yshift=.5cm] {Reinsurance broker};
\node (Retro) [Calculs, below of=Reass, node distance = 3cm] {RETROASSIGNEE(S)};
\draw [arrow] (Assure.south) -- (Assureur.north);
\draw [arrow] (Assureur.south) -- (Reass.north);
\draw [arrow] (Assureur.south) -- (Reass.north);
\draw [arrow] (Reass.south) -- (Retro.north);
%
\end{tikzpicture}}

\end{f}
\hrule


%https://www.reinsurancene.ws/top-50-reinsurance-groups/

\begin{f}[Key words of reinsurance]

\textbf{Cedant :} client of the reinsurer, i.e., the direct insurer, who transfers (cedes) risks to the reinsurer in exchange for the payment
of a \textbf{reinsurance premium}.

\textbf{Cession :} transfer of risks by the direct insurer to the reinsurer.

\textbf{Value Exposure} : limit of the amount of risk covered by a (re)insurance contract.

\textbf{Proportional reinsurance :} proportional participation of the reinsurer in the premiums and claims of the direct insurer.

\textbf{Quota Share :} type of proportional reinsurance where the reinsurer participates in a given percentage of all risks underwritten by a direct insurer in a given line of business.

\textbf{Surplus Share :} A type of proportional reinsurance where the reinsurer covers risks beyond the direct insurer's 
full retention amount. This ratio is calculated on the capacity of the risk subscribed (\(\approx\) Maximum possible loss).

\textbf{Reinsurance commission :} remuneration that the reinsurer grants to the insurer or brokers as compensation for the costs of acquiring and managing insurance contracts.

\textbf{Excess Reinsurance :}
 coverage by the reinsurer of claims exceeding a certain amount, against payment by the direct insurer of a specific reinsurance premium.

\textbf{Retrocession :}\index{R\'eassurance! R\'etrocession}  share of risks that the reinsurer cedes to other reinsurers.

\textbf{Co-insurance :}\index{R\'eassurance! Coassurance} participation of several direct insurers in the same risk.

We then use the expression \textbf{reinsurance pool}.
The main reinsurer is called \textbf{leading reinsurer}.

\textbf{Reinsurance treaty :} contract concluded between the direct insurer and the reinsurer on one or more of the insurer's portfolios.

\textbf{Facultative reinsurance :}
It differs from the reinsurance treaty by underwriting risk by risk (or policy by policy) (case by case, one risk at a time).

\end{f}
\hrule




\begin{f}
	[The economic role of reinsurance]

Insurance and reinsurance share the same purpose : the pooling of risks.
Reinsurance intervenes in particular on risks :
\begin{itemize}
	\item independent, but unitarily expensive (airplane, ship, industrial sites\ldots),
	\item small amounts (breakage, car, ...) but correlated during large-scale events, resulting in expensive accumulations,
	\item aggregated within a portfolio of insurance policies,
	\item little known or new.
\end{itemize}
	

Reinsurance allows to increase the business issuing capacity, ensure the financial stability of the insurer, especially in the event of disasters, reduce their capital requirements, and benefit from the expertise of the reinsurer.

\end{f}	
\hrule



\begin{f}
	[Types of reinsurance agreements]

{\color{white}.}	

\begin{center}
		\resizebox{0.90\linewidth}{!}{%!TeX spellcheck = en_GB
%!TEX root = ActuarialFormSheet_MBFA-en.tex

%%Created by jPicEdt 1.4.1_03: mixed JPIC-XML/LaTeX format
%%Tue Jul 03 17:58:32 CEST 2012
%%Begin JPIC-XML
%<?xml version="1.0" standalone="yes"?>
%<jpic x-min="5" x-max="100" y-min="2.5" y-max="37.5" auto-bounding="true">
%<multicurve points= "(40,30);(40,30);(40,20);(40,20)"
%	 fill-style= "none"
%	 />
%<multicurve points= "(20,30);(20,30);(60,30);(60,30)"
%	 fill-style= "none"
%	 />
%<multicurve points= "(60,35);(60,35);(60,30);(60,30)"
%	 fill-style= "none"
%	 />
%<multicurve points= "(20,30);(20,30);(20,35);(20,35)"
%	 fill-style= "none"
%	 />
%<multicurve points= "(20,20);(20,20);(20,15);(20,15)"
%	 fill-style= "none"
%	 />
%<multicurve points= "(20,20);(20,20);(70,20);(70,20)"
%	 fill-style= "none"
%	 />
%<multicurve points= "(70,20);(70,20);(70,15);(70,15)"
%	 fill-style= "none"
%	 />
%<multicurve points= "(5,15);(5,15);(30,15);(30,15)"
%	 fill-style= "none"
%	 />
%<multicurve points= "(30,10);(30,10);(30,15);(30,15)"
%	 fill-style= "none"
%	 />
%<multicurve points= "(5,15);(5,15);(5,10);(5,10)"
%	 fill-style= "none"
%	 />
%<multicurve points= "(55,15);(55,15);(55,10);(55,10)"
%	 fill-style= "none"
%	 />
%<multicurve points= "(55,15);(55,15);(100,15);(100,15)"
%	 fill-style= "none"
%	 />
%<multicurve points= "(75,15);(75,15);(75,10);(75,10)"
%	 fill-style= "none"
%	 />
%<multicurve points= "(100,15);(100,15);(100,10);(100,10)"
%	 fill-style= "none"
%	 />
%<text text-vert-align= "center-v"
%	 anchor-point= "(5,7.5)"
%	 fill-style= "none"
%	 text-frame= "noframe"
%	 text-hor-align= "center-h"
%	 >
%Quote-Part
%</text>
%<text text-vert-align= "center-v"
%	 anchor-point= "(30,7.5)"
%	 fill-style= "none"
%	 text-frame= "noframe"
%	 text-hor-align= "center-h"
%	 >
%Excédent
%</text>
%<text text-vert-align= "center-v"
%	 anchor-point= "(30,2.5)"
%	 fill-style= "none"
%	 text-frame= "noframe"
%	 text-hor-align= "center-h"
%	 >
%de plein
%</text>
%<text text-vert-align= "center-v"
%	 anchor-point= "(55,7.5)"
%	 fill-style= "none"
%	 text-frame= "noframe"
%	 text-hor-align= "center-h"
%	 >
%Excédent
%</text>
%<text text-vert-align= "center-v"
%	 anchor-point= "(75,7.5)"
%	 fill-style= "none"
%	 text-frame= "noframe"
%	 text-hor-align= "center-h"
%	 >
%Excédent
%</text>
%<text text-vert-align= "center-v"
%	 anchor-point= "(100,7.5)"
%	 fill-style= "none"
%	 text-frame= "noframe"
%	 text-hor-align= "center-h"
%	 >
%Excédent Annuel
%</text>
%<text text-vert-align= "center-v"
%	 anchor-point= "(70,22.5)"
%	 fill-style= "none"
%	 text-frame= "noframe"
%	 text-hor-align= "center-h"
%	 >
%Non proportionnelle
%</text>
%<text text-vert-align= "center-v"
%	 anchor-point= "(20,22.5)"
%	 fill-style= "none"
%	 text-frame= "noframe"
%	 text-hor-align= "center-h"
%	 >
%Proportionnelle
%</text>
%<text text-vert-align= "center-v"
%	 anchor-point= "(20,37.5)"
%	 fill-style= "none"
%	 text-frame= "noframe"
%	 text-hor-align= "center-h"
%	 >
%Facultative
%</text>
%<text text-vert-align= "center-v"
%	 anchor-point= "(60,37.5)"
%	 fill-style= "none"
%	 text-frame= "noframe"
%	 text-hor-align= "center-h"
%	 >
%Traité
%</text>
%<text text-vert-align= "center-v"
%	 anchor-point= "(55,2.5)"
%	 fill-style= "none"
%	 text-frame= "noframe"
%	 text-hor-align= "center-h"
%	 >
%par sinistre
%</text>
%<text text-vert-align= "center-v"
%	 anchor-point= "(75,2.5)"
%	 fill-style= "none"
%	 text-frame= "noframe"
%	 text-hor-align= "center-h"
%	 >
%par événement
%</text>
%<text text-vert-align= "center-v"
%	 anchor-point= "(100,2.5)"
%	 fill-style= "none"
%	 text-frame= "noframe"
%	 text-hor-align= "center-h"
%	 >
%Stop Loss
%</text>
%</jpic>
%%End JPIC-XML
%LaTeX-picture environment using emulated lines and arcs
%You can rescale the whole picture (to 80% for instance) by using the command \def\JPicScale{0.8}
\ifx\JPicScale\undefined\def\JPicScale{1}\fi
\unitlength \JPicScale mm
\begin{picture}(100,45)(5,0)
\linethickness{0.3mm}
\put(40,20){\line(0,1){10}}
\linethickness{0.3mm}
\put(20,30){\line(1,0){40}}
\linethickness{0.3mm}
\put(60,30){\line(0,1){5}}
\linethickness{0.3mm}
\put(20,30){\line(0,1){5}}
\linethickness{0.3mm}
\put(20,15){\line(0,1){5}}
\linethickness{0.3mm}
\put(20,20){\line(1,0){50}}
\linethickness{0.3mm}
\put(70,15){\line(0,1){5}}
\linethickness{0.3mm}
\put(10,15){\line(1,0){20}}
\linethickness{0.3mm}
\put(30,10){\line(0,1){5}}
\linethickness{0.3mm}
\put(10,10){\line(0,1){5}}
\linethickness{0.3mm}
\put(50,10){\line(0,1){5}}
\linethickness{0.3mm}
\put(50,15){\line(1,0){45}}
\linethickness{0.3mm}
\put(70,10){\line(0,1){5}}
\linethickness{0.3mm}
\put(95,10){\line(0,1){5}}
\put(10,7.5){\makebox(0,0)[cc]{Quota}}

\put(30,7.5){\makebox(0,0)[cc]{Full}}

\put(30,2.5){\makebox(0,0)[cc]{surplus}}

\put(50,7.5){\makebox(0,0)[cc]{Surplus }}

\put(70,7.5){\makebox(0,0)[cc]{Surplus per }}

\put(95,7.5){\makebox(0,0)[cc]{Excess losses}}

\put(70,22.5){\makebox(0,0)[cc]{Exccess}}

\put(20,22.5){\makebox(0,0)[cc]{Proportional}}

\put(20,37.5){\makebox(0,0)[cc]{Optional}}

\put(60,37.5){\makebox(0,0)[cc]{Treaty}}

\put(50,2.5){\makebox(0,0)[cc]{per claim}}

\put(70,2.5){\makebox(0,0)[cc]{event}}

\put(95,2.5){\makebox(0,0)[cc]{losses}}

\end{picture}
}
\end{center}
\end{f}
\hrule


\begin{f}[Types of reinsurance through an example]
	
Our insurer reinsures \(N=30\) insurance policies, with a total premium of 10M\EUR{} (\(P=\sum_{i=1\ldots N}P_i\)).
The total capacity is 180M\EUR{} (\(\sum_{i=1\ldots30}K_i\)).
\(S_r\) will be the total share of the loss covered by the insurer and \(P_r\) the total reinsurance premium.
Here are the \(n=8\) policies affected by losses (\(1\geq i \geq n\)), the losses of the other policies being zero (\(S_i=0, \forall i>n\)) :
	
	\begin{center}\footnotesize
		\renewcommand{\arraystretch}{1.25}
		\begin{tabular}{|l|rrrrrrrr|}
			\hline
			\rowcolor{BleuProfondIRA!40}         Claim number	& 1 & 2 & 3 & 4 & 5 & 6 & 7 & 8 \\ \hline 
			Prime  (k\EUR{})		& 500 & 200 & 100 & 100 & 50 & 200 & 500 & 200 \\ 
			Value Exposure  (M\EUR{})   	& 8 & 5 & 3 & 2 & 3 & 5 & 8 & 8 \\ 
			Claims  (M\EUR{})   	& 1 & 1 & 1 & 2 & 3 & 3 & 5 & 8 \\ \hline
		\end{tabular}
		\renewcommand{\arraystretch}{1}
	\end{center}
	
	The \(S/P\) is 240\%.
	
	
	\begin{center}
		\begin{tikzpicture}
			\begin{axis}[axis x line=bottom, axis y line = left,ymin=0,ymax=24,xmin=0, xmax=10, height=7cm,width=5cm, xticklabel style ={font=\footnotesize,align=center},  yticklabel style ={font=\footnotesize}, 
				legend style={at={(0.5,-0.15)},anchor=north,legend columns=-1,font=\tiny}]
				\addplot [ybar,fill=BleuProfondIRA!40,mark=none,draw=none] table [x index=0, y index=1] {..\\_Common\\ReinsuranceClaim.dat} ;
				\legend{Ordered Claims}
			\end{axis}	
		\end{tikzpicture}
		\begin{tikzpicture}
			\begin{axis}[axis x line=center,axis y line = left,ymin=0,ymax=24,xmin=0, xmax=10,height=7cm,width=5cm, xticklabel style ={font=\footnotesize,align=center},  yticklabel style ={font=\footnotesize}, 
				legend style={at={(0.5,-0.15)},
					anchor=north,legend columns=-1,font=\tiny}]
				\addplot [ybar,fill=BleuProfondIRA!40,mark=none,draw=none] table [x index=0, y index=2] {..\\_Common\\ReinsuranceClaim.dat} ;
				\legend{Cumulative Claims}
			\end{axis}	
		\end{tikzpicture}
		%\includegraphics{../../Reinsurance_M2/Graph/ReinsuranceClaim.pdf}
		%\includegraphics{../../Reinsurance_M2/Graph/ReinsuranceClaimCum.pdf}
	\end{center}
\medskip	


\textbf{Quota:}
\[
S_r=\alpha \sum_{i=1\ldots n}S_i\quad \ \ P_r=\alpha\sum_{i=1\ldots N}P_i 
\]
where \(\alpha\ \in [0,1]\) (25\% in the figure) is the share transferred in Quota.
	
	%\setlength{\tabcolsep}{1cm}

%\begin{tabular}{|l|rrrrrrrr|}
%	\hline
%	\rowcolor{BleuProfondIRA!40}         Num de sinistre		& 1 & 2 & 3 & 4 & 5 & 6 & 7 & 8 \\ \hline \hline
%	Sinistres  (M\EUR{})   	& 1 & 1 & 1 & 2 & 3 & 3 & 5 & 8 \\ 
%	Cédante  (M\EUR{})   	& 0,75 & 0,75 & 0,75 &
%	1,50 & 2,25 & 2,25 & 
%	3,75 & 6,00\\
%	Réassurance  (M\EUR{})  & 0,25 & 0,25 & 0,25 &
%	0,50 & 0,75 & 0,75 & 
%	1,25 & 2,00\\
%	Prime portef conservée  & \multicolumn{2}{r}{\cellcolor{mbfaulmbleu!10}7,5 M\EUR{}} & & & & & &\\
%	Sinistre conservé  	& \multicolumn{2}{r}{18} & & & & & & \\
%	S/P conservé  		& \multicolumn{2}{r}{\cellcolor{mbfaulmbleu!10}240\%} & & & & & & \\
%	\hline
%\end{tabular}
	
\begin{center}
		\begin{tikzpicture}
		\begin{axis}[axis x line=bottom, axis y line = left,ymin=0,ymax=24,xmin=0, xmax=10, height=7cm,width=5cm, xticklabel style ={font=\footnotesize,align=center},  yticklabel style ={font=\footnotesize}, 
			legend style={at={(0.5,-0.15)},anchor=north,legend columns=1,font=\tiny}]
			\addplot [ybar ,fill=OrangeMoyenIRA!40,mark=none,draw=OrangeMoyenIRA!40] table [x index=0, y index=1] {..\\_Common\\ReinsuranceClaim.dat} ;
			\addplot [ybar ,fill=BleuProfondIRA!40,mark=none,draw=BleuProfondIRA!40] table [x index=0, y index=6] {..\\_Common\\ReinsuranceClaim.dat} ;
			\legend{Claims Transfered,Claims Retained Quote}
		\end{axis}	
	\end{tikzpicture}
	%	
	\begin{tikzpicture}
		\begin{axis}[axis x line=center,axis y line = left,ymin=0,ymax=24,xmin=0, xmax=10,height=7cm,width=5cm, xticklabel style ={font=\footnotesize,align=center},  yticklabel style ={font=\footnotesize}, 
			legend style={at={(0.5,-0.15)},
				anchor=north,legend columns=1,font=\tiny}]
			\addplot [ybar ,fill=OrangeMoyenIRA!40,mark=none,draw=OrangeMoyenIRA!40] table [x index=0, y index=2] {..\\_Common\\ReinsuranceClaim.dat} ;
			\addplot [ybar ,fill=BleuProfondIRA!40,mark=none,draw=BleuProfondIRA!40] table [x index=0, y index=7] {..\\_Common\\ReinsuranceClaim.dat} ;
			\legend{Claims Transfered,Claims Retained Quote}
		\end{axis}	
	\end{tikzpicture}
	%\includegraphics{../../Reinsurance_M2/Graph/ReinsuranceClaimQuotePart.pdf}
	%\includegraphics{../../Reinsurance_M2/Graph/ReinsuranceClaimCumQuotePart.pdf}

\end{center}
\medskip

\textbf{Full surplus}, the full is noted \(\boldsymbol{K}\) (2M\EUR{} in the example), \(\alpha_i\) represents the cession rate of policy \(i\).
	\[
	S_r= \sum_{i=1\ldots n}\underbrace{\left(\frac{\left(K_i-\boldsymbol{K} \right)_+ }{K_i} \right)}_{\alpha_i} S_i\quad \ \ P_r=\sum_{i=1\ldots N}\left(\frac{\left(K_i-\boldsymbol{K} \right)_+ }{K_i} \right)P_i 
	\]
	%\renewcommand{\arraystretch}{1.25}
	%\rowcolors{1}{sectionColor}{white}
	%\setlength{\tabcolsep}{1cm}
%\begin{tabular}{|l|rrrrrrrr|}
%\hline
%\rowcolor{BleuProfondIRA!40}       Num de sinistre		& 1 & 2 & 3 & 4 & 5 & 6 & 7 & 8 \\ \hline \hline
%Sinistres  (M\EUR{})   	& 1 & 1 & 1 & 2 & 3 & 3 & 5 & 8 \\ 
%Capacité  (M\EUR{})   	& 8 & 5 & 3 & 2 & 3 & 5 & 8 & 8 \\ 
%Capacité cédante	& \multicolumn{2}{r}{60M\EUR{}} & & & & & & \\
%Prime cédante		& \multicolumn{2}{r}{\cellcolor{mbfaulmbleu!40}3M\EUR{}} & & & & & & \\
%Part cédée  (\%)   & 75 & 60 & 33 &
%0 & 33 & 60 & 
%75 & 75\\
%Cédante  (M\EUR{})  & 0,25 & 0,4 & 0,67 &
%2,00 & 2,00 & 1,20 & 
%1,25 & 2,00\\
%Réassurance  (M\EUR{})  & 0,75 & 0,60 & 0,33 &
%0,00 & 1,00 & 1,80 & 
%3,75 & 6,00\\
%Sinistre conservé  	& \multicolumn{2}{r}{\cellcolor{mbfaulmbleu!40}9,77} & & & & & & \\
%Sinistre cédé  		& \multicolumn{2}{r}{14,23} & & & & & & \\
%S/P conservée  		& \multicolumn{2}{r}{\cellcolor{mbfaulmbleu!40}325\%} & & & & & & \\
%\hline
%\end{tabular}


\begin{center}
		%\includegraphics{../../Reinsurance_M2/Graph/ReinsuranceClaimEPlein.pdf}
	%\includegraphics{../../Reinsurance_M2/Graph/ReinsuranceClaimCumEPlein.pdf}
	\begin{tikzpicture}
		%format de la date yyyy-mm-dd , pas de nom de colonne vide,
		\begin{axis}[axis x line=bottom, axis y line = left,ymin=0,ymax=24,xmin=0, xmax=10, height=7cm,width=5cm, xticklabel style ={font=\footnotesize,align=center},  yticklabel style ={font=\footnotesize}, 
			legend style={at={(0.5,-0.15)},anchor=north,legend columns=2,font=\tiny}]
			\addplot [ybar ,mark=none,draw=OrangeMoyenIRA!40,fill=OrangeMoyenIRA!40] table [x index=0, y index=1] {..\\_Common\\ReinsuranceClaim.dat} ;
			\addplot [ybar ,mark=none,draw=GrisLogoIRA,mark=none] table [x index=0, y index=11] {..\\_Common\\ReinsuranceClaim.dat} ;
			\addplot [draw=GrisLogoIRA] table [x index=0, y index=4] {..\\_Common\\ReinsuranceClaim.dat} ;
			\addplot [ybar ,fill=BleuProfondIRA!40,mark=none,draw=BleuProfondIRA!40] table [x index=0, y index=12] {..\\_Common\\ReinsuranceClaim.dat} ;
			\legend{Claims Transf,Value Exposure,Full, Claims Retained}
		\end{axis}	
	\end{tikzpicture}
	%	
	\begin{tikzpicture}
		%format de la date yyyy-mm-dd , pas de nom de colonne vide,
		\begin{axis}[axis x line=center,axis y line = left,ymin=0,ymax=24,xmin=0, xmax=10,height=7cm,width=5cm, xticklabel style ={font=\footnotesize,align=center},  yticklabel style ={font=\footnotesize}, 
			legend style={at={(0.5,-0.15)},
				anchor=north,legend columns=1,font=\tiny}]
			\addplot [ybar ,fill=OrangeMoyenIRA!40,mark=none,draw=OrangeMoyenIRA!40] table [x index=0, y index=2] {..\\_Common\\ReinsuranceClaim.dat} ;
			\addplot [ybar ,fill=BleuProfondIRA!40,mark=none,draw=BleuProfondIRA!40] table [x index=0, y index=13] {..\\_Common\\ReinsuranceClaim.dat} ;
			%		\addplot [ybar interval,fill=BleuProfondIRA,mark=none,draw=BleuProfondIRA] table [x index=0, y index=7] {..\\_Common\\ReinsuranceClaim.dat} ;
			\legend{Claims Transf, Claims Retained}
		\end{axis}	
	\end{tikzpicture}
\end{center}
\medskip

\textbf{Surplus per claim}
	%\renewcommand{\arraystretch}{1.25}
	%\rowcolors{1}{sectionColor}{white}
	%\setlength{\tabcolsep}{1cm}
	
The insurer sets the priority \(a\) and the scope \(b\) (respectively 2M\EUR{} and 4M\EUR{} in the figure).
\[
S_r= \sum_{i=1\ldots n} \min\left( \left( S_i-a\right)^+,b\right)  
\]
The premium is set by the reinsurer, based on its estimate of \(\mathbb{E}[S_r]\).
%\begin{tabular}{|l|rrrrrrrr|}
%	\hline
%	\rowcolor{BleuProfondIRA!40}        Num de sinistre		& 1 & 2 & 3 & 4 & 5 & 6 & 7 & 8 \\ \hline \hline
%	Sinistres  (M\EUR{})   	& 1 & 1 & 1 & 2 & 3 & 3 & 5 & 8 \\ 
%	Cédante  (M\EUR{})   	& 1 & 1 & 1 &
%	2 & 2 & 2 & 
%	2 & 4\\
%	Réassurance  (M\EUR{})  & 0 & 0 & 0 &
%	0 & 1 & 1 & 
%	3 & 4\\
%	\hline
%\end{tabular}

\begin{center}
		%\includegraphics{../../Reinsurance_M2/Graph/ReinsuranceClaimXSsin.pdf}
	%\includegraphics{../../Reinsurance_M2/Graph/ReinsuranceClaimCumXSsin.pdf}
	\begin{tikzpicture}
		%format de la date yyyy-mm-dd , pas de nom de colonne vide,
		\begin{axis}[axis x line=bottom, axis y line = left,ymin=0,ymax=24,xmin=0, xmax=10, height=7cm,width=5cm, xticklabel style ={font=\footnotesize,align=center},  yticklabel style ={font=\footnotesize}, 
			legend style={at={(0.5,-0.15)},anchor=north,legend columns=2,font=\tiny}]
			\addplot [ybar ,mark=none,draw=BleuProfondIRA!40,fill=BleuProfondIRA!40] table [x index=0, y index=1] {..\\_Common\\ReinsuranceClaim.dat} ;
			\addplot [ybar ,mark=none,draw=OrangeMoyenIRA!40,fill=OrangeMoyenIRA!40] table [x index=0, y index=10] {..\\_Common\\ReinsuranceClaim.dat} ;
			\addplot [ybar ,mark=none,draw=BleuProfondIRA!40,fill=BleuProfondIRA!40] table [x index=0, y index=8] {..\\_Common\\ReinsuranceClaim.dat} ;
			\addplot [draw=GrisLogoIRA] table [x index=0, y index=4] {..\\_Common\\ReinsuranceClaim.dat} ;
			\addplot [draw=GrisLogoIRA] table [x index=0, y index=5] {..\\_Common\\ReinsuranceClaim.dat} ;
			\legend{Claims Retained, Claims Transf, Priority €2M, Increased €4M }
		\end{axis}	
	\end{tikzpicture}
	%	
	\begin{tikzpicture}
		%format de la date yyyy-mm-dd , pas de nom de colonne vide,
		\begin{axis}[axis x line=center,axis y line = left,ymin=0,ymax=24,xmin=0, xmax=10,height=7cm,width=5cm, xticklabel style ={font=\footnotesize,align=center},  yticklabel style ={font=\footnotesize}, 
			legend style={at={(0.5,-0.15)},
				anchor=north,legend columns=1,font=\tiny}]
			\addplot [ybar ,fill=OrangeMoyenIRA!40,mark=none,draw=OrangeMoyenIRA!40] table [x index=0, y index=2] {..\\_Common\\ReinsuranceClaim.dat} ;
			\addplot [ybar ,fill=BleuProfondIRA!40,mark=none,draw=BleuProfondIRA!40] table [x index=0, y index=9] {..\\_Common\\ReinsuranceClaim.dat} ;
			%		\addplot [ybar ,fill=BleuProfondIRA,mark=none,draw=BleuProfondIRA] table [x index=0, y index=7] {..\\_Common\\ReinsuranceClaim.dat} ;
			\legend{Claims Transfered, Claims Retained}
		\end{axis}	
	\end{tikzpicture}
	\emph {WXL-R = Working XL per Risk}
\end{center}
	\medskip


\textbf{Surplus per event}
	%\renewcommand{\arraystretch}{1.25}
	%\rowcolors{1}{sectionColor}{white}
	%\setlength{\tabcolsep}{1cm}
	
	\[
	S_r= \sum_{\begin{array}{c}
			Cat_j,\\ i=1\ldots N
	\end{array}} \min\left( \mathds{1}_{i\in Cat_j}\times \left( S_i-a\right)^+,b\right)  
	\]
	
	In the illustration, claims refer to a single event, with a priority of 5M\EUR{} and a scope of 10M\EUR{}.
	
%\begin{tabular}{|l|rrrrrrrr|}
%	\hline
%	\rowcolor{BleuProfondIRA!40}        Num de sinistre		& 1 & 2 & 3 & 4 & 5 & 6 & 7 & 8 \\ \hline \hline
%	Sinistres  (M\EUR{})   	& 1 & 1 & 1 & 2 & 3 & 3 & 5 & 8 \\ 
%	Sinistres en Cumulé   	& 1 & 2 & 3 & 5 & 8 & 11 & 16 & 24 \\ 
%	Cédante en Cumulé   	& 1 & 2 & 3 &
%	5 & 5 & 5 & 
%	6 & 14\\
%	Réassurance en Cumulé  & 0 & 0 & 0 &
%	0 & 3 & 6 & 
%	10 & 10\\
%	\hline
%\end{tabular}

	
	%{../../Reinsurance_M2/Graph/ReinsuranceClaimXScat.pdf}
	%\includegraphics{../../Reinsurance_M2/Graph/ReinsuranceClaimCumXScat.pdf}
\begin{center}
		\begin{tikzpicture}
		\begin{axis}[axis x line=center,axis y line = left,ymin=0,ymax=24,xmin=0, xmax=10,height=7cm,width=5cm, xticklabel style ={font=\footnotesize,align=center},  yticklabel style ={font=\footnotesize}, 
			legend style={at={(0.5,-0.15)},
				anchor=north,legend columns=1,font=\tiny}]
			\addplot [ybar ,fill=OrangeMoyenIRA!40,mark=none,draw=OrangeMoyenIRA!40] table [x index=0, y index=1] {..\\_Common\\ReinsuranceClaim.dat} ;
			\addplot [ybar ,fill=BleuProfondIRA!40,mark=none,draw=BleuProfondIRA!40] table [x index=0, y index=18] {..\\_Common\\ReinsuranceClaim.dat} ;
			\legend{Claims Transfered, Claims Retained};
		\end{axis}	
	\end{tikzpicture}
	%	
	\begin{tikzpicture}
		\begin{axis}[axis x line=bottom, axis y line = left,ymin=0,ymax=24,xmin=0, xmax=10, height=7cm,width=5cm, xticklabel style ={font=\footnotesize,align=center},  yticklabel style ={font=\footnotesize}, 
			legend style={at={(0.5,-0.15)},anchor=north,legend columns=2,font=\tiny}]
			\addplot [ybar ,mark=none,draw=BleuProfondIRA!40,fill=BleuProfondIRA!40] table [x index=0, y index=2] {..\\_Common\\ReinsuranceClaim.dat} ;
			\addplot [ybar ,mark=none,draw=OrangeMoyenIRA!40,fill=OrangeMoyenIRA!40] table [x index=0, y index=16] {..\\_Common\\ReinsuranceClaim.dat} ;
			\addplot [ybar ,mark=none,draw=BleuProfondIRA!40,fill=BleuProfondIRA!40] table [x index=0, y index=17] {..\\_Common\\ReinsuranceClaim.dat} ;
			\addplot [draw=GrisLogoIRA] table [x index=0, y index=14] {..\\_Common\\ReinsuranceClaim.dat} ;
			\addplot [draw=GrisLogoIRA] table [x index=0, y index=15] {..\\_Common\\ReinsuranceClaim.dat} ;
			\legend{Claims Retained, Claims Transf, Priority €2M, Increased €4M};
		\end{axis}	
	\end{tikzpicture}
	\emph{ Cat-XL = Catastrophe XL }
\end{center}
\medskip

	
	
	%\renewcommand{\arraystretch}{1.25}
	%\rowcolors{1}{sectionColor}{white}
	%\setlength{\tabcolsep}{1cm}
	
%	L'assureur fixe la priorité à 5M\EUR{}  et la portée à 10M\EUR{}, soit un plafond de 15M\EUR{}.
%	Dans notre exemple supposons que les 8 sinistres font référence à \textbf{deux événements, une première tempête en octobre, une deuxième en décembre}.
	
%\begin{tabular}{|l|rrrrrrrr|}
%	\hline
%	\rowcolor{BleuProfondIRA!40}         Num de sinistre		& 1 & 2 & 3 & 4 & 5 & 6 & 7 & 8 \\ \hline \hline
%	Événement		&Oct & Oct & Déc & Oct & Déc & Oct & Déc & Déc \\ \hline \hline
%	Sinistres  (M\EUR{})   	& 1 & 1 & 1 & 2 & 3 & 3 & 5 & 8 \\ 
%	\hline
%\end{tabular}
%\begin{tabular}{|l|rr|r|}
%	\hline
%	\rowcolor{BleuProfondIRA!40}        Événement		&Oct &  Déc  & Total\\ \hline \hline
%	Som de sinistre		& 7 & 17& 24\\ \hline \hline
%	Cédante  	& 5 & 7& 12 \\
%	Réassurance en Cumulé  & 2 & 10& 12 \\
%	\hline
%\end{tabular}



\textbf{Annual excess losses}
	
	
Stop Loss occurs when the cumulative annual losses deteriorate.
It is expressed on the basis of the ratio \(S/P\) with a priority and a scope of \(XL\) expressed in \%.
	
\[
S_r= \min\left( \left( \sum_{i=1\ldots N} S_i - a  P\right)^+,bP\right)  
\]

\end{f}
\hrule
	
\begin{f}[The main clauses in reinsurance]

\textbf{The deductible} \(a^{ag}\) and the \textbf{aggregate limit} \(b^{ag}\) apply after the calculation of \(S_r\).

\[
S_r^{ag} = \min\left( \left(S_r - a^{ag}  \right)^+,b^{ag}\right)  
\]
\medskip

The objective of the \textbf{indexation clause} is to maintain the \underline{terms of the treaty} over several successive financial years. The treaty limits are aligned with an economic index (salary, currency, price index, ...).
\medskip

With \textbf{stabilization clause}, when the claim suffers from a \underline{long settlement}, or even a very long one (at least \(\geq 1\) year), the treaty limits are updated in the calculation of the \(S_r\) so that the respective shares of the reinsurer and the ceding company initially planned are generally respected.
\medskip
	

With the \textbf{interest sharing clause}, if in a transaction or court judgment a distinction has been made \underline{between compensation and interest}, the interest accrued between the date of the loss and the date of actual payment of the compensation will be divided between the ceding company and the reinsurer in proportion to their respective burden resulting from the application of the treaty excluding interest.


\medskip


The \textbf{guarantee reinstatement clause}
only concerns \emph{processed in excess of loss by risk or by event} which could be triggered several times during the year.
The reinsurer limits its benefit to \(N\) times the scope of the \(XS\), in return for the payment of an additional premium.
Reconstitution can be done pro rata temporis (time remaining until the expiry date of the treaty) or pro rata to the absorbed capital, or both (double pro rata).




\textbf{Interlocking Clause} is used in event-based XS treaties, which operate by subscription exercise and not \underline{by occurrence exercise}.
The interlocking clause will have the effect of recalculating the treaty limits, because the same event can trigger the treaty for both \(n\) and \(N-1\) subscriptions.


\end{f}
\hrule

\begin{f}[Public reinsurance]
The \href{http://www.ccr.fr}{Caisse Centrale de Réassurance (CCR)} offers, with the State guarantee, unlimited coverage for branches specific to the French market.
\begin{itemize}
\item   exceptional risks linked to transport,
\item   liability insurance for operators of nuclear vessels and installations,
\item   the risks of natural disasters,
\item   the risks of attacks and acts of terrorism,
\item   the Public Credit Insurance Supplement (CAP).
\end{itemize}
It also manages certain Public Funds on behalf of the State, in particular the Cat Nat regime.

Also, the \href{http://www.gareat.com/}{GAREAT} is a non-profit Economic Interest Group (GIE), mandated by its members, which manages reinsurance of risks of attacks and acts of terrorism with the support of the State via the CCR.
\end{f}
\hrule

\begin{f}[Securitization / CatBonds]
\label{CatBond}
Why? The financial capacity of all insurers and reinsurers combined does not cover the damage of a major earthquake in the United States. (\(\geq\) 200 B\EUR{}). 
This amount corresponds to less than 1\% of the capitalization of American financial markets.

Securitization transforms an insurance risk into a negotiable security, often into bonds called Cat-Bonds.
It consists of an exchange of principal for periodic payment of coupons, in which the payment of coupons and/or the repayment of principal are conditional on the occurrence of a triggering event defined a priori.
The rates on these bonds are increased based on the risk, not of default or counterparty, but of the occurrence of the event (less than 1\%). The structure dedicated to this transformation is called a Special Purpose Vehicle (SPV).


The trigger can be directly linked to the results of the predecessor (Compensation), depend on a loss index, a measurable parameter (sum of excess rainfall, Richter scale, mortality rate), or a model (RMS \& Equecat Storm modeling).



	\renewcommand{\arraystretch}{1.25}
\begin{center}\small
\begin{tabular}{|m{20mm}|*{4}{>{\centering\arraybackslash}m{14mm}|}}
\hline \rowcolor{BleuProfondIRA!40}
\textbf{Criteria} & \textbf{Compensative} & \textbf{Hint} & \textbf{Parametric} & \textbf{Model} \\
\hline
Transparency & \(\ominus\) & \(\oplus\) & \(\oplus\) & \(\oplus\) \\
\hline
Basis risk & \(\oplus\) & \(\ominus\) & \(\ominus\) & \(\oplus\) \\
\hline
Moral hazard & \(\ominus\) & \(\oplus\) & \(\oplus\) & \(\oplus\) \\
\hline
Universality of perils & \(\oplus\) & \(\oplus\) & \(\ominus\) & \(\oplus\) \\
\hline
Trigger delay & \(\ominus\) & \(\ominus\) & \(\oplus\) & \(\oplus\) \\
\hline
\end{tabular}
\end{center}	\renewcommand{\arraystretch}{1}
\end{f}







\end{multicols}

\newpage
\begin{center}
	\section*{Extreme statistics and reinsurance pricing}
	\medskip
\end{center}

\begin{multicols}{2}
	% !TeX root = ActuarialFormSheet_MBFA-en.tex
% !TeX spellcheck = fr_FR

%\subsection*{Lois utilisées dans la théorie des extrêmes}


\begin{f}[Loi de Pareto]
Soit la variable aléatoire $X$ qui suit une loi de Pareto de paramètres $(x_{\mathrm{m}},k)$, $k$ est l'indice de Pareto:
$$
\mathbb{P}(X>x)=\left(\frac{x}{x_{\mathrm{m}}}\right)^{-k}\ \mbox{avec}\ x \geq x_{\mathrm{m}} 
$$


$$
f_{k,x_\mathrm{m}}(x) = k\,\frac{x_\mathrm{m}^k}{x^{k+1}}\ \mbox{pour}\ x \ge x_\mathrm{m}
$$

\textbf{	Loi de Pareto généralisée} (GPD) a 3 paramètres $\mu$, $\sigma$ et $\xi$.


$$
F_{\xi,\mu,\sigma}(x) = \begin{cases} 1 - \left(1+ \frac{\xi(x-\mu)}{\sigma}\right)^{-1/\xi} & \text{for }\xi \neq 0, \\ 1 - \exp \left(-\frac{x-\mu}{\sigma}\right) & \text{for }\xi = 0. \end{cases} 
$$
pour  $x \geq \mu$  quand $\xi \geq 0$  et $ \mu \leq x \leq \mu - \sigma /\xi$  quand $ \xi < 0$ et où $\mu\in\mathbb{R}$ est la localisation, $\sigma>0$ l'échelle et  $\xi\in\mathbb{R}$ la forme. 
Notez que certaines références donnent le \og{}paramètre de forme\fg{}, comme $\kappa = - \xi$.

$$
f_{\xi,\mu,\sigma}(x) = \frac{1}{\sigma}\left(1 + \frac{\xi (x-\mu)}{\sigma}\right)^{\left(-\frac{1}{\xi} - 1\right)}
=
 \frac{\sigma^{\frac{1}{\xi}}}{\left(\sigma + \xi (x-\mu)\right)^{\frac{1}{\xi}+1}}
$$
%à nouveau, pour  $x \geq \mu$, et $x \leq \mu - \sigma /\xi$ où $\xi < 0$. 

\medskip

\begin{center}
		{\shorthandoff{:!;}	
\tikzset{
	declare function={
		dParetoGen(\x,\mu,\sigma,\xi) = 
		(\x>\mu)*(1/\sigma*(1 + \xi* (\x-\mu)/\sigma)^(-1/\xi - 1));}
}  
\begin{tikzpicture}[mark=none,samples=100,smooth,domain=0:15]
	\begin{axis}[ ytick=\empty,xtick pos=left,
		axis y line=middle,
		axis x line=bottom,
		height=.5*\linewidth,width=0.95*\linewidth, ticklabel style ={font=\footnotesize}, legend pos=north east,
		legend style={font=\tiny}]
		\addplot[BleuProfondIRA] 	({\x},{dParetoGen({\x},1,1,1)}) ;
		\addplot[FushiaIRA] 	({\x},{dParetoGen({\x},1,1,5)}) ;
		\addplot[VertIRA]  	({\x},{dParetoGen({\x},1,1,20)}) ;
		\addplot[OrangeProfondIRA] 	({\x},{dParetoGen({\x},3,2,5)}) ;
		%
		\legend{$\mu$ =  1 $\sigma=1$ $\xi = 1$,$\mu$ =  1  $\sigma=2$ $\xi =5$,$\mu$ =  1  $\sigma=3$ $\xi =20$,$\mu$ = 3  $\sigma=2$ $\xi = 5$};
	\end{axis} 
\end{tikzpicture}
}
\end{center}


%\begin{center}
%	{\shorthandoff{:!;}	
%	\tikzset{
%	declare function={
%		pParetoGen(\x,\mu,\sigma,\xi) =  (\x > \mu)*( 1-(1+ (\xi*(\x-\mu))/\sigma)^(-1/\xi));}
%}  
%\begin{tikzpicture}[mark=none,samples=100,smooth,domain=0:15]
%	\begin{axis}[ ytick=\empty,xtick pos=left,
%		axis y line=middle,
%		axis x line=bottom,
%		height=.5*\linewidth,width=0.95*\linewidth, ticklabel style ={font=\footnotesize}, legend pos=north east,
%		legend style={font=\tiny}]
%		\addplot[BleuProfondIRA] 
%		({\x},{pParetoGen({\x},1,1,1)}) ;
%		\addplot[FushiaIRA] 
%		({\x},{pParetoGen({\x},1,1,5)}) ;
%		\addplot[VertIRA] 
%		({\x},{pParetoGen({\x},1,1,20)}) ;
%		\addplot[OrangeProfondIRA] 
%		({\x},{pParetoGen({\x},3,2,5)}) ;
%		%
%		\legend{$\mu$ =  1 $\sigma=1$ $\xi = 1$,$\mu$ =  1  $\sigma=2$ $\xi =5$,$\mu$ =  1  $\sigma=3$ $\xi =20$,$\mu$ = 3  $\sigma=2$ $\xi = 5$};
%	\end{axis}	 
%\end{tikzpicture}
%}
%\end{center}

\end{f}
\hrule

\begin{f}[Loi des valeurs extrême généralisée]
La fonction de répartition de la loi des extrêmes généralisée est
$$
F_{\mu,\sigma,\xi}(x) = \exp\left\{-\left[1+\xi\left(\frac{x-\mu}{\sigma}\right)\right]^{-1/\xi}\right\}
$$
pour $1+\xi(x-\mu)/\sigma>0$, où $\mu\in\mathbb{R}$ est la localisation, $\sigma>0$ d'échelle et  $\xi\in\mathbb{R}$ la forme. Pour $\xi = 0$ l'expression est définie par sa limite en 0.

\begin{align*}
	f_{\mu,\sigma,\xi}(x) =& \frac{1}{\sigma}\left[1+\xi\left(\frac{x-\mu}{\sigma}\right)\right]^{(-1/\xi)-1}\\ &\times\exp\left\{-\left[1+\xi\left(\frac{x-\mu}{\sigma}\right)\right]^{-1/\xi}\right\}
\end{align*}

$$
f(x;\mu ,\sigma ,0)={\frac {1}{\sigma }}\exp \left(-{\frac {x-\mu }{\sigma }}\right)\exp \left[-\exp \left(-{\frac {x-\mu }{\sigma }}\right)\right]
$$

\end{f}
\hrule



\begin{f}[Loi de Gumbel]

La fonction de répartition de la \textbf{loi de Gumbel} est :
$$
F_{\mu,\sigma}(x) = e^{-e^{(\mu-x)/\sigma}}.\,
$$
Pour $\mu = 0 $ et $\sigma = 1$, on obtient la loi standard de Gumbel.
La loi de Gumbel est un cas particulier de la GEV (avec $\xi=0$).

Sa densité :
$$
f_{\mu,\sigma}(x)=\frac{1}{\sigma}e^{\left(\frac{x-\mu}{\sigma}-e^{-(x-\mu)/\sigma}\right)}
$$


\begin{center}
\begin{tikzpicture}[mark=none,samples=100,smooth,domain=-5:15]
\begin{axis}[ ytick=\empty,xtick pos=left,
	axis y line=middle,
	axis x line=bottom,
	height=.5*\linewidth,width=0.95*\linewidth, ticklabel style ={font=\footnotesize}, legend pos=north east,
	legend style={font=\tiny}]
	\addplot[BleuProfondIRA] 
	({\x},{dGumbel({\x},0.5,1)}) ;
	\addplot[FushiaIRA] 
	plot ({\x},{dGumbel({\x},1,2)}) ;
	\addplot[VertIRA] 
	plot ({\x},{dGumbel({\x},1.5,3)}) ;
	\addplot[OrangeProfondIRA] 
	plot ({\x},{dGumbel({\x},3,4)}) ;
	%
	\legend{$\mu$ =  0.5 $\gamma = 1$,$\mu$ =  1 $\gamma =2$,$\mu$ =  1.5 $\gamma =3$,$\mu$ =  3 $\gamma = 4$};
\end{axis}	 
\end{tikzpicture}
\end{center}


%\begin{center}
%\begin{tikzpicture}[mark=none,samples=100,smooth,domain=-5:15]
%\begin{axis}[ ytick=\empty,xtick pos=left,
%	axis y line=middle,
%	axis x line=bottom,
%	height=.5*\linewidth,width=0.95*\linewidth, ticklabel style ={font=\footnotesize}, legend pos=north west,
%	legend style={font=\tiny}]
%	\addplot[BleuProfondIRA] 
%	({\x},{pGumbel({\x},0.5,1)}) ;
%	\addplot[FushiaIRA] 
%	plot ({\x},{pGumbel({\x},1,2)}) ;
%	\addplot[VertIRA] 
%	plot ({\x},{pGumbel({\x},1.5,3)}) ;
%	\addplot[OrangeProfondIRA] 
%	plot ({\x},{pGumbel({\x},3,4)}) ;
%	%
%	\legend{$\mu$ =  0.5 $\gamma = 1$,$\mu$ =  1 $\gamma =2$,$\mu$ =  1.5 $\gamma =3$,$\mu$ =  3 $\gamma = 4$};
%\end{axis}	 
%\end{tikzpicture}
%\end{center}

%\subsubsubsection{Loi de Weibull}\label{label}
\end{f}
\hrule

\begin{f}[Loi de Weibull]


La \textbf{loi de Weibull} a pour fonction de répartition est définie par :
$$
F_{\alpha,\mu, \sigma}(x) = 1- e^{-((x-\mu)/\sigma)^\alpha}\,
$$
où $x >\mu$.
Sa densité de probabilité est :
$$
f_{\alpha,\mu,\sigma}(x) = (\alpha/\sigma) ((x-\mu)/\sigma)^{(\alpha-1)} e^{-((x-\mu)/\sigma)^\alpha}\,
$$
où  $\mu\in\mathbb{R}$ est la localisation, $\sigma>0$ d'échelle et  $\alpha=-1/\xi>0 $ la forme.

La distribution de Weibull est souvent utilisée dans le domaine de l'analyse de la durée de vie. C'est un cas particulier de la GEV lorsque $\xi<0$.

Si le taux de pannes diminue au cours du temps alors, $\alpha<1$. Si le taux de panne est constant dans le temps alors, $\alpha=1$. Si le taux de panne augmente avec le temps alors, $\alpha>1$. La compréhension du taux de pannes peut fournir une indication au sujet de la cause des pannes.


\begin{center}
	{\shorthandoff{:!;}	\tikzset{
	declare function={
		dWeibull(\x,\sigma,\alpha) = (\alpha/\sigma)*(\x/\sigma)^(\alpha-1)*exp(-(\x/\sigma)^\alpha);}
}  
\begin{tikzpicture}[mark=none,samples=100,smooth,domain=0:5]
	\begin{axis}[ ytick=\empty,xtick pos=left,
		axis y line=middle,
		axis x line=bottom,
		height=.5*\linewidth,width=0.95*\linewidth, ticklabel style ={font=\footnotesize}, legend pos=north east,
		legend style={font=\tiny}]
		\addplot[BleuProfondIRA] 
		({\x},{dWeibull({\x},0.5,1)}) ;
		\addplot[FushiaIRA] 
		plot ({\x},{dWeibull({\x},1,2)}) ;
		\addplot[VertIRA] 
		plot ({\x},{dWeibull({\x},1.5,3)}) ;
		\addplot[OrangeProfondIRA] 
		plot ({\x},{dWeibull({\x},3,4)}) ;
		%
		\legend{$\mu$ =  0.5 $\gamma = 1$ ($\alpha=-1/\xi>0$),$\mu$ =  1 $\gamma =2$,$\mu$ =  1.5 $\gamma =3$,$\mu$ =  3 $\gamma = 4$};
	\end{axis}	 
\end{tikzpicture}
}
%	{\shorthandoff{:!;}	
%		\tikzset{
%	declare function={
%		pWeibull(\x,\sigma,\alpha) = 1- exp(-(\x/\sigma)^\alpha);}
%}  
%\begin{tikzpicture}[mark=none,samples=100,smooth,domain=0:5]
%	\begin{axis}[ ytick=\empty,xtick pos=left,
%		axis y line=middle,
%		axis x line=bottom,
%		height=.5*\linewidth,width=0.95*\linewidth, ticklabel style ={font=\footnotesize}, legend pos= south east,
%		legend style={font=\tiny}]
%		\addplot[BleuProfondIRA] 
%		({\x},{pWeibull({\x},0.5,1)}) ;
%		\addplot[FushiaIRA] 
%		plot ({\x},{pWeibull({\x},1,2)}) ;
%		\addplot[VertIRA] 
%		plot ({\x},{pWeibull({\x},1.5,3)}) ;
%		\addplot[OrangeProfondIRA] 
%		plot ({\x},{pWeibull({\x},3,4)}) ;
%		%
%		\legend{$\mu$ =  0.5 $\gamma = 1$ ($\alpha=-1/\xi>0$),$\mu$ =  1 $\gamma =2$,$\mu$ =  1.5 $\gamma =3$,$\mu$ =  3 $\gamma = 4$};
%	\end{axis}	 
%\end{tikzpicture}
%}
\end{center}
\end{f}


\begin{f}[Loi de Fréchet]
Sa fonction de répartition de la \textbf{loi de Frechet}\index{D\'efinition! loi de Frechet} est donnée par 
$$
F_{\alpha,\mu,\sigma}(x)=\mathbb{P}(X \le x)=\begin{cases} e^{-\left(\frac{x-\mu}{\sigma}\right)^{-\alpha}} & \text{ si } x>\mu \\ 0 &\text{ sinon.}\end{cases} 
$$
où  $\mu\in\mathbb{R}$ est la localisation, $\sigma>0$ l'échelle et  $\alpha=1/\xi>0$ la forme. 
C'est un cas particulier de la GEV lorsque $\xi>0$.

$$
f_{\alpha,\mu,\sigma}(x)=\frac{\alpha}{\sigma} \left(\frac{x-\mu}{\sigma}\right)^{-1-\alpha}  e^{-(\frac{x-\mu}{\sigma})^{-\alpha}}
$$

\begin{center}
%	\includegraphics[width=0.90\textwidth]{../Graph/LoiFrechet.pdf}
	{\shorthandoff{:!;}	
		\tikzset{
declare function={
	dFrechet(\x,\m,\sigma,\alpha) = (\alpha/\sigma) * (((\x-\m)/\sigma)^(-1-\alpha)) * exp(- ((\x-\m)/\sigma)^(-\alpha) );}
}  
\begin{tikzpicture}[mark=none,samples=100,smooth,domain=0:15]
\begin{axis}[ ytick=\empty,xtick pos=left,
	axis y line=middle,
	axis x line=bottom,
	height=.5*\linewidth,width=0.95*\linewidth, ticklabel style ={font=\footnotesize}, legend pos= north east,
	legend style={font=\tiny}]
	\addplot[BleuProfondIRA] 
	({\x},{dFrechet({\x},0,1,1)}) ;
	\addplot[FushiaIRA] 
	({\x},{dFrechet({\x},0,2,1)}) ;
	\addplot[VertIRA] 
	({\x},{dFrechet({\x},0,3,1)}) ;
	\addplot[OrangeProfondIRA] 
	({\x},{dFrechet({\x},0,2,2)}) ;
	%
	\legend{$\mu$ =  0 $\alpha=1$ $\sigma = 1$,$\mu$ =  0  $\alpha=2$ $\sigma =1$,$\mu$ =  0  $\alpha=3$ $\sigma =1$,$\mu$ = 0  $\alpha=2$ $\sigma = 2$};
\end{axis}	 
\end{tikzpicture}
}\end{center}


%\begin{center}
%%	\includegraphics[width=0.90\textwidth]{../Graph/FLoiFrechet.pdf}
%	{\shorthandoff{:!;}	
%\tikzset{
%	declare function={
%		pFrechet(\x,\m,\sigma,\alpha) = exp( -((\x-\m)/\sigma)^(-\alpha));}
%}  
%\begin{tikzpicture}[mark=none,samples=100,smooth,domain=0:15]
%	\begin{axis}[ ytick=\empty,xtick pos=left,
%		axis y line=middle,
%		axis x line=bottom,
%		height=.5*\linewidth,width=0.95*\linewidth, ticklabel style ={font=\footnotesize}, legend pos= south east,
%		legend style={font=\tiny}]
%		\addplot[BleuProfondIRA] 
%		({\x},{pFrechet({\x},0,1,1)}) ;
%		\addplot[FushiaIRA] 
%		({\x},{pFrechet({\x},0,2,1)}) ;
%		\addplot[VertIRA] 
%		({\x},{pFrechet({\x},0,3,1)}) ;
%		\addplot[OrangeProfondIRA] 
%		({\x},{pFrechet({\x},0,2,2)}) ;
%		%
%		\legend{$\mu$ =  0 $\alpha=1$ $\sigma = 1$,$\mu$ =  0  $\alpha=2$ $\sigma =1$,$\mu$ =  0  $\alpha=3$ $\sigma =1$,$\mu$ = 0  $\alpha=2$ $\sigma = 2$};
%	\end{axis}	 
%\end{tikzpicture}
%}
%\end{center}
\end{f}

\begin{f}[Lien entre GEV, Gumbel, Fréchet et Weibull]

Le paramètre de forme $\xi$ gouverne le comportement de la queue de distribution. 
Les sous-familles définies par $\xi= 0$, $\xi>0$ et $\xi<0$ correspondent respectivement aux familles de Gumbel, Fréchet et Weibull~:
\begin{itemize}
\item 	 Gumbel ou loi des valeurs extrêmes de type I
\item     Fréchet ou loi des valeurs extrêmes de type II, si $\xi  = \alpha^{-1}$ avec $\alpha>0$,
\item    Reversed Weibull ($\overline{F}$) ou loi des valeurs extrêmes de type III, si $\xi =-\alpha^{-1} $, avec $\alpha>0$.
\end{itemize}
\end{f}
\hrule 

\begin{f}[Théorème général des valeurs extrêmes] 

	Soit $X_1, \dots, X_n$  $iid$, $X$ de fonction de répartition $F_X$ et soit $M_n =\max(X_1,\dots,X_n)$.
	
	La théorie donne la distribution exacte du maximum :
\begin{align*}
		\mathcal{P}(M_n \leq z) = &\Pr(X_1 \leq z, \dots, X_n \leq z) \\
		 = &\mathcal{P}(X_1 \leq z) \cdots \mathcal{P}(X_n \leq z) = (F_X(z))^n. 
\end{align*}
	S'il existe une séquence de paire de nombres réels $(a_n, b_n)$ de telle sorte que $a_n>0$ et $\lim_{n \to \infty}\mathcal{P}\left(\frac{M_n-b_n}{a_n}\leq x\right) = F_X(x)$, où $F_X$ est une fonction de répartition non dégénérée, alors la limite de la fonction $F_X$ appartient à la famille des lois $GEV$. 

\end{f}

\begin{f}[Densité sous-exponentielle]
	
	\textbf{Cas des puissances} 
	
	Si $\overline{F}_{X}(x)=\mathbb{P}(X > x)\sim c\ x^{-\alpha}$
		quand $x \to \infty $ pour un $\alpha > 0 $ et une constante $c > 0 $ alors la loi de $X$
		est sous-exponentielle.
	
		Si $F_X$ est une fonction de répartition continue d'espérance $\mathbb{E}[X]$ finie, on appelle l'indice des grands risques par
		$$
		D_{F_X}(p)=\frac{1}{\mathbb{E}[X]}\int_{1-p}^{1} F_X^{-1}(t)dt,\, \, p\in [0,1]
		$$

	Cette distribution en excès décroit moins vite que n'importe quelle distribution exponentielle.
	Il est possible de considérer cette statistique ~:
	$$
	T_n(p)=\frac{X_{(1:n)}+X_{(2:n)}+\ldots + X_{(np:n)}}{\sum_{1\leq i\leq n}(X_i)} \mbox{ où } \frac{1}{n}\leq p\leq 1
	$$
	$X_{(i:n)}$ désigne le $i^e$ $\max $ des $X_i$.

\end{f}
\begin{f}[Théorème de Pickands–Balkema–de Haan (loi des excès)]
	Soit $X$ de distribution $F_X$, et soit $u$ un seuil élevé. Alors, pour une large classe de lois $F_X$, la loi conditionnelle des excès
	\[
	X_u := X - u \mid X > u
	\]
	est approximable, pour $u$ suffisamment grand, par une loi de Pareto généralisée (GPD) :
	
	\[
	\mathbb{P}(X - u \le y \mid X > u) \approx G_{\xi, \sigma, \mu=0}(y) :=  1 - \left(1+ \frac{\xi(x-\mu)}{\sigma}\right)^{-1/\xi} 
	\]
	
	$\quad y \ge 0$. Autrement dit, pour $u \to x_F := \sup\{x : F(x) < 1\}$,
	\[
	\sup_{0 \le y < x_F - u} \left| \mathbb{P}(X - u \le y \mid X > u) - G_{\xi,\sigma,\mu=0}(y) \right| \to 0.
	\]
	
	Ce théorème justifie l’utilisation de la \textit{loi de Pareto (généralisée)} pour modéliser les excès au-delà d’un seuil, ce qui est précisément le cadre des traités de réassurance en \textit{excess of loss} par risque, par événement ou de cumul annuel. 
\end{f}
\hrule


\begin{f}
	[Les données en réassurance ]

Comme la réassurance indemnise des agrégations de sinistre ou des sinistres extrêmes, elle utilise souvent des historiques qui devront être utilisé avec prudence~:
\begin{itemize}
	\item l'actualisation des données (impact de l'inflation monétaire).
	\item la revalorisation prend en compte l'évolution du risque :
	\begin{itemize}
		\item l'évolution des taux de prime, garanties et modalités des contrats,
		\item l'évolution des coûts des sinistres (indice des coûts de la construction, indices des coûts de réparation automobile,\ldots)
		\item l'évolution de l'environnement juridique.
	\end{itemize}
	\item le redressement de la statistique pour prendre en compte l'évolution de la base  portefeuille :
	\begin{itemize}
		\item profil des polices (nombre, capitaux,\ldots),
		\item natures des garanties (évolution des franchises, des exclusions\ldots)
	\end{itemize}
\end{itemize}
Après ces corrections, les données sont dites \og as if \fg (en économie, on utilise l'expression contre-factuel).

\end{f}


\begin{f}[La prime \emph{Burning Cost}]
	
	$X_{i}^{j}$ désigne le $i^{e}$ sinistre  de l'année $j$ \og{}as if\fg{} actualisé, revalorisé et  redressé,
$n^j$ le nombre de sinistres l'année $j$, $c^{j}$ la charge de l'assureur $c$.
Le taux pur  par la méthode de \textbf{Burning Cost} est donné par la formule~:
$$
BC_{pur}=\frac{1}{s}\sum_{j=1}^{n}\frac{c^{j}}{a_{j}}
$$ 
Le Burning Cost n'est qu'une moyenne des ratios $S/P$ croisés~: les sinistres à la charge du réassureur sur les primes reçues par la cédante.
La prime Burning Cost est alors : $P_{pure}=BC_{pur}\times a_{s+1}$.

Dans le cas d'un $p$ XS $f$), $$
c^{j}=\sum_{i=1}^{n^j}\max\left(  \left( X_{i}^{j}-f\right),p\right) 1\!\!1_{x^{j}\geq f}
$$
Si l'assurance vie calcule des taux de prime en référence au capital, l'assurance non vie utilise comme  référence à la valeur assurée, la réassurance prend elle comme référence le total des primes de la cédante, appelée \textbf{assiette}.
On note $a_{j}$ désigne l'assiette de prime à l'année $j$ et $a_{s+1}^{*}$ désigne l'assiette estimée de l'année à venir et où $s$ désigne le nombre d'années d'historique.
\end{f}



\begin{f}[Le modèle Poisson-Pareto]
[Prime de l'XS ou de l'XL]
Soit $p$ et $f$ respectivement la portée et la priorité (franchise) de l'XS, avec la limite $l=p+f$ ($p$ XS $f$).

La prime XS correspond à :
$$
\mathbb{E}\left[S_N\right]=\mathbb{E}\left[\sum_{i=1}^{N} Y_{i}\right]=\mathbb{E}[N]\times \mathbb{E}[Y]
$$
où
$$
\mathbb{E}[Y]= l \mathbb{P}[X>l] - f \times \mathbb{P}[X\geq f] + \mathbb{E}[X\mid f\geq x\geq l]
$$


Si $l=\infty$ et $\alpha \neq 1$ :
$$
\mathbb{E}[S_N] = \lambda \frac{x_\mathrm{m}^{\alpha}}{\alpha -1}f^{1-\alpha} 
$$

si  $l=\infty$ et $\alpha = 1$ il n'y a pas de solution.

Si $l<\infty$ et $\alpha \neq 1$ :
$$
\mathbb{E}[S_N] = \lambda \frac{x_\mathrm{m}^{\alpha}}{\alpha -1}\left( f^{1-\alpha} -l^{1-\alpha} \right)
$$

Si $l<\infty$ et $\alpha = 1$ :
$$
\mathbb{E}[S_N] = \lambda x_\mathrm{m} \ln \left(  \frac{1}{f}\right)
$$
\end{f}



\begin{f}
[Le modèle Poisson-LogNormal]

Si $X$ suit une $\mathcal{L}\mathcal{N}orm(x_\mathrm{m}, \mu, \sigma)$ alors $X-x_\mathrm{m}$ suit une $\mathcal{L}\mathcal{N}orm(\mu, \sigma)$
Il vient :
$$
\mathbb{P}[X>f]=\mathbb{P}[X-x_\mathrm{m}>f-x_\mathrm{m}]=1-\Phi\left(\frac{\ln(f-x_\mathrm{m})-\mu}{\sigma}\right)
$$
\begin{align*}
	\mathbb{E}[X\mid X>f] \\ 
	= & \mathbb{E}\left[ X-x_\mathrm{m}\mid X-x_\mathrm{m}>f-x_\mathrm{m} \right]+x_\mathrm{m} \mathbb{P}[X>f]\\
	= & e ^{m+\sigma^2/2} \left[1-\Phi\left(\frac{\ln(f-x_\mathrm{m})-(\mu+\sigma^2)}{\sigma}\right) \right]\\
	& +x_\mathrm{m} \left( 1-\Phi\left(\frac{\ln(f-x_\mathrm{m})-\mu}{\sigma}\right) \right)
\end{align*}
Avec franchise et sans limite :
\begin{align*}
	\mathbb{E}[S_N] \\ 
	= & \lambda \left(\mathbb{E}\left[ X-x_\mathrm{m}\mid X-x_\mathrm{m}>f-x_\mathrm{m} \right]+x_\mathrm{m} \mathbb{P}[X>f]-f\mathbb{P}[X>f]\right)\\
	= & \lambda \left(  e ^{m+\sigma^2/2} \left[1-\Phi\left(\frac{\ln(f-x_\mathrm{m})-(\mu+\sigma^2)}{\sigma}\right) \right]\right)\\
	& +\lambda (x_\mathrm{m}-l) \left( 1-\Phi\left(\frac{\ln(f-x_\mathrm{m})-\mu}{\sigma}\right) \right)
\end{align*}

\end{f}
\end{multicols}

\newpage

\begin{center}
\section*{High School/Prep Mathematics Reminders}
\medskip
\end{center}

\begin{multicols}{2}	
	% !TeX root = ActuarialFormSheet_MBFA-en.tex
% !TeX spellcheck = en_GB

\begin{f}[Pythagoras]
In a right triangle, the square of the hypotenuse is equal to the sum of the squares of the other two sides. If $ABC$ is right-angled at $C$, then

%
\begin{tikzpicture}[scale=.85]
	% Triangle
	\coordinate [label=above left:$C$] (C) at (0,0);
	\coordinate [label=above right:$B$] (B) at (4,0);
	\coordinate [label=below left:$A$] (A) at (0,3);
	
	%\draw[thick] (A) -- (B) -- (C) -- cycle;
	
	% Carré sur AC
	\draw[BleuProfondIRA!60] (A) -- (C) ;
	\node at (-0.6,1.5) {\color{BleuProfondIRA}$AC^2$};
	
	% Carré sur BC
	\draw[VertIRA] (B) -- (C);
	\node at (2,0.6) {\color{VertIRA}$BC^2$};
	
	% Carré sur AB
	\draw[OrangeProfondIRA] (A) -- (B) ;
	\node at (2.5,1.75) {\color{OrangeProfondIRA}$AB^2$};
	
	% Angle droit
	\draw (C) rectangle +(0.3,0.3);
	\node[right] at (3.75,1.75) {$
		\color{OrangeProfondIRA}AB^2 = \color{VertIRA}AC^2 +\color{BleuProfondIRA} BC^2
		$};
\end{tikzpicture}
\end{f}

\begin{f}[Thales]
Let two lines \textbf{intersect at a point} \( A \), and let two lines \( (BC) \) and \( (DE) \) \textbf{parallel}, intersecting the two lines at \( B, D \) and \( C, E \), then :
\[
\frac{AB}{AD} = \frac{AC}{AE} = \frac{BC}{DE}
\]


\begin{center}
	\begin{tikzpicture}[scale=.8]
		% Points
		\coordinate (A) at (0,0);
		\coordinate (B) at (2,0.5);
		\coordinate (C) at (6,1.5);
		\coordinate (D) at (1.5,1);
		\coordinate (E) at (4.5,3);
		
		% Segments
		\draw[thick] (A) -- (C);
		\draw[thick] (A) -- (E);
%		\draw[thick] (B) -- (C);
%		\draw[thick] (D) -- (E);
		\draw[dashed] (B) -- (D);
		\draw[dashed] (E) -- (C);
		
		% Points
		\filldraw[black] (A) circle (2pt) node[below left] {\(A\)};
		\filldraw[black] (B) circle (2pt) node[below right] {\(B\)};
		\filldraw[black] (C) circle (2pt) node[right] {\(C\)};
		\filldraw[black] (D) circle (2pt) node[above left] {\(D\)};
		\filldraw[black] (E) circle (2pt) node[right] {\(E\)};
		
%		% Parallèles
%		\draw[<->,blue,thick] (0.8,2.2) -- (3.2,2.2) node[right] {\small $\ell$};
%		\draw[<->,blue,thick] (0.8,0.8) -- (3.2,0.8) node[right] {\small $\ell'$};
%		\node at (3.6,1.5) {\small $\ell \parallel \ell'$};
	\end{tikzpicture}
\end{center}
\end{f}
\hrule

\begin{f} [Quadratic equation] 
     \[
    ax^2 + bx + c = 0
    \]

    The discriminant is defined by :
    
    \[
    \Delta = b^2 - 4ac
    \]
       
    \begin{itemize}
        \item If \(\Delta > 0\), the equation has two distinct solutions :
        \[
        x_1 = \frac{-b + \sqrt{\Delta}}{2a}, \quad x_2 = \frac{-b - \sqrt{\Delta}}{2a}
        \]
        \item If \(\Delta = 0\), the equation has a double solution :
        \[
        x = \frac{-b}{2a}
        \]
        \item If \(\Delta < 0\), the equation has a solution in the imaginary
        \[
        x_1 = \frac{-b + i\sqrt{\Delta}}{2a}, \quad x_2 = \frac{-b - i\sqrt{\Delta}}{2a}
        \]
    \end{itemize}
\end{f}
\hrule
\begin{f} [Factorial, Counting and Gamma Functions] 

The \textbf{factorial} function (of $\mathbb{N}$ in $\mathbb{N}$ ) is defined by $0!=1$ and $n!=n \times(n-1) \times \cdots \times 2 \times 1=$ permutations of $n$ elements $\displaystyle C_n^k=\binom{k}{n}=\frac{n!}{k!(n-k)!}=$ choice of $k$ elements among $n$ the $C_n^k$ are also calculated by Pascal's triangle and verify:

$$
C_n^k=C_n^{n-k}, C_n^k+C_n^{k+1}=C_{n+1}^{k+1} .
$$


Let $E$ be a set of cardinal $\operatorname{Card}(E)$ and parts $\mathcal{P}(E)$ :

$$
\begin{aligned}
\operatorname{Card}(\mathcal{P}(E)) & =2^{\operatorname{Card}(E)} \\
\operatorname{Card}(A \times B) & =\operatorname{Card}(A) \times \operatorname{Card}(B) \\
\operatorname{Card}(A \cup B) & =\operatorname{Card}(A)+\operatorname{Card}(B)-\operatorname{Card}(A \cap B)
\end{aligned}
$$
\[ \Gamma(n) = \int_0^\infty t^{n-1} e^{-t} \, dt \]
The function $\Gamma$ can be seen as the extension of the factorial : $\Gamma(n+1)=n!$.
\end{f}

\hrule
\begin{f}
	[Binomial expansion] 
	
	For a positive integer $n$,
	$$
	(x+y)^n=\sum_{i=0}^n\binom{n}{i} x^i y^{n-i}
	$$
	
\end{f}
\hrule
\begin{f}[Sequences]  
	
	Arithmetic sequences of reason $r$
	
	$$
	\left\{\begin{array} { r l } 
		{ u _ { n + 1 } } & { = u _ { n } + r } \\
		{ u _ { 0 } } & { \in \mathbb { R } }
	\end{array} \Rightarrow \left\{\begin{array}{rl}
		u_n & =n r+u_0 \\
		\sum_{k=0}^n u_k & =\frac{(n+1)\left(2 u_0+n r\right)}{2}
	\end{array}\right.\right.
	$$
	
	geometric sequences of reason $q\left\{\begin{array}{rll}u_{n+1} & = & q \times u_n \\ u_0 & \in & \mathbb{R}\end{array}\right.$
	
	$$
	\Rightarrow\left\{\begin{array}{rlr}
		u_n & =u_0 \times q^n \\
		\sum_{k=0}^n u_k & =\left\{\begin{array}{rl}
			(n+1) u_0 & \text { if } \\
			u_0 \frac{1-q^{n+1}}{1-q} & \text { otherwise }
		\end{array} \quad q=1\right.
	\end{array}\right.
	$$
	
\end{f}
\hrule
\begin{f}[Exponential and Logarithm]
	The exponential function \( e^x \) can be defined by the following power series expansion :
	
	\[
	e^x = \sum_{n=0}^{\infty} \frac{x^n}{n!} = 1 + x + \frac{x^2}{2!} + \frac{x^3}{3!} + \cdots
	\]
	
	This series converges for all \( x \in \mathbb{R} \) and allows us to define the exponential as an infinite sum.
	
	
	The natural logarithm function \( \ln(x) \) is defined as the antiderivative of the function \( \frac{1}{x} \). In other words :
	
	\[
	\frac{d}{dx} \ln(x) = \frac{1}{x}
	\]
	
	with the condition \( \ln(1) = 0 \). This definition allows us to establish the link between the exponential and the logarithm via inversion : \( e^{\ln(x)} = x \) for \( x > 0 \).
	
\end{f}



\hrule
\begin{f}[Congruence relationship]
	Let $m > 0$. We say that two real numbers $a$ and $b$ are congruent modulo $m$ if there exists a relative integer $k \in \mathbb{Z}$ such that :
	\[
	a = b + km.
	\]
	On note $a \equiv b \pmod{m}$.
	
	In trigonometry, we often choose $m = 2\pi$ or $m = \pi$.
\end{f}

\begin{f}
	[Trigonometric circle — sine, cosine, tangent]
	
		\begin{animateinline}[poster=first, autoplay,loop]{1}
			\multiframe{24}{i=0+1}{%
				\begin{tikzpicture}[scale=3,yrange=-4:4]
\node[text width=9cm, align=left] at (0,3.75) {\ $\displaystyle M = (\cos \theta, \sin \theta)$};
\node[text width=9cm, align=left] at (0,3.5) {\ If $\theta \not\equiv \frac{\pi}{2} \pmod{\pi}$, we define :
	$\displaystyle
	\tan \theta = \frac{\sin \theta}{\cos \theta}
	$};
\node[text width=9cm, align=left] at (0,3.25) {\ $\displaystyle \cos^2 \theta + \sin^2 \theta = 1 $};
\node[text width=9cm, align=left] at (0,2.5) {\renewcommand{\arraystretch}{1.5}\begin{tabular}{|c|c|c|c|c|c|}
		\hline\rowcolor{BleuProfondIRA!40}
		$\theta$ & $0$ & $\frac{\pi}{6}$ & $\frac{\pi}{4}$ & $\frac{\pi}{3}$ & $\frac{\pi}{2}$ \\
		\hline
		$\cos \theta$ & $1$ & $\frac{\sqrt{3}}{2}$ & $\frac{\sqrt{2}}{2}$ & $\frac{1}{2}$ & $0$ \\
		\hline
		$\sin \theta$ & $0$ & $\frac{1}{2}$ & $\frac{\sqrt{2}}{2}$ & $\frac{\sqrt{3}}{2}$ & $1$ \\
		\hline
		$\tan \theta$ & $0$ & $\frac{\sqrt{3}}{3}$ & $1$ & $\sqrt{3}$ & indéfini \\
		\hline
\end{tabular}};
\node[text width=9cm, align=left] at (0,-2.75) {\small\renewcommand{\arraystretch}{1.5}	
\begin{tabular}{|c|c|c|}
	\hline\rowcolor{BleuProfondIRA!40}
	$-\theta$ & $\theta + \pi$ & $\pi - \theta$ \\
	\hline
	$\cos \theta$ & $-\cos \theta$ & $-\cos \theta$ \\
	\hline
	$-\sin \theta$ & $-\sin \theta$ & $\sin \theta$ \\
	\hline
	\hline\rowcolor{BleuProfondIRA!40}
	$\theta + 2\pi$ & $\frac{\pi}{2} - \theta$ & $\frac{\pi}{2} + \theta$ \\
	\hline
	$\cos \theta$ & $\sin \theta$ & $-\sin \theta$ \\
	\hline
	$\sin \theta$ & $\cos \theta$ & $\cos \theta$ \\
	\hline
\end{tabular}
 \\ \vspace{2mm}
\ $\cos x = \cos y \iff \begin{cases}
	x \equiv y \pmod{2\pi} \text{ or }\\ x \equiv -y \pmod{2\pi}
\end{cases} $ \\
\ $\sin x = \sin y \iff \begin{cases}
	 x \equiv y \pmod{2\pi} \text{ or }\\ x \equiv \pi - y \pmod{2\pi}
\end{cases}$\\
\ $\tan x = \tan y \iff x \equiv y \pmod{\pi}$
};


	\def\radius{1}
	\pgfmathsetmacro{\angle}{15*\i+45} % degr\'es
	\pgfmathsetmacro{\cosx}{cos(\angle)}
	\pgfmathsetmacro{\siny}{sin(\angle)}
	
	% Coordonn\'ees de M
	\coordinate (M) at ({\cosx}, {\siny});
	
	% Axes
	\draw[->] (-0.2,0) -- (1.25,0) node[anchor=west] {$x$};
	\draw[->] (0,-1.5) -- (0,1.25) node[anchor=south] {$y$};
	
	% Cercle
	\draw[thick] (0,0) circle(\radius);
	
	% OM
	\draw[->, thick, BleuProfondIRA] (0,0) -- (M) node[above right] {$M$};
	
	% Projections cosinus et sinus
	\draw[dashed] (M) -- ({\cosx}, 0);
	\draw[dashed] (M) -- (0, {\siny});
	
	\draw[<->, thick, OrangeProfondIRA] (0, -0.1) -- ({\cosx}, -0.1);
	\node[below, OrangeProfondIRA] at ({\cosx/2}, -0.1) {$\cos\theta$};
	
	\draw[<->, thick, VertIRA] (-0.1, 0) -- (-0.1, {\siny});
	\node[left, VertIRA] at (-0.1, {\siny/2}) {$\sin\theta$};
	
	% Angle
	\draw[->, FushiaIRA] (0.5,0) arc[start angle=0, end angle=\angle, radius=0.5];
	\node at (0.7,0.15) {$\theta$};
	
	% Tangente (affich\'ee sauf si angle = 90\textdegree{} ou 270\textdegree{})
	\pgfmathtruncatemacro{\angleInt}{\angle} % angle en entier
	\ifnum\angleInt=90
	\draw[dashed] (1,-4) -- (1,4);
	\node[right] at (1,1.3) {$x=1$};
	\else\ifnum\angleInt=270
	\draw[dashed] (1,-4) -- (1,4);
	\node[right] at (1,1.3) {$x=1$};
	\else
	\pgfmathsetmacro{\tany}{tan(\angle)}
	\draw[dashed, gray] (0,0) -- (1.05, {\tany*1.05});
	\draw[dashed] (1,-4) -- (1,4);
	\coordinate (T) at (1, {\tany});
	\draw[fill=black] (T) circle(0.015);
	\node[above right] at (T) {$T$};
	\node[below right] at (T) {$(1,\tan\theta)$};
	\node[right] at (1,1.3) {$x=1$};
	\fi\fi
\end{tikzpicture}
			}%
		\end{animateinline}
	
\end{f}
\renewcommand{\arraystretch}{1}

\columnbreak
\begin{f}
	[Properties of the congruence relation]
	
	Let $m > 0$ et $a, b, c, d \in \mathbb{R}$. Then :
	
	\begin{itemize}
		\item \textbf{Reflexivity} : $a \equiv a \pmod{m}$.
		\item \textbf{Symmetry} : $a \equiv b \pmod{m} \iff b \equiv a \pmod{m}$.
		\item \textbf{Transitivity} : if $a \equiv b \pmod{m}$ and $b \equiv c \pmod{m}$, then $a \equiv c \pmod{m}$.
		\item \textbf{Additivity} : if $a \equiv b \pmod{m}$ and $c \equiv d \pmod{m}$, then $a + c \equiv b + d \pmod{m}$.
	\end{itemize}
	
\end{f}	

\hrule
\begin{f}[Derivatives and primitives] 

\[
\begin{aligned}
& f \text { continues in } x_\mathrm{m} \Leftrightarrow \lim _{x \rightarrow x_\mathrm{m}} f(x)=f\left(x_\mathrm{m}\right) \\
& f \text { derivable in } x_\mathrm{m} \Leftrightarrow \exists \lim _{h \rightarrow 0} \frac{f\left(x_\mathrm{m}+h\right)-f\left(x_\mathrm{m}\right)}{h}=: f^{\prime}\left(x_\mathrm{m}\right)
\end{aligned}
\]
 
The \textbf{Riemann integral} of a function \(f(x)\) on an interval \([a, b]\) is the limit, if it exists, of the sum of the areas of the rectangles approaching the area under the curve, given by :  

\[
\int_a^b f(x) \, dx = \lim_{n \to \infty} \sum_{i=1}^n f(x_i^*) \Delta x_i,
\]
\begin{itemize}
\item \( [x_{i-1}, x_i] \) is a subdivision of \([a, b]\),  
\item \( \Delta x_i = x_i - x_{i-1} \) is the width of the subinterval,  
\item \( x_i^* \in [x_{i-1}, x_i] \) is an arbitrarily chosen point in each subinterval.
\end{itemize}
  
\newcounter{numberofdiscretisation} \setcounter{numberofdiscretisation}{1}
\begin{center}
	    \begin{animateinline}[autoplay,loop,poster=15]{2}
      \whiledo{\thenumberofdiscretisation<20}{
        \begin{tikzpicture}[line cap=round, line join=round, >=triangle 45,
                            x=2.6cm, y=1.0cm, scale=1]
           \draw [->,color=OrangeProfondIRA] (-0.1,0) -- (2.5,0);
          \foreach \x in {1,2}
            \draw [shift={(\x,0)}, color=OrangeProfondIRA] (0pt,2pt)
                  -- (0pt,-2pt) node [below] {\footnotesize $\x$};
          \draw [color=OrangeProfondIRA] (2.5,0) node [below] {$x$};
          \draw [->,color=OrangeProfondIRA] (0,-0.1) -- (0,4.5);
          \foreach \y in {1,2,3,4}
            \draw [shift={(0,\y)}, color=OrangeProfondIRA] (2pt,0pt)
                  -- (-2pt,0pt) node[left] {\footnotesize $\y$};
          \draw [color=OrangeProfondIRA] (0,4.5) node [right] {$y$};
          \draw [color=OrangeProfondIRA] (0pt,-10pt) node [left] {\footnotesize $0$};
          \draw [color=OrangeProfondIRA, domain=0:2.2, line width=1.0pt] plot (\x,{(\x)^2});
          \clip(0,-0.5) rectangle (3,5);
          \draw (2,0) -- (2,4);
          \foreach \i in {1,...,\thenumberofdiscretisation}
            \draw [draw=BleuProfondIRA,fill=BleuProfondIRA,fill opacity=0.3, smooth,samples=50] ({1+(\i-1)/\thenumberofdiscretisation},{(1+(\i)/\thenumberofdiscretisation)^2})
                  --({1+(\i)/\thenumberofdiscretisation},{(1+(\i)/\thenumberofdiscretisation)^2})
                  --  ({1+(\i)/\thenumberofdiscretisation},0)
                  -- ({1+(\i-1)/\thenumberofdiscretisation},0)
                  -- cycle;
        \end{tikzpicture}
        %
        \stepcounter{numberofdiscretisation}
        \ifthenelse{\thenumberofdiscretisation<20}{ \newframe }{\end{animateinline} }
      }
      
\end{center}
      Example of a Riemann integral (upper)\href{https://texample.net/media/tikz/examples/TEX/upper-riemann-sum.tex}{$^*$}
 
 

The \textbf{Lebesgue integral} of a function \( f(x) \) on a set \( E \) is defined by measuring the area under the curve as a function of the values taken by \( f \), given by :  
\[
\int_E f \, d\mu = \int_0^\infty \mu(\{x \in E : f(x) > t\}) \, dt,
\]
\begin{itemize}
    \item \( \mu \) is a measure (often the Lebesgue measure),  
\item \( \{x \in E : f(x) > t\} \) represents the set of points where \( f(x) \) exceeds \( t \).  
\end{itemize} 
Unlike Riemann, Lebesgue groups points according to their values rather than their position.

\renewcommand{\arraystretch}{1.5}
\[
\begin{array}{|c|c|c|}
\hline \rowcolor{BleuProfondIRA!40} \text { function }(n \in \mathbb{R}) & \text { derivative } & \text { primitive } \\
\hline x & 1 & \frac{x^2}{2}+C \\
\hline x^2 & 2 x & \frac{x^3}{3}+C \\
\hline 1 / x & -1 / x^2 & \ln (x)+C \\
\hline \sqrt{x}=x^{1 / 2} & \frac{1}{2 \sqrt{x}} & \frac{2}{3} x^{3 / 2}+C \\
\hline x^n, n \neq-1 & n x^{n-1} & \frac{x^{n+1}}{n+1}+C \\
\hline \ln (x) & 1 / x & x \ln (x)-x+C \\
\hline e^x & e^x & e^x+C \\
\hline a^x=e^{x \ln (a)} & \ln (a) \times a^x & a^x / \ln (a)+C \\
\hline \sin (x) & \cos (x) & -\cos (x)+C \\
\hline \cos (x) & -\sin (x) & \sin (x)+C \\
\hline \tan (x) & 1+\tan (x) & -\ln (|\cos (x)|)+C \\
\hline 1 /\left(1+x^2\right) & -2 x /\left(1+x^2\right)^2 & \arctan (x)+C \\
\hline
\end{array}
\]
\renewcommand{\arraystretch}{1}

\[
    \begin{array}{rl|rl}
(u+v)^{\prime} & =u^{\prime}+v^{\prime} & \left(\frac{1}{u}\right)^{\prime} & =-\frac{u^{\prime}}{u^2} \\
(k u)^{\prime} & =k u^{\prime} & (\ln (u))^{\prime} & =\frac{u^{\prime}}{u} \\
(u \times v)^{\prime} & =u^{\prime} v+u v^{\prime} & (\exp (u))^{\prime} & =\exp (u) \times u^{\prime} \\
\left(\frac{u}{v}\right)^{\prime} & =\frac{u^{\prime} v-u v^{\prime}}{v^2} & (f(u))^{\prime} & =f^{\prime}(u) \times u^{\prime} \\
\left(u^n\right)^{\prime} & =n u^{n-1} \times u^{\prime} & (f \circ u)^{\prime} & =\left(f^{\prime} \circ u\right) \times u^{\prime}
\end{array}
\]
\end{f}

\hrule

\begin{f} [Integration by parts] 

Either \( u(x) \) et \( v(x) \) two continuously differentiable functions on the interval \([a, b]\), then

\[
\int_a^b u(x) v'(x) \, dx = \left[ u(x) v(x) \right]_a^b - \int_a^b u'(x) v(x) \, dx
\]

or :
\begin{itemize}
    \item \( u(x) \) is a function whose derivative is known \( u'(x) \),
    \item \( v'(x) \) is a function whose primitive is known \( v(x) \).
\end{itemize}

\end{f}
\hrule
\begin{f}[Integration with change of variable] 

Either \( f(x) \) a continuous function and \( x = \phi(t) \) a change of variable, where \( \phi \) is a differentiable function. Then :

\[
\int_{a}^{b} f(x) \, dx = \int_{\phi^{-1}(a)}^{\phi^{-1}(b)} f(\phi(t)) \phi'(t) \, dt
\]

or :
\begin{itemize}
    \item \( x = \phi(t) \) represents the change of variable,
    \item \( \phi'(t) \) is the derivative of \( \phi(t) \),
    \item the limits of the integral are adjusted according to the change of variable.
\end{itemize}
\end{f}

\hrule
\begin{f}
 [Taylor formula] 
 
Either \( f(x) \) a function \( n \) - times differentiable at a point \( a \). Taylor's development of \( f(x) \) around \( a \) is given by :

\begin{align*}
 f(x) =& f(a) + f'(a)(x - a) + \frac{f''(a)}{2!}(x - a)^2 \\
 &+ \cdots + \frac{f^{(n)}(a)}{n!}(x - a)^n + \mathcal{O}_n(x)
\end{align*}


où :
\begin{itemize}
    \item \( f^{(n)}(a) \) is the \( n \)\ieme{} derived from \( f \) evaluated in \( a \),
    \item \( \mathcal{O}_n(x) \) is the remainder of the Taylor development, representing the approximation error when truncating the series after the order term \( n \), then 
    \[
    \lim_{x\rightarrow 0}\frac{\mathcal{O}_n(x) }{x^n}\Rightarrow 0
    \]
\end{itemize}
\end{f}
\hrule
\begin{f}[Intermediate Value Theorem] 

Either \( f \) a continuous function on a closed interval \([a, b]\) and \( f(a) \neq f(b) \). The intermediate value theorem states that for any real number \( c \) between \( f(a) \) and \( f(b) \), there is a point \( x \in [a, b] \) such as :

\[
f(x) = c
\]

In other words, if a function is continuous on an interval, it takes all the values between \( f(a) \) and \( f(b) \) at least once.
\end{f}
\hrule
\begin{f}[Matrices and properties]
  
\textbf{Diagonal matrices :}
A matrix is said to be diagonal if all elements outside the main diagonal are zero. For a matrix \( A \in \mathbb{R}^{n \times n} \), it is written :
\[
A = \text{diag}(a_1, a_2, \dots, a_n)
\]
where \( a_i \) are the diagonal elements.

\textbf{Triangular matrices :}
A matrix is upper triangular if all elements below the diagonal are zero, that is, \( A_{ij} = 0 \) for \( i > j \). Conversely, it is lower triangular if \( A_{ij} = 0 \) for \( i < j \).
\end{f}
\begin{f}[Determinant of a matrix]
  
The determinant of a square matrix \( A \in \mathbb{R}^{n \times n} \) is a scalar, denoted \( \det(A) \) :
\begin{itemize}
    \item If \( A \) is a square matrix \( n \times n \), then \( A \) is invertible if and only if \( \det(A) \neq 0 \).
    \item The determinant of an upper or lower triangular matrix or a diagonal matrix :
    \[
    \det(A) = \prod_{i=1}^{n} A_{ii}
    \]
    \item
    \[
    \det(AB) = \det(A) \cdot \det(B),\quad \det(\lambda B) = \lambda\det(B), 
    \]
    \item 
    \[
    \det(A^T) = \det(A)
    \]
    \item If a matrix \( A \) contains two identical rows or columns, then \( \det(A) = 0 \).
\end{itemize}

\textbf{Calculation of the determinant :}
The determinant of a matrix \( 2 \times 2 \) is simply calculated by :
\[
\det\begin{pmatrix} a & b \\ c & d \end{pmatrix} = ad - bc
\]
For a matrix \( 3 \times 3 \), it is given by :
\[
\det\begin{pmatrix} a & b & c \\ d & e & f \\ g & h & i \end{pmatrix} = a(ei - fh) - b(di - fg) + c(dh - eg)
\]
For higher-dimensional matrices, the determinant can be calculated by cofactors or via a reduction method (e.g., Gauss's method).

\end{f}
\hrule
\begin{f}[Invertibility of a matrix] 
A matrix \( A \in \mathbb{R}^{n \times n} \) is invertible if there exists a matrix \( A^{-1} \) such as :
\[
A A^{-1} = A^{-1} A = I_n
\]
or \( I_n \) is the identity matrix. The invertibility of a matrix is guaranteed by \( \det(A) \neq 0 \).

\textbf{Trace :}
The trace of a square matrix \( A \), noted \( \text{Tr}(A) \), is the sum of its diagonal elements :
\[
\text{Tr}(A) = \sum_{i=1}^{n} A_{ii}
\]
It often represents quantities linked to the sum of the eigenvalues of a matrix.

\textbf{Cholesky decomposition :}
Cholesky decomposition is applicable to positive definite symmetric matrices. It allows a matrix to be factorized \( A \in \mathbb{R}^{n \times n} \) into a product of the form :
\[
A = LL^T
\]
or \( L \) is a lower triangular matrix. This decomposition is useful in numerical calculations and optimization algorithms.

\end{f}
\hrule

\begin{f}[Gradient and Hessian matrix]
	For a function \( f : \mathbb{R}^n \rightarrow \mathbb{R} \) of class \( C^2 \), we define :
	
	\begin{itemize}
		\item The \textbf{gradient} \( \nabla f(x) \) as the vector of first partial derivatives :
		$$
		\nabla f(x) = 
		\begin{pmatrix}
			\frac{\partial f}{\partial x_1}(x) \\
			\vdots \\
			\frac{\partial f}{\partial x_n}(x)
		\end{pmatrix}
		$$
		
		\item The \textbf{Hessian matrix} \( \nabla^2 f(x) \) as the symmetric matrix of second derivatives :
		$$
		\nabla^2 f(x) =
		\begin{pmatrix}
			\frac{\partial^2 f}{\partial x_1^2}(x) & \cdots & \frac{\partial^2 f}{\partial x_1 \partial x_n}(x) \\
			\vdots & \ddots & \vdots \\
			\frac{\partial^2 f}{\partial x_n \partial x_1}(x) & \cdots & \frac{\partial^2 f}{\partial x_n^2}(x)
		\end{pmatrix}
		$$
	\end{itemize}
	
\end{f}

\begin{f}[Implicit Function Theorem]
	Either \( F : \mathbb{R}^2 \rightarrow \mathbb{R} \) a class function \( C^1 \), and suppose that \( F(a^\star, b) = 0 \) for a certain couple \( (a^\star, b) \in \mathbb{R}^2 \). If
	\[
	\frac{\partial F}{\partial y}(a^\star, b) \neq 0,
	\]
	so there is a real \( h > 0 \) and a single function \( \varphi \), defined on a neighbourhood \( (a^\star - h, a^\star + h) \), such as
	\[
	\varphi(a^\star) = b \quad \text{et} \quad \forall x \in (a^\star - h, a^\star + h), \quad F(x, \varphi(x)) = 0
	\]
	
	In addition, the implicit function \( \varphi \) is of class \( C^1 \) and its derivative is given by :
	\[
	\varphi'(x) = - \left. \frac{\partial F / \partial x}{\partial F / \partial y} \right|_{y = \varphi(x)}
	\]
	

\end{f}

\end{multicols}

\begin{center}
\medskip

	\section*{Mes notes personnelles}
	\medskip
\end{center}

\begin{multicols}{2}	

\foreach \i in {1,...,67}{%
	\noindent\textcolor{gray!40}{\rule{\linewidth}{0.4pt}}\\[1.25em]
}	

\end{multicols}

\begin{tikzpicture}[remember picture, overlay]
	\node[anchor=south east, xshift=-0.5cm, yshift=2cm,
	draw=OrangeProfondIRA, rounded corners=4pt, line width=1pt,
	inner sep=0pt, text width=15.5cm, font=\small]
	at (current page.south east)
	{
		% Titre avec fond coloré inversé
		\begin{minipage}{\linewidth}
			\begin{tcolorbox}[beamerblock, title=About the authors]
				\href{https://sites.google.com/site/martialphelippeguinvarch}{\color{OrangeProfondIRA}Martial Ph\'elipp\'e-Guinvarc'h\ } 
				\href{mailto:Martial.Phelippe-GuinvarcH@univ-lemans.fr?subject=ActuarialFormSheet}{\color{OrangeProfondIRA}\faAt\ \faEnvelope} 
				\href{https://www.linkedin.com/in/martial-phelippe-guinvarc-h}{\color{OrangeProfondIRA}\faLinkedinSquare}
				\href{https://orcid.org/0000-0002-9894-803X}{\tikz[scale=0.75, baseline=-1ex]{\fill[OrangeProfondIRA] (0, 0) circle (0.25); 
						\node[white,font=\sffamily\bfseries\scriptsize] at (0.175, -0.1)  {ID};}}
				\href{https://github.com/MartialP-G}{\color{OrangeProfondIRA} \large \faGithub}  
				has been an actuary and lecturer at Le Mans University since 2012 and has also taught at EURIA since 2003. He is an actuary (EURIA, 2006), a member of the Institute of Actuaries (IA) and a member of the European Association of Agricultural Economists (EAAE).
			Martial Phélippé-Guinvarc'h teaches statistics, data analysis, actuarial modeling, programming, particularly in SAS and SAS IML, financial mathematics, commodity markets, market finance and derivatives, non-life actuarial science, life actuarial science, provisioning, solvency,  reinsurance, insurance economics, risk management, and asset-liability management. He is a SAS Protor and has obtained the SAS Join certificate \href{https://www.sas.com/fr_fr/learn/academic-programs/specializations/programs.html}{Insurance and  Economic Analytics} for the MBFA Master's degree from Le Mans University.
			
				
				\href{https://servicios.urjc.es/pdi/ver/marcelo.moreno}{\color{OrangeProfondIRA}Marcelo Moreno Porras} 
				\href{mailto:marcelo.moreno@urjc.es?subject=ActuarialFormSheet}{\color{OrangeProfondIRA}\faAt\ \faEnvelope} 
				\href{https://www.linkedin.com/in/marcelomorenop/}{\color{OrangeProfondIRA}\faLinkedinSquare}
				\href{https://orcid.org/0009-0007-1714-6507}{\tikz[scale=0.75, baseline=-1ex]{\fill[OrangeProfondIRA] (0, 0) circle (0.25); 
						\node[white,font=\sffamily\bfseries\scriptsize] at (0.175, -0.1) {ID};}}
				\href{https://github.com/marcelomijas}{\color{OrangeProfondIRA} \large \faGithub} 
			 is an Associate Professor at the Department of Applied Economics I and Economic History and Institutions at Universidad Rey Juan Carlos (Madrid, Spain) since 2023, specializing in econometrics, macroeconomics, and programming. He is also a Software Developer and IT specialist at a private company in Mijas (Spain). He is the author of several contributions to teaching innovation, co-author of publications on incorporating Sustainable Development Goals in economics education, and has technical expertise on multiple programming languages and statistical tools. His academic background includes a Bachelor’s degree in Economics (URJC, top 1% national ranking), a Master’s in Modern Economic Analysis (URJC), and a Master’s in Applied Statistics (Nebrija University).%

			\end{tcolorbox}
		\end{minipage}
	};
\end{tikzpicture}
\end{document}
