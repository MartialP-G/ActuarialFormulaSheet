% !TeX root = ActuarialFormSheet_MBFA-en.tex
% !TeX spellcheck = en_GB


\begin{f}[Transfer/retrocession schemes]

\tikzstyle{Sources} = [rectangle, rounded corners, minimum width=5cm, minimum height=1cm,text centered, text width=5cm, draw=black, fill=BleuProfondIRA!40]
\tikzstyle{projections} = [ellipse, trapezium left angle=70, trapezium right angle=110, minimum height=1cm, text centered, draw=black, fill=FushiaIRA!30]
\tikzstyle{Calculs} = [rectangle, minimum width=5cm, minimum height=1cm, text centered, text width=5cm, draw=black, fill=OrangeProfondIRA!30]
\tikzstyle{modalite} = [ellipse, minimum width=3cm, minimum height=1cm, text centered, draw=black, fill=BleuProfondIRA!40]
\tikzstyle{arrow} = [thick,->,>=stealth]


\resizebox{\linewidth}{!}{			%
\begin{tikzpicture}[node distance=1.5cm]
\node (Assure) [Sources] {\begin{tabular}{c} INSURED\\ \rowcolor{BleuProfondIRA!40}\tiny subscriber\end{tabular}};
\node (Ca) [projections, below left = of Assure, xshift=2.5cm, yshift=.5cm] {Insurance contract};
\node (AG) [modalite, below right = of Assure, xshift=-2.5cm, yshift=.5cm] {General Agent / Broker};
\node (Assureur) [Calculs, below of=Assure, node distance = 3cm] {\begin{tabular}{c}DIRECT INSURER\\
\rowcolor{OrangeProfondIRA!30}  \tiny  transferor\end{tabular} };
\node (ConvR) [projections, below left = of Assureur, xshift=2.5cm, yshift=.5cm] {Reinsurance agreement};
\node (Courtr) [modalite, below right = of Assureur, xshift=-2.5cm, yshift=.5cm] {Reinsurance broker};
\node (Reass) [Calculs, below of=Assureur, node distance = 3cm] {REINSURER(S)};
\node (ConvR2) [projections, below left = of Reass, xshift=2.5cm, yshift=.5cm] {Reinsurance agreement};
\node (Courtr2) [modalite, below right = of Reass, xshift=-2.5cm, yshift=.5cm] {Reinsurance broker};
\node (Retro) [Calculs, below of=Reass, node distance = 3cm] {RETROASSIGNEE(S)};
\draw [arrow] (Assure.south) -- (Assureur.north);
\draw [arrow] (Assureur.south) -- (Reass.north);
\draw [arrow] (Assureur.south) -- (Reass.north);
\draw [arrow] (Reass.south) -- (Retro.north);
%
\end{tikzpicture}}

\end{f}
\hrule


%https://www.reinsurancene.ws/top-50-reinsurance-groups/

\begin{f}[Key words of reinsurance]

\textbf{Cedant :} client of the reinsurer, i.e., the direct insurer, who transfers (cedes) risks to the reinsurer in exchange for the payment
of a \textbf{reinsurance premium}.

\textbf{Cession :} transfer of risks by the direct insurer to the reinsurer.

\textbf{Value Exposure} : limit of the amount of risk covered by a (re)insurance contract.

\textbf{Proportional reinsurance :} proportional participation of the reinsurer in the premiums and claims of the direct insurer.

\textbf{Quota Share :} type of proportional reinsurance where the reinsurer participates in a given percentage of all risks underwritten by a direct insurer in a given line of business.

\textbf{Surplus Share :} A type of proportional reinsurance where the reinsurer covers risks beyond the direct insurer's 
full retention amount. This ratio is calculated on the capacity of the risk subscribed (\(\approx\) Maximum possible loss).

\textbf{Reinsurance commission :} remuneration that the reinsurer grants to the insurer or brokers as compensation for the costs of acquiring and managing insurance contracts.

\textbf{Excess Reinsurance :}
 coverage by the reinsurer of claims exceeding a certain amount, against payment by the direct insurer of a specific reinsurance premium.

\textbf{Retrocession :}\index{R\'eassurance! R\'etrocession}  share of risks that the reinsurer cedes to other reinsurers.

\textbf{Co-insurance :}\index{R\'eassurance! Coassurance} participation of several direct insurers in the same risk.

We then use the expression \textbf{reinsurance pool}.
The main reinsurer is called \textbf{leading reinsurer}.

\textbf{Reinsurance treaty :} contract concluded between the direct insurer and the reinsurer on one or more of the insurer's portfolios.

\textbf{Facultative reinsurance :}
It differs from the reinsurance treaty by underwriting risk by risk (or policy by policy) (case by case, one risk at a time).

\end{f}
\hrule




\begin{f}
	[The economic role of reinsurance]

Insurance and reinsurance share the same purpose : the pooling of risks.
Reinsurance intervenes in particular on risks :
\begin{itemize}
	\item independent, but unitarily expensive (airplane, ship, industrial sites\ldots),
	\item small amounts (breakage, car, ...) but correlated during large-scale events, resulting in expensive accumulations,
	\item aggregated within a portfolio of insurance policies,
	\item little known or new.
\end{itemize}
	

Reinsurance allows to increase the business issuing capacity, ensure the financial stability of the insurer, especially in the event of disasters, reduce their capital requirements, and benefit from the expertise of the reinsurer.

\end{f}	
\hrule



\begin{f}
	[Types of reinsurance agreements]

{\color{white}.}	

\begin{center}
		\resizebox{0.90\linewidth}{!}{%!TeX spellcheck = en_GB
%!TEX root = ActuarialFormSheet_MBFA-en.tex

%%Created by jPicEdt 1.4.1_03: mixed JPIC-XML/LaTeX format
%%Tue Jul 03 17:58:32 CEST 2012
%%Begin JPIC-XML
%<?xml version="1.0" standalone="yes"?>
%<jpic x-min="5" x-max="100" y-min="2.5" y-max="37.5" auto-bounding="true">
%<multicurve points= "(40,30);(40,30);(40,20);(40,20)"
%	 fill-style= "none"
%	 />
%<multicurve points= "(20,30);(20,30);(60,30);(60,30)"
%	 fill-style= "none"
%	 />
%<multicurve points= "(60,35);(60,35);(60,30);(60,30)"
%	 fill-style= "none"
%	 />
%<multicurve points= "(20,30);(20,30);(20,35);(20,35)"
%	 fill-style= "none"
%	 />
%<multicurve points= "(20,20);(20,20);(20,15);(20,15)"
%	 fill-style= "none"
%	 />
%<multicurve points= "(20,20);(20,20);(70,20);(70,20)"
%	 fill-style= "none"
%	 />
%<multicurve points= "(70,20);(70,20);(70,15);(70,15)"
%	 fill-style= "none"
%	 />
%<multicurve points= "(5,15);(5,15);(30,15);(30,15)"
%	 fill-style= "none"
%	 />
%<multicurve points= "(30,10);(30,10);(30,15);(30,15)"
%	 fill-style= "none"
%	 />
%<multicurve points= "(5,15);(5,15);(5,10);(5,10)"
%	 fill-style= "none"
%	 />
%<multicurve points= "(55,15);(55,15);(55,10);(55,10)"
%	 fill-style= "none"
%	 />
%<multicurve points= "(55,15);(55,15);(100,15);(100,15)"
%	 fill-style= "none"
%	 />
%<multicurve points= "(75,15);(75,15);(75,10);(75,10)"
%	 fill-style= "none"
%	 />
%<multicurve points= "(100,15);(100,15);(100,10);(100,10)"
%	 fill-style= "none"
%	 />
%<text text-vert-align= "center-v"
%	 anchor-point= "(5,7.5)"
%	 fill-style= "none"
%	 text-frame= "noframe"
%	 text-hor-align= "center-h"
%	 >
%Quote-Part
%</text>
%<text text-vert-align= "center-v"
%	 anchor-point= "(30,7.5)"
%	 fill-style= "none"
%	 text-frame= "noframe"
%	 text-hor-align= "center-h"
%	 >
%Excédent
%</text>
%<text text-vert-align= "center-v"
%	 anchor-point= "(30,2.5)"
%	 fill-style= "none"
%	 text-frame= "noframe"
%	 text-hor-align= "center-h"
%	 >
%de plein
%</text>
%<text text-vert-align= "center-v"
%	 anchor-point= "(55,7.5)"
%	 fill-style= "none"
%	 text-frame= "noframe"
%	 text-hor-align= "center-h"
%	 >
%Excédent
%</text>
%<text text-vert-align= "center-v"
%	 anchor-point= "(75,7.5)"
%	 fill-style= "none"
%	 text-frame= "noframe"
%	 text-hor-align= "center-h"
%	 >
%Excédent
%</text>
%<text text-vert-align= "center-v"
%	 anchor-point= "(100,7.5)"
%	 fill-style= "none"
%	 text-frame= "noframe"
%	 text-hor-align= "center-h"
%	 >
%Excédent Annuel
%</text>
%<text text-vert-align= "center-v"
%	 anchor-point= "(70,22.5)"
%	 fill-style= "none"
%	 text-frame= "noframe"
%	 text-hor-align= "center-h"
%	 >
%Non proportionnelle
%</text>
%<text text-vert-align= "center-v"
%	 anchor-point= "(20,22.5)"
%	 fill-style= "none"
%	 text-frame= "noframe"
%	 text-hor-align= "center-h"
%	 >
%Proportionnelle
%</text>
%<text text-vert-align= "center-v"
%	 anchor-point= "(20,37.5)"
%	 fill-style= "none"
%	 text-frame= "noframe"
%	 text-hor-align= "center-h"
%	 >
%Facultative
%</text>
%<text text-vert-align= "center-v"
%	 anchor-point= "(60,37.5)"
%	 fill-style= "none"
%	 text-frame= "noframe"
%	 text-hor-align= "center-h"
%	 >
%Traité
%</text>
%<text text-vert-align= "center-v"
%	 anchor-point= "(55,2.5)"
%	 fill-style= "none"
%	 text-frame= "noframe"
%	 text-hor-align= "center-h"
%	 >
%par sinistre
%</text>
%<text text-vert-align= "center-v"
%	 anchor-point= "(75,2.5)"
%	 fill-style= "none"
%	 text-frame= "noframe"
%	 text-hor-align= "center-h"
%	 >
%par événement
%</text>
%<text text-vert-align= "center-v"
%	 anchor-point= "(100,2.5)"
%	 fill-style= "none"
%	 text-frame= "noframe"
%	 text-hor-align= "center-h"
%	 >
%Stop Loss
%</text>
%</jpic>
%%End JPIC-XML
%LaTeX-picture environment using emulated lines and arcs
%You can rescale the whole picture (to 80% for instance) by using the command \def\JPicScale{0.8}
\ifx\JPicScale\undefined\def\JPicScale{1}\fi
\unitlength \JPicScale mm
\begin{picture}(100,45)(5,0)
\linethickness{0.3mm}
\put(40,20){\line(0,1){10}}
\linethickness{0.3mm}
\put(20,30){\line(1,0){40}}
\linethickness{0.3mm}
\put(60,30){\line(0,1){5}}
\linethickness{0.3mm}
\put(20,30){\line(0,1){5}}
\linethickness{0.3mm}
\put(20,15){\line(0,1){5}}
\linethickness{0.3mm}
\put(20,20){\line(1,0){50}}
\linethickness{0.3mm}
\put(70,15){\line(0,1){5}}
\linethickness{0.3mm}
\put(10,15){\line(1,0){20}}
\linethickness{0.3mm}
\put(30,10){\line(0,1){5}}
\linethickness{0.3mm}
\put(10,10){\line(0,1){5}}
\linethickness{0.3mm}
\put(50,10){\line(0,1){5}}
\linethickness{0.3mm}
\put(50,15){\line(1,0){45}}
\linethickness{0.3mm}
\put(70,10){\line(0,1){5}}
\linethickness{0.3mm}
\put(95,10){\line(0,1){5}}
\put(10,7.5){\makebox(0,0)[cc]{Quota}}

\put(30,7.5){\makebox(0,0)[cc]{Full}}

\put(30,2.5){\makebox(0,0)[cc]{surplus}}

\put(50,7.5){\makebox(0,0)[cc]{Surplus }}

\put(70,7.5){\makebox(0,0)[cc]{Surplus per }}

\put(95,7.5){\makebox(0,0)[cc]{Excess losses}}

\put(70,22.5){\makebox(0,0)[cc]{Exccess}}

\put(20,22.5){\makebox(0,0)[cc]{Proportional}}

\put(20,37.5){\makebox(0,0)[cc]{Optional}}

\put(60,37.5){\makebox(0,0)[cc]{Treaty}}

\put(50,2.5){\makebox(0,0)[cc]{per claim}}

\put(70,2.5){\makebox(0,0)[cc]{event}}

\put(95,2.5){\makebox(0,0)[cc]{losses}}

\end{picture}
}
\end{center}
\end{f}
\hrule


\begin{f}[Types of reinsurance through an example]
	
Our insurer reinsures \(N=30\) insurance policies, with a total premium of 10M\EUR{} (\(P=\sum_{i=1\ldots N}P_i\)).
The total capacity is 180M\EUR{} (\(\sum_{i=1\ldots30}K_i\)).
\(S_r\) will be the total share of the loss covered by the insurer and \(P_r\) the total reinsurance premium.
Here are the \(n=8\) policies affected by losses (\(1\leq i \leq n\)), the losses of the other policies being zero (\(S_i=0, \forall i>n\)) :
	
	\begin{center}\footnotesize
		\renewcommand{\arraystretch}{1.25}
		\begin{tabular}{|l|rrrrrrrr|}
			\hline
			\rowcolor{BleuProfondIRA!40}         Claim number	& 1 & 2 & 3 & 4 & 5 & 6 & 7 & 8 \\ \hline 
			Prime  (k\EUR{})		& 500 & 200 & 100 & 100 & 50 & 200 & 500 & 200 \\ 
			Value Exposure  (M\EUR{})   	& 8 & 5 & 3 & 2 & 3 & 5 & 8 & 8 \\ 
			Claims  (M\EUR{})   	& 1 & 1 & 1 & 2 & 3 & 3 & 5 & 8 \\ \hline
		\end{tabular}
		\renewcommand{\arraystretch}{1}
	\end{center}
	
	The \(S/P\) is 240\%.
	
	
	\begin{center}
		\begin{tikzpicture}
			\begin{axis}[axis x line=bottom, axis y line = left,ymin=0,ymax=24,xmin=0, xmax=10, height=7cm,width=5cm, xticklabel style ={font=\footnotesize,align=center},  yticklabel style ={font=\footnotesize}, 
				legend style={at={(0.5,-0.15)},anchor=north,legend columns=-1,font=\tiny}]
				\addplot [ybar,fill=BleuProfondIRA!40,mark=none,draw=none] table [x index=0, y index=1] {..\\_Common\\ReinsuranceClaim.dat} ;
				\legend{Ordered Claims}
			\end{axis}	
		\end{tikzpicture}
		\begin{tikzpicture}
			\begin{axis}[axis x line=center,axis y line = left,ymin=0,ymax=24,xmin=0, xmax=10,height=7cm,width=5cm, xticklabel style ={font=\footnotesize,align=center},  yticklabel style ={font=\footnotesize}, 
				legend style={at={(0.5,-0.15)},
					anchor=north,legend columns=-1,font=\tiny}]
				\addplot [ybar,fill=BleuProfondIRA!40,mark=none,draw=none] table [x index=0, y index=2] {..\\_Common\\ReinsuranceClaim.dat} ;
				\legend{Cumulative Claims}
			\end{axis}	
		\end{tikzpicture}
		%\includegraphics{../../Reinsurance_M2/Graph/ReinsuranceClaim.pdf}
		%\includegraphics{../../Reinsurance_M2/Graph/ReinsuranceClaimCum.pdf}
	\end{center}
\medskip	


\textbf{Quota:}
\[
S_r=\alpha \sum_{i=1\ldots n}S_i\quad \ \ P_r=\alpha\sum_{i=1\ldots N}P_i 
\]
where \(\alpha\ \in [0,1]\) (25\% in the figure) is the share transferred in Quota.
	
	%\setlength{\tabcolsep}{1cm}

%\begin{tabular}{|l|rrrrrrrr|}
%	\hline
%	\rowcolor{BleuProfondIRA!40}         Num de sinistre		& 1 & 2 & 3 & 4 & 5 & 6 & 7 & 8 \\ \hline \hline
%	Sinistres  (M\EUR{})   	& 1 & 1 & 1 & 2 & 3 & 3 & 5 & 8 \\ 
%	Cédante  (M\EUR{})   	& 0,75 & 0,75 & 0,75 &
%	1,50 & 2,25 & 2,25 & 
%	3,75 & 6,00\\
%	Réassurance  (M\EUR{})  & 0,25 & 0,25 & 0,25 &
%	0,50 & 0,75 & 0,75 & 
%	1,25 & 2,00\\
%	Prime portef conservée  & \multicolumn{2}{r}{\cellcolor{mbfaulmbleu!10}7,5 M\EUR{}} & & & & & &\\
%	Sinistre conservé  	& \multicolumn{2}{r}{18} & & & & & & \\
%	S/P conservé  		& \multicolumn{2}{r}{\cellcolor{mbfaulmbleu!10}240\%} & & & & & & \\
%	\hline
%\end{tabular}
	
\begin{center}
		\begin{tikzpicture}
		\begin{axis}[axis x line=bottom, axis y line = left,ymin=0,ymax=24,xmin=0, xmax=10, height=7cm,width=5cm, xticklabel style ={font=\footnotesize,align=center},  yticklabel style ={font=\footnotesize}, 
			legend style={at={(0.5,-0.15)},anchor=north,legend columns=1,font=\tiny}]
			\addplot [ybar ,fill=OrangeMoyenIRA!40,mark=none,draw=OrangeMoyenIRA!40] table [x index=0, y index=1] {..\\_Common\\ReinsuranceClaim.dat} ;
			\addplot [ybar ,fill=BleuProfondIRA!40,mark=none,draw=BleuProfondIRA!40] table [x index=0, y index=6] {..\\_Common\\ReinsuranceClaim.dat} ;
			\legend{Claims Transfered,Claims Retained Quote}
		\end{axis}	
	\end{tikzpicture}
	%	
	\begin{tikzpicture}
		\begin{axis}[axis x line=center,axis y line = left,ymin=0,ymax=24,xmin=0, xmax=10,height=7cm,width=5cm, xticklabel style ={font=\footnotesize,align=center},  yticklabel style ={font=\footnotesize}, 
			legend style={at={(0.5,-0.15)},
				anchor=north,legend columns=1,font=\tiny}]
			\addplot [ybar ,fill=OrangeMoyenIRA!40,mark=none,draw=OrangeMoyenIRA!40] table [x index=0, y index=2] {..\\_Common\\ReinsuranceClaim.dat} ;
			\addplot [ybar ,fill=BleuProfondIRA!40,mark=none,draw=BleuProfondIRA!40] table [x index=0, y index=7] {..\\_Common\\ReinsuranceClaim.dat} ;
			\legend{Claims Transfered,Claims Retained Quote}
		\end{axis}	
	\end{tikzpicture}
	%\includegraphics{../../Reinsurance_M2/Graph/ReinsuranceClaimQuotePart.pdf}
	%\includegraphics{../../Reinsurance_M2/Graph/ReinsuranceClaimCumQuotePart.pdf}

\end{center}
\medskip

\textbf{Full surplus}, the full is noted \(\boldsymbol{K}\) (2M\EUR{} in the example), \(\alpha_i\) represents the cession rate of policy \(i\).
	\[
	S_r= \sum_{i=1\ldots n}\underbrace{\left(\frac{\left(K_i-\boldsymbol{K} \right)_+ }{K_i} \right)}_{\alpha_i} S_i\quad \ \ P_r=\sum_{i=1\ldots N}\left(\frac{\left(K_i-\boldsymbol{K} \right)_+ }{K_i} \right)P_i 
	\]
	%\renewcommand{\arraystretch}{1.25}
	%\rowcolors{1}{sectionColor}{white}
	%\setlength{\tabcolsep}{1cm}
%\begin{tabular}{|l|rrrrrrrr|}
%\hline
%\rowcolor{BleuProfondIRA!40}       Num de sinistre		& 1 & 2 & 3 & 4 & 5 & 6 & 7 & 8 \\ \hline \hline
%Sinistres  (M\EUR{})   	& 1 & 1 & 1 & 2 & 3 & 3 & 5 & 8 \\ 
%Capacité  (M\EUR{})   	& 8 & 5 & 3 & 2 & 3 & 5 & 8 & 8 \\ 
%Capacité cédante	& \multicolumn{2}{r}{60M\EUR{}} & & & & & & \\
%Prime cédante		& \multicolumn{2}{r}{\cellcolor{mbfaulmbleu!40}3M\EUR{}} & & & & & & \\
%Part cédée  (\%)   & 75 & 60 & 33 &
%0 & 33 & 60 & 
%75 & 75\\
%Cédante  (M\EUR{})  & 0,25 & 0,4 & 0,67 &
%2,00 & 2,00 & 1,20 & 
%1,25 & 2,00\\
%Réassurance  (M\EUR{})  & 0,75 & 0,60 & 0,33 &
%0,00 & 1,00 & 1,80 & 
%3,75 & 6,00\\
%Sinistre conservé  	& \multicolumn{2}{r}{\cellcolor{mbfaulmbleu!40}9,77} & & & & & & \\
%Sinistre cédé  		& \multicolumn{2}{r}{14,23} & & & & & & \\
%S/P conservée  		& \multicolumn{2}{r}{\cellcolor{mbfaulmbleu!40}325\%} & & & & & & \\
%\hline
%\end{tabular}


\begin{center}
		%\includegraphics{../../Reinsurance_M2/Graph/ReinsuranceClaimEPlein.pdf}
	%\includegraphics{../../Reinsurance_M2/Graph/ReinsuranceClaimCumEPlein.pdf}
	\begin{tikzpicture}
		%format de la date yyyy-mm-dd , pas de nom de colonne vide,
		\begin{axis}[axis x line=bottom, axis y line = left,ymin=0,ymax=24,xmin=0, xmax=10, height=7cm,width=5cm, xticklabel style ={font=\footnotesize,align=center},  yticklabel style ={font=\footnotesize}, 
			legend style={at={(0.5,-0.15)},anchor=north,legend columns=2,font=\tiny}]
			\addplot [ybar ,mark=none,draw=OrangeMoyenIRA!40,fill=OrangeMoyenIRA!40] table [x index=0, y index=1] {..\\_Common\\ReinsuranceClaim.dat} ;
			\addplot [ybar ,mark=none,draw=GrisLogoIRA,mark=none] table [x index=0, y index=11] {..\\_Common\\ReinsuranceClaim.dat} ;
			\addplot [draw=GrisLogoIRA] table [x index=0, y index=4] {..\\_Common\\ReinsuranceClaim.dat} ;
			\addplot [ybar ,fill=BleuProfondIRA!40,mark=none,draw=BleuProfondIRA!40] table [x index=0, y index=12] {..\\_Common\\ReinsuranceClaim.dat} ;
			\legend{Claims Transf,Value Exposure,Full, Claims Retained}
		\end{axis}	
	\end{tikzpicture}
	%	
	\begin{tikzpicture}
		%format de la date yyyy-mm-dd , pas de nom de colonne vide,
		\begin{axis}[axis x line=center,axis y line = left,ymin=0,ymax=24,xmin=0, xmax=10,height=7cm,width=5cm, xticklabel style ={font=\footnotesize,align=center},  yticklabel style ={font=\footnotesize}, 
			legend style={at={(0.5,-0.15)},
				anchor=north,legend columns=1,font=\tiny}]
			\addplot [ybar ,fill=OrangeMoyenIRA!40,mark=none,draw=OrangeMoyenIRA!40] table [x index=0, y index=2] {..\\_Common\\ReinsuranceClaim.dat} ;
			\addplot [ybar ,fill=BleuProfondIRA!40,mark=none,draw=BleuProfondIRA!40] table [x index=0, y index=13] {..\\_Common\\ReinsuranceClaim.dat} ;
			%		\addplot [ybar interval,fill=BleuProfondIRA,mark=none,draw=BleuProfondIRA] table [x index=0, y index=7] {..\\_Common\\ReinsuranceClaim.dat} ;
			\legend{Claims Transf, Claims Retained}
		\end{axis}	
	\end{tikzpicture}
\end{center}
\medskip

\textbf{Surplus per claim}
	%\renewcommand{\arraystretch}{1.25}
	%\rowcolors{1}{sectionColor}{white}
	%\setlength{\tabcolsep}{1cm}
	
The insurer sets the priority \(a\) and the scope \(b\) (respectively 2M\EUR{} and 4M\EUR{} in the figure).
\[
S_r= \sum_{i=1\ldots n} \min\left( \left( S_i-a\right)^+,b\right)  
\]
The premium is set by the reinsurer, based on its estimate of \(\mathbb{E}[S_r]\).
%\begin{tabular}{|l|rrrrrrrr|}
%	\hline
%	\rowcolor{BleuProfondIRA!40}        Num de sinistre		& 1 & 2 & 3 & 4 & 5 & 6 & 7 & 8 \\ \hline \hline
%	Sinistres  (M\EUR{})   	& 1 & 1 & 1 & 2 & 3 & 3 & 5 & 8 \\ 
%	Cédante  (M\EUR{})   	& 1 & 1 & 1 &
%	2 & 2 & 2 & 
%	2 & 4\\
%	Réassurance  (M\EUR{})  & 0 & 0 & 0 &
%	0 & 1 & 1 & 
%	3 & 4\\
%	\hline
%\end{tabular}

\begin{center}
		%\includegraphics{../../Reinsurance_M2/Graph/ReinsuranceClaimXSsin.pdf}
	%\includegraphics{../../Reinsurance_M2/Graph/ReinsuranceClaimCumXSsin.pdf}
	\begin{tikzpicture}
		%format de la date yyyy-mm-dd , pas de nom de colonne vide,
		\begin{axis}[axis x line=bottom, axis y line = left,ymin=0,ymax=24,xmin=0, xmax=10, height=7cm,width=5cm, xticklabel style ={font=\footnotesize,align=center},  yticklabel style ={font=\footnotesize}, 
			legend style={at={(0.5,-0.15)},anchor=north,legend columns=2,font=\tiny}]
			\addplot [ybar ,mark=none,draw=BleuProfondIRA!40,fill=BleuProfondIRA!40] table [x index=0, y index=1] {..\\_Common\\ReinsuranceClaim.dat} ;
			\addplot [ybar ,mark=none,draw=OrangeMoyenIRA!40,fill=OrangeMoyenIRA!40] table [x index=0, y index=10] {..\\_Common\\ReinsuranceClaim.dat} ;
			\addplot [ybar ,mark=none,draw=BleuProfondIRA!40,fill=BleuProfondIRA!40] table [x index=0, y index=8] {..\\_Common\\ReinsuranceClaim.dat} ;
			\addplot [draw=GrisLogoIRA] table [x index=0, y index=4] {..\\_Common\\ReinsuranceClaim.dat} ;
			\addplot [draw=GrisLogoIRA] table [x index=0, y index=5] {..\\_Common\\ReinsuranceClaim.dat} ;
			\legend{Claims Retained, Claims Transf, Priority €2M, Increased €4M }
		\end{axis}	
	\end{tikzpicture}
	%	
	\begin{tikzpicture}
		%format de la date yyyy-mm-dd , pas de nom de colonne vide,
		\begin{axis}[axis x line=center,axis y line = left,ymin=0,ymax=24,xmin=0, xmax=10,height=7cm,width=5cm, xticklabel style ={font=\footnotesize,align=center},  yticklabel style ={font=\footnotesize}, 
			legend style={at={(0.5,-0.15)},
				anchor=north,legend columns=1,font=\tiny}]
			\addplot [ybar ,fill=OrangeMoyenIRA!40,mark=none,draw=OrangeMoyenIRA!40] table [x index=0, y index=2] {..\\_Common\\ReinsuranceClaim.dat} ;
			\addplot [ybar ,fill=BleuProfondIRA!40,mark=none,draw=BleuProfondIRA!40] table [x index=0, y index=9] {..\\_Common\\ReinsuranceClaim.dat} ;
			%		\addplot [ybar ,fill=BleuProfondIRA,mark=none,draw=BleuProfondIRA] table [x index=0, y index=7] {..\\_Common\\ReinsuranceClaim.dat} ;
			\legend{Claims Transfered, Claims Retained}
		\end{axis}	
	\end{tikzpicture}
	\emph {WXL-R = Working XL per Risk}
\end{center}
	\medskip


\textbf{Surplus per event}
	%\renewcommand{\arraystretch}{1.25}
	%\rowcolors{1}{sectionColor}{white}
	%\setlength{\tabcolsep}{1cm}
	
	\[
	S_r= \sum_{\begin{array}{c}
			Cat_j,\\ i=1\ldots N
	\end{array}} \min\left( \mathds{1}_{i\in Cat_j}\times \left( S_i-a\right)^+,b\right)  
	\]
	
	In the illustration, claims refer to a single event, with a priority of 5M\EUR{} and a scope of 10M\EUR{}.
	
%\begin{tabular}{|l|rrrrrrrr|}
%	\hline
%	\rowcolor{BleuProfondIRA!40}        Num de sinistre		& 1 & 2 & 3 & 4 & 5 & 6 & 7 & 8 \\ \hline \hline
%	Sinistres  (M\EUR{})   	& 1 & 1 & 1 & 2 & 3 & 3 & 5 & 8 \\ 
%	Sinistres en Cumulé   	& 1 & 2 & 3 & 5 & 8 & 11 & 16 & 24 \\ 
%	Cédante en Cumulé   	& 1 & 2 & 3 &
%	5 & 5 & 5 & 
%	6 & 14\\
%	Réassurance en Cumulé  & 0 & 0 & 0 &
%	0 & 3 & 6 & 
%	10 & 10\\
%	\hline
%\end{tabular}

	
	%{../../Reinsurance_M2/Graph/ReinsuranceClaimXScat.pdf}
	%\includegraphics{../../Reinsurance_M2/Graph/ReinsuranceClaimCumXScat.pdf}
\begin{center}
		\begin{tikzpicture}
		\begin{axis}[axis x line=center,axis y line = left,ymin=0,ymax=24,xmin=0, xmax=10,height=7cm,width=5cm, xticklabel style ={font=\footnotesize,align=center},  yticklabel style ={font=\footnotesize}, 
			legend style={at={(0.5,-0.15)},
				anchor=north,legend columns=1,font=\tiny}]
			\addplot [ybar ,fill=OrangeMoyenIRA!40,mark=none,draw=OrangeMoyenIRA!40] table [x index=0, y index=1] {..\\_Common\\ReinsuranceClaim.dat} ;
			\addplot [ybar ,fill=BleuProfondIRA!40,mark=none,draw=BleuProfondIRA!40] table [x index=0, y index=18] {..\\_Common\\ReinsuranceClaim.dat} ;
			\legend{Claims Transfered, Claims Retained};
		\end{axis}	
	\end{tikzpicture}
	%	
	\begin{tikzpicture}
		\begin{axis}[axis x line=bottom, axis y line = left,ymin=0,ymax=24,xmin=0, xmax=10, height=7cm,width=5cm, xticklabel style ={font=\footnotesize,align=center},  yticklabel style ={font=\footnotesize}, 
			legend style={at={(0.5,-0.15)},anchor=north,legend columns=2,font=\tiny}]
			\addplot [ybar ,mark=none,draw=BleuProfondIRA!40,fill=BleuProfondIRA!40] table [x index=0, y index=2] {..\\_Common\\ReinsuranceClaim.dat} ;
			\addplot [ybar ,mark=none,draw=OrangeMoyenIRA!40,fill=OrangeMoyenIRA!40] table [x index=0, y index=16] {..\\_Common\\ReinsuranceClaim.dat} ;
			\addplot [ybar ,mark=none,draw=BleuProfondIRA!40,fill=BleuProfondIRA!40] table [x index=0, y index=17] {..\\_Common\\ReinsuranceClaim.dat} ;
			\addplot [draw=GrisLogoIRA] table [x index=0, y index=14] {..\\_Common\\ReinsuranceClaim.dat} ;
			\addplot [draw=GrisLogoIRA] table [x index=0, y index=15] {..\\_Common\\ReinsuranceClaim.dat} ;
			\legend{Claims Retained, Claims Transf, Priority €2M, Increased €4M};
		\end{axis}	
	\end{tikzpicture}
	\emph{ Cat-XL = Catastrophe XL }
\end{center}
\medskip

	
	
	%\renewcommand{\arraystretch}{1.25}
	%\rowcolors{1}{sectionColor}{white}
	%\setlength{\tabcolsep}{1cm}
	
%	L'assureur fixe la priorité à 5M\EUR{}  et la portée à 10M\EUR{}, soit un plafond de 15M\EUR{}.
%	Dans notre exemple supposons que les 8 sinistres font référence à \textbf{deux événements, une première tempête en octobre, une deuxième en décembre}.
	
%\begin{tabular}{|l|rrrrrrrr|}
%	\hline
%	\rowcolor{BleuProfondIRA!40}         Num de sinistre		& 1 & 2 & 3 & 4 & 5 & 6 & 7 & 8 \\ \hline \hline
%	Événement		&Oct & Oct & Déc & Oct & Déc & Oct & Déc & Déc \\ \hline \hline
%	Sinistres  (M\EUR{})   	& 1 & 1 & 1 & 2 & 3 & 3 & 5 & 8 \\ 
%	\hline
%\end{tabular}
%\begin{tabular}{|l|rr|r|}
%	\hline
%	\rowcolor{BleuProfondIRA!40}        Événement		&Oct &  Déc  & Total\\ \hline \hline
%	Som de sinistre		& 7 & 17& 24\\ \hline \hline
%	Cédante  	& 5 & 7& 12 \\
%	Réassurance en Cumulé  & 2 & 10& 12 \\
%	\hline
%\end{tabular}



\textbf{Annual excess losses}
	
	
Stop Loss occurs when the cumulative annual losses deteriorate.
It is expressed on the basis of the ratio \(S/P\) with a priority and a scope of \(XL\) expressed in \%.
	
\[
S_r= \min\left( \left( \sum_{i=1\ldots N} S_i - a  P\right)^+,bP\right)  
\]

\end{f}
\hrule
	
\begin{f}[The main clauses in reinsurance]

\textbf{The deductible} \(a^{ag}\) and the \textbf{aggregate limit} \(b^{ag}\) apply after the calculation of \(S_r\).

\[
S_r^{ag} = \min\left( \left(S_r - a^{ag}  \right)^+,b^{ag}\right)  
\]
\medskip

The objective of the \textbf{indexation clause} is to maintain the \underline{terms of the treaty} over several successive financial years. The treaty limits are aligned with an economic index (salary, currency, price index, ...).
\medskip

With \textbf{stabilization clause}, when the claim suffers from a \underline{long settlement}, or even a very long one (at least \(\geq 1\) year), the treaty limits are updated in the calculation of the \(S_r\) so that the respective shares of the reinsurer and the ceding company initially planned are generally respected.
\medskip
	

With the \textbf{interest sharing clause}, if in a transaction or court judgment a distinction has been made \underline{between compensation and interest}, the interest accrued between the date of the loss and the date of actual payment of the compensation will be divided between the ceding company and the reinsurer in proportion to their respective burden resulting from the application of the treaty excluding interest.


\medskip


The \textbf{guarantee reinstatement clause}
only concerns \emph{processed in excess of loss by risk or by event} which could be triggered several times during the year.
The reinsurer limits its benefit to \(N\) times the scope of the \(XS\), in return for the payment of an additional premium.
Reconstitution can be done pro rata temporis (time remaining until the expiry date of the treaty) or pro rata to the absorbed capital, or both (double pro rata).




\textbf{Interlocking Clause} is used in event-based XS treaties, which operate by subscription exercise and not \underline{by occurrence exercise}.
The interlocking clause will have the effect of recalculating the treaty limits, because the same event can trigger the treaty for both \(n\) and \(N-1\) subscriptions.


\end{f}
\hrule

\begin{f}[Public reinsurance]
The \href{http://www.ccr.fr}{Caisse Centrale de Réassurance (CCR)} offers, with the State guarantee, unlimited coverage for branches specific to the French market.
\begin{itemize}
\item   exceptional risks linked to transport,
\item   liability insurance for operators of nuclear vessels and installations,
\item   the risks of natural disasters,
\item   the risks of attacks and acts of terrorism,
\item   the Public Credit Insurance Supplement (CAP).
\end{itemize}
It also manages certain Public Funds on behalf of the State, in particular the Cat Nat regime.

Also, the \href{http://www.gareat.com/}{GAREAT} is a non-profit Economic Interest Group (GIE), mandated by its members, which manages reinsurance of risks of attacks and acts of terrorism with the support of the State via the CCR.
\end{f}
\hrule

\begin{f}[Securitization / CatBonds]
\label{CatBond}
Why? The financial capacity of all insurers and reinsurers combined does not cover the damage of a major earthquake in the United States. (\(\geq\) 200 B\EUR{}). 
This amount corresponds to less than 1\% of the capitalization of American financial markets.

Securitization transforms an insurance risk into a negotiable security, often into bonds called Cat-Bonds.
It consists of an exchange of principal for periodic payment of coupons, in which the payment of coupons and/or the repayment of principal are conditional on the occurrence of a triggering event defined a priori.
The rates on these bonds are increased based on the risk, not of default or counterparty, but of the occurrence of the event (less than 1\%). The structure dedicated to this transformation is called a Special Purpose Vehicle (SPV).


The trigger can be directly linked to the results of the predecessor (Compensation), depend on a loss index, a measurable parameter (sum of excess rainfall, Richter scale, mortality rate), or a model (RMS \& Equecat Storm modeling).



	\renewcommand{\arraystretch}{1.25}
\begin{center}\small
\begin{tabular}{|m{20mm}|*{4}{>{\centering\arraybackslash}m{14mm}|}}
\hline \rowcolor{BleuProfondIRA!40}
\textbf{Criteria} & \textbf{Compen\-sative} & \textbf{Hint} & \textbf{Para\-metric} & \textbf{Model} \\
\hline
Transparency & \(\ominus\) & \(\oplus\) & \(\oplus\) & \(\oplus\) \\
\hline
Basis risk & \(\oplus\) & \(\ominus\) & \(\ominus\) & \(\oplus\) \\
\hline
Moral hazard & \(\ominus\) & \(\oplus\) & \(\oplus\) & \(\oplus\) \\
\hline
Universality of perils & \(\oplus\) & \(\oplus\) & \(\ominus\) & \(\oplus\) \\
\hline
Trigger delay & \(\ominus\) & \(\ominus\) & \(\oplus\) & \(\oplus\) \\
\hline
\end{tabular}
\end{center}	\renewcommand{\arraystretch}{1}
\end{f}






