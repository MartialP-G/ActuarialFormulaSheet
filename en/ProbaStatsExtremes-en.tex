% !TeX root = ActuarialFormSheet_MBFA-en.tex
% !TeX spellcheck = en_GB

%\subsection*{Lois utilisées dans la théorie des extrêmes}


\begin{f}[Pareto's Law]
Let the random variable $X$ follow a Pareto distribution with parameters $(x_{\mathrm{m}},k)$, $k$ is the Pareto index:
$$
\mathbb{P}(X>x)=\left(\frac{x}{x_{\mathrm{m}}}\right)^{-k}\ \mbox{with}\ x \geq x_{\mathrm{m}} 
$$


$$
f_{k,x_\mathrm{m}}(x) = k\,\frac{x_\mathrm{m}^k}{x^{k+1}}\ \mbox{for}\ x \ge x_\mathrm{m}
$$

\textbf{	Generalized Pareto Law} (GPD) has 3 parameters $\mu$, $\sigma$ and $\xi$.


$$
F_{\xi,\mu,\sigma}(x) = \begin{cases} 1 - \left(1+ \frac{\xi(x-\mu)}{\sigma}\right)^{-1/\xi} & \text{for }\xi \neq 0, \\ 1 - \exp \left(-\frac{x-\mu}{\sigma}\right) & \text{for }\xi = 0. \end{cases} 
$$
for  $x \geq \mu$  when $\xi \geq 0$  and $ \mu \leq x \leq \mu - \sigma /\xi$  when $ \xi < 0$ et où $\mu\in\mathbb{R}$ is the location, $\sigma>0$ the scale and  $\xi\in\mathbb{R}$ the form. 
Note that some references give the \og{}shape parameter\fg{}, as $\kappa = - \xi$.

$$
f_{\xi,\mu,\sigma}(x) = \frac{1}{\sigma}\left(1 + \frac{\xi (x-\mu)}{\sigma}\right)^{\left(-\frac{1}{\xi} - 1\right)}
=
 \frac{\sigma^{\frac{1}{\xi}}}{\left(\sigma + \xi (x-\mu)\right)^{\frac{1}{\xi}+1}}
$$
%à nouveau, pour  $x \geq \mu$, et $x \leq \mu - \sigma /\xi$ où $\xi < 0$. 

\medskip

\begin{center}
		{\shorthandoff{:!;}	
\tikzset{
	declare function={
		dParetoGen(\x,\mu,\sigma,\xi) = 
		(\x>\mu)*(1/\sigma*(1 + \xi* (\x-\mu)/\sigma)^(-1/\xi - 1));}
}  
\begin{tikzpicture}[mark=none,samples=100,smooth,domain=0:15]
	\begin{axis}[ ytick=\empty,xtick pos=left,
		axis y line=middle,
		axis x line=bottom,
		height=.5*\linewidth,width=0.95*\linewidth, ticklabel style ={font=\footnotesize}, legend pos=north east,
		legend style={font=\tiny}]
		\addplot[BleuProfondIRA] 	({\x},{dParetoGen({\x},1,1,1)}) ;
		\addplot[FushiaIRA] 	({\x},{dParetoGen({\x},1,1,5)}) ;
		\addplot[VertIRA]  	({\x},{dParetoGen({\x},1,1,20)}) ;
		\addplot[OrangeProfondIRA] 	({\x},{dParetoGen({\x},3,2,5)}) ;
		%
		\legend{$\mu$ =  1 $\sigma=1$ $\xi = 1$,$\mu$ =  1  $\sigma=2$ $\xi =5$,$\mu$ =  1  $\sigma=3$ $\xi =20$,$\mu$ = 3  $\sigma=2$ $\xi = 5$};
	\end{axis} 
\end{tikzpicture}
}
\end{center}


%\begin{center}
%	{\shorthandoff{:!;}	
%	\tikzset{
%	declare function={
%		pParetoGen(\x,\mu,\sigma,\xi) =  (\x > \mu)*( 1-(1+ (\xi*(\x-\mu))/\sigma)^(-1/\xi));}
%}  
%\begin{tikzpicture}[mark=none,samples=100,smooth,domain=0:15]
%	\begin{axis}[ ytick=\empty,xtick pos=left,
%		axis y line=middle,
%		axis x line=bottom,
%		height=.5*\linewidth,width=0.95*\linewidth, ticklabel style ={font=\footnotesize}, legend pos=north east,
%		legend style={font=\tiny}]
%		\addplot[BleuProfondIRA] 
%		({\x},{pParetoGen({\x},1,1,1)}) ;
%		\addplot[FushiaIRA] 
%		({\x},{pParetoGen({\x},1,1,5)}) ;
%		\addplot[VertIRA] 
%		({\x},{pParetoGen({\x},1,1,20)}) ;
%		\addplot[OrangeProfondIRA] 
%		({\x},{pParetoGen({\x},3,2,5)}) ;
%		%
%		\legend{$\mu$ =  1 $\sigma=1$ $\xi = 1$,$\mu$ =  1  $\sigma=2$ $\xi =5$,$\mu$ =  1  $\sigma=3$ $\xi =20$,$\mu$ = 3  $\sigma=2$ $\xi = 5$};
%	\end{axis}	 
%\end{tikzpicture}
%}
%\end{center}

\end{f}
\hrule

\begin{f}[Generalized extreme value law]
The distribution function of the generalized law of extremes is
$$
F_{\mu,\sigma,\xi}(x) = \exp\left\{-\left[1+\xi\left(\frac{x-\mu}{\sigma}\right)\right]^{-1/\xi}\right\}
$$
for $1+\xi(x-\mu)/\sigma>0$, or $\mu\in\mathbb{R}$ is the location, $\sigma>0$ of scale and  $\xi\in\mathbb{R}$ the form. For $\xi = 0$ the expression is defined by its limit at 0.

\begin{align*}
	f_{\mu,\sigma,\xi}(x) =& \frac{1}{\sigma}\left[1+\xi\left(\frac{x-\mu}{\sigma}\right)\right]^{(-1/\xi)-1}\\ &\times\exp\left\{-\left[1+\xi\left(\frac{x-\mu}{\sigma}\right)\right]^{-1/\xi}\right\}
\end{align*}

$$
f(x;\mu ,\sigma ,0)={\frac {1}{\sigma }}\exp \left(-{\frac {x-\mu }{\sigma }}\right)\exp \left[-\exp \left(-{\frac {x-\mu }{\sigma }}\right)\right]
$$

\end{f}
\hrule



\begin{f}[Gumbel's Law]

The distribution function of Gumbel's law is :
$$
F_{\mu,\sigma}(x) = e^{-e^{(\mu-x)/\sigma}}.\,
$$
For $\mu = 0 $ et $\sigma = 1$, we obtain the standard Gumbel law.
Gumbel's law is a special case of GEV (with $\xi=0$).

Its density :
$$
f_{\mu,\sigma}(x)=\frac{1}{\sigma}e^{\left(\frac{x-\mu}{\sigma}-e^{-(x-\mu)/\sigma}\right)}
$$


\begin{center}
\begin{tikzpicture}[mark=none,samples=100,smooth,domain=-5:15]
\begin{axis}[ ytick=\empty,xtick pos=left,
	axis y line=middle,
	axis x line=bottom,
	height=.5*\linewidth,width=0.95*\linewidth, ticklabel style ={font=\footnotesize}, legend pos=north east,
	legend style={font=\tiny}]
	\addplot[BleuProfondIRA] 
	({\x},{dGumbel({\x},0.5,1)}) ;
	\addplot[FushiaIRA] 
	plot ({\x},{dGumbel({\x},1,2)}) ;
	\addplot[VertIRA] 
	plot ({\x},{dGumbel({\x},1.5,3)}) ;
	\addplot[OrangeProfondIRA] 
	plot ({\x},{dGumbel({\x},3,4)}) ;
	%
	\legend{$\mu$ =  0.5 $\gamma = 1$,$\mu$ =  1 $\gamma =2$,$\mu$ =  1.5 $\gamma =3$,$\mu$ =  3 $\gamma = 4$};
\end{axis}	 
\end{tikzpicture}
\end{center}


%\begin{center}
%\begin{tikzpicture}[mark=none,samples=100,smooth,domain=-5:15]
%\begin{axis}[ ytick=\empty,xtick pos=left,
%	axis y line=middle,
%	axis x line=bottom,
%	height=.5*\linewidth,width=0.95*\linewidth, ticklabel style ={font=\footnotesize}, legend pos=north west,
%	legend style={font=\tiny}]
%	\addplot[BleuProfondIRA] 
%	({\x},{pGumbel({\x},0.5,1)}) ;
%	\addplot[FushiaIRA] 
%	plot ({\x},{pGumbel({\x},1,2)}) ;
%	\addplot[VertIRA] 
%	plot ({\x},{pGumbel({\x},1.5,3)}) ;
%	\addplot[OrangeProfondIRA] 
%	plot ({\x},{pGumbel({\x},3,4)}) ;
%	%
%	\legend{$\mu$ =  0.5 $\gamma = 1$,$\mu$ =  1 $\gamma =2$,$\mu$ =  1.5 $\gamma =3$,$\mu$ =  3 $\gamma = 4$};
%\end{axis}	 
%\end{tikzpicture}
%\end{center}

%\subsubsubsection{Loi de Weibull}\label{label}
\end{f}
\hrule

\begin{f}[Weibull's Law]


The \textbf{Weibull distribution} has the distribution function defined by :
$$
F_{\alpha,\mu, \sigma}(x) = 1- e^{-((x-\mu)/\sigma)^\alpha}\,
$$
or $x >\mu$.
Its probability density is :
$$
f_{\alpha,\mu,\sigma}(x) = (\alpha/\sigma) ((x-\mu)/\sigma)^{(\alpha-1)} e^{-((x-\mu)/\sigma)^\alpha}\,
$$
or  $\mu\in\mathbb{R}$ is the location, $\sigma>0$ of scale and  $\alpha=-1/\xi>0 $ the form.

The Weibull distribution is often used in the field of lifetime analysis. It is a special case of the GEV when $\xi<0$.

If the failure rate decreases over time then, $\alpha<1$. If the failure rate is constant over time then, $\alpha=1$. If the failure rate increases over time then, $\alpha>1$. Understanding the failure rate can provide insight into the cause of failures.


\begin{center}
	{\shorthandoff{:!;}	\tikzset{
	declare function={
		dWeibull(\x,\sigma,\alpha) = (\alpha/\sigma)*(\x/\sigma)^(\alpha-1)*exp(-(\x/\sigma)^\alpha);}
}  
\begin{tikzpicture}[mark=none,samples=100,smooth,domain=0:5]
	\begin{axis}[ ytick=\empty,xtick pos=left,
		axis y line=middle,
		axis x line=bottom,
		height=.5*\linewidth,width=0.95*\linewidth, ticklabel style ={font=\footnotesize}, legend pos=north east,
		legend style={font=\tiny}]
		\addplot[BleuProfondIRA] 
		({\x},{dWeibull({\x},0.5,1)}) ;
		\addplot[FushiaIRA] 
		plot ({\x},{dWeibull({\x},1,2)}) ;
		\addplot[VertIRA] 
		plot ({\x},{dWeibull({\x},1.5,3)}) ;
		\addplot[OrangeProfondIRA] 
		plot ({\x},{dWeibull({\x},3,4)}) ;
		%
		\legend{$\mu$ =  0.5 $\gamma = 1$ ($\alpha=-1/\xi>0$),$\mu$ =  1 $\gamma =2$,$\mu$ =  1.5 $\gamma =3$,$\mu$ =  3 $\gamma = 4$};
	\end{axis}	 
\end{tikzpicture}
}
%	{\shorthandoff{:!;}	
%		\tikzset{
%	declare function={
%		pWeibull(\x,\sigma,\alpha) = 1- exp(-(\x/\sigma)^\alpha);}
%}  
%\begin{tikzpicture}[mark=none,samples=100,smooth,domain=0:5]
%	\begin{axis}[ ytick=\empty,xtick pos=left,
%		axis y line=middle,
%		axis x line=bottom,
%		height=.5*\linewidth,width=0.95*\linewidth, ticklabel style ={font=\footnotesize}, legend pos= south east,
%		legend style={font=\tiny}]
%		\addplot[BleuProfondIRA] 
%		({\x},{pWeibull({\x},0.5,1)}) ;
%		\addplot[FushiaIRA] 
%		plot ({\x},{pWeibull({\x},1,2)}) ;
%		\addplot[VertIRA] 
%		plot ({\x},{pWeibull({\x},1.5,3)}) ;
%		\addplot[OrangeProfondIRA] 
%		plot ({\x},{pWeibull({\x},3,4)}) ;
%		%
%		\legend{$\mu$ =  0.5 $\gamma = 1$ ($\alpha=-1/\xi>0$),$\mu$ =  1 $\gamma =2$,$\mu$ =  1.5 $\gamma =3$,$\mu$ =  3 $\gamma = 4$};
%	\end{axis}	 
%\end{tikzpicture}
%}
\end{center}
\end{f}


\begin{f}[Fréchet's Law]
Its distribution function of \textbf{Frechet's law}\index{D\'efinition! loi de Frechet} is given by
$$
F_{\alpha,\mu,\sigma}(x)=\mathbb{P}(X \le x)=\begin{cases} e^{-\left(\frac{x-\mu}{\sigma}\right)^{-\alpha}} & \text{ if } x>\mu \\ 0 &\text{ otherwise.}\end{cases} 
$$
or  $\mu\in\mathbb{R}$ is the location, $\sigma>0$ the scale and  $\alpha=1/\xi>0$ the form. 
This is a special case of GEV when $\xi>0$.

$$
f_{\alpha,\mu,\sigma}(x)=\frac{\alpha}{\sigma} \left(\frac{x-\mu}{\sigma}\right)^{-1-\alpha}  e^{-(\frac{x-\mu}{\sigma})^{-\alpha}}
$$

\begin{center}
%	\includegraphics[width=0.90\textwidth]{../Graph/LoiFrechet.pdf}
	{\shorthandoff{:!;}	
		\tikzset{
declare function={
	dFrechet(\x,\m,\sigma,\alpha) = (\alpha/\sigma) * (((\x-\m)/\sigma)^(-1-\alpha)) * exp(- ((\x-\m)/\sigma)^(-\alpha) );}
}  
\begin{tikzpicture}[mark=none,samples=100,smooth,domain=0:15]
\begin{axis}[ ytick=\empty,xtick pos=left,
	axis y line=middle,
	axis x line=bottom,
	height=.5*\linewidth,width=0.95*\linewidth, ticklabel style ={font=\footnotesize}, legend pos= north east,
	legend style={font=\tiny}]
	\addplot[BleuProfondIRA] 
	({\x},{dFrechet({\x},0,1,1)}) ;
	\addplot[FushiaIRA] 
	({\x},{dFrechet({\x},0,2,1)}) ;
	\addplot[VertIRA] 
	({\x},{dFrechet({\x},0,3,1)}) ;
	\addplot[OrangeProfondIRA] 
	({\x},{dFrechet({\x},0,2,2)}) ;
	%
	\legend{$\mu$ =  0 $\alpha=1$ $\sigma = 1$,$\mu$ =  0  $\alpha=2$ $\sigma =1$,$\mu$ =  0  $\alpha=3$ $\sigma =1$,$\mu$ = 0  $\alpha=2$ $\sigma = 2$};
\end{axis}	 
\end{tikzpicture}
}\end{center}


%\begin{center}
%%	\includegraphics[width=0.90\textwidth]{../Graph/FLoiFrechet.pdf}
%	{\shorthandoff{:!;}	
%\tikzset{
%	declare function={
%		pFrechet(\x,\m,\sigma,\alpha) = exp( -((\x-\m)/\sigma)^(-\alpha));}
%}  
%\begin{tikzpicture}[mark=none,samples=100,smooth,domain=0:15]
%	\begin{axis}[ ytick=\empty,xtick pos=left,
%		axis y line=middle,
%		axis x line=bottom,
%		height=.5*\linewidth,width=0.95*\linewidth, ticklabel style ={font=\footnotesize}, legend pos= south east,
%		legend style={font=\tiny}]
%		\addplot[BleuProfondIRA] 
%		({\x},{pFrechet({\x},0,1,1)}) ;
%		\addplot[FushiaIRA] 
%		({\x},{pFrechet({\x},0,2,1)}) ;
%		\addplot[VertIRA] 
%		({\x},{pFrechet({\x},0,3,1)}) ;
%		\addplot[OrangeProfondIRA] 
%		({\x},{pFrechet({\x},0,2,2)}) ;
%		%
%		\legend{$\mu$ =  0 $\alpha=1$ $\sigma = 1$,$\mu$ =  0  $\alpha=2$ $\sigma =1$,$\mu$ =  0  $\alpha=3$ $\sigma =1$,$\mu$ = 0  $\alpha=2$ $\sigma = 2$};
%	\end{axis}	 
%\end{tikzpicture}
%}
%\end{center}
\end{f}

\begin{f}[Link between GEV, Gumbel, Fréchet and Weibull]

The shape parameter $\xi$ governs the behavior of the distribution tail.
The subfamilies defined by $\xi= 0$, $\xi>0$ and $\xi<0$ correspond respectively to the Gumbel, Fréchet and Weibull families :
\begin{itemize}
\item 	 Gumbel or type I extreme value law
\item     Fréchet or type II extreme value law, if $\xi  = \alpha^{-1}$ with $\alpha>0$,
\item    Reversed Weibull ($\overline{F}$) or type III extreme value law, if $\xi =-\alpha^{-1} $, with $\alpha>0$.
\end{itemize}
\end{f}
\hrule 

\begin{f}[General Extreme Value Theorem] 

	Either $X_1, \dots, X_n$  $iid$, $X$ distribution function $F_X$ and either $M_n =\max(X_1,\dots,X_n)$.
	
	The theory gives the exact distribution of the maximum :
\begin{align*}
		\mathcal{P}(M_n \leq z) = &\Pr(X_1 \leq z, \dots, X_n \leq z) \\
		 = &\mathcal{P}(X_1 \leq z) \cdots \mathcal{P}(X_n \leq z) = (F_X(z))^n. 
\end{align*}
	If there exists a sequence of pairs of real numbers $(a_n, b_n)$ such that $a_n>0$ and $\lim_{n \to \infty}\mathcal{P}\left(\frac{M_n-b_n}{a_n}\leq x\right) = F_X(x)$, or $F_X$ is a non-degenerate distribution function, then the limit of the function $F_X$ belongs to the family of $GEV$ laws. 

\end{f}

\begin{f}[Sub-exponential density]
	
	\textbf{Case of powers} 
	
	If $\overline{F}_{X}(x)=\mathbb{P}(X > x)\sim c\ x^{-\alpha}$
		when $x \to \infty $ for one $\alpha > 0 $ and a constant $c > 0 $ then the law of $X$
		is sub-exponential.
	
		If $F_X$ is a continuous distribution function of expectation $\mathbb{E}[X]$ finished, we call the index of major risks by
		$$
		D_{F_X}(p)=\frac{1}{\mathbb{E}[X]}\int_{1-p}^{1} F_X^{-1}(t)dt,\, \, p\in [0,1]
		$$

	This excess distribution decays less quickly than any exponential distribution.
	It is possible to consider this statistic :
	$$
	T_n(p)=\frac{X_{(1:n)}+X_{(2:n)}+\ldots + X_{(np:n)}}{\sum_{1\leq i\leq n}(X_i)} \mbox{ où } \frac{1}{n}\leq p\leq 1
	$$
	$X_{(i:n)}$ designates the $i^e$ $\max $ of the $X_i$.

\end{f}
\begin{f}[Pickands-Balkema-de Haan theorem (law of excesses)]
	Let $X$ be of distribution $F_X$, and let $u$ be a high threshold. Then, for a large class of $F_X$ distributions, the conditional excess distribution
	\[
	X_u := X - u \mid X > u
	\]
	is approximated, for $u$ sufficiently large, by a generalized Pareto distribution (GPD) :
	
	\[
	\mathbb{P}(X - u \le y \mid X > u) \approx G_{\xi, \sigma, \mu=0}(y) :=  1 - \left(1+ \frac{\xi(x-\mu)}{\sigma}\right)^{-1/\xi} 
	\]
	
	$\quad y \ge 0$. In other words, for $u \to x_F := \sup\{x : F(x) < 1\}$,
	\[
	\sup_{0 \le y < x_F - u} \left| \mathbb{P}(X - u \le y \mid X > u) - G_{\xi,\sigma,\mu=0}(y) \right| \to 0.
	\]
	
	This theorem justifies the use of the \textit{Pareto law (generalized)} to model excesses beyond a threshold, which is precisely the framework of reinsurance treaties in \textit{excess of loss} by risk, by event or annual accumulation.
\end{f}
\hrule


\begin{f}
	[Reinsurance data ]

As reinsurance compensates for aggregations of losses or extreme losses, it often uses historical data which should be used with caution :
\begin{itemize}
	\item updating of data (impact of monetary inflation).
	\item the revaluation takes into account the evolution of the risk :
	\begin{itemize}
		\item changes in premium rates, guarantees and terms of contracts,
		\item the evolution of claims costs (construction cost index, car repair cost index,\ldots)
		\item the evolution of the legal environment.
	\end{itemize}
	\item the adjustment of the statistics to take into account the evolution of the portfolio base :
	\begin{itemize}
		\item font profile (number, capitals, \ldots),
		\item nature of guarantees (evolution of deductibles, exclusions\ldots)
	\end{itemize}
\end{itemize}
After these corrections, the data are said to be \og as if \fg (in economics, we use the expression counterfactual).

\end{f}


\begin{f}[Burning Cost]
	
	$X_{i}^{j}$ designates the $i^{e}$ disaster of the year $j$ \og{}as if\fg{} updated, revalued and corrected,
$n^j$ the number of claims per year $j$, $c^{j}$ the insurer's responsibility $c$.
The pure rate by the \textbf{Burning Cost} method is given by the formula :
$$
BC_{pur}=\frac{1}{s}\sum_{j=1}^{n}\frac{c^{j}}{a_{j}}
$$ 
The Burning Cost is only an average of the crossed $S/P$ ratios : the claims payable by the reinsurer on the premiums received by the ceding company.
The Burning Cost premium is then : $P_{pure}=BC_{pur}\times a_{s+1}$.

In the case of a $p$ XS $f$), $$
c^{j}=\sum_{i=1}^{n^j}\max\left(  \left( X_{i}^{j}-f\right),p\right) 1\!\!1_{x^{j}\geq f}
$$
While life insurance calculates premium rates with reference to capital, non-life insurance uses the insured value as a reference, reinsurance takes the total premiums of the ceding company as a reference, called the \textbf{base}.
We note $a_{j}$ denotes the premium base for year $j$ and $a_{s+1}^{*}$ denotes the estimated base for the coming year and where $s$ denotes the number of years of history.
\end{f}



\begin{f}[The Poisson-Pareto model]
[Bonus of XS or XL]
Let $p$ and $f$ be the scope and priority (franchise) of the XS, respectively, with the limit $l=p+f$ ($p$ XS $f$).

The XS bonus corresponds to :
$$
\mathbb{E}\left[S_N\right]=\mathbb{E}\left[\sum_{i=1}^{N} Y_{i}\right]=\mathbb{E}[N]\times \mathbb{E}[Y]
$$
où
$$
\mathbb{E}[Y]= l \mathbb{P}[X>l] - f \times \mathbb{P}[X\geq f] + \mathbb{E}[X\mid f\geq x\geq l]
$$


If $l=\infty$ and $\alpha \neq 1$ :
$$
\mathbb{E}[S_N] = \lambda \frac{x_\mathrm{m}^{\alpha}}{\alpha -1}f^{1-\alpha} 
$$

if  $l=\infty$ and $\alpha = 1$ there is no solution.

If $l<\infty$ and $\alpha \neq 1$ :
$$
\mathbb{E}[S_N] = \lambda \frac{x_\mathrm{m}^{\alpha}}{\alpha -1}\left( f^{1-\alpha} -l^{1-\alpha} \right)
$$

If $l<\infty$ and $\alpha = 1$ :
$$
\mathbb{E}[S_N] = \lambda x_\mathrm{m} \ln \left(  \frac{1}{f}\right)
$$
\end{f}



\begin{f}
[The Poisson-LogNormal Model]

If $X$ follows a $\mathcal{L}\mathcal{N}orm(x_\mathrm{m}, \mu, \sigma)$ then $X-x_\mathrm{m}$ follows a $\mathcal{L}\mathcal{N}orm(\mu, \sigma)$
It comes :
$$
\mathbb{P}[X>f]=\mathbb{P}[X-x_\mathrm{m}>f-x_\mathrm{m}]=1-\Phi\left(\frac{\ln(f-x_\mathrm{m})-\mu}{\sigma}\right)
$$
\begin{align*}
	\mathbb{E}[X\mid X>f] \\ 
	= & \mathbb{E}\left[ X-x_\mathrm{m}\mid X-x_\mathrm{m}>f-x_\mathrm{m} \right]+x_\mathrm{m} \mathbb{P}[X>f]\\
	= & e ^{m+\sigma^2/2} \left[1-\Phi\left(\frac{\ln(f-x_\mathrm{m})-(\mu+\sigma^2)}{\sigma}\right) \right]\\
	& +x_\mathrm{m} \left( 1-\Phi\left(\frac{\ln(f-x_\mathrm{m})-\mu}{\sigma}\right) \right)
\end{align*}
With frankness and without limits :
\begin{align*}
	\mathbb{E}[S_N] \\ 
	= & \lambda \left(\mathbb{E}\left[ X-x_\mathrm{m}\mid X-x_\mathrm{m}>f-x_\mathrm{m} \right]+x_\mathrm{m} \mathbb{P}[X>f]-f\mathbb{P}[X>f]\right)\\
	= & \lambda \left(  e ^{m+\sigma^2/2} \left[1-\Phi\left(\frac{\ln(f-x_\mathrm{m})-(\mu+\sigma^2)}{\sigma}\right) \right]\right)\\
	& +\lambda (x_\mathrm{m}-l) \left( 1-\Phi\left(\frac{\ln(f-x_\mathrm{m})-\mu}{\sigma}\right) \right)
\end{align*}

\end{f}