% !TeX root = ActuarialFormSheet_MBFA-en.tex
% !TeX spellcheck = fr_FR

%\subsection*{Lois utilisées dans la théorie des extrêmes}


\begin{f}[Loi de Pareto]
Soit la variable aléatoire $X$ qui suit une loi de Pareto de paramètres $(x_{\mathrm{m}},k)$, $k$ est l'indice de Pareto:
$$
\mathbb{P}(X>x)=\left(\frac{x}{x_{\mathrm{m}}}\right)^{-k}\ \mbox{avec}\ x \geq x_{\mathrm{m}} 
$$


$$
f_{k,x_\mathrm{m}}(x) = k\,\frac{x_\mathrm{m}^k}{x^{k+1}}\ \mbox{pour}\ x \ge x_\mathrm{m}
$$

\textbf{	Loi de Pareto généralisée} (GPD) a 3 paramètres $\mu$, $\sigma$ et $\xi$.


$$
F_{\xi,\mu,\sigma}(x) = \begin{cases} 1 - \left(1+ \frac{\xi(x-\mu)}{\sigma}\right)^{-1/\xi} & \text{for }\xi \neq 0, \\ 1 - \exp \left(-\frac{x-\mu}{\sigma}\right) & \text{for }\xi = 0. \end{cases} 
$$
pour  $x \geq \mu$  quand $\xi \geq 0$  et $ \mu \leq x \leq \mu - \sigma /\xi$  quand $ \xi < 0$ et où $\mu\in\mathbb{R}$ est la localisation, $\sigma>0$ l'échelle et  $\xi\in\mathbb{R}$ la forme. 
Notez que certaines références donnent le \og{}paramètre de forme\fg{}, comme $\kappa = - \xi$.

$$
f_{\xi,\mu,\sigma}(x) = \frac{1}{\sigma}\left(1 + \frac{\xi (x-\mu)}{\sigma}\right)^{\left(-\frac{1}{\xi} - 1\right)}
=
 \frac{\sigma^{\frac{1}{\xi}}}{\left(\sigma + \xi (x-\mu)\right)^{\frac{1}{\xi}+1}}
$$
%à nouveau, pour  $x \geq \mu$, et $x \leq \mu - \sigma /\xi$ où $\xi < 0$. 

\medskip

\begin{center}
		{\shorthandoff{:!;}	
\tikzset{
	declare function={
		dParetoGen(\x,\mu,\sigma,\xi) = 
		(\x>\mu)*(1/\sigma*(1 + \xi* (\x-\mu)/\sigma)^(-1/\xi - 1));}
}  
\begin{tikzpicture}[mark=none,samples=100,smooth,domain=0:15]
	\begin{axis}[ ytick=\empty,xtick pos=left,
		axis y line=middle,
		axis x line=bottom,
		height=.5*\linewidth,width=0.95*\linewidth, ticklabel style ={font=\footnotesize}, legend pos=north east,
		legend style={font=\tiny}]
		\addplot[BleuProfondIRA] 	({\x},{dParetoGen({\x},1,1,1)}) ;
		\addplot[FushiaIRA] 	({\x},{dParetoGen({\x},1,1,5)}) ;
		\addplot[VertIRA]  	({\x},{dParetoGen({\x},1,1,20)}) ;
		\addplot[OrangeProfondIRA] 	({\x},{dParetoGen({\x},3,2,5)}) ;
		%
		\legend{$\mu$ =  1 $\sigma=1$ $\xi = 1$,$\mu$ =  1  $\sigma=2$ $\xi =5$,$\mu$ =  1  $\sigma=3$ $\xi =20$,$\mu$ = 3  $\sigma=2$ $\xi = 5$};
	\end{axis} 
\end{tikzpicture}
}
\end{center}


%\begin{center}
%	{\shorthandoff{:!;}	
%	\tikzset{
%	declare function={
%		pParetoGen(\x,\mu,\sigma,\xi) =  (\x > \mu)*( 1-(1+ (\xi*(\x-\mu))/\sigma)^(-1/\xi));}
%}  
%\begin{tikzpicture}[mark=none,samples=100,smooth,domain=0:15]
%	\begin{axis}[ ytick=\empty,xtick pos=left,
%		axis y line=middle,
%		axis x line=bottom,
%		height=.5*\linewidth,width=0.95*\linewidth, ticklabel style ={font=\footnotesize}, legend pos=north east,
%		legend style={font=\tiny}]
%		\addplot[BleuProfondIRA] 
%		({\x},{pParetoGen({\x},1,1,1)}) ;
%		\addplot[FushiaIRA] 
%		({\x},{pParetoGen({\x},1,1,5)}) ;
%		\addplot[VertIRA] 
%		({\x},{pParetoGen({\x},1,1,20)}) ;
%		\addplot[OrangeProfondIRA] 
%		({\x},{pParetoGen({\x},3,2,5)}) ;
%		%
%		\legend{$\mu$ =  1 $\sigma=1$ $\xi = 1$,$\mu$ =  1  $\sigma=2$ $\xi =5$,$\mu$ =  1  $\sigma=3$ $\xi =20$,$\mu$ = 3  $\sigma=2$ $\xi = 5$};
%	\end{axis}	 
%\end{tikzpicture}
%}
%\end{center}

\end{f}
\hrule

\begin{f}[Loi des valeurs extrême généralisée]
La fonction de répartition de la loi des extrêmes généralisée est
$$
F_{\mu,\sigma,\xi}(x) = \exp\left\{-\left[1+\xi\left(\frac{x-\mu}{\sigma}\right)\right]^{-1/\xi}\right\}
$$
pour $1+\xi(x-\mu)/\sigma>0$, où $\mu\in\mathbb{R}$ est la localisation, $\sigma>0$ d'échelle et  $\xi\in\mathbb{R}$ la forme. Pour $\xi = 0$ l'expression est définie par sa limite en 0.

\begin{align*}
	f_{\mu,\sigma,\xi}(x) =& \frac{1}{\sigma}\left[1+\xi\left(\frac{x-\mu}{\sigma}\right)\right]^{(-1/\xi)-1}\\ &\times\exp\left\{-\left[1+\xi\left(\frac{x-\mu}{\sigma}\right)\right]^{-1/\xi}\right\}
\end{align*}

$$
f(x;\mu ,\sigma ,0)={\frac {1}{\sigma }}\exp \left(-{\frac {x-\mu }{\sigma }}\right)\exp \left[-\exp \left(-{\frac {x-\mu }{\sigma }}\right)\right]
$$

\end{f}
\hrule



\begin{f}[Loi de Gumbel]

La fonction de répartition de la \textbf{loi de Gumbel} est :
$$
F_{\mu,\sigma}(x) = e^{-e^{(\mu-x)/\sigma}}.\,
$$
Pour $\mu = 0 $ et $\sigma = 1$, on obtient la loi standard de Gumbel.
La loi de Gumbel est un cas particulier de la GEV (avec $\xi=0$).

Sa densité :
$$
f_{\mu,\sigma}(x)=\frac{1}{\sigma}e^{\left(\frac{x-\mu}{\sigma}-e^{-(x-\mu)/\sigma}\right)}
$$


\begin{center}
\begin{tikzpicture}[mark=none,samples=100,smooth,domain=-5:15]
\begin{axis}[ ytick=\empty,xtick pos=left,
	axis y line=middle,
	axis x line=bottom,
	height=.5*\linewidth,width=0.95*\linewidth, ticklabel style ={font=\footnotesize}, legend pos=north east,
	legend style={font=\tiny}]
	\addplot[BleuProfondIRA] 
	({\x},{dGumbel({\x},0.5,1)}) ;
	\addplot[FushiaIRA] 
	plot ({\x},{dGumbel({\x},1,2)}) ;
	\addplot[VertIRA] 
	plot ({\x},{dGumbel({\x},1.5,3)}) ;
	\addplot[OrangeProfondIRA] 
	plot ({\x},{dGumbel({\x},3,4)}) ;
	%
	\legend{$\mu$ =  0.5 $\gamma = 1$,$\mu$ =  1 $\gamma =2$,$\mu$ =  1.5 $\gamma =3$,$\mu$ =  3 $\gamma = 4$};
\end{axis}	 
\end{tikzpicture}
\end{center}


%\begin{center}
%\begin{tikzpicture}[mark=none,samples=100,smooth,domain=-5:15]
%\begin{axis}[ ytick=\empty,xtick pos=left,
%	axis y line=middle,
%	axis x line=bottom,
%	height=.5*\linewidth,width=0.95*\linewidth, ticklabel style ={font=\footnotesize}, legend pos=north west,
%	legend style={font=\tiny}]
%	\addplot[BleuProfondIRA] 
%	({\x},{pGumbel({\x},0.5,1)}) ;
%	\addplot[FushiaIRA] 
%	plot ({\x},{pGumbel({\x},1,2)}) ;
%	\addplot[VertIRA] 
%	plot ({\x},{pGumbel({\x},1.5,3)}) ;
%	\addplot[OrangeProfondIRA] 
%	plot ({\x},{pGumbel({\x},3,4)}) ;
%	%
%	\legend{$\mu$ =  0.5 $\gamma = 1$,$\mu$ =  1 $\gamma =2$,$\mu$ =  1.5 $\gamma =3$,$\mu$ =  3 $\gamma = 4$};
%\end{axis}	 
%\end{tikzpicture}
%\end{center}

%\subsubsubsection{Loi de Weibull}\label{label}
\end{f}
\hrule

\begin{f}[Loi de Weibull]


La \textbf{loi de Weibull} a pour fonction de répartition est définie par :
$$
F_{\alpha,\mu, \sigma}(x) = 1- e^{-((x-\mu)/\sigma)^\alpha}\,
$$
où $x >\mu$.
Sa densité de probabilité est :
$$
f_{\alpha,\mu,\sigma}(x) = (\alpha/\sigma) ((x-\mu)/\sigma)^{(\alpha-1)} e^{-((x-\mu)/\sigma)^\alpha}\,
$$
où  $\mu\in\mathbb{R}$ est la localisation, $\sigma>0$ d'échelle et  $\alpha=-1/\xi>0 $ la forme.

La distribution de Weibull est souvent utilisée dans le domaine de l'analyse de la durée de vie. C'est un cas particulier de la GEV lorsque $\xi<0$.

Si le taux de pannes diminue au cours du temps alors, $\alpha<1$. Si le taux de panne est constant dans le temps alors, $\alpha=1$. Si le taux de panne augmente avec le temps alors, $\alpha>1$. La compréhension du taux de pannes peut fournir une indication au sujet de la cause des pannes.


\begin{center}
	{\shorthandoff{:!;}	\tikzset{
	declare function={
		dWeibull(\x,\sigma,\alpha) = (\alpha/\sigma)*(\x/\sigma)^(\alpha-1)*exp(-(\x/\sigma)^\alpha);}
}  
\begin{tikzpicture}[mark=none,samples=100,smooth,domain=0:5]
	\begin{axis}[ ytick=\empty,xtick pos=left,
		axis y line=middle,
		axis x line=bottom,
		height=.5*\linewidth,width=0.95*\linewidth, ticklabel style ={font=\footnotesize}, legend pos=north east,
		legend style={font=\tiny}]
		\addplot[BleuProfondIRA] 
		({\x},{dWeibull({\x},0.5,1)}) ;
		\addplot[FushiaIRA] 
		plot ({\x},{dWeibull({\x},1,2)}) ;
		\addplot[VertIRA] 
		plot ({\x},{dWeibull({\x},1.5,3)}) ;
		\addplot[OrangeProfondIRA] 
		plot ({\x},{dWeibull({\x},3,4)}) ;
		%
		\legend{$\mu$ =  0.5 $\gamma = 1$ ($\alpha=-1/\xi>0$),$\mu$ =  1 $\gamma =2$,$\mu$ =  1.5 $\gamma =3$,$\mu$ =  3 $\gamma = 4$};
	\end{axis}	 
\end{tikzpicture}
}
%	{\shorthandoff{:!;}	
%		\tikzset{
%	declare function={
%		pWeibull(\x,\sigma,\alpha) = 1- exp(-(\x/\sigma)^\alpha);}
%}  
%\begin{tikzpicture}[mark=none,samples=100,smooth,domain=0:5]
%	\begin{axis}[ ytick=\empty,xtick pos=left,
%		axis y line=middle,
%		axis x line=bottom,
%		height=.5*\linewidth,width=0.95*\linewidth, ticklabel style ={font=\footnotesize}, legend pos= south east,
%		legend style={font=\tiny}]
%		\addplot[BleuProfondIRA] 
%		({\x},{pWeibull({\x},0.5,1)}) ;
%		\addplot[FushiaIRA] 
%		plot ({\x},{pWeibull({\x},1,2)}) ;
%		\addplot[VertIRA] 
%		plot ({\x},{pWeibull({\x},1.5,3)}) ;
%		\addplot[OrangeProfondIRA] 
%		plot ({\x},{pWeibull({\x},3,4)}) ;
%		%
%		\legend{$\mu$ =  0.5 $\gamma = 1$ ($\alpha=-1/\xi>0$),$\mu$ =  1 $\gamma =2$,$\mu$ =  1.5 $\gamma =3$,$\mu$ =  3 $\gamma = 4$};
%	\end{axis}	 
%\end{tikzpicture}
%}
\end{center}
\end{f}


\begin{f}[Loi de Fréchet]
Sa fonction de répartition de la \textbf{loi de Frechet}\index{D\'efinition! loi de Frechet} est donnée par 
$$
F_{\alpha,\mu,\sigma}(x)=\mathbb{P}(X \le x)=\begin{cases} e^{-\left(\frac{x-\mu}{\sigma}\right)^{-\alpha}} & \text{ si } x>\mu \\ 0 &\text{ sinon.}\end{cases} 
$$
où  $\mu\in\mathbb{R}$ est la localisation, $\sigma>0$ l'échelle et  $\alpha=1/\xi>0$ la forme. 
C'est un cas particulier de la GEV lorsque $\xi>0$.

$$
f_{\alpha,\mu,\sigma}(x)=\frac{\alpha}{\sigma} \left(\frac{x-\mu}{\sigma}\right)^{-1-\alpha}  e^{-(\frac{x-\mu}{\sigma})^{-\alpha}}
$$

\begin{center}
%	\includegraphics[width=0.90\textwidth]{../Graph/LoiFrechet.pdf}
	{\shorthandoff{:!;}	
		\tikzset{
declare function={
	dFrechet(\x,\m,\sigma,\alpha) = (\alpha/\sigma) * (((\x-\m)/\sigma)^(-1-\alpha)) * exp(- ((\x-\m)/\sigma)^(-\alpha) );}
}  
\begin{tikzpicture}[mark=none,samples=100,smooth,domain=0:15]
\begin{axis}[ ytick=\empty,xtick pos=left,
	axis y line=middle,
	axis x line=bottom,
	height=.5*\linewidth,width=0.95*\linewidth, ticklabel style ={font=\footnotesize}, legend pos= north east,
	legend style={font=\tiny}]
	\addplot[BleuProfondIRA] 
	({\x},{dFrechet({\x},0,1,1)}) ;
	\addplot[FushiaIRA] 
	({\x},{dFrechet({\x},0,2,1)}) ;
	\addplot[VertIRA] 
	({\x},{dFrechet({\x},0,3,1)}) ;
	\addplot[OrangeProfondIRA] 
	({\x},{dFrechet({\x},0,2,2)}) ;
	%
	\legend{$\mu$ =  0 $\alpha=1$ $\sigma = 1$,$\mu$ =  0  $\alpha=2$ $\sigma =1$,$\mu$ =  0  $\alpha=3$ $\sigma =1$,$\mu$ = 0  $\alpha=2$ $\sigma = 2$};
\end{axis}	 
\end{tikzpicture}
}\end{center}


%\begin{center}
%%	\includegraphics[width=0.90\textwidth]{../Graph/FLoiFrechet.pdf}
%	{\shorthandoff{:!;}	
%\tikzset{
%	declare function={
%		pFrechet(\x,\m,\sigma,\alpha) = exp( -((\x-\m)/\sigma)^(-\alpha));}
%}  
%\begin{tikzpicture}[mark=none,samples=100,smooth,domain=0:15]
%	\begin{axis}[ ytick=\empty,xtick pos=left,
%		axis y line=middle,
%		axis x line=bottom,
%		height=.5*\linewidth,width=0.95*\linewidth, ticklabel style ={font=\footnotesize}, legend pos= south east,
%		legend style={font=\tiny}]
%		\addplot[BleuProfondIRA] 
%		({\x},{pFrechet({\x},0,1,1)}) ;
%		\addplot[FushiaIRA] 
%		({\x},{pFrechet({\x},0,2,1)}) ;
%		\addplot[VertIRA] 
%		({\x},{pFrechet({\x},0,3,1)}) ;
%		\addplot[OrangeProfondIRA] 
%		({\x},{pFrechet({\x},0,2,2)}) ;
%		%
%		\legend{$\mu$ =  0 $\alpha=1$ $\sigma = 1$,$\mu$ =  0  $\alpha=2$ $\sigma =1$,$\mu$ =  0  $\alpha=3$ $\sigma =1$,$\mu$ = 0  $\alpha=2$ $\sigma = 2$};
%	\end{axis}	 
%\end{tikzpicture}
%}
%\end{center}
\end{f}

\begin{f}[Lien entre GEV, Gumbel, Fréchet et Weibull]

Le paramètre de forme $\xi$ gouverne le comportement de la queue de distribution. 
Les sous-familles définies par $\xi= 0$, $\xi>0$ et $\xi<0$ correspondent respectivement aux familles de Gumbel, Fréchet et Weibull~:
\begin{itemize}
\item 	 Gumbel ou loi des valeurs extrêmes de type I
\item     Fréchet ou loi des valeurs extrêmes de type II, si $\xi  = \alpha^{-1}$ avec $\alpha>0$,
\item    Reversed Weibull ($\overline{F}$) ou loi des valeurs extrêmes de type III, si $\xi =-\alpha^{-1} $, avec $\alpha>0$.
\end{itemize}
\end{f}
\hrule 

\begin{f}[Théorème général des valeurs extrêmes] 

	Soit $X_1, \dots, X_n$  $iid$, $X$ de fonction de répartition $F_X$ et soit $M_n =\max(X_1,\dots,X_n)$.
	
	La théorie donne la distribution exacte du maximum :
\begin{align*}
		\mathcal{P}(M_n \leq z) = &\Pr(X_1 \leq z, \dots, X_n \leq z) \\
		 = &\mathcal{P}(X_1 \leq z) \cdots \mathcal{P}(X_n \leq z) = (F_X(z))^n. 
\end{align*}
	S'il existe une séquence de paire de nombres réels $(a_n, b_n)$ de telle sorte que $a_n>0$ et $\lim_{n \to \infty}\mathcal{P}\left(\frac{M_n-b_n}{a_n}\leq x\right) = F_X(x)$, où $F_X$ est une fonction de répartition non dégénérée, alors la limite de la fonction $F_X$ appartient à la famille des lois $GEV$. 

\end{f}

\begin{f}[Densité sous-exponentielle]
	
	\textbf{Cas des puissances} 
	
	Si $\overline{F}_{X}(x)=\mathbb{P}(X > x)\sim c\ x^{-\alpha}$
		quand $x \to \infty $ pour un $\alpha > 0 $ et une constante $c > 0 $ alors la loi de $X$
		est sous-exponentielle.
	
		Si $F_X$ est une fonction de répartition continue d'espérance $\mathbb{E}[X]$ finie, on appelle l'indice des grands risques par
		$$
		D_{F_X}(p)=\frac{1}{\mathbb{E}[X]}\int_{1-p}^{1} F_X^{-1}(t)dt,\, \, p\in [0,1]
		$$

	Cette distribution en excès décroit moins vite que n'importe quelle distribution exponentielle.
	Il est possible de considérer cette statistique ~:
	$$
	T_n(p)=\frac{X_{(1:n)}+X_{(2:n)}+\ldots + X_{(np:n)}}{\sum_{1\leq i\leq n}(X_i)} \mbox{ où } \frac{1}{n}\leq p\leq 1
	$$
	$X_{(i:n)}$ désigne le $i^e$ $\max $ des $X_i$.

\end{f}
\begin{f}[Théorème de Pickands–Balkema–de Haan (loi des excès)]
	Soit $X$ de distribution $F_X$, et soit $u$ un seuil élevé. Alors, pour une large classe de lois $F_X$, la loi conditionnelle des excès
	\[
	X_u := X - u \mid X > u
	\]
	est approximable, pour $u$ suffisamment grand, par une loi de Pareto généralisée (GPD) :
	
	\[
	\mathbb{P}(X - u \le y \mid X > u) \approx G_{\xi, \sigma, \mu=0}(y) :=  1 - \left(1+ \frac{\xi(x-\mu)}{\sigma}\right)^{-1/\xi} 
	\]
	
	$\quad y \ge 0$. Autrement dit, pour $u \to x_F := \sup\{x : F(x) < 1\}$,
	\[
	\sup_{0 \le y < x_F - u} \left| \mathbb{P}(X - u \le y \mid X > u) - G_{\xi,\sigma,\mu=0}(y) \right| \to 0.
	\]
	
	Ce théorème justifie l’utilisation de la \textit{loi de Pareto (généralisée)} pour modéliser les excès au-delà d’un seuil, ce qui est précisément le cadre des traités de réassurance en \textit{excess of loss} par risque, par événement ou de cumul annuel. 
\end{f}
\hrule


\begin{f}
	[Les données en réassurance ]

Comme la réassurance indemnise des agrégations de sinistre ou des sinistres extrêmes, elle utilise souvent des historiques qui devront être utilisé avec prudence~:
\begin{itemize}
	\item l'actualisation des données (impact de l'inflation monétaire).
	\item la revalorisation prend en compte l'évolution du risque :
	\begin{itemize}
		\item l'évolution des taux de prime, garanties et modalités des contrats,
		\item l'évolution des coûts des sinistres (indice des coûts de la construction, indices des coûts de réparation automobile,\ldots)
		\item l'évolution de l'environnement juridique.
	\end{itemize}
	\item le redressement de la statistique pour prendre en compte l'évolution de la base  portefeuille :
	\begin{itemize}
		\item profil des polices (nombre, capitaux,\ldots),
		\item natures des garanties (évolution des franchises, des exclusions\ldots)
	\end{itemize}
\end{itemize}
Après ces corrections, les données sont dites \og as if \fg (en économie, on utilise l'expression contre-factuel).

\end{f}


\begin{f}[La prime \emph{Burning Cost}]
	
	$X_{i}^{j}$ désigne le $i^{e}$ sinistre  de l'année $j$ \og{}as if\fg{} actualisé, revalorisé et  redressé,
$n^j$ le nombre de sinistres l'année $j$, $c^{j}$ la charge de l'assureur $c$.
Le taux pur  par la méthode de \textbf{Burning Cost} est donné par la formule~:
$$
BC_{pur}=\frac{1}{s}\sum_{j=1}^{n}\frac{c^{j}}{a_{j}}
$$ 
Le Burning Cost n'est qu'une moyenne des ratios $S/P$ croisés~: les sinistres à la charge du réassureur sur les primes reçues par la cédante.
La prime Burning Cost est alors : $P_{pure}=BC_{pur}\times a_{s+1}$.

Dans le cas d'un $p$ XS $f$), $$
c^{j}=\sum_{i=1}^{n^j}\max\left(  \left( X_{i}^{j}-f\right),p\right) 1\!\!1_{x^{j}\geq f}
$$
Si l'assurance vie calcule des taux de prime en référence au capital, l'assurance non vie utilise comme  référence à la valeur assurée, la réassurance prend elle comme référence le total des primes de la cédante, appelée \textbf{assiette}.
On note $a_{j}$ désigne l'assiette de prime à l'année $j$ et $a_{s+1}^{*}$ désigne l'assiette estimée de l'année à venir et où $s$ désigne le nombre d'années d'historique.
\end{f}



\begin{f}[Le modèle Poisson-Pareto]
[Prime de l'XS ou de l'XL]
Soit $p$ et $f$ respectivement la portée et la priorité (franchise) de l'XS, avec la limite $l=p+f$ ($p$ XS $f$).

La prime XS correspond à :
$$
\mathbb{E}\left[S_N\right]=\mathbb{E}\left[\sum_{i=1}^{N} Y_{i}\right]=\mathbb{E}[N]\times \mathbb{E}[Y]
$$
où
$$
\mathbb{E}[Y]= l \mathbb{P}[X>l] - f \times \mathbb{P}[X\geq f] + \mathbb{E}[X\mid f\geq x\geq l]
$$


Si $l=\infty$ et $\alpha \neq 1$ :
$$
\mathbb{E}[S_N] = \lambda \frac{x_\mathrm{m}^{\alpha}}{\alpha -1}f^{1-\alpha} 
$$

si  $l=\infty$ et $\alpha = 1$ il n'y a pas de solution.

Si $l<\infty$ et $\alpha \neq 1$ :
$$
\mathbb{E}[S_N] = \lambda \frac{x_\mathrm{m}^{\alpha}}{\alpha -1}\left( f^{1-\alpha} -l^{1-\alpha} \right)
$$

Si $l<\infty$ et $\alpha = 1$ :
$$
\mathbb{E}[S_N] = \lambda x_\mathrm{m} \ln \left(  \frac{1}{f}\right)
$$
\end{f}



\begin{f}
[Le modèle Poisson-LogNormal]

Si $X$ suit une $\mathcal{L}\mathcal{N}orm(x_\mathrm{m}, \mu, \sigma)$ alors $X-x_\mathrm{m}$ suit une $\mathcal{L}\mathcal{N}orm(\mu, \sigma)$
Il vient :
$$
\mathbb{P}[X>f]=\mathbb{P}[X-x_\mathrm{m}>f-x_\mathrm{m}]=1-\Phi\left(\frac{\ln(f-x_\mathrm{m})-\mu}{\sigma}\right)
$$
\begin{align*}
	\mathbb{E}[X\mid X>f] \\ 
	= & \mathbb{E}\left[ X-x_\mathrm{m}\mid X-x_\mathrm{m}>f-x_\mathrm{m} \right]+x_\mathrm{m} \mathbb{P}[X>f]\\
	= & e ^{m+\sigma^2/2} \left[1-\Phi\left(\frac{\ln(f-x_\mathrm{m})-(\mu+\sigma^2)}{\sigma}\right) \right]\\
	& +x_\mathrm{m} \left( 1-\Phi\left(\frac{\ln(f-x_\mathrm{m})-\mu}{\sigma}\right) \right)
\end{align*}
Avec franchise et sans limite :
\begin{align*}
	\mathbb{E}[S_N] \\ 
	= & \lambda \left(\mathbb{E}\left[ X-x_\mathrm{m}\mid X-x_\mathrm{m}>f-x_\mathrm{m} \right]+x_\mathrm{m} \mathbb{P}[X>f]-f\mathbb{P}[X>f]\right)\\
	= & \lambda \left(  e ^{m+\sigma^2/2} \left[1-\Phi\left(\frac{\ln(f-x_\mathrm{m})-(\mu+\sigma^2)}{\sigma}\right) \right]\right)\\
	& +\lambda (x_\mathrm{m}-l) \left( 1-\Phi\left(\frac{\ln(f-x_\mathrm{m})-\mu}{\sigma}\right) \right)
\end{align*}

\end{f}