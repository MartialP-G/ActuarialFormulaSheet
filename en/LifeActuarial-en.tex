% !TeX root = ActuarialFormSheet_MBFA-en.tex
% !TeX spellcheck = fr_FR

\begin{f}[Notations sur les tables de survie]

L'âge $x$, $y$, $z$...    

$l_x$ est le nombre de personnes vivantes, par rapport à une cohorte initiale, à l'âge $x$

$\omega$ est l'âge limite des tables de mortalité.

$d_x=l_x-l_{x+1}$ est le nombre de personnes qui meurent entre l'âge $x$ et l'âge $x+1$.

$q_x$ est la probabilité de décès entre les âges de $x$ et l'âge $x+1$.
$$
\,q_x = d_x / l_x 
$$

$p_x$ est la probabilité que l'individu agé de $x$ survive à l'âge $x+1$.
$$
\,p_x+q_x=1 
$$

De même, 
$\,_nd_x = d_x + d_{x+1} + \cdots + d_{x+n-1} = l_x - l_{x+n}$ montre le nombre de personnes qui meurent entre l'âge $x$ et l'âge $x+n$.

$\,_nq_x$ est la probabilité de décès entre les âges de $x$ et l'âge $x+n$.

$$
\,_nq_x = {}_nd_x / l_x
$$
$\,_np_x$ est la probabilité d'une personne d'âge $x$ de survivre à l'âge $x+n$.
$$
\,_np_x = l_{x+n} / l_x 
$$


${}_{m|}q_{x}$, la probabilité que l'individu d'âge $x$ meurt dans la ${m+1}^e$ année.
$${}_{m|}q_{x}=\frac{d_{x+m}}{l_x}=\frac{l_{x+m}-l_{x+m+1}}{l_x}$$

$\,e_x$  est l'espérance de vie pour une personne encore en vie à l'âge $x$. 
C'est le nombre espéré d'anniversaires à vivre.
$$
\,e_x = \sum_{t=1}^{\infty} \ _tp_x 
$$
\end{f}
\hrule

\begin{f}[Coefficient ou commutations]


Ces coefficients ou commutations établies par des fonctions actuarielles qui dépendent d'une table de mortalité et d'un taux $i$ ($v=1/(1+i)$) pour établir la table actuarielle. 
$$
D_x=l_x .v^x
$$
peut être vu "comme" le nombre de survivants actualisés. Les sommes 

$$
N_x=\sum_{k\geq 0} D_{x+k}=\sum_{k= 0}^{\omega-x} D_{x+k}
$$

$$
S_x=\sum_{k\geq 0} N_{x+k}=\sum_{k\geq 0}(k+1). D_{x+k}
$$
seront utilisés pour simplifier les calculs.
De même
$$
C_x = d_x v^{ x+1} 
$$
peut être vu "comme" le nombre de décès  actualisés à l'âge $x$. Les sommes

$$
M_x=\sum_{k= 0}^{\omega-x} C_{x+k}
$$
$$
R_x=\sum_{k= 0}^{\omega-x} M_{x+k}
$$
seront utilisés pour simplifier les calculs.

Les  coefficients $D_x$ $N_x$ et $S_x$ seront utilisés pour les calculs sur les opérations en cas de vie et $C_x$ $M_x$ et $R_x$  pour les opérations en cas de décès.

\end{f} 
\hrule

\begin{f}[Les annuités viagères ou rentes]


\medskip
	

\begin{tikzpicture}[scale=0.75]
    % Draw the x-axis and y-axis.
    \def\w{11}
    \def\n{6}
    \node[left] at (-.5,0) {${}_{}a_x$};
    
    \begin{scope}[shift={(3.75,.25)}]
        \draw[color=OrangeProfondIRA,scale=0.2,fill=OrangeProfondIRA] \Cerceuil;
    \end{scope}
    \foreach \y in  {0,...,3} {
        \draw (\y,0) -- (\y,-0.1);
        \ifthenelse{\y>0 }{	\node[below] at (\y,-0.1) {\tiny $ \scriptstyle x+\y$};
            \draw[ line width=1, color=OrangeProfondIRA, arrows={-Stealth[length=4, inset=0]}] (\y,0) -- (\y,1);}{
            \node[below] at (\y,-0.1) {\tiny $ \scriptstyle x$};}
    }
    \draw (\n,0) -- (\n,-0.1);
    \node[below] at (\n,-0.1) {\tiny $\scriptstyle  x+n$};
    \foreach \y in  {1,...,3} {
        \draw (\y+\n,0) -- (\y+\n,-0.1);
        \node[below] at (\y+\n,-0.1) {\tiny $\scriptstyle x+n+\y$};
        %		\draw[ line width=1, color=OrangeProfondIRA, arrows={-Stealth[length=4, inset=0]}] (\y+\n,0) -- (\y+\n,1);
    }
    \draw[arrows={-Stealth[length=4, inset=0]}, line width=1] (-.5,0) -- (\w,0);
\end{tikzpicture}

\begin{tikzpicture}[scale=0.75]
    % Draw the x-axis and y-axis.
    \def\w{11}
    \def\n{6}
    \node[left] at (-.5,0) {${}_{}a_x$};
    
    \begin{scope}[shift={(\n+.5+3,.25)}]
        \draw[color=OrangeProfondIRA,scale=0.2,fill=OrangeProfondIRA] \Cerceuil;
    \end{scope}
    \foreach \y in  {0,...,3} {
        \draw (\y,0) -- (\y,-0.1);
        \ifthenelse{\y>0 }{	\node[below] at (\y,-0.1) {\tiny $ \scriptstyle x+\y$};
            \draw[ line width=1, color=OrangeProfondIRA, arrows={-Stealth[length=4, inset=0]}] (\y,0) -- (\y,1);}{
            \node[below] at (\y,-0.1) {\tiny $ \scriptstyle x$};}
    }
    \foreach \y in  {0,...,4} {
        \draw (\y+\n,0) -- (\y+\n,-0.1);
        \ifthenelse{\y>0 }{\node[below] at (\y+\n,-0.1) {\tiny $\scriptstyle x+n+\y$};}{
            \node[below] at (\y+\n,-0.1) {\tiny $\scriptstyle x+n$};}
        \ifthenelse{\y<4 }{	\draw[ line width=1, color=OrangeProfondIRA, arrows={-Stealth[length=4, inset=0]}] (\y+\n,0) -- (\y+\n,1);}
    }
    \draw[arrows={-Stealth[length=4, inset=0]}, line width=1] (-.5,0) -- (\w,0);
\end{tikzpicture}
\begin{tikzpicture}[scale=0.75]
    % Draw the x-axis and y-axis.
    \def\w{11}
    \def\n{6}
    \node[left] at (-.5,0) {$\ddot{a}_x$};
    
    \begin{scope}[shift={(3.75,.25)}]
        \draw[color=OrangeProfondIRA,scale=0.2,fill=OrangeProfondIRA] \Cerceuil;
    \end{scope}
    \foreach \y in  {0,...,3} {
        \draw (\y,0) -- (\y,-0.1);
        \ifthenelse{\y>0 }{	\node[below] at (\y,-0.1) {\tiny $ \scriptstyle x+\y$};}{
            \node[below] at (\y,-0.1) {\tiny $ \scriptstyle x$};}
        \draw[ line width=1, color=OrangeProfondIRA, arrows={-Stealth[length=4, inset=0]}] (\y,0) -- (\y,1);
    }
    \draw (\n,0) -- (\n,-0.1);
    \node[below] at (\n,-0.1) {\tiny $\scriptstyle  x+n$};
    \foreach \y in  {1,...,4} {
        \draw (\y+\n,0) -- (\y+\n,-0.1);
        \node[below] at (\y+\n,-0.1) {\tiny $\scriptstyle x+n+\y$};
        %		\draw[ line width=1, color=OrangeProfondIRA, arrows={-Stealth[length=4, inset=0]}] (\y+\n,0) -- (\y+\n,1);
    }
    \draw[arrows={-Stealth[length=4, inset=0]}, line width=1] (-.5,0) -- (\w,0);
\end{tikzpicture}
\begin{tikzpicture}[scale=0.75]
    % Draw the x-axis and y-axis.
    \def\w{11}
    \def\n{6}
    \node[left] at (-.5,0) {$\ddot{a}_x$};
    
    \begin{scope}[shift={(\n+.5+3,.25)}]
        \draw[color=OrangeProfondIRA,scale=0.2,fill=OrangeProfondIRA] \Cerceuil;
    \end{scope}
    \foreach \y in  {0,...,3} {
        \draw (\y,0) -- (\y,-0.1);
        \ifthenelse{\y>0 }{	\node[below] at (\y,-0.1) {\tiny $ \scriptstyle x+\y$};}{
            \node[below] at (\y,-0.1) {\tiny $ \scriptstyle x$};}
        \draw[ line width=1, color=OrangeProfondIRA, arrows={-Stealth[length=4, inset=0]}] (\y,0) -- (\y,1);
    }
    \foreach \y in  {0,...,4} {
        \draw (\y+\n,0) -- (\y+\n,-0.1);
        \ifthenelse{\y>0 }{\node[below] at (\y+\n,-0.1) {\tiny $\scriptstyle x+n+\y$};}{
            \node[below] at (\y+\n,-0.1) {\tiny $\scriptstyle x+n$};}
        \ifthenelse{\y<4 }{	\draw[ line width=1, color=OrangeProfondIRA, arrows={-Stealth[length=4, inset=0]}] (\y+\n,0) -- (\y+\n,1);}
    }
    \draw[arrows={-Stealth[length=4, inset=0]}, line width=1] (-.5,0) -- (\w,0);
\end{tikzpicture}	
$$a_x %=\frac{N_{x+1}}{D_x} 
%=\sum_{k=1}^{\infty}{}_{k|}q_{x} \ddot{a}_{\lcroof{k+1}}
=\sum_{k=1}^{\infty}{}_{k}p_{x} v^{k}=\ddot{a}_x -1
=\frac{N_{x+1}}{D_{x}}
$$

$$\ddot{a}_x 
%\frac{N_x}{D_x}=
%=\sum_{k=0}^{\infty}{}_{k|}q_{x} \ddot{a}_{\lcroof{k+1}}
=\sum_{k=0}^{\infty}{}_{k}p_{x} v^{k}
=	\frac{N_{x}}{D_{x}} 
$$

Si la périodicité correspond à $m$ période par an:
$$\ddot{a}_{x}^{(m)} 
=\sum_{k=0}^{\infty}\frac{1}{m}{}_{\frac{k}{m}}p_{x} v^{\frac{k}{m}}\approx\ddot{a}_x -\frac{m-1}{2m}
$$
De même, s'il paie $1/m$ en début des $m$ périodes
$$a_{x}^{(m)}\approx a_x +\frac{m-1}{2m}
$$


\textbf{Les annuités viagères temporaires} (\engl{Whole life annuity guaranteed for n years})
$$
a_{x:\lcroof{n}} =
\sum_{k=1}^{n}{}_{k}p_{x} v^{k}
=\frac{N_{x+1}-N_{x+n+1}}{D_{x}}
$$
	
$$
\ddot{a}_{x:\lcroof{n}} =%\frac{N_x - N_{x+n}}{D_x}=
\sum_{k=0}^{n-1}{}_{k}p_{x} v^{k}
=\frac{N_{x}-N_{x+n}}{D_{x}}
$$


\textbf{Les annuités viagères différées}
${}_{m|}a_{x}$ (\engl{Deferred life annuity}) représentent les rentes sur l'individu d'âge $x$ différée $m$ années. Le premier paiement intervient dans $m+1$ ans en cas de vie.

	%	\includegraphics[width=1\linewidth]{../../LifeActuarial/Graph/RenteViagereDifferee}
\begin{tikzpicture}[scale=0.75]
    % Draw the x-axis and y-axis.
    \def\w{11}
    \def\m{6}
    \node[left] at (-.5,0) {${}_{m|}a_x$};
    
    \begin{scope}[shift={(4.25,.25)}]
        \draw[color=OrangeProfondIRA,scale=0.2,fill=OrangeProfondIRA] \Cerceuil;
    \end{scope}
    \draw[dashed, color=BleuProfondIRA,arrows={Stealth[length=4, inset=0]-Stealth[length=4, inset=0]},  line width=1] (0,.3) -- (\m,.3) node [pos=0.5, above] {$m$};		
    \draw (0,0) -- (0,-0.1);
    \node[below] at (0,-0.1) {\tiny $x$};
    \foreach \y in  {1,...,3} {
        \draw (\y,0) -- (\y,-0.1);
        \node[below] at (\y,-0.1) {\tiny $ \scriptstyle x+\y$};
    }
    \draw (\m,0) -- (\m,-0.1);
    \node[below] at (\m,-0.1) {\tiny $\scriptstyle  x+m$};
    \foreach \y in  {1,...,3} {
        \draw (\y+\m,0) -- (\y+\m,-0.1);
        \node[below] at (\y+\m,-0.1) {\tiny $\scriptstyle x+m+\y$};
        %		\draw[ line width=1, color=OrangeProfondIRA, arrows={-Stealth[length=4, inset=0]}] (\y+\m,0) -- (\y+\m,1);
        \draw[arrows={-Stealth[length=4, inset=0]}, line width=1] (-.5,0) -- (\w,0);
    }
\end{tikzpicture}

\begin{tikzpicture}[scale=0.75]
    % Draw the x-axis and y-axis.
    \def\w{11}
    \def\m{6}
    \node[left] at (-.5,0) {${}_{m|}a_x$};
    
    \begin{scope}[shift={(\m+.5+3,.25)}]
        \draw[color=OrangeProfondIRA,scale=0.2,fill=OrangeProfondIRA] \Cerceuil;
    \end{scope}
    \draw[dashed, color=BleuProfondIRA,arrows={Stealth[length=4, inset=0]-Stealth[length=4, inset=0]},  line width=1] (0,.3) -- (\m,.3) node [pos=0.5, above] {$m$};		
    \draw (0,0) -- (0,-0.1);
    \node[below] at (0,-0.1) {\tiny $x$};
    \foreach \y in  {1,...,3} {
        \draw (\y,0) -- (\y,-0.1);
        \node[below] at (\y,-0.1) {\tiny $ \scriptstyle x+\y$};
    }
    \draw (\m,0) -- (\m,-0.1);
    \node[below] at (\m,-0.1) {\tiny $\scriptstyle  x+m$};
    \foreach \y in  {1,...,3} {
        \draw (\y+\m,0) -- (\y+\m,-0.1);
        \node[below] at (\y+\m,-0.1) {\tiny $\scriptstyle x+m+\y$};
        \draw[ line width=1, color=OrangeProfondIRA, arrows={-Stealth[length=4, inset=0]}] (\y+\m,0) -- (\y+\m,1);
    }
    \draw[arrows={-Stealth[length=4, inset=0]}, line width=1] (-.5,0) -- (\w,0);
\end{tikzpicture}


  
\end{f}
\hrule

\begin{f}[Capitaux décès ou survie]

% !TeX spellcheck = fr_FR
\textbf{Les capitaux décès}(\engl{Whole life insurance } \engl{noted} ${SP}_{x}$ or ${A}_{x}$)

$A_x$ indique une prestation au décès à la fin de l'année de la mort (montant de 1), quelque que soit la date de survenance, pour un individu assuré à l'âge $x$ lors de la souscription.

$A_{x:\lcroof{n}}$ désigne un capital versé au décès s'il survient et  au plus tard dans $n$ années (\engl{Endowment}).

$\lcterm{A}{x}{n}$ % ou $\termins{x}{n}$ 
désigne un capital décès versé si $x$ décède dans les $n$ années à venir (\engl{Term insurance}).


$A_x^{(12)}$ indique une prestation   payable à la fin du mois du décès.

$\overline{A}_x$  indique une prestation  payée à la date du décès.
\begin{tikzpicture}[scale=0.75]
% Draw the x-axis and y-axis.
\def\w{11}
\def\n{7}
\node[left] at (-.5,0) {$A_x$};

\begin{scope}[shift={(3.75,.25)}]
    \draw[color=OrangeProfondIRA,scale=0.2,fill=OrangeProfondIRA] \Cerceuil;
\end{scope}
\draw[ line width=1, color=OrangeProfondIRA, arrows={-Stealth[length=4, inset=0]}] (4,0) -- (4,1);
\foreach \y in  {0,...,4} {
    \draw (\y,0) -- (\y,-0.1);
    \ifthenelse{\y>0 }{	\node[below] at (\y,-0.1) {\tiny $ \scriptstyle x+\y$};}{
        \node[below] at (\y,-0.1) {\tiny $ \scriptstyle x$};}
}
\draw (\n,0) -- (\n,-0.1);
\node[below] at (\n,-0.1) {\tiny $\scriptstyle  x+n$};
\foreach \y in  {1,...,3} {
    \draw (\y+\n,0) -- (\y+\n,-0.1);
    \node[below] at (\y+\n,-0.1) {\tiny $\scriptstyle x+n+\y$};
    %		\draw[ line width=1, color=OrangeProfondIRA, arrows={-Stealth[length=4, inset=0]}] (\y+\n,0) -- (\y+\n,1);
}
\draw[arrows={-Stealth[length=4, inset=0]}, line width=1] (-.5,0) -- (\w,0);
\end{tikzpicture}
\begin{tikzpicture}[scale=0.75]
% Draw the x-axis and y-axis.
\def\w{11}
\def\n{7}
\node[left] at (-.5,0) {$A_x$};

\begin{scope}[shift={(\n+.5+2,.25)}]
    \draw[color=OrangeProfondIRA,scale=0.2,fill=OrangeProfondIRA] \Cerceuil;
\end{scope}
\draw[ line width=1, color=OrangeProfondIRA, arrows={-Stealth[length=4, inset=0]}] (\n+3,0) -- (\n+3,1);
\foreach \y in  {0,...,4} {
    \draw (\y,0) -- (\y,-0.1);
    \ifthenelse{\y>0 }{	\node[below] at (\y,-0.1) {\tiny $ \scriptstyle x+\y$};}{
        \node[below] at (\y,-0.1) {\tiny $ \scriptstyle x$};}
}
\draw (\n,0) -- (\n,-0.1);
\node[below] at (\n,-0.1) {\tiny $\scriptstyle  x+n$};
\foreach \y in  {1,...,3} {
    \draw (\y+\n,0) -- (\y+\n,-0.1);
    \node[below] at (\y+\n,-0.1) {\tiny $\scriptstyle x+n+\y$};
    %		\draw[ line width=1, color=OrangeProfondIRA, arrows={-Stealth[length=4, inset=0]}] (\y+\n,0) -- (\y+\n,1);
}
\draw[arrows={-Stealth[length=4, inset=0]}, line width=1] (-.5,0) -- (\w,0);
\end{tikzpicture}
\begin{tikzpicture}[scale=0.75]
% Draw the x-axis and y-axis.
\def\w{11}
\def\n{7}
\node[left] at (-.5,0) {$\overline{A}_x$};

\begin{scope}[shift={(3.75,.25)}]
    \draw[color=OrangeProfondIRA,scale=0.2,fill=OrangeProfondIRA] \Cerceuil;
\end{scope}
\draw[ line width=1, color=black, arrows={-Stealth[length=4, inset=0]}] (3.75,0) -- (3.75,1);
\foreach \y in  {0,...,4} {
    \draw (\y,0) -- (\y,-0.1);
    \ifthenelse{\y>0 }{	\node[below] at (\y,-0.1) {\tiny $ \scriptstyle x+\y$};}{
        \node[below] at (\y,-0.1) {\tiny $ \scriptstyle x$};}
}
\draw (\n,0) -- (\n,-0.1);
\node[below] at (\n,-0.1) {\tiny $\scriptstyle  x+n$};
\foreach \y in  {1,...,3} {
    \draw (\y+\n,0) -- (\y+\n,-0.1);
    \node[below] at (\y+\n,-0.1) {\tiny $\scriptstyle x+n+\y$};
    %		\draw[ line width=1, color=OrangeProfondIRA, arrows={-Stealth[length=4, inset=0]}] (\y+\n,0) -- (\y+\n,1);
}
\draw[arrows={-Stealth[length=4, inset=0]}, line width=1] (-.5,0) -- (\w,0);
\end{tikzpicture}
\begin{tikzpicture}[scale=0.75]
% Draw the x-axis and y-axis.
\def\w{11}
\def\n{7}
\node[left] at (-.5,0) {$\overline{A}_x$};

\begin{scope}[shift={(\n+.5+2,.25)}]
    \draw[color=OrangeProfondIRA,scale=0.2,fill=OrangeProfondIRA] \Cerceuil;
\end{scope}
\draw[ line width=1, color=black, arrows={-Stealth[length=4, inset=0]}] (\n+.5+2,0) -- (\n+.5+2,1);
\foreach \y in  {0,...,4} {
    \draw (\y,0) -- (\y,-0.1);
    \ifthenelse{\y>0 }{	\node[below] at (\y,-0.1) {\tiny $ \scriptstyle x+\y$};}{
        \node[below] at (\y,-0.1) {\tiny $ \scriptstyle x$};}
}
\draw (\n,0) -- (\n,-0.1);
\node[below] at (\n,-0.1) {\tiny $\scriptstyle  x+n$};
\foreach \y in  {1,...,3} {
    \draw (\y+\n,0) -- (\y+\n,-0.1);
    \node[below] at (\y+\n,-0.1) {\tiny $\scriptstyle x+n+\y$};
    %		\draw[ line width=1, color=OrangeProfondIRA, arrows={-Stealth[length=4, inset=0]}] (\y+\n,0) -- (\y+\n,1);
}
\draw[arrows={-Stealth[length=4, inset=0]}, line width=1] (-.5,0) -- (\w,0);
\end{tikzpicture}

\medskip

Capital décès (\engl{Whole life})
$$A_{x}=\sum_{k=0}^{\infty} {}_{k|}q_x\ \nu^{k+1}=\frac{M_x}{D_x}$$
	
$$\termins{x}{n}=\sum_{k=0}^{n-1} {}_{k|}q_x\ \nu^{k+1}=\frac{M_x-M_{x+n}}{D_x}
$$



\medskip    
\textbf{Capital différé (\engl{Pure Endowment}, capital unique en cas de survie)} noté $\lcend{A}{x}{n}$ %$\pureend{x}{n}$ 
ou ${}_n E_x$.

\begin{tikzpicture}[scale=0.75]
% Draw the x-axis and y-axis.
\def\w{11}
\def\n{7}
\node[left] at (-.5,0) {${}_n E_x$};

\begin{scope}[shift={(3.75,.25)}]
    \draw[color=OrangeProfondIRA,scale=0.2,fill=OrangeProfondIRA] \Cerceuil;
\end{scope}
%		\draw[ line width=1, color=OrangeProfondIRA, arrows={-Stealth[length=4, inset=0]}] (4,0) -- (4,1);
\foreach \y in  {0,...,4} {
    \draw (\y,0) -- (\y,-0.1);
    \ifthenelse{\y>0 }{	\node[below] at (\y,-0.1) {\tiny $ \scriptstyle x+\y$};}{
        \node[below] at (\y,-0.1) {\tiny $ \scriptstyle x$};}
}
\draw (\n,0) -- (\n,-0.1);
\node[below] at (\n,-0.1) {\tiny $\scriptstyle  x+n$};
\foreach \y in  {1,...,3} {
    \draw (\y+\n,0) -- (\y+\n,-0.1);
    \node[below] at (\y+\n,-0.1) {\tiny $\scriptstyle x+n+\y$};
    %		\draw[ line width=1, color=OrangeProfondIRA, arrows={-Stealth[length=4, inset=0]}] (\y+\n,0) -- (\y+\n,1);
}
\draw[arrows={-Stealth[length=4, inset=0]}, line width=1] (-.5,0) -- (\w,0);
\end{tikzpicture}
\begin{tikzpicture}[scale=0.75]
% Draw the x-axis and y-axis.
\def\w{11}
\def\n{7}
\node[left] at (-.5,0) {${}_n E_x$};

\begin{scope}[shift={(\n+.5+2,.25)}]
    \draw[color=OrangeProfondIRA,scale=0.2,fill=OrangeProfondIRA] \Cerceuil;
\end{scope}
\draw[ line width=1, color=OrangeProfondIRA, arrows={-Stealth[length=4, inset=0]}] (\n,0) -- (\n,1);
\foreach \y in  {0,...,4} {
    \draw (\y,0) -- (\y,-0.1);
    \ifthenelse{\y>0 }{	\node[below] at (\y,-0.1) {\tiny $ \scriptstyle x+\y$};}{
        \node[below] at (\y,-0.1) {\tiny $ \scriptstyle x$};}
}
\draw (\n,0) -- (\n,-0.1);
\node[below] at (\n,-0.1) {\tiny $\scriptstyle  x+n$};
\foreach \y in  {1,...,3} {
    \draw (\y+\n,0) -- (\y+\n,-0.1);
    \node[below] at (\y+\n,-0.1) {\tiny $\scriptstyle x+n+\y$};
    %		\draw[ line width=1, color=OrangeProfondIRA, arrows={-Stealth[length=4, inset=0]}] (\y+\n,0) -- (\y+\n,1);
}
\draw[arrows={-Stealth[length=4, inset=0]}, line width=1] (-.5,0) -- (\w,0);
\end{tikzpicture}
	$$
	={}_n E_x={}_n p_x .v^n=\frac{l_{x+n}}{l_x} . v^n =\frac{D_{x+n}}{D_{x}}
	$$
Capital décès avec versement du capital en cas de survie (\engl{Endowment})
$$A_{x:\lcroof{n}}=\termins{x}{n}+\lcend{A}{x}{n}$$

\end{f}
\hrule

\begin{f}[L'assurance vie sur plusieurs individus] 

$a_{xyz}$ est une rente annuelle, payée dès la fin de la première année et tant que vivent $(x)$, $(y)$ et $(z)$.

$a_{\overline{xyz}}$ est une rente annuelle, payée dès la fin de la première année et tant que vivent $(x)$, $(y)$ ou $(z)$.

$$
a_{\overline{xy}}=a_{y}+a_{x}-a_{xy}
$$

$A_{xyz}$ est une assurance qui intervient à la fin de l'année du premier décès de $(x)$, $(y)$ et $(z)$.

La barre verticale indique la conditionnalité :

$a_{x|y}$ est une rente de réversion qui profite à $(x)$ après le décès de $(y)$.

$A_{x|yz}$ est une assurance au premier décès de  $(y)$ et $(z)$.	


\begin{tikzpicture}[scale=0.85]
    % Draw the x-axis and y-axis.
    \def\w{11}
    \def\n{6}
    \node[left] at (-.5,0) {$a_{{\color{OrangeProfondIRA}x}|{\color{BleuProfondIRA}y}}$};
    
    \begin{scope}[shift={(9.75,.25)}]
        \draw[color=OrangeProfondIRA,scale=0.2,fill=OrangeProfondIRA] \Cerceuil;
    \end{scope}
    \begin{scope}[shift={(7.55,.25)}]
        \draw[color=BleuProfondIRA,scale=0.2,fill=BleuProfondIRA] \Cerceuil;
    \end{scope}
    \draw[ line width=1, color=OrangeProfondIRA, arrows={-Stealth[length=4, inset=0]}] (8,0) -- (8,1);
    \draw[ line width=1, color=OrangeProfondIRA, arrows={-Stealth[length=4, inset=0]}] (9,0) -- (9,1);
    \foreach \y in  {0,...,3} {
        \draw (\y,0) -- (\y,-0.1);
        \ifthenelse{\y>0 }{	\node[below] at (\y,-0.1) {\tiny $ \scriptstyle x+\y$};}{
            \node[below] at (\y,-0.1) {\tiny $ \scriptstyle x$};}
    }
    \draw (\n,0) -- (\n,-0.1);
    \node[below] at (\n,-0.1) {\tiny $\scriptstyle  x+n$};
    \foreach \y in  {1,...,3} {
        \draw (\y+\n,0) -- (\y+\n,-0.1);
        \node[below] at (\y+\n,-0.1) {\tiny $\scriptstyle x+n+\y$};
    }
    \draw[arrows={-Stealth[length=4, inset=0]}, line width=1] (-.5,0) -- (\w,0);
\end{tikzpicture}
\begin{tikzpicture}[scale=0.85]
    % Draw the x-axis and y-axis.
    \def\w{11}
    \def\n{6}
    \node[left] at (-.5,0) {$a_{{\color{OrangeProfondIRA}x}|{\color{BleuProfondIRA}y}}$};
    
    \begin{scope}[shift={(8.4,.25)}]
        \draw[color=OrangeProfondIRA,scale=0.2,fill=OrangeProfondIRA] \Cerceuil;
    \end{scope}
    \begin{scope}[shift={(2.75,.25)}]
        \draw[color=BleuProfondIRA,scale=0.2,fill=BleuProfondIRA] \Cerceuil;
    \end{scope}
    \foreach \y in  {3,...,8} {
        \draw[ line width=1, color=OrangeProfondIRA, arrows={-Stealth[length=4, inset=0]}] (\y,0) -- (\y,1);
    }
    \foreach \y in  {0,...,3} {
        \draw (\y,0) -- (\y,-0.1);
        \ifthenelse{\y>0 }{	\node[below] at (\y,-0.1) {\tiny $ \scriptstyle x+\y$};}{
            \node[below] at (\y,-0.1) {\tiny $ \scriptstyle x$};}
    }
    \foreach \y in  {0,...,4} {
        \draw (\y+\n,0) -- (\y+\n,-0.1);
        \ifthenelse{\y>0 }{\node[below] at (\y+\n,-0.1) {\tiny $\scriptstyle x+n+\y$};}{
            \node[below] at (\y+\n,-0.1) {\tiny $\scriptstyle x+n$};}
    }
    \draw[arrows={-Stealth[length=4, inset=0]}, line width=1] (-.5,0) -- (\w,0);
\end{tikzpicture}
\begin{tikzpicture}[scale=0.85]
    % Draw the x-axis and y-axis.
    \def\w{11}
    \def\n{6}
    \node[left] at (-.5,0) {$a_{{\color{OrangeProfondIRA}x}|{\color{BleuProfondIRA}y}}$};
    
    \begin{scope}[shift={(3.75,.25)}]
        \draw[color=OrangeProfondIRA,scale=0.2,fill=OrangeProfondIRA] \Cerceuil;
    \end{scope}
    \begin{scope}[shift={(7.55,.25)}]
        \draw[color=BleuProfondIRA,scale=0.2,fill=BleuProfondIRA] \Cerceuil;
    \end{scope}
    \foreach \y in  {0,...,3} {
        \draw (\y,0) -- (\y,-0.1);
        \ifthenelse{\y>0 }{	\node[below] at (\y,-0.1) {\tiny $ \scriptstyle x+\y$};}{
            \node[below] at (\y,-0.1) {\tiny $ \scriptstyle x$};}
    }
    \draw (\n,0) -- (\n,-0.1);
    \node[below] at (\n,-0.1) {\tiny $\scriptstyle  x+n$};
    \foreach \y in  {1,...,4} {
        \draw (\y+\n,0) -- (\y+\n,-0.1);
        \node[below] at (\y+\n,-0.1) {\tiny $\scriptstyle x+n+\y$};
        %		\draw[ line width=1, color=OrangeProfondIRA, arrows={-Stealth[length=4, inset=0]}] (\y+\n,0) -- (\y+\n,1);
    }
    \draw[arrows={-Stealth[length=4, inset=0]}, line width=1] (-.5,0) -- (\w,0);
\end{tikzpicture}
\end{f}

\begin{f}[Schémas simplifié de tarification des primes périodiques et provisions]
	
\tikzstyle{startstop} = [rectangle, rounded corners, text width=4cm, minimum height=1cm,text centered, draw=black, fill=BleuProfondIRA!40]
\tikzstyle{process} = [rectangle, minimum width=4cm, minimum height=1cm, text centered, text width=3cm, draw=black, fill=OrangeProfondIRA!40]
\tikzstyle{arrow} = [thick,->,>=stealth]
\ \medskip

\centering
\begin{tikzpicture}[node distance=35mm and 100mm]
	\node (pu) [startstop] {Prime Unique de la prestation\\  notée $PU(.)$ ou $PV_0(.)$};
	\node  [below left of=pu, text width=55mm, node distance=18mm, xshift=12mm] {Le point est $A$ ou $a$ ou $E$ ou ...\\
	avec $x,k,n,m,K...$};
	\node (puu) [startstop, right of=pu, node distance=50mm] {Valeur d'une prime périodique unitaire\\  notée $PPU_0$ ou $VP_P$};
	\node  [below of=puu, text width=3cm, node distance=12mm] { $PPU=_{m'|}\ddot{a}_{x:\lcroof{n'}}^{(k')} $};
	\node (ppp) [startstop, below of=pu] {Prime Pure Périodique\\ $PPP$};
	\node  [below right of=ppp, text width=3cm, node distance=25mm, yshift=6mm] {$\displaystyle PPP=\frac{PU(.)}{PPU_0}$};
	\node (ppc) [startstop, right of=ppp, node distance=50mm] {Prime Périodique Chargée $PPC$};
	\node  [below of=ppc, node distance=12mm] {$\displaystyle PPC=\frac{PPP(1+\lambda)}{1-com}$};
	\node (Prov) [startstop, below of=ppp] {Provision mathématique $t^e$ année\\  notée $PM_t$ ou $V_t$ };
	\node (provt) [below right of=Prov,text width=10cm, node distance=30mm] {$V_t=PV_t(.)-PPP\times PPU_t$ \\
	$PV_t(.)$ recalculée avec  $x+t,k,n,m-t,K...\ si\ t\geq m$ \\
	ou $x+t,k,n-(t-m),m=0,K...\ si\ t>m$\\
	de même	$PPU_t=_{(m'-t)^+|}\ddot{a}_{x+t:\lcroof{n'-(t-m)^+}}^{(k')} $};
%	\node (pro2b) [startstop, right of=dec1, xshift=2cm] {Process 2b};
%	\node (stop) [startstop, below of=pro2b] {Stop};
	
	\draw [arrow] (pu) -- (ppp);
	\draw [arrow] (puu) -- (ppp);
	\draw [arrow] (ppp) -- (ppc);
	\draw [arrow] (ppp) -- node[anchor=east] {$t$} (Prov);
%	\draw [arrow] (dec1) -- node[anchor=south] {no} (pro2b);
%	\draw [arrow] (pro2b) |- (ppp);
%	\draw [arrow] (ppp) -- (stop);
	
\end{tikzpicture}
\end{f}