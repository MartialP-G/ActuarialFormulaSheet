% !TeX root = ActuarialFormSheet_MBFA-en.tex
% !TeX spellcheck = en_US

\begin{f}[Life Table Notations]

Age $x$, $y$, $z$...    

$l_x$ is the number of people alive, relative to an initial cohort, at age $x$ (or $y$, $z$...)

$\omega$ is the age limit of mortality tables.

$d_x=l_x-l_{x+1}$ is the number of people who die between the age $x$ and age $x+1$.

$q_x$ is the probability of death between the ages of $x$ et age $x+1$.
$$
\,q_x = d_x / l_x 
$$

$p_x$ is the probability that the individual aged $x$ survives age $x+1$.
$$
\,p_x+q_x=1 
$$

Likewise,
$\,_nd_x = d_x + d_{x+1} + \cdots + d_{x+n-1} = l_x - l_{x+n}$ shows the number of people who die between the age $x$ and age $x+n$.

$\,_nq_x$ is the probability of death between the ages of $x$ and age $x+n$.

$$
\,_nq_x = {}_nd_x / l_x
$$
$\,_np_x$ is the probability of a person of age $x$ to survive the age $x+n$.
$$
\,_np_x = l_{x+n} / l_x 
$$


${}_{m|}q_{x}$, the probability that the individual of age $x$ dies in the ${m+1}^e$ year.
$${}_{m|}q_{x}=\frac{d_{x+m}}{l_x}=\frac{l_{x+m}-l_{x+m+1}}{l_x}$$

$\,e_x$ is the life expectancy for a person still alive at the age $x$. 
This is the number of birthdays you hope to live.
$$
\,e_x = \sum_{t=1}^{\infty} \ _tp_x 
$$
\end{f}
\hrule

\begin{f}[Coefficient or commutations]


These coefficients or commutations established by actuarial functions which depend on a mortality table and a rate $i$ ($v=1/(1+i)$) to establish the actuarial table.
$$
D_x=l_x .v^x
$$
can be seen "as" the actualized number of survivors. The sums

$$
N_x=\sum_{k\geq 0} D_{x+k}=\sum_{k= 0}^{\omega-x} D_{x+k}
$$

$$
S_x=\sum_{k\geq 0} N_{x+k}=\sum_{k\geq 0}(k+1). D_{x+k}
$$
will be used to simplify the calculations.
Likewise
$$
C_x = d_x v^{ x+1} 
$$
can be seen "as" the number of deaths discounted to age $x$. The sums

$$
M_x=\sum_{k= 0}^{\omega-x} C_{x+k}
$$
$$
R_x=\sum_{k= 0}^{\omega-x} M_{x+k}
$$
will be used to simplify the calculations.

The coefficients $D_x$ $N_x$ and $S_x$ will be used for calculations on operations in case of life and $C_x$ $M_x$ and $R_x$ for operations in case of death.

\end{f} 
\hrule

\begin{f}[Life annuities or annuities]


\medskip
	

\begin{tikzpicture}[scale=0.75]
    % Draw the x-axis and y-axis.
    \def\w{11}
    \def\n{6}
    \node[left] at (-.5,0) {${}_{}a_x$};
    
    \begin{scope}[shift={(3.75,.25)}]
        \draw[color=OrangeProfondIRA,scale=0.2,fill=OrangeProfondIRA] \Cerceuil;
    \end{scope}
    \foreach \y in  {0,...,3} {
        \draw (\y,0) -- (\y,-0.1);
        \ifthenelse{\y>0 }{	\node[below] at (\y,-0.1) {\tiny $ \scriptstyle x+\y$};
            \draw[ line width=1, color=OrangeProfondIRA, arrows={-Stealth[length=4, inset=0]}] (\y,0) -- (\y,1);}{
            \node[below] at (\y,-0.1) {\tiny $ \scriptstyle x$};}
    }
    \draw (\n,0) -- (\n,-0.1);
    \node[below] at (\n,-0.1) {\tiny $\scriptstyle  x+n$};
    \foreach \y in  {1,...,3} {
        \draw (\y+\n,0) -- (\y+\n,-0.1);
        \node[below] at (\y+\n,-0.1) {\tiny $\scriptstyle x+n+\y$};
        %		\draw[ line width=1, color=OrangeProfondIRA, arrows={-Stealth[length=4, inset=0]}] (\y+\n,0) -- (\y+\n,1);
    }
    \draw[arrows={-Stealth[length=4, inset=0]}, line width=1] (-.5,0) -- (\w,0);
\end{tikzpicture}

\begin{tikzpicture}[scale=0.75]
    % Draw the x-axis and y-axis.
    \def\w{11}
    \def\n{6}
    \node[left] at (-.5,0) {${}_{}a_x$};
    
    \begin{scope}[shift={(\n+.5+3,.25)}]
        \draw[color=OrangeProfondIRA,scale=0.2,fill=OrangeProfondIRA] \Cerceuil;
    \end{scope}
    \foreach \y in  {0,...,3} {
        \draw (\y,0) -- (\y,-0.1);
        \ifthenelse{\y>0 }{	\node[below] at (\y,-0.1) {\tiny $ \scriptstyle x+\y$};
            \draw[ line width=1, color=OrangeProfondIRA, arrows={-Stealth[length=4, inset=0]}] (\y,0) -- (\y,1);}{
            \node[below] at (\y,-0.1) {\tiny $ \scriptstyle x$};}
    }
    \foreach \y in  {0,...,4} {
        \draw (\y+\n,0) -- (\y+\n,-0.1);
        \ifthenelse{\y>0 }{\node[below] at (\y+\n,-0.1) {\tiny $\scriptstyle x+n+\y$};}{
            \node[below] at (\y+\n,-0.1) {\tiny $\scriptstyle x+n$};}
        \ifthenelse{\y<4 }{	\draw[ line width=1, color=OrangeProfondIRA, arrows={-Stealth[length=4, inset=0]}] (\y+\n,0) -- (\y+\n,1);}
    }
    \draw[arrows={-Stealth[length=4, inset=0]}, line width=1] (-.5,0) -- (\w,0);
\end{tikzpicture}
\begin{tikzpicture}[scale=0.75]
    % Draw the x-axis and y-axis.
    \def\w{11}
    \def\n{6}
    \node[left] at (-.5,0) {$\ddot{a}_x$};
    
    \begin{scope}[shift={(3.75,.25)}]
        \draw[color=OrangeProfondIRA,scale=0.2,fill=OrangeProfondIRA] \Cerceuil;
    \end{scope}
    \foreach \y in  {0,...,3} {
        \draw (\y,0) -- (\y,-0.1);
        \ifthenelse{\y>0 }{	\node[below] at (\y,-0.1) {\tiny $ \scriptstyle x+\y$};}{
            \node[below] at (\y,-0.1) {\tiny $ \scriptstyle x$};}
        \draw[ line width=1, color=OrangeProfondIRA, arrows={-Stealth[length=4, inset=0]}] (\y,0) -- (\y,1);
    }
    \draw (\n,0) -- (\n,-0.1);
    \node[below] at (\n,-0.1) {\tiny $\scriptstyle  x+n$};
    \foreach \y in  {1,...,4} {
        \draw (\y+\n,0) -- (\y+\n,-0.1);
        \node[below] at (\y+\n,-0.1) {\tiny $\scriptstyle x+n+\y$};
        %		\draw[ line width=1, color=OrangeProfondIRA, arrows={-Stealth[length=4, inset=0]}] (\y+\n,0) -- (\y+\n,1);
    }
    \draw[arrows={-Stealth[length=4, inset=0]}, line width=1] (-.5,0) -- (\w,0);
\end{tikzpicture}
\begin{tikzpicture}[scale=0.75]
    % Draw the x-axis and y-axis.
    \def\w{11}
    \def\n{6}
    \node[left] at (-.5,0) {$\ddot{a}_x$};
    
    \begin{scope}[shift={(\n+.5+3,.25)}]
        \draw[color=OrangeProfondIRA,scale=0.2,fill=OrangeProfondIRA] \Cerceuil;
    \end{scope}
    \foreach \y in  {0,...,3} {
        \draw (\y,0) -- (\y,-0.1);
        \ifthenelse{\y>0 }{	\node[below] at (\y,-0.1) {\tiny $ \scriptstyle x+\y$};}{
            \node[below] at (\y,-0.1) {\tiny $ \scriptstyle x$};}
        \draw[ line width=1, color=OrangeProfondIRA, arrows={-Stealth[length=4, inset=0]}] (\y,0) -- (\y,1);
    }
    \foreach \y in  {0,...,4} {
        \draw (\y+\n,0) -- (\y+\n,-0.1);
        \ifthenelse{\y>0 }{\node[below] at (\y+\n,-0.1) {\tiny $\scriptstyle x+n+\y$};}{
            \node[below] at (\y+\n,-0.1) {\tiny $\scriptstyle x+n$};}
        \ifthenelse{\y<4 }{	\draw[ line width=1, color=OrangeProfondIRA, arrows={-Stealth[length=4, inset=0]}] (\y+\n,0) -- (\y+\n,1);}
    }
    \draw[arrows={-Stealth[length=4, inset=0]}, line width=1] (-.5,0) -- (\w,0);
\end{tikzpicture}	
$$a_x %=\frac{N_{x+1}}{D_x} 
%=\sum_{k=1}^{\infty}{}_{k|}q_{x} \ddot{a}_{\lcroof{k+1}}
=\sum_{k=1}^{\infty}{}_{k}p_{x} v^{k}=\ddot{a}_x -1
=\frac{N_{x+1}}{D_{x}}
$$

$$\ddot{a}_x 
%\frac{N_x}{D_x}=
%=\sum_{k=0}^{\infty}{}_{k|}q_{x} \ddot{a}_{\lcroof{k+1}}
=\sum_{k=0}^{\infty}{}_{k}p_{x} v^{k}
=	\frac{N_{x}}{D_{x}} 
$$

If the periodicity corresponds to $m$ periods per year:
$$\ddot{a}_{x}^{(m)} 
=\sum_{k=0}^{\infty}\frac{1}{m}{}_{\frac{k}{m}}p_{x} v^{\frac{k}{m}}\approx\ddot{a}_x -\frac{m-1}{2m}
$$
Similarly, if he pays $1/m$ at the start of the $m$ periods
$$a_{x}^{(m)}\approx a_x +\frac{m-1}{2m}
$$


\textbf{Temporary life annuities}. Whole life annuity guaranteed for n years
$$
a_{x:\lcroof{n}} =
\sum_{k=1}^{n}{}_{k}p_{x} v^{k}
=\frac{N_{x+1}-N_{x+n+1}}{D_{x}}
$$
	
$$
\ddot{a}_{x:\lcroof{n}} =%\frac{N_x - N_{x+n}}{D_x}=
\sum_{k=0}^{n-1}{}_{k}p_{x} v^{k}
=\frac{N_{x}-N_{x+n}}{D_{x}}
$$


\textbf{Deferred life annuities}
${}_{m|}a_{x}$ represent the annuities on the individual of age $x$ deferred $m$ years. The first payment occurs in $m+1$ years in the case of life.

	%	\includegraphics[width=1\linewidth]{../../LifeActuarial/Graph/RenteViagereDifferee}
\begin{tikzpicture}[scale=0.75]
    % Draw the x-axis and y-axis.
    \def\w{11}
    \def\m{6}
    \node[left] at (-.5,0) {${}_{m|}a_x$};
    
    \begin{scope}[shift={(4.25,.25)}]
        \draw[color=OrangeProfondIRA,scale=0.2,fill=OrangeProfondIRA] \Cerceuil;
    \end{scope}
    \draw[dashed, color=BleuProfondIRA,arrows={Stealth[length=4, inset=0]-Stealth[length=4, inset=0]},  line width=1] (0,.3) -- (\m,.3) node [pos=0.5, above] {$m$};		
    \draw (0,0) -- (0,-0.1);
    \node[below] at (0,-0.1) {\tiny $x$};
    \foreach \y in  {1,...,3} {
        \draw (\y,0) -- (\y,-0.1);
        \node[below] at (\y,-0.1) {\tiny $ \scriptstyle x+\y$};
    }
    \draw (\m,0) -- (\m,-0.1);
    \node[below] at (\m,-0.1) {\tiny $\scriptstyle  x+m$};
    \foreach \y in  {1,...,3} {
        \draw (\y+\m,0) -- (\y+\m,-0.1);
        \node[below] at (\y+\m,-0.1) {\tiny $\scriptstyle x+m+\y$};
        %		\draw[ line width=1, color=OrangeProfondIRA, arrows={-Stealth[length=4, inset=0]}] (\y+\m,0) -- (\y+\m,1);
        \draw[arrows={-Stealth[length=4, inset=0]}, line width=1] (-.5,0) -- (\w,0);
    }
\end{tikzpicture}

\begin{tikzpicture}[scale=0.75]
    % Draw the x-axis and y-axis.
    \def\w{11}
    \def\m{6}
    \node[left] at (-.5,0) {${}_{m|}a_x$};
    
    \begin{scope}[shift={(\m+.5+3,.25)}]
        \draw[color=OrangeProfondIRA,scale=0.2,fill=OrangeProfondIRA] \Cerceuil;
    \end{scope}
    \draw[dashed, color=BleuProfondIRA,arrows={Stealth[length=4, inset=0]-Stealth[length=4, inset=0]},  line width=1] (0,.3) -- (\m,.3) node [pos=0.5, above] {$m$};		
    \draw (0,0) -- (0,-0.1);
    \node[below] at (0,-0.1) {\tiny $x$};
    \foreach \y in  {1,...,3} {
        \draw (\y,0) -- (\y,-0.1);
        \node[below] at (\y,-0.1) {\tiny $ \scriptstyle x+\y$};
    }
    \draw (\m,0) -- (\m,-0.1);
    \node[below] at (\m,-0.1) {\tiny $\scriptstyle  x+m$};
    \foreach \y in  {1,...,3} {
        \draw (\y+\m,0) -- (\y+\m,-0.1);
        \node[below] at (\y+\m,-0.1) {\tiny $\scriptstyle x+m+\y$};
        \draw[ line width=1, color=OrangeProfondIRA, arrows={-Stealth[length=4, inset=0]}] (\y+\m,0) -- (\y+\m,1);
    }
    \draw[arrows={-Stealth[length=4, inset=0]}, line width=1] (-.5,0) -- (\w,0);
\end{tikzpicture}


  
\end{f}
\hrule

\begin{f}[Death or survival benefits]

% !TeX spellcheck = en_US
\textbf{Death benefits} (Whole life insurance noted ${SP}_{x}$ or ${A}_{x}$)

$A_x$ indicates a death benefit at the end of the year of death (amount of 1), regardless of the date of occurrence, for an individual insured at age $x$ at the time of subscription.

$A_{x:\lcroof{n}}$ denotes a capital paid upon death if it occurs and at the latest in $n$ years (Endowment).

$\lcterm{A}{x}{n}$ % ou $\termins{x}{n}$ 
denotes a death benefit paid if $x$ dies within the next $n$ years (Term insurance).


$A_x^{(12)}$ indicates a benefit payable at the end of the month of death.

$\overline{A}_x$ indicates a benefit paid on the date of death.
\begin{tikzpicture}[scale=0.75]
% Draw the x-axis and y-axis.
\def\w{11}
\def\n{7}
\node[left] at (-.5,0) {$A_x$};

\begin{scope}[shift={(3.75,.25)}]
    \draw[color=OrangeProfondIRA,scale=0.2,fill=OrangeProfondIRA] \Cerceuil;
\end{scope}
\draw[ line width=1, color=OrangeProfondIRA, arrows={-Stealth[length=4, inset=0]}] (4,0) -- (4,1);
\foreach \y in  {0,...,4} {
    \draw (\y,0) -- (\y,-0.1);
    \ifthenelse{\y>0 }{	\node[below] at (\y,-0.1) {\tiny $ \scriptstyle x+\y$};}{
        \node[below] at (\y,-0.1) {\tiny $ \scriptstyle x$};}
}
\draw (\n,0) -- (\n,-0.1);
\node[below] at (\n,-0.1) {\tiny $\scriptstyle  x+n$};
\foreach \y in  {1,...,3} {
    \draw (\y+\n,0) -- (\y+\n,-0.1);
    \node[below] at (\y+\n,-0.1) {\tiny $\scriptstyle x+n+\y$};
    %		\draw[ line width=1, color=OrangeProfondIRA, arrows={-Stealth[length=4, inset=0]}] (\y+\n,0) -- (\y+\n,1);
}
\draw[arrows={-Stealth[length=4, inset=0]}, line width=1] (-.5,0) -- (\w,0);
\end{tikzpicture}
\begin{tikzpicture}[scale=0.75]
% Draw the x-axis and y-axis.
\def\w{11}
\def\n{7}
\node[left] at (-.5,0) {$A_x$};

\begin{scope}[shift={(\n+.5+2,.25)}]
    \draw[color=OrangeProfondIRA,scale=0.2,fill=OrangeProfondIRA] \Cerceuil;
\end{scope}
\draw[ line width=1, color=OrangeProfondIRA, arrows={-Stealth[length=4, inset=0]}] (\n+3,0) -- (\n+3,1);
\foreach \y in  {0,...,4} {
    \draw (\y,0) -- (\y,-0.1);
    \ifthenelse{\y>0 }{	\node[below] at (\y,-0.1) {\tiny $ \scriptstyle x+\y$};}{
        \node[below] at (\y,-0.1) {\tiny $ \scriptstyle x$};}
}
\draw (\n,0) -- (\n,-0.1);
\node[below] at (\n,-0.1) {\tiny $\scriptstyle  x+n$};
\foreach \y in  {1,...,3} {
    \draw (\y+\n,0) -- (\y+\n,-0.1);
    \node[below] at (\y+\n,-0.1) {\tiny $\scriptstyle x+n+\y$};
    %		\draw[ line width=1, color=OrangeProfondIRA, arrows={-Stealth[length=4, inset=0]}] (\y+\n,0) -- (\y+\n,1);
}
\draw[arrows={-Stealth[length=4, inset=0]}, line width=1] (-.5,0) -- (\w,0);
\end{tikzpicture}
\begin{tikzpicture}[scale=0.75]
% Draw the x-axis and y-axis.
\def\w{11}
\def\n{7}
\node[left] at (-.5,0) {$\overline{A}_x$};

\begin{scope}[shift={(3.75,.25)}]
    \draw[color=OrangeProfondIRA,scale=0.2,fill=OrangeProfondIRA] \Cerceuil;
\end{scope}
\draw[ line width=1, color=black, arrows={-Stealth[length=4, inset=0]}] (3.75,0) -- (3.75,1);
\foreach \y in  {0,...,4} {
    \draw (\y,0) -- (\y,-0.1);
    \ifthenelse{\y>0 }{	\node[below] at (\y,-0.1) {\tiny $ \scriptstyle x+\y$};}{
        \node[below] at (\y,-0.1) {\tiny $ \scriptstyle x$};}
}
\draw (\n,0) -- (\n,-0.1);
\node[below] at (\n,-0.1) {\tiny $\scriptstyle  x+n$};
\foreach \y in  {1,...,3} {
    \draw (\y+\n,0) -- (\y+\n,-0.1);
    \node[below] at (\y+\n,-0.1) {\tiny $\scriptstyle x+n+\y$};
    %		\draw[ line width=1, color=OrangeProfondIRA, arrows={-Stealth[length=4, inset=0]}] (\y+\n,0) -- (\y+\n,1);
}
\draw[arrows={-Stealth[length=4, inset=0]}, line width=1] (-.5,0) -- (\w,0);
\end{tikzpicture}
\begin{tikzpicture}[scale=0.75]
% Draw the x-axis and y-axis.
\def\w{11}
\def\n{7}
\node[left] at (-.5,0) {$\overline{A}_x$};

\begin{scope}[shift={(\n+.5+2,.25)}]
    \draw[color=OrangeProfondIRA,scale=0.2,fill=OrangeProfondIRA] \Cerceuil;
\end{scope}
\draw[ line width=1, color=black, arrows={-Stealth[length=4, inset=0]}] (\n+.5+2,0) -- (\n+.5+2,1);
\foreach \y in  {0,...,4} {
    \draw (\y,0) -- (\y,-0.1);
    \ifthenelse{\y>0 }{	\node[below] at (\y,-0.1) {\tiny $ \scriptstyle x+\y$};}{
        \node[below] at (\y,-0.1) {\tiny $ \scriptstyle x$};}
}
\draw (\n,0) -- (\n,-0.1);
\node[below] at (\n,-0.1) {\tiny $\scriptstyle  x+n$};
\foreach \y in  {1,...,3} {
    \draw (\y+\n,0) -- (\y+\n,-0.1);
    \node[below] at (\y+\n,-0.1) {\tiny $\scriptstyle x+n+\y$};
    %		\draw[ line width=1, color=OrangeProfondIRA, arrows={-Stealth[length=4, inset=0]}] (\y+\n,0) -- (\y+\n,1);
}
\draw[arrows={-Stealth[length=4, inset=0]}, line width=1] (-.5,0) -- (\w,0);
\end{tikzpicture}

\medskip

Whole life benefit
$$A_{x}=\sum_{k=0}^{\infty} {}_{k|}q_x\ \nu^{k+1}=\frac{M_x}{D_x}$$
	
$$\termins{x}{n}=\sum_{k=0}^{n-1} {}_{k|}q_x\ \nu^{k+1}=\frac{M_x-M_{x+n}}{D_x}
$$



\medskip    
\textbf{Deferred capital (Pure Endowment, unique capital in the event of survival)} noted $\lcend{A}{x}{n}$ %$\pureend{x}{n}$
or ${}_n E_x$.

\begin{tikzpicture}[scale=0.75]
% Draw the x-axis and y-axis.
\def\w{11}
\def\n{7}
\node[left] at (-.5,0) {${}_n E_x$};

\begin{scope}[shift={(3.75,.25)}]
    \draw[color=OrangeProfondIRA,scale=0.2,fill=OrangeProfondIRA] \Cerceuil;
\end{scope}
%		\draw[ line width=1, color=OrangeProfondIRA, arrows={-Stealth[length=4, inset=0]}] (4,0) -- (4,1);
\foreach \y in  {0,...,4} {
    \draw (\y,0) -- (\y,-0.1);
    \ifthenelse{\y>0 }{	\node[below] at (\y,-0.1) {\tiny $ \scriptstyle x+\y$};}{
        \node[below] at (\y,-0.1) {\tiny $ \scriptstyle x$};}
}
\draw (\n,0) -- (\n,-0.1);
\node[below] at (\n,-0.1) {\tiny $\scriptstyle  x+n$};
\foreach \y in  {1,...,3} {
    \draw (\y+\n,0) -- (\y+\n,-0.1);
    \node[below] at (\y+\n,-0.1) {\tiny $\scriptstyle x+n+\y$};
    %		\draw[ line width=1, color=OrangeProfondIRA, arrows={-Stealth[length=4, inset=0]}] (\y+\n,0) -- (\y+\n,1);
}
\draw[arrows={-Stealth[length=4, inset=0]}, line width=1] (-.5,0) -- (\w,0);
\end{tikzpicture}
\begin{tikzpicture}[scale=0.75]
% Draw the x-axis and y-axis.
\def\w{11}
\def\n{7}
\node[left] at (-.5,0) {${}_n E_x$};

\begin{scope}[shift={(\n+.5+2,.25)}]
    \draw[color=OrangeProfondIRA,scale=0.2,fill=OrangeProfondIRA] \Cerceuil;
\end{scope}
\draw[ line width=1, color=OrangeProfondIRA, arrows={-Stealth[length=4, inset=0]}] (\n,0) -- (\n,1);
\foreach \y in  {0,...,4} {
    \draw (\y,0) -- (\y,-0.1);
    \ifthenelse{\y>0 }{	\node[below] at (\y,-0.1) {\tiny $ \scriptstyle x+\y$};}{
        \node[below] at (\y,-0.1) {\tiny $ \scriptstyle x$};}
}
\draw (\n,0) -- (\n,-0.1);
\node[below] at (\n,-0.1) {\tiny $\scriptstyle  x+n$};
\foreach \y in  {1,...,3} {
    \draw (\y+\n,0) -- (\y+\n,-0.1);
    \node[below] at (\y+\n,-0.1) {\tiny $\scriptstyle x+n+\y$};
    %		\draw[ line width=1, color=OrangeProfondIRA, arrows={-Stealth[length=4, inset=0]}] (\y+\n,0) -- (\y+\n,1);
}
\draw[arrows={-Stealth[length=4, inset=0]}, line width=1] (-.5,0) -- (\w,0);
\end{tikzpicture}
	$$
	={}_n E_x={}_n p_x .v^n=\frac{l_{x+n}}{l_x} . v^n =\frac{D_{x+n}}{D_{x}}
	$$
Death benefit with payment of the capital in the event of survival (Endowment)
$$A_{x:\lcroof{n}}=\termins{x}{n}+\lcend{A}{x}{n}$$

\end{f}
\hrule

\begin{f}[Life insurance on several individuals] 

$a_{xyz}$ is an annual annuity, paid at the end of the first year and for as long as they live $(x)$, $(y)$ and $(z)$.

$a_{\overline{xyz}}$ is an annual annuity, paid at the end of the first year and for as long as they live $(x)$, $(y)$ or $(z)$.

$$
a_{\overline{xy}}=a_{y}+a_{x}-a_{xy}
$$

$A_{xyz}$ is an insurance that comes into effect at the end of the year of the first death of $(x)$, $(y)$ and $(z)$.

The vertical bar indicates conditionality :

$a_{x|y}$ is a survivor's annuity which benefits $(x)$ after the death of $(y)$.

$A_{x|yz}$ is a first-to-die insurance $(y)$ and $(z)$.	


\begin{tikzpicture}[scale=0.85]
    % Draw the x-axis and y-axis.
    \def\w{11}
    \def\n{6}
    \node[left] at (-.5,0) {$a_{{\color{OrangeProfondIRA}x}|{\color{BleuProfondIRA}y}}$};
    
    \begin{scope}[shift={(9.75,.25)}]
        \draw[color=OrangeProfondIRA,scale=0.2,fill=OrangeProfondIRA] \Cerceuil;
    \end{scope}
    \begin{scope}[shift={(7.55,.25)}]
        \draw[color=BleuProfondIRA,scale=0.2,fill=BleuProfondIRA] \Cerceuil;
    \end{scope}
    \draw[ line width=1, color=OrangeProfondIRA, arrows={-Stealth[length=4, inset=0]}] (8,0) -- (8,1);
    \draw[ line width=1, color=OrangeProfondIRA, arrows={-Stealth[length=4, inset=0]}] (9,0) -- (9,1);
    \foreach \y in  {0,...,3} {
        \draw (\y,0) -- (\y,-0.1);
        \ifthenelse{\y>0 }{	\node[below] at (\y,-0.1) {\tiny $ \scriptstyle x+\y$};}{
            \node[below] at (\y,-0.1) {\tiny $ \scriptstyle x$};}
    }
    \draw (\n,0) -- (\n,-0.1);
    \node[below] at (\n,-0.1) {\tiny $\scriptstyle  x+n$};
    \foreach \y in  {1,...,3} {
        \draw (\y+\n,0) -- (\y+\n,-0.1);
        \node[below] at (\y+\n,-0.1) {\tiny $\scriptstyle x+n+\y$};
    }
    \draw[arrows={-Stealth[length=4, inset=0]}, line width=1] (-.5,0) -- (\w,0);
\end{tikzpicture}
\begin{tikzpicture}[scale=0.85]
    % Draw the x-axis and y-axis.
    \def\w{11}
    \def\n{6}
    \node[left] at (-.5,0) {$a_{{\color{OrangeProfondIRA}x}|{\color{BleuProfondIRA}y}}$};
    
    \begin{scope}[shift={(8.4,.25)}]
        \draw[color=OrangeProfondIRA,scale=0.2,fill=OrangeProfondIRA] \Cerceuil;
    \end{scope}
    \begin{scope}[shift={(2.75,.25)}]
        \draw[color=BleuProfondIRA,scale=0.2,fill=BleuProfondIRA] \Cerceuil;
    \end{scope}
    \foreach \y in  {3,...,8} {
        \draw[ line width=1, color=OrangeProfondIRA, arrows={-Stealth[length=4, inset=0]}] (\y,0) -- (\y,1);
    }
    \foreach \y in  {0,...,3} {
        \draw (\y,0) -- (\y,-0.1);
        \ifthenelse{\y>0 }{	\node[below] at (\y,-0.1) {\tiny $ \scriptstyle x+\y$};}{
            \node[below] at (\y,-0.1) {\tiny $ \scriptstyle x$};}
    }
    \foreach \y in  {0,...,4} {
        \draw (\y+\n,0) -- (\y+\n,-0.1);
        \ifthenelse{\y>0 }{\node[below] at (\y+\n,-0.1) {\tiny $\scriptstyle x+n+\y$};}{
            \node[below] at (\y+\n,-0.1) {\tiny $\scriptstyle x+n$};}
    }
    \draw[arrows={-Stealth[length=4, inset=0]}, line width=1] (-.5,0) -- (\w,0);
\end{tikzpicture}
\begin{tikzpicture}[scale=0.85]
    % Draw the x-axis and y-axis.
    \def\w{11}
    \def\n{6}
    \node[left] at (-.5,0) {$a_{{\color{OrangeProfondIRA}x}|{\color{BleuProfondIRA}y}}$};
    
    \begin{scope}[shift={(3.75,.25)}]
        \draw[color=OrangeProfondIRA,scale=0.2,fill=OrangeProfondIRA] \Cerceuil;
    \end{scope}
    \begin{scope}[shift={(7.55,.25)}]
        \draw[color=BleuProfondIRA,scale=0.2,fill=BleuProfondIRA] \Cerceuil;
    \end{scope}
    \foreach \y in  {0,...,3} {
        \draw (\y,0) -- (\y,-0.1);
        \ifthenelse{\y>0 }{	\node[below] at (\y,-0.1) {\tiny $ \scriptstyle x+\y$};}{
            \node[below] at (\y,-0.1) {\tiny $ \scriptstyle x$};}
    }
    \draw (\n,0) -- (\n,-0.1);
    \node[below] at (\n,-0.1) {\tiny $\scriptstyle  x+n$};
    \foreach \y in  {1,...,4} {
        \draw (\y+\n,0) -- (\y+\n,-0.1);
        \node[below] at (\y+\n,-0.1) {\tiny $\scriptstyle x+n+\y$};
        %		\draw[ line width=1, color=OrangeProfondIRA, arrows={-Stealth[length=4, inset=0]}] (\y+\n,0) -- (\y+\n,1);
    }
    \draw[arrows={-Stealth[length=4, inset=0]}, line width=1] (-.5,0) -- (\w,0);
\end{tikzpicture}
\end{f}

\begin{f}[Simplified pricing schemes for periodic premiums and reserves]
	
\tikzstyle{startstop} = [rectangle, rounded corners, text width=4cm, minimum height=1cm,text centered, draw=black, fill=BleuProfondIRA!40]
\tikzstyle{process} = [rectangle, minimum width=4cm, minimum height=1cm, text centered, text width=3cm, draw=black, fill=OrangeProfondIRA!40]
\tikzstyle{arrow} = [thick,->,>=stealth]
\ \medskip

\centering
\begin{tikzpicture}[node distance=35mm and 100mm]
	\node (pu) [startstop] {Single Premium of the service\\ noted $PU(.)$ or $PV_0(.)$};
	\node  [below left of=pu, text width=55mm, node distance=18mm, xshift=12mm] {The point is $A$ or $a$ or $E$ or ...\\
	avec $x,k,n,m,K...$};
	\node (puu) [startstop, right of=pu, node distance=50mm] {Value of a unitary periodic premium\\ noted $PPU_0$ or $VP_P$};
	\node  [below of=puu, text width=3cm, node distance=12mm] { $PPU=_{m'|}\ddot{a}_{x:\lcroof{n'}}^{(k')} $};
	\node (ppp) [startstop, below of=pu] {Pure Periodic Premium\\ $PPP$};
	\node  [below right of=ppp, text width=3cm, node distance=25mm, yshift=6mm] {$\displaystyle PPP=\frac{PU(.)}{PPU_0}$};
	\node (ppc) [startstop, right of=ppp, node distance=50mm] {Periodic Premium Charged $PPC$};
	\node  [below of=ppc, node distance=12mm] {$\displaystyle PPC=\frac{PPP(1+\lambda)}{1-com}$};
	\node (Prov) [startstop, below of=ppp] {Mathematical reserves $t^e$ year\\ noted $PM_t$ or $V_t$};
	\node (provt) [below right of=Prov,text width=10cm, node distance=30mm] {$V_t=PV_t(.)-PPP\times PPU_t$ \\
	$PV_t(.)$ recalculated with  $x+t,k,n,m-t,K...\ if\ t\geq m$ \\
	or $x+t,k,n-(t-m),m=0,K...\ si\ t>m$\\
	likewise $PPU_t=_{(m'-t)^+|}\ddot{a}_{x+t:\lcroof{n'-(t-m)^+}}^{(k')} $};
%	\node (pro2b) [startstop, right of=dec1, xshift=2cm] {Process 2b};
%	\node (stop) [startstop, below of=pro2b] {Stop};
	
	\draw [arrow] (pu) -- (ppp);
	\draw [arrow] (puu) -- (ppp);
	\draw [arrow] (ppp) -- (ppc);
	\draw [arrow] (ppp) -- node[anchor=east] {$t$} (Prov);
%	\draw [arrow] (dec1) -- node[anchor=south] {no} (pro2b);
%	\draw [arrow] (pro2b) |- (ppp);
%	\draw [arrow] (ppp) -- (stop);
	
\end{tikzpicture}
\end{f}