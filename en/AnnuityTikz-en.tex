\ \newline


	Certain annuity \(a_{\lcroof{n}}\) (or \(a_{\lcroof{n}\;i}\) if the interest rate \(i\) needs to be specified): this is the default case in financial mathematics. Its payments are, for example, guaranteed by a contract.
	
\begin{center}
	\begin{tikzpicture}[scale=0.75]
	% Draw the x-axis and y-axis.
	\def\w{11}
	\def\n{9}
	\node[left] at (-.5,0) {\(a_{\lcroof{n}}\)};
	\foreach \y in  {0,...,3} {
	\draw (\y,0) -- (\y,-0.1);
	\ifthenelse{\y>0 }{	\node[below] at (\y,-0.1) {\tiny \( \scriptstyle \y\)};
		\draw[ line width=1, color=OrangeProfondIRA, arrows={-Stealth[length=4, inset=0]}] (\y,0) -- (\y,1);}{
		\node[below] at (\y,-0.1) {\tiny 0};}
}
\foreach \y in  {-3,...,1} {
	\draw (\y+\n,0) -- (\y+\n,-0.1);
	\ifthenelse{\y<0 }{\node[below] at (\y+\n,-0.1) {\tiny \(\scriptstyle n \y\)};}{}
	\ifthenelse{\y=0 }{\node[below] at (\y+\n,-0.1) {\tiny \(\scriptstyle n \)};}{}
	\ifthenelse{\y>0}{	\node[below] at (\y+\n,-0.1) {\tiny \(\scriptstyle n+\y\)};}{}
	\ifthenelse{\y<1 }{	\draw[ line width=1, color=OrangeProfondIRA, arrows={-Stealth[length=4, inset=0]}] (\y+\n,0) -- (\y+\n,1);}
}
\draw[ line width=1] (-.25,0) -- (4,0);
\draw[ line width=1, dashed] (4,0) -- (5,0);
\draw[arrows={-Stealth[length=4, inset=0]}, line width=1] (5,0) -- (\w,0);
\end{tikzpicture}
\end{center}
\[
\ddot{a}_{\lcroof{n}}=1+v+\cdots+v^{n-1}=\frac{1-v^{n}}{1-v}=\frac{1-v^{n}}{d}
\]


Contingent annuity \(\ddot{a}_{x}\): its payments are conditional on a random event, such as a life annuity of an individual aged \(x\). In this example, payments continue until death occurs:

\begin{center}
	\begin{tikzpicture}[scale=0.75]
	% Draw the x-axis and y-axis.
	\def\w{11}
	\def\n{6}
	\node[left] at (-.5,0) {\(a_{x}\)};
	
	\begin{scope}[shift={(\n+.5+3,.25)}]
		\draw[color=OrangeProfondIRA,scale=0.2,fill=OrangeProfondIRA] \Cerceuil;
	\end{scope}
	\foreach \y in  {0,...,3} {
		\draw (\y,0) -- (\y,-0.1);
		\ifthenelse{\y>0 }{	\node[below] at (\y,-0.1) {\tiny \( \scriptstyle x+\y\)};
			\draw[ line width=1, color=OrangeProfondIRA, arrows={-Stealth[length=4, inset=0]}] (\y,0) -- (\y,1);}{
			\node[below] at (\y,-0.1) {\tiny \( \scriptstyle x\)};}
	}
	\foreach \y in  {0,...,4} {
		\draw (\y+\n,0) -- (\y+\n,-0.1);
		\ifthenelse{\y>0 }{\node[below] at (\y+\n,-0.1) {\tiny \(\scriptstyle x+n+\y\)};}{
			\node[below] at (\y+\n,-0.1) {\tiny \(\scriptstyle x+n\)};}
		\ifthenelse{\y<4 }{	\draw[ line width=1, color=OrangeProfondIRA, arrows={-Stealth[length=4, inset=0]}] (\y+\n,0) -- (\y+\n,1);}
	}
\draw[ line width=1] (-.25,0) -- (4,0);
\draw[ line width=1, dashed] (4,0) -- (5,0);
\draw[arrows={-Stealth[length=4, inset=0]}, line width=1] (5,0) -- (\w,0);
\end{tikzpicture}	

\end{center}
The date of death is represented here by a small coffin. This type of annuity will be extensively studied in the life actuarial section.

Annuity in arrears (immediate) \(a_{\lcroof{n}}\): its periodic payments are made at the end of each payment period, as with a salary paid at the end of the month. This is the default case, previously illustrated for the certain annuity.
\[
\ddot{a}_{\lcroof{n}}=1+v+\cdots+v^{n-1}=\frac{1-v^{n}}{1-v}=\frac{1-v^{n}}{d}
\]
%
\[
	\mathrm{PV}_{\lcroof{n}}^{\text {due }}=K \ddot{a}_{\lcroof{n}}=K \frac{1-v^{n}}{d} 
\]

Annuity in advance (due) \(\ddot{a}_{\lcroof{n}}\): its periodic payments are made at the beginning of each payment period, as with rent payments, for example.

\begin{center}
	\begin{tikzpicture}[scale=0.75]
		% Draw the x-axis and y-axis.
		\def\w{11}
		\def\n{9}
		\node[left] at (-.5,0) {\(\ddot{a}_{\lcroof{n}}\)};
		\foreach \y in  {0,...,3} {
			\draw (\y,0) -- (\y,-0.1);
				\node[below] at (\y,-0.1) {\tiny \( \scriptstyle \y\)};
				\draw[ line width=1, color=OrangeProfondIRA, arrows={-Stealth[length=4, inset=0]}] (\y,0) -- (\y,1);
		}
		\foreach \y in  {-3,...,1} {
			\draw (\y+\n,0) -- (\y+\n,-0.1);
			\ifthenelse{\y<0 }{\node[below] at (\y+\n,-0.1) {\tiny \(\scriptstyle n \y\)};}{}
			\ifthenelse{\y=0 }{\node[below] at (\y+\n,-0.1) {\tiny \(\scriptstyle n \)};}{}
			\ifthenelse{\y>0}{	\node[below] at (\y+\n,-0.1) {\tiny \(\scriptstyle n+\y\)};}{}
			\ifthenelse{\y<0 }{	\draw[ line width=1, color=OrangeProfondIRA, arrows={-Stealth[length=4, inset=0]}] (\y+\n,0) -- (\y+\n,1);}{}
		}
		\draw[ line width=1] (-.25,0) -- (4,0);
		\draw[ line width=1, dashed] (4,0) -- (5,0);
		\draw[arrows={-Stealth[length=4, inset=0]}, line width=1] (5,0) -- (\w,0);
	\end{tikzpicture}
\end{center}
Also denoted \(\mathrm{PV}^{\mathrm{im}}\) :
\[
a_{\lcroof{n}}=v+v^{2}+\cdots+v^{n}=\frac{1-v^{n}}{i}=v \frac{1-v^{n}}{1-v}
\]
%
\[
	\mathrm{PV}_{\lcroof{n}}^{\mathrm{im}}=K a_{\lcroof{n}}=K \frac{1-v^{n}}{i} 
\]
Perpetuity \(a\) or \(a_{\lcroof{\infty}}\):
\[
a=1/i
\]
Deferred annuity \(_{m|}a_{\lcroof{n}}\): its payments do not start in the first period but after \(m\) periods, with \(m\) fixed in advance.
\begin{center}
	\begin{tikzpicture}[scale=0.75]
		% Draw the x-axis and y-axis.
\def\w{10}
\def\n{9}
\def\m{2}
\node[left] at (-.5,0) {\(_{m|}a_{\lcroof{n}}\)};
\foreach \y in  {0,...,2} {
	\draw (\y+\m,0) -- (\y+\m,-0.1);
	\ifthenelse{\y<0 }{\node[below] at (\y+\m,-0.1) {\tiny \(\scriptstyle m \y\)};}{}
\ifthenelse{\y=0 }{\node[below] at (\y+\m,-0.1) {\tiny \(\scriptstyle m \)};}{}
\ifthenelse{\y>0}{	\node[below] at (\y+\m,-0.1) {\tiny \(\scriptstyle m+\y\)};
		\draw[ line width=1, color=OrangeProfondIRA, arrows={-Stealth[length=4, inset=0]}] (\y+\m,0) -- (\y+\m,1);
		}
}
\foreach \y in  {-3,...,0} {
	\draw (\y+\n,0) -- (\y+\n,-0.1);
	\ifthenelse{\y<0 }{\node[below] at (\y+\n,-0.1) {\tiny \(\scriptstyle m+n \y\)};}{}
	\ifthenelse{\y=0 }{\node[below] at (\y+\n,-0.1) {\tiny \(\scriptstyle m+n \)};}{}
	\ifthenelse{\y>0}{	\node[below] at (\y+\n,-0.1) {\tiny \(\scriptstyle m+n+\y\)};}{}
	\ifthenelse{\y<1 }{	\draw[ line width=1, color=OrangeProfondIRA, arrows={-Stealth[length=4, inset=0]}] (\y+\n,0) -- (\y+\n,1);}
}
\draw[ line width=1] (-.25,0) -- (0.5,0);
\draw[ line width=1, dashed] (0.5,0) -- (1.5,0);
\draw[ line width=1] (1.5,0) -- (4,0);
\draw[ line width=1, dashed] (4,0) -- (5,0);
		\draw[arrows={-Stealth[length=4, inset=0]}, line width=1] (5,0) -- (\w,0);
	\end{tikzpicture}	
	
\end{center}


Periodic / monthly annuity \(a^{(m)}\) : the default periodicity is one year, but the unit payment can also be spread over \(m\) periods within the year.

\begin{center}
	\begin{tikzpicture}[scale=0.75]
		% Draw the x-axis and y-axis.
		\def\w{10}
		\def\n{9}
		\def\m{6}
		\node[left] at (-.5,0) {\(a_{\lcroof{n}}^{(m)}\)};
		\foreach \y in  {0,...,3} {
			\draw (\y,0) -- (\y,-0.1);
			\ifthenelse{\y>0 }{	\node[below] at (\y,-0.1) {\tiny \( \scriptstyle \y\)};}{
				\node[below] at (\y,-0.1) {\tiny 0};}
		}
		\pgfmathparse{3.5*\m} 
		\foreach \y in  {1,...,\pgfmathresult} {
			\ifthenelse{\y>0 }{	
				\draw[ line width=1, color=OrangeProfondIRA, arrows={-Stealth[length=4, inset=0]}] (\y/\m,0) -- (\y/\m,2/\m);}{}
		}
		\foreach \y in  {-3,...,0} {
			\draw (\y+\n,0) -- (\y+\n,-0.1);
			\ifthenelse{\y<0 }{\node[below] at (\y+\n,-0.1) {\tiny \(\scriptstyle n \y\)};}{}
			\ifthenelse{\y=0 }{\node[below] at (\y+\n,-0.1) {\tiny \(\scriptstyle n \)};}{}
			\ifthenelse{\y>0}{	\node[below] at (\y+\n,-0.1) {\tiny \(\scriptstyle n+\y\)};}{}
		}
		\pgfmathparse{3.5*\m} 
		\foreach \y  in  {1,...,\pgfmathresult} {
		\draw[ line width=1, color=OrangeProfondIRA, arrows={-Stealth[length=4, inset=0]}] (-\y/\m+\n,0) -- (-\y/\m+\n,2/\m);
		}
		\draw[ line width=1] (-.25,0) -- (3.5,0);
		\draw[ line width=1, dashed] (3.5,0) -- (5.5,0);
		\draw[arrows={-Stealth[length=4, inset=0]}, line width=1] (5.5,0) -- (\w,0);
	\end{tikzpicture}
\end{center}

If \(i^{(m)}\) represents the equivalent nominal (annual) interest rate with \(m\) periods per year, then \(i^{(m)}= m\left((1+i)^{1 / m}-1\right)\) .

Similarly, \(d^{(m)} \) is the nominal discount rate consistent with \(d\) and \(m\) : \(d^{(m)}= m\left(1-(1-d)^{1 / m}\right)\).

\[
\ddot{a}_{\lcroof{n}}^{(m)}=\frac{1}{m} \sum_{k=0}^{m n-1} v^{\frac{k}{m}}=\frac{d}{d^{(m)}} \ddot{a}_{\lcroof{n}}=\frac{1-v^{n}}{d^{(m)}} \approx \ddot{a}_{\lcroof{n}}+\frac{m-1}{2 m}\left(1-v^{n}\right)
\]
\[
a_{\lcroof{n}}^{(m)}=\frac{1}{m} \sum_{k=1}^{m n} v^{\frac{k}{m}}=\frac{i}{i^{(m)}} a_{\lcroof{n}}=\frac{1-v^{n}}{i^{(m)}} \approx a_{\lcroof{n}}-\frac{m-1}{2 m}\left(1-v^{n}\right)
\]
Unit annuity \(a\): it is used when constructing annuity formulas.
For a constant annuity, the total amount paid each year is 1, regardless of \(m\). 

Dynamic annuity, increasing/decreasing \(Ia\)/\(Da\): in its simplest form, it pays an amount that starts at 1 (\(n\)) and increases (decreases) each period arithmetically or geometrically. In the following example, the progression is arithmetic.
The prefix \(I\) (increasing) is used to indicate increasing annuities and \(D\) (decreasing) for decreasing annuities.
	
\begin{center}
	\begin{tikzpicture}[scale=0.75]
		% Draw the x-axis and y-axis.
		\def\w{11}
		\def\n{9}
		\node[left] at (-.5,0) {\(Ia_{\lcroof{n}}\)};
		\foreach \y in  {0,...,3} {
			\draw (\y,0) -- (\y,-0.1);
			\ifthenelse{\y>0 }{	\node[below] at (\y,-0.1) {\tiny \( \scriptstyle \y\)};
				\draw[ line width=1, color=OrangeProfondIRA, arrows={-Stealth[length=4, inset=0]}] (\y,0) -- (\y,\y/3);}{
				\node[below] at (\y,-0.1) {\tiny 0};}
		}
		\foreach \y in  {-3,...,1} {
			\draw (\y+\n,0) -- (\y+\n,-0.1);
			\ifthenelse{\y<0 }{\node[below] at (\y+\n,-0.1) {\tiny \(\scriptstyle n \y\)};}{}
			\ifthenelse{\y=0 }{\node[below] at (\y+\n,-0.1) {\tiny \(\scriptstyle n \)};}{}
			\ifthenelse{\y>0}{	\node[below] at (\y+\n,-0.1) {\tiny \(\scriptstyle n+\y\)};}{}
			\ifthenelse{\y<1 }{	\draw[ line width=1, color=OrangeProfondIRA, arrows={-Stealth[length=4, inset=0]}] (\y+\n,0) -- (\y+\n,\y/3+\n/3);}
		}
		\draw[ line width=1] (-.25,0) -- (4,0);
		\draw[ line width=1, dashed] (4,0) -- (5,0);
		\draw[arrows={-Stealth[length=4, inset=0]}, line width=1] (5,0) -- (\w,0);
	\end{tikzpicture}
	\begin{tikzpicture}[scale=0.75]
	% Draw the x-axis and y-axis.
	\def\w{11}
	\def\n{9}
	\node[left] at (-.5,0) {\(Da_{\lcroof{n}}\)};
	\foreach \y in  {0,...,3} {
		\draw (\y,0) -- (\y,-0.1);
		\ifthenelse{\y>0 }{	\node[below] at (\y,-0.1) {\tiny \( \scriptstyle \y\)};
			\draw[ line width=1, color=OrangeProfondIRA, arrows={-Stealth[length=4, inset=0]}] (\y,0) -- (\y,\n/3-\y/3);}{
			\node[below] at (\y,-0.1) {\tiny 0};}
	}
	\foreach \y in  {-3,...,1} {
		\draw (\y+\n,0) -- (\y+\n,-0.1);
		\ifthenelse{\y<0 }{\node[below] at (\y+\n,-0.1) {\tiny \(\scriptstyle n \y\)};}{}
		\ifthenelse{\y=0 }{\node[below] at (\y+\n,-0.1) {\tiny \(\scriptstyle n \)};}{}
		\ifthenelse{\y>0}{	\node[below] at (\y+\n,-0.1) {\tiny \(\scriptstyle n+\y\)};}{}
		\ifthenelse{\y<1 }{	\draw[ line width=1, color=OrangeProfondIRA, arrows={-Stealth[length=4, inset=0]}] (\y+\n,0) -- (\y+\n,1/3-\y/3);}
	}
	\draw[ line width=1] (-.25,0) -- (4,0);
	\draw[ line width=1, dashed] (4,0) -- (5,0);
	\draw[arrows={-Stealth[length=4, inset=0]}, line width=1] (5,0) -- (\w,0);
\end{tikzpicture}
\end{center}

\begin{equation}
(I \ddot{a})_{\lcroof{n}}=1+2 v+\cdots+n v^{n-1}=\frac{1}{d}\left(\ddot{a}_{\lcroof{n}}-n v^{n}\right)
\label{Ian}
\end{equation}
with, we recall,  \(d=i/(1+i)\) and in arrears (immediate)
\[
(I a)_{\lcroof{n}}=v+2 v^{2}+\cdots+n v^{n}=\frac{1}{i}\left(\ddot{a}_{\lcroof{n}}-n v^{n}\right)
\]
\[
(D \ddot{a})_{\lcroof{n}}=n+(n-1) v+\cdots+v^{n-1}=\frac{1}{d}\left(n-a_{\lcroof{n}}\right)
\]
and in arrears :
\[
(D a)_{\lcroof{n}}=n v+(n-1) v^{2}+\cdots+v^{n}=\frac{1}{i}\left(n-a_{\lcroof{n}}\right)
\]



