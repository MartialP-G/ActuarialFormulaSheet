% !TeX root = ActuarialFormSheet_MBFA-en.tex
% !TeX spellcheck = en_GB


\begin{f}[Non-life insurance pricing]
	
A general approach, but not exhaustive, because there are many possibilities : \vspace{4mm}

	
\resizebox{\linewidth}{!}
{	\begin{tikzpicture}[every node/.style={draw, align=center, rounded corners,fill=BleuProfondIRA!30, 
			font=\footnotesize, minimum height=1cm, text width=3cm},
		every path/.style={->, thick},
		node distance=.5cm and .5cm]
		% Noeuds
		\node[] (data) {Data collection \\ (claims, production, ...)};
		\node[ right=of data] (prep) {Cleaning and treatment \\ data};
		\node[ fill=OrangePastelIRA!30, right=of prep] (method) {Choice of methods};
		%
		\node[below =of method] (glm) {GLM / Logistics Models\\ (Poisson, Gamma,...)};
		\node[below =of glm] (Seg) {Segmentation \\ (CART, Lasso, ...)};
		\node[below =of Seg] (ml) {GAM, Machine learning\\ (Random Forest, Gradian Boosting,...) };
		%
		\node[left=of Seg] (fit) {Adjusting models};
		\node[left=of fit] (eval) {Validation :\\ residues, Gini, AIC, etc.};
		\node[below=of eval] (pure) {Pure premium estimate};
		\node[below=of pure] (loadings) {Charges: fees, margin, taxes};
		\node[right=of loadings] (final) {Commercial bonus};
		%
		% Flèches
		\draw (data) -- (prep);
		\draw (prep) -- (method);
		\draw (method.east) |- +(.5,0) |-   (Seg.east);
		\draw (method.east) |- +(.5,0) |-  (glm.east);
		\draw (method.east) |- +(.5,0) |- (ml.east);
		%
		\draw (glm.west) |- +(-.25,0) |- (fit.east);
		\draw (Seg.west) -- (fit.east);
		\draw (ml.west) |- +(-.25,0) |- (fit.east);
		%
		\draw (fit) -- (eval);
		\draw (eval) -- (pure);
		\draw (pure) -- (loadings);
		\draw (loadings) -- (final);
	\end{tikzpicture}
}
\end{f}


\begin{f}[General structure of insurance data]

	
A classic data structure in insurance. Here again, the possibilities are numerous. :



	\tikzstyle{NoeudR}=[cylinder, shape border rotate=90, draw,minimum height=1.5cm,shape aspect=.25,align=center]
		\begin{tikzpicture} %[node distance=5cm]
	
	\node (db01) at (-2,2) [fill=BleuProfondIRA!30, minimum width=1cm,NoeudR] {\footnotesize  Insured};
	\node (db02) at (0,2) [minimum width=1cm,fill=BleuProfondIRA!30,NoeudR] {\footnotesize  Entity \\ \footnotesize  Insured};
	\node (db03) at (2,2) [fill=BleuProfondIRA!30,minimum width=1cm,NoeudR] {\footnotesize  Risks \\ \footnotesize  Insured};
	
	\node (db1) at (0,0) [ fill=BleuProfondIRA!30,minimum width=2cm, NoeudR] { Contract \\ Production};
	\node (db2) at (3,0) [minimum height=1cm,fill=OrangeProfondIRA!30,	minimum width=2cm, NoeudR] { Claim};
	\node (db3) at (6,0) [ fill=OrangePastelIRA!20,minimum width=2cm,NoeudR,densely dotted] {\small External \\ \small data};
	\node (db4) at (3,-2) [fill=VertIRA!30,minimum width=3cm,NoeudR] {\small Database \\ \small  Pricing};
	
	\node (TB) at (-1.2,-2) [rectangle,fill=FushiaIRA!30, minimum width=2cm,
	shape border rotate=90, draw,minimum height=1.5cm,	shape aspect=.25,align=center] {\small Dashboards};
	\draw[->,>=latex] (db01.south) -- (db1);
	\draw[->,>=latex] (db02) -- (db1);
	\draw[->,>=latex] (db03.south) -- (db1);
	\draw[->,>=latex] (db1.south) -- (db4);
	\draw[->,>=latex] (db2.south) -- (db4);
	\draw[->,>=latex,densely dotted] (db3.south) -- (db4);
	\draw[<->,>=latex] (db4) -- (TB);
\end{tikzpicture}
\end{f}


%		\begin{tikzpicture} %[node distance=5cm]
%	%	
%	\draw[thick,<-,>=latex,BleuProfondIRA] (-3,0) -- (10,0) ;
%	\draw[thick,->,>=latex,BleuProfondIRA,densely dotted] (10,0) -- (15,0) ;
%	\node (A) at (-2,1) [NoeudR] {Incured \\date \\\footnotesize  15/10/\NN};
%	\node (B) at (0,1) [NoeudR] {Reported \\date\\\footnotesize  19/10/\NN};
%	\node (C) at (4,1.2) [NoeudR] {First\\ payment \\ date\\\footnotesize  25/11/\NN};
%	\node (D) at (10,1) [NoeudR] {Payment \\ date \\ \footnotesize 8/01/\N};
%	\node (E) at (12,1.2) [NoeudR,densely dotted] {New \\ information \\ date\\ \footnotesize 15/02/\N};
%	\node (F) at (14,1.2) [NoeudR,densely dotted] {Last \\ Payment \\ date\\ \footnotesize 31/04/\N};
%	\draw[ BleuProfondIRA] (A.south) -- ++(0,-1);
%	\draw[ BleuProfondIRA] (B.south) -- ++(0,-1);
%	\draw[ BleuProfondIRA] (C.south) -- ++(0,-1);
%	\draw[ BleuProfondIRA] (D.south) -- ++(0,-1);
%	\draw[ BleuProfondIRA] (E.south) -| ++(0,-1);
%	\draw[ BleuProfondIRA] (F.south) -- ++(0,-0.1) -| ++(-1,-0.9);
%	\draw[thick,<->,>=latex,OrangeProfondIRA] (\((A.south) +(0,-1)\)) -- (\((B.south) +(0,-1)\)) node [below, midway] {\footnotesize Not reported} ;
%	\draw[thick,<->,>=latex,BleuProfondIRA] (\((B.south) +(0,-1)\)) -- (\((D.south) +(0,-1)\)) node [below, midway] {\footnotesize Open} ;
%	\draw[thick,<->,>=latex,BleuProfondIRA] (\((B.south) +(0,-1)\)) -- (\((D.south) +(0,-1)\)) node [below, midway] {\footnotesize Open} ;
%	\draw[thick,<->,>=latex,GrisLogoIRA] (\((C.south) +(0,-1.6)\)) -- (\((D.south) +(0,-1.6)\)) node [below, midway] {\footnotesize Partially paid} ;
%	\draw[thick,<->,>=latex,GrisLogoIRA] (\((C.south) +(0,-1.6)\)) -- (\((C.south) +(-1,-1.6)\)) node [below, midway] {\footnotesize Not Paid} ;
%	\draw[thick,<->,>=latex,GrisLogoIRA] (\((B.south) +(0,-1.6)\)) -- (\((C.south) +(-1,-1.6)\)) node [below, midway] {\footnotesize Not Valued} ;
%	\draw[very thick,<->,>=latex,FushiaIRA] (\((D.south) +(0,-2.2)\)) -- (\((C.south) +(-1,-2.2)\)) node [below, midway] {\footnotesize Reserved};
%	\draw[thick,<->,>=latex,VertIRA] (\((D.south) +(0,-1)\)) -- (\((E.south) +(0,-1)\)) node [below, midway] {\footnotesize  Closed} ;
%	\draw[thick,<->,>=latex,OrangeProfondIRA,densely dotted] (\((E.south) +(0,-1)\)) -- (\((F.south) +(-1,-1)\)) node [below, midway,align=right] 
%	{\footnotesize Reopen } ;
%	\draw[thick,<->,>=latex,VertIRA] (\((F.south) +(-1,-1)\)) -- (\((F.south) +(1,-1)\)) node [below, midway] {\footnotesize  Closed} ;
%	%	
%\end{tikzpicture}

\hrule

\begin{f}[Provision]
	
The non-life actuary mainly assesses
the following provisions:
\begin{itemize}
	\item Reserves for claims reported but not settel (RBNS).
	\item Reserve for claims incurred but not reported (IBNR).
	\item Reserves for unearned premiums.
	\item Reserves for outstanding risks (non-life).
	
\end{itemize}

\end{f}




\begin{f}[Deterministic Chain Ladder]
	%
	Let \(C_{i k}\) be the amount, cumulative up to development year \(k\), of claims occurring in accident year \(i\), for \(1 \leq i, k \leq n\). \(C_{i k}\) may represent either the amount paid or the total estimated cost (payment already made plus reserve) of the claim. The amounts \(C_{i k}\) are known for \(i+k \leq n+1\) and we seek to estimate the values of the \(C_{i k}\) for \(i+k>n+1,\) and in particular the ultimate values \(C_{i n}\) for \(2 \leq i \leq n\). These notations are illustrated in the following triangle:
	%	\[
	%	C=\left(\begin{array}{ccccc}
		%	\rowcolor{white}	C_{1,1} & C_{1,2} & \cdots & C_{1, n-1} & C_{1, n} \\
		%	\rowcolor{white}	C_{2,1} & C_{2,2} & \cdots & C_{2, n-1} & \\
		%	\rowcolor{white}	\vdots & \vdots & \ddots & & \\
		%	\rowcolor{white}	C_{n-1,1} & C_{n-1,2} & & &  \\
		%	\rowcolor{white}	C_{n, 1} & & & 
		%	\end{array}\right)
	%	\]
	
	The Chain Ladder method estimates the unknown amounts, \(C_{i k}\) for \(i+k>n+1,\) by
	\begin{equation}\label{CL1}
		\hat{C}_{i k}=C_{i, n+1-i} \cdot \hat{f}_{n+1-i} \cdots \hat{f}_{k-1} \quad i+k>n+1
	\end{equation}	or
	\begin{equation}\label{CL2}
		\hat{f}_{k}=\frac{\color{OrangeProfondIRA}\sum_{i=1}^{n-k} C_{i, k+1}}{\color{BleuProfondIRA}\sum_{i=1}^{n-k} C_{i k}} \quad 1 \leq k \leq n-1 .
	\end{equation}
	The claim reserve for the year of the accident (\(R_{i}\), \(2 \leq i \leq n\)), is then estimated by
	\begin{align*}
		\hat{R}_{i}=&C_{in }-C_{i, n+1-i}\\
		&=C_{i, n+1-i} \cdot \hat{f}_{n+1-i} \cdots \hat{f}_{n-1}-C_{i, n+1-i} 
	\end{align*}
	
	\tikzset{BarreStyle/.style =   {opacity=.3,line width=15 mm,line cap=round,color=#1}}
	\begin{tikzpicture}[baseline=(A.center)]
		%	
		\matrix (A) [matrix of math nodes,%
		left delimiter  = (,%
		right delimiter =)]%
		{%
			C_{1,1} & C_{1,2} & \cdots &C_{1, n+1-i} & \cdots & C_{1, n-1} & \node (A-1-7) {\color{BleuProfondIRA}C_{1, n}}; \\
			C_{2,1} & C_{2,2} & \cdots &C_{2, n+1-i} & \cdots & \node (A-2-6) {\color{BleuProfondIRA}C_{2, n-1} };& \\
			\vdots & \vdots & \cdots & \vdots & \ddots & & \\
			C_{i,1} & 	C_{i,2} & \cdots & \node (A-4-4){\color{BleuProfondIRA} C_{i,n+1-i}}; &  & & \\
			\vdots & \vdots & \ddots & & \\
			C_{n-1,1} &\node (A-6-2) {\color{BleuProfondIRA}C_{n-1,2}}; & & & & & \\
			\node (A-7-1) {\color{BleuProfondIRA}C_{n, 1}}; & & & & & &\\
		};
		\node [draw,above=10pt] at (A.north) 	{ Payment deadlines};
		\node [draw,left=20pt,rotate=90, align=right, xshift=-1cm] at (A.north west) 	{ Years of origin \(i\)};
		\draw[BarreStyle=BleuProfondIRA] (A-7-1.south west) to  (A-1-7.north east) ;
		\draw  (A.south east) node [ left,color=BleuProfondIRA, align=right] {Regulations of the year \(n\)\\ (or \(i+j=n+1\))};
		\draw[color=BleuProfondIRA,thick] (A-7-1) to[bend right] (A-6-2) to[bend right=40] node[below right, pos=1] {\(\sum\)}  (A-4-4) to[bend right=40]   (A-2-6) to[bend right]   (A-1-7);
	\end{tikzpicture}
	
\end{f}


\begin{f}[Mack's Method]

The first two hypotheses are:
%
\begin{equation}\label{CL3}
E\left(C_{i, k+1} \mid C_{i 1}, \ldots, C_{i k}\right)=C_{i k} f_{k} \quad 1 \leq i \leq n, 1 \leq k \leq n-1
\end{equation}	
\begin{equation}\label{CL4}
\left\{C_{i 1}, \ldots, C_{i n}\right\},\left\{C_{j 1}, \ldots, C_{j n}\right\} \quad \forall i, j \quad \text{are independent}
\end{equation}


Mack demonstrates that if we estimate the parameters of the model (\ref{CL3}) par (\ref{CL2}) so this stochastic model (\ref{CL3}), combined with the assumption (\ref{CL4}) provides exactly the same caveats as the original Chain Ladder method (\ref{CL1}).

With the notation \(f_{i, k}= \frac{C_{i, k+1}}{C_{i, k}}\),  \(\hat{f}_{k}\) is the average of the \({\color{OrangeProfondIRA}f_{i, k}}\) weighted by the \({\color{BleuProfondIRA} C_{i, k}}\):
\begin{align*}
	\hat{f}_{k}&=\frac{\sum_{i=1}^{n-k}{\color{BleuProfondIRA} C_{i, k}}\times \color{OrangeProfondIRA}f_{i, k}}{\color{BleuProfondIRA}\sum_{i=1}^{n-k} C_{i k}}
\end{align*}	
The variance is written :
\begin{align*}
	\hat{\sigma}_{k}^{2}&=\frac{1}{n-k-1} \sum_{i=1}^{n-k} C_{i k}\left(\frac{C_{i, k+1}}{C_{i k}}-\hat{f}_{k}\right)^{2} \\
	&=\frac{1}{n-k-1} \sum_{i=1}^{n-k}  \left(\frac{\color{BleuProfondIRA} C_{i, k+1}-{\color{BleuProfondIRA} C_{i, k}} \hat{f}_{k}}{\color{BleuProfondIRA} \sqrt{C_{i, k}}}\right)^{2} 
\end{align*}		

%\begin{center}
%	\begin{tikzpicture}
%		%\draw[Bracket-Bracket] (0,0) -- (2,0);
%		%\draw[{Bracket[reversed]-Bracket[reversed]}] (0,1) -- (2,1);
%		%\draw[{Parenthesis-Parenthesis[reversed]}] (0,2) -- (2,2);
%		\draw [color=BleuProfondIRA,decorate,decoration={brace,amplitude=10pt,raise=1pt},yshift=0pt] (0,0)	-- (0,1);
%		\draw[gray!40] (0,0) -- (0,1) -- (1,1) -- cycle node[below] {\(C_{i, k+1}\)};
%		\draw[color=BleuProfondIRA] (0.2,0.2) -- (0.2,1) -- (1,1) -- cycle node {};
%		\node at (1.2,0.3) {\(-\)};
%		\draw[gray!40]  (2.3,0.2) --  (2.3,1) -- (1.5,0.2) -- cycle node[left=6pt, above] {\(0\)};
%		\draw[gray!40] (1.5,0) -- (1.5,1) -- (2.5,1) -- cycle node[below] {\(C_{i, k}\)};
%		\draw[color=BleuProfondIRA] (1.5,0.2) -- (1.5,1) -- (2.3,1) -- cycle node {};
%		\node at (2.5,0.3) {\(\times\)};
%		\draw[color=BleuProfondIRA] (3,1) -- (3.2,1) -- (3.8,0.2) -- (3.6,0.2) -- cycle node[below] {\(f_{ k}\)};	
%		\draw [color=BleuProfondIRA,decorate,decoration={brace,amplitude=10pt,mirror,raise=0pt},yshift=0pt] (3.8,0)	-- (3.8,1);
%		\node at (4.3,0.3) {\(\oslash\)};
%		\draw[gray!40] (4.5,0) -- (4.5,1) -- (5.5,1) -- cycle node[below] {\(\sqrt{C_{i, k}}\)};
%		\draw[color=BleuProfondIRA] (4.5,0.2) -- (4.5,1) -- (5.3,1) -- cycle node {};
%		\draw [color=BleuProfondIRA!30,decorate,decoration={brace,amplitude=10pt,raise=1pt,mirror},yshift=-15] (0,0)	-- (5.5,0);
%		\draw[color=BleuProfondIRA,->] (2.75,-1) -- (2.75,-2) node[pos=0.5,right,align=left] {\tiny\(\sum\) of square\\\tiny by col};	
%		\draw[BleuProfondIRA]  (2.25,-2) rectangle (3.25,-2.3)
%		node[below] {\(\hat{\sigma}_{k}^{2}\)}; 
%	\end{tikzpicture}
%\end{center}
%où \(\oslash\) désigne la division de Hamadard désigné par \(/\) en iml (ou division élément par élément de deux matrices).
%
%Il n'y a pas besoin d'enlever les valeurs de la diagonale de la première matrice car on ajoute des valeurs manquantes avec la deuxième matrice.

		

The third assumption concerns the distribution of \(R_{i}\) to be able to easily construct confidence intervals on the estimated reserves. If the distribution is normal, with mean the estimated value \(\hat{R}_{i}\) and standard deviation given by the standard error \(\operatorname{se}\left(\hat{R}_{i}\right)\). A confidence interval at \(95 \%\) is then given by \(\left[\hat{R}_{i}-2 \operatorname{se}\left(\hat{R}_{i}\right), \hat{R}_{i}+2 \operatorname{se}\left(\hat{R}_{i}\right)\right]\).

If the distribution is assumed to be lognormal, the bounds of a \(95\%\) confidence interval will then be given by
\[
\left[\hat{R}_{i} \exp \left(\frac{-\sigma_{i}^{2}}{2}-2 \sigma_{i}\right), \hat{R}_{i} \exp \left(\frac{-\sigma_{i}^{2}}{2}+2 \sigma_{i}\right)\right]
\]
		
\end{f}
\hrule

\begin{f}[The collective risk model]
Is the collective model the basic model in non-life actuarial science?
	 \(X_{i}\) denotes the amount of the \(i^th\) claim, \(N\) denotes the number of claims and \(S\) the total amount over a year
	\[
	S=\sum_{i=1}^{N} X_{i}
	\]
	knowing that \(S=0\) when \(N=0\) and that \(\left\{X_{i}\right\}_{i=1}^{\infty}\) is a sequence \(iid\) and  \(N \perp \left\{X_{i}\right\}_{i=1}^{\infty}\). 
The difficulty is to obtain the distribution of \(S\), even though \(\E [N]\) is not large in the TCL sense.
\end{f}


\begin{f}[The distribution of \(S\)]
	
Either \(G(x)=\mathbb{P}(S \leq x)\), \(F(x)=\mathbb{P}\left(X_{1} \leq x\right)\) , and \(p_{n}=\mathbb{P}(N=n)\) so that \(\left\{p_{n}\right\}_{n=0}^{\infty}\) be the probability function for the number of claims.
	
	\[
	\{S \leq x\}=\bigcup_{n=0}^{\infty}\{S \leq x \text { and } N=n\}
	\]
	\[
	\mathbb{P}(S \leq x \mid N=n)=\mathbb{P}\left(\sum_{i=1}^{n} X_{i} \leq x\right)=F^{n *}(x)
	\]
	So, for \(x \geq 0\)
	
	\begin{equation*}\label{GxCollectif}
		G(x)=\sum_{n=0}^{\infty} p_{n} F^{n *}(x)
	\end{equation*}
	where \(F^{n *}\) denotes the \(n^{th}\) convolution, unfortunately it does not exist in closed form for many distributions.
	
If \(E[X]=m\)
	\[
	E[S]=E\left[N m\right]=E[N] m
	\]
	This result is very interesting because it indicates that the total expected amount of claims is the product of the expected number of claims and the expected amount of each claim. Similarly, using the fact that \(\left\{X_{i}\right\}_{i=1}^{\infty}\) are independent random variables,
	\[
	V[S \mid N=n]=V\left[\sum_{i=1}^{n} X_{i}\right]=\sum_{i=1}^{n} V\left[X_{i}\right]
	\]
	\[
	\begin{aligned}
		V[S] &=E[V(S \mid N)]+V[E(S \mid N)] \\
		&=E[N] V\left[X_{i}\right]+V[N] m^{2}
	\end{aligned}
	\]    
\end{f}

\hrule


\begin{f}[The distribution class \((a, b, 0)\)]
	
	A counting distribution is said to be \((a, b, 0)\) if its probability function \(\left\{p_{n}\right\}_{n=0}^{\infty}\) can be calculated recursively from the formula
	\[
	p_{n}=\left(a+\frac{b}{n}\right) p_{n-1}
	\]
	for \(n=1,2,3, \ldots,\) where \(a\) and \(b\) are constants.
	
	There are exactly three non-trivial distributions in the class \((a, b, 0)\), namely Poisson, binomial and negative binomial. Here are the values of \(a\) and \(b\) for the main distributions \((a, b, 0)\):
	\begin{center}
		\begin{tabular}{ccc} 
			& \(a\) & \(b\) \\
			\hline\(\mathcal{P}_{ois}(\lambda)\) & 0 & \(\lambda\) \\
			\(\mathcal{B}_{in}(n, q)\) & \(-q /(1-q)\) & \((n+1) q /(1-q)\) \\
			\(\mathcal{N}\mathcal{B}_{in}(k, q)\) & \(1-q\) & \((1-q)(k-1)\) \\
			\(\mathcal{G}_{eo}( q)\) & \(1-q\) & 0 \\
			\textit{    Panjer's distribution} & \(\frac{\lambda}{\alpha+\lambda}\) &    \(\frac {(\alpha -1)\lambda }{\alpha +\lambda }\) \\
			\hline 
		\end{tabular}
		
	\end{center}
	
	{\footnotesize\color{OrangeProfondIRA} The geometric law is a special case of the negative binomial where k=1.}
\end{f}

\begin{f}[Panjer's aggregation algorithm]
	
	The \textbf{Panjer algorithm} aims at estimating the distribution of a composite cost-frequency law under particular conditions.
	\begin{itemize}
		\item \((X_i)_{i=1}^{N}\) \(iid\) discrete defined on \(\{0,h,2h,3h...\}\)
		\item the law of numbers in the so-called class \((a,b,0)\)
	\end{itemize}
	

Since we now assume that the individual claim amounts are distributed over the non-negative integers, it follows that \(S\) is also distributed over the non-negative integers.
As \(S=\sum_{i=1}^{N} X_{i}\), it follows that \(S=0\) if \(N=0\) or if \(N=n\) and \(\sum_{i=1}^{n} X_{i}=0 . \) As \(\sum_{i=1}^{n} X_{i}=0\) only if each \(X_{i}=0,\) it follows by independence that
	\[
	\mathbb{P}\left(\sum_{i=1}^{n} X_{i}=0\right)=f_{0}^{n}
	\]
	
\begin{equation*}
		\begin{cases}\label{Panjer0}\displaystyle
		g_{0}=p_{0}+\sum_{n=1}^{\infty} p_{n} f_{0}^{n}=P_{N}\left(f_{0}\right){\color{OrangeProfondIRA}\text{ if }a \ne 0},\\
	g_0=p_0\cdot \exp(f_0 b){\color{OrangeProfondIRA}\text{ if }a = 0,}\\
		g_{k}=\frac{1}{1-a f_{0}} \sum_{j=1}^{k}\left(a+\frac{b j}{k}\right) f_{j} g_{k-j}
		\end{cases}	
	\end{equation*}
\(g{x}\) is expressed in terms of \(g{0}, g{1}, \ldots, g_{x-1},\) so that the calculation of the probability function is recursive. In all practical applications of this formula, a computer is required to perform the calculations. However, the advantage of Panjer's recursion formula over the formula for \(g_{x}\) is that there is no need to compute convolutions, which is much more computationally efficient.
Panjer's algorithm requires the discretization of the variable \(X_i\).
\end{f}



%\begin{f}{Algorithme de Panjer}{Exemple}
%\begin{figure}
%	\includegraphics[width=0.90\textwidth]{../../Reinsurance_M2/Graph/Expba07.pdf}
%\end{figure}
%The following example shows the approximated density of 
%\(\scriptstyle S = \sum_{i=1}^N X_i\) where \(\scriptstyle N \sim \text{NegBin}(3.5,0.3)\) and \(\scriptstyle X \sim \text{Frechet}(1.7,1)\) with lattice width \(h = 0.04\).
%\end{f}

\begin{f}[Panjer and Poisson's Law]
	When the frequency follows a Poisson distribution, this implies that {\color{OrangeProfondIRA} \(a=0\) and \(b=\lambda\).}
	The result is simplified :
	\[
\begin{cases}
		g_0= e^{-\lambda (1-f_0)}
	\\
	g_k=\frac{\lambda}{k}\sum_{j=1}^k j . f_j  . g_{k-j}
\end{cases}	
\]
	
\end{f}

\begin{f}[Panjer and Pollaczeck-Khinchina-Beekman]
	
	Let \(\tau_1\) be the first instant where \(R_t<\kappa(=\kappa_0)\). We then set \(L_1=\kappa -R_{\tau_1}\).
	We restart the process with \(\kappa_1=\kappa_0-R_{\tau_1}\) to find \(\tau_2\) and \(L_2=\kappa_1 -R_{\tau_2}\).
	Continuing in this way, we see that :
	\[
	M=\sup_{t\geq 0}\{ S_t -ct \} =\sum_{k=1}^{K}L_k
	\]
	where \(K\sim \mathcal{G}eo(q)\) with \(q=1-\psi(0)\).  
	Noting that the variables \(\left( L_k\right)_{1\leq k\leq K}\) are \(iid\) (\(F\)), we then have \(\psi(k)=\mathbb{P}[M>\kappa]\) given by the Pollaczeck-Khinchine-Beekman formula.
	
	
	The representation
	\[
	\psi(\kappa)=\mathbb{P}\left[\sum_{j = 1}^{K}L_j>\kappa \right] 
	\]
	allows to evaluate the probability of ruin on an infinite horizon using the Panjer algorithm.
	
\end{f}





