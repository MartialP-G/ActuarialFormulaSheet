% !TeX root = ActuarialFormSheet_MBFA-en.tex
% !TeX spellcheck = en_US
\def\scaleBS{.95}

\begin{f}[Market Functioning]
	
The \textbf{Exchange} – a place of exchange – enables, in fact, the physical meeting between capital demanders and suppliers.
The main listings concern \textbf{equities}, \textbf{bonds} (\engl{Fixed Income}), and \textbf{commodities}.
Listed are \textbf{securities} such as stocks or bonds, \textbf{funds} (\engl{Exchange Traded Funds} that replicate equity indices, ETC or ETN that replicate more specific indices or commodities, SICAV or FCP, subscription warrants), \textbf{futures contracts}, \textbf{options}, \textbf{swaps}, and \textbf{structured products}.

The \textbf{Financial Markets Authority} (\href{https://www.amf-france.org/fr}{AMF}) oversees:
\begin{itemize}
	\item the protection of invested savings; 
	\item the information of investors; 
	\item the proper functioning of the markets.
\end{itemize}

\textbf{\href{https://www.euronext.com/fr}{Euronext}} (including
\href{https://www.euronext.com/en/markets/amsterdam}{Amsterdam}, 
\href{https://www.euronext.com/en/markets/brussels}{Brussels}, 
\href{https://www.euronext.com/en/markets/lisbon}{Lisbon}, 
and \href{https://www.euronext.com/en/markets/paris}{Paris}) is the main stock exchange in France.
%
Its competitors include \textbf{\href{https://www.deutsche-boerse.com/dbg-en/}{Deutsche Börse}} 
(which includes \href{https://www.eurex.com/}{Eurex}, 
\href{https://www.eex.com/en/}{EEX}) in Europe,
%
or \textbf{\href{https://www.ice.com/}{ICE}} (which includes
\href{https://www.nyse.com/index}{NYSE (2012)}, 
\href{https://www.ice.com/about/history}{NYBOT (2005)}, 
\href{https://www.ice.com/futures-europe}{IPE (2001), LIFFE})
%
and \textbf{\href{https://www.cmegroup.com/}{CME Group}} (including
\href{https://www.cmegroup.com/company/cbot.html}{CBOT}, 
\href{https://www.cmegroup.com/company/nymex.html}{NYMEX}, 
\href{https://www.cmegroup.com/company/comex.html}{COMEX}) in the United States.

The \textbf{over-the-counter market} (\textbf{OTC}, \textit{Over-The-Counter}) represents a major share of volumes traded outside organized markets.
Since the Pittsburgh G20 (2009), certain standardized OTC derivatives must be cleared through a central entity.
These \textbf{CCPs} (\textit{Central Counterparties}) thus play the role of \textbf{clearinghouses}: they replace the bilateral contract with two contracts between each party and the CCP.

\begin{tikzpicture}[scale=.47]
	\node[businessman,minimum size=1.75cm,monogramtext=MPG,tie=OrangeProfondIRA, shirt=BleuProfondIRA,skin=white,hair=OrangeProfondIRA] at (-8,0) (MPG){Short position};
	\node[businessman,mirrored,minimum size=1.75cm,monogramtext=SPG,shirt=FushiaIRA, tie=black,skin=FushiaIRA, hair=black] at (8,0) (SPG){Long position};
	%  	
	\begin{scope}[yscale=-0.03, xscale=0.03, shift={(-150,-200)}]
		%  		
		%Usine
		\draw [OrangeMoyenIRA!40,fill] (65,132) rectangle (195,225)  node [pos=0.5,text width=1.8cm, align=center] {\color{black}\bfseries Clearinghouse} ;
		%Straight Lines [id:da13081779728777443] 
		%	\draw (65,272) -- (474,272) -- (474,320);
		%Shape: Polygon [id:ds025531371089408395] 
		\draw   [GrisLogoIRA] (195,272) -- (65,272) -- (65,129) -- (85,129) -- (85,20) -- (117,20)  -- (117,52)-- (175,52) -- (175,129) -- (195,129) -- cycle ;
		%Shape: Grid [id:dp09099790819019526] 
		%	\draw  [draw opacity=0] (117,129) -- (195,129) -- (195,169) -- (117,169) -- cycle ; 
		\draw  [GrisLogoIRA] (95,100) -- (95,60)(105,100) -- (105,60)(115,100) -- (115,60)(125,100) -- (125,60)(135,100) -- (135,60)(145,100) -- (145,60)(155,100) -- (155,60)(165,100) -- (165,60) ; 
		\draw  [GrisLogoIRA]  (85,100) -- (175,100)(85,90) -- (175,90)(85,80) -- (175,80)(85,70) -- (175,70)(85,60) -- (175,60) ; 
		%
		\draw  (65,175) node (a) {} ;
		\draw  (195,175) node (b) {} ;
		%Shape: Rectangle [id:dp30583722611654374] 
	\end{scope}
	% 	
	\draw[->,>=latex, thick] (MPG) to [bend left]  node[pos=0.4, above]{Short} (a);
	\draw[<-,>=latex, thick,OrangeProfondIRA] (MPG) to [bend right]  node[pos=0.4, below=.2]{payment} (a);
	\draw[->,>=latex, thick,FushiaIRA] (SPG) to [bend left] node[pos=0.4, below=.2]{No payment}  (b);
	\draw[->,>=latex, thick] (SPG) to [bend right] node[pos=0.4, above]{Long} (b);
	%  	
\end{tikzpicture}
\medskip

\end{f}
\hrule

\begin{f}[The Money Market]
Short-term interest-bearing securities, traded on money markets, are generally at \textbf{discounted interest}. 
Nominal rates are then annual and calculations use \textbf{proportional rates} to adjust for durations less than one year.
These securities are quoted or valued according to the discount principle and with a Euro-30/360 calendar convention.

In the American market, public debt securities are called:
Treasury Bills (T-bills): ZC < 1 year, Treasury Notes (T-notes): ZC < 10 years,
Treasury Bonds (T-bonds): coupon bonds with maturity > 10 years.


They are mainly:
\begin{itemize}
	\item \textbf{BTF (fixed-rate Treasury bills, France):} issued at 13, 26, 52 weeks, discounted rate, weekly auction, nominal 1~€, settlement at T+2.
	
	\item \textbf{Treasury bills > 1 year:} same rules as bonds (see next section).
	
	\item \textbf{Certificates of deposit (CDN):} securities issued by banks at fixed/discounted rate (short term) or variable/post-discounted rate (long term), also called BMTN.
	
	\item \textbf{Eurodollars:} USD deposits outside the USA, formerly indexed on LIBOR, now declining.
	
	\item \textbf{Commercial paper:} unsecured short-term securities issued by large companies to finance their cash flow.
\end{itemize}

\textbf{Price calculations of a fixed-rate Treasury bill with discounted interest}

In the case of a discounted interest security according to the Euro-30/360 convention, the discount $D$ is expressed as :
$$
D=F \cdot d \cdot \frac{k}{360}
$$
where $F$ denotes the nominal value, $d$ the annual discount rate used to value the discounted security, and $k$ the maturity in days.

If the discount rate $d$ is known, then the price $P$ is expressed as :
$$
P=F-D=F\left(1-d \cdot \frac{k}{360}\right)
$$
Similarly, if the price $P$ is known, then the discount rate $d$ is derived as :
$$
d=\frac{F-P}{F} \cdot \frac{360}{k}
$$


The main Futures Contracts: Federal Funds Futures (US),
Three-Month SOFR Futures (US),
ESTR Futures (EU),
SONIA Futures (UK),
Euribor Futures (EU).

\end{f}
\hrule



\begin{f}[Bond Market]
	
\textbf{Bonds} are long-term debt securities in which the issuer (central or local government, bank, borrowing company) promises the bondholder (the lender) to pay interest (\textbf{coupons}) periodically and to repay the \textbf{nominal value} (or face value, or principal) at maturity.
As mentioned in the previous section, \textbf{Treasury bills with a maturity greater than one year} will be treated as bonds with maturities under 5 years because their functioning is similar.


%Les \textbf{coupons} : sont payés régulièrement à la fin des périodes de coupon (annuelles ou semestrielles) jusqu'à la date d'échéance.

\textbf{Zero-coupon bonds}: pay only the nominal value at maturity. With $E$ the issue price and $R$ its redemption value:
	
\begin{center}
\begin{tikzpicture}[scale=0.85]
    % Draw the x-axis and y-axis.
    \def\w{11}
    \def\n{9}
    \draw[ line width=1, color=OrangeProfondIRA, arrows={-Stealth[length=4, inset=0]}] (0,0) -- (0,-0.4909) node[above left] (A) {$E$};	
    \foreach \y in  {0,...,3} {
        \draw (\y,0) -- (\y,-0.1);
        \ifthenelse{\y>0 }{	\node[below] at (\y,-0.1) {\tiny $ \scriptstyle \y$};
        }{
            \node[above] at (\y,-0.1) {\tiny 0};}
    }
    \foreach \y in  {-3,...,1} {
        \draw (\y+\n,0) -- (\y+\n,-0.1);
        \ifthenelse{\y<0 }{\node[below] at (\y+\n,-0.1) {\tiny $\scriptstyle n \y$};}{}
        \ifthenelse{\y=0 }{\node[below] at (\y+\n,-0.1) {\tiny $\scriptstyle n $};
            \draw[ line width=1, color=OrangeProfondIRA, arrows={-Stealth[length=4, inset=0]}] (\y+\n,0) -- (\y+\n,1) node[below right] (B) {$R$};}{}
        \ifthenelse{\y>0}{	\node[below] at (\y+\n,-0.1) {\tiny $\scriptstyle n+\y$};}{}
    }
    \draw[ line width=1] (-.25,0) -- (4,0);
    \draw[ line width=1, dashed] (4,0) -- (5,0);
    \draw[arrows={-Stealth[length=4, inset=0]}, line width=1] (5,0) -- (\w,0);
\end{tikzpicture}
\end{center}

\textbf{Coupon bonds}:
\textbf{Fixed-rate bonds} have a coupon rate that remains constant until maturity. Assuming a \emph{bullet} repayment, with $E$ the issue price, $c$ the coupons, and $R$ the redemption value, it can be illustrated as follows:
\begin{center}
\begin{tikzpicture}[scale=0.85]
    % Draw the x-axis and y-axis.
    \def\w{11}
    \def\n{9}
    \draw[ line width=1, color=OrangeProfondIRA, arrows={-Stealth[length=4, inset=0]}] (0,0) -- (0,-1.8) node[above left] (A) {$E$};	
    \foreach \y in  {0,...,3} {
        \draw (\y,0) -- (\y,-0.1);
        \ifthenelse{\y>0 }{	\node[below] at (\y,-0.1) {\tiny $ \scriptstyle \y$};
        \draw[ line width=1, color=OrangeProfondIRA, arrows={-Stealth[length=4, inset=0]}] (\y,0) -- (\y,.25) node [above] {$c$};
        }{
            \node[above] at (\y,-0.1) {\tiny 0};}
        }
    \foreach \y in  {-3,...,1} {
        \draw (\y+\n,0) -- (\y+\n,-0.1);
        \ifthenelse{\y<0 }{\node[below] at (\y+\n,-0.1) {\tiny $\scriptstyle n \y$};
        \draw[ line width=1, color=OrangeProfondIRA, arrows={-Stealth[length=4, inset=0]}] (\y+\n,0) -- (\y+\n,.25) node [above] {$c$};}{}
        \ifthenelse{\y=0 }{\node[below] at (\y+\n,-0.1) {\tiny $\scriptstyle n $};
        \draw[ line width=1, color=OrangeProfondIRA, arrows={-Stealth[length=4, inset=0]}] (\y+\n,0) -- (\y+\n,2) node[below right] (B) {$R$};
        \draw[ line width=1, color=OrangeProfondIRA, arrows={-Stealth[length=4, inset=0]}] (\y+\n,2) -- (\y+\n,2.25) node[right] (B) {$c$};}{}
        \ifthenelse{\y>0}{	\node[below] at (\y+\n,-0.1) {\tiny $\scriptstyle n+\y$};}{}
    }
    \draw[ line width=1] (-.25,0) -- (4,0);
    \draw[ line width=1, dashed] (4,0) -- (5,0);
    \draw[arrows={-Stealth[length=4, inset=0]}, line width=1] (5,0) -- (\w,0);
\end{tikzpicture}
\end{center}

\textbf{Indexed bonds} (inflation-linked bonds) have coupons and sometimes also the nominal value indexed to inflation or another economic indicator, such as Treasury Assimilable Bonds
indexed to inflation (OATi). The values of $c$ vary.
        
Bonds with \textbf{floating rate}, \textbf{variable rate}, or \textbf{resettable rate}: have a coupon rate linked to a reference interest rate (for example, the euro short-term rate (\href{https://www.cmegroup.com/markets/interest-rates/stirs/euro-short-term-rate.quotes.html}{€STR})).

\textbf{Perpetual bonds} have no maturity date; the principal is never repaid.

A distinction is often made between government bonds (Treasury bonds) and
corporate bonds issued by private companies.



A bond is mainly defined by a \textbf{nominal value} $F$ (Face Value), the \textbf{nominal rate} $i$, its duration or \textbf{maturity} $n$.
In the default case, the bondholder lends the amount $E=F$ at issuance at time 0, receives each year a coupon $c = i \times F$, and at $n$, the principal or capital $R=F$ is returned.
When $E=F$, the issue is said to be at par, and when $R=F$, the redemption is said to be at par.

The price of a bond is determined by the present value of the expected future cash flows (coupons and principal repayment) discounted at the market yield rate $r$.

The price calculation of bonds simply relies on the present value formula:
\[
VP = \sum_{k=1}^{n} \frac{c}{(1 + r)^k} + \frac{F}{(1 + i)^n}
 \]
where:
\begin{itemize}
    \item $PV$: price or present value\index{Present Value} of the bond,
	\item $r$: market interest rate for the relevant maturity.
\end{itemize}


For bonds with periodic coupons, the coupon is divided by the number of periods ($m$) per year and the formula becomes:
\[ 
PV = \sum_{k=1}^{mn} \frac{c/m}{(1 + r^{(m)})^k} + \frac{R}{(1 + + r^{(m)})^{mn}}
 \]
where $c/m$ represents the periodic coupon payment and $r^{(m)}$ the periodic interest rate.

The bond yield is the value $r^{(m)}$, the equivalent rate of $r$ over $m$ periods in the year, which equates the present value $VP_r$ with the current or market price of this bond.

\textbf{The quotation of a bond} is given as a percentage. Thus, a quotation of 97.9 on Euronext indicates a quoted value of $97.9 / 100 \times F$.  
It is quoted excluding \textbf{accrued coupons}, the portion of the next coupon to which the seller is entitled if the bond is sold before the payment of that coupon.
\end{f}
\hrule

\begin{f}[Duration \& Convexity]
	
The Macaulay duration:
\[ 		
D = \sum_{t=1}^{n} t \cdot w_t, \quad \text{où} \quad w_t = \frac{PV(C_t)}{P}.
 \]	
If the payment frequency is $k$ per year, the duration expressed in years is obtained by dividing by $k$.
The modified duration $D^*$:
\[ 	
D^* = \frac{D}{1 + i}.
 \]
Which allows approximating the portfolio change $\Delta P$ in case of interest rate changes $\Delta_i$
\[ 
\Delta P \approx -P\ D^* \Delta_i 
 \]
Similarly, the convexity
\[ 	
C = \frac{1}{P(i)} \times \frac{d^2 P(i)}{di^2},
 \]
which allows refining the approximation of $\Delta P$
\[ 	
P(i + \Delta_i) \approx P(i) \left( 1 -D^*\Delta_i + \frac{1}{2} C (\Delta_i)^2 \right).
 \]
\end{f}
\hrule



\begin{f}[CAPM]
	Capital Asset Pricing Model:

\[
E(r_i) = r_f + \beta_i (E(r_m) - r_f)
\]

\begin{itemize}
	\item \( E(r_i) \) is the expected return of asset \( i \),
	\item \( r_f \) is the risk-free rate,
	\item \( E(r_m) \) is the expected market return,
	\item \( \beta_i \) is the sensitivity coefficient of asset \( i \) with respect to market variations.
\end{itemize}


The coefficient \( \beta_i \) measures the volatility of asset \( i \) relative to the overall market.

\end{f}
\hrule


\begin{f}[Marché des dérivés]

Un \textbf{contrat dérivé} (ou actif contingent) est un instrument financier dont la valeur dépend d’un actif ou d’une variable sous-jacente. Les options font partie des contrats dérivés.


Une \textbf{option} est un contrat donnant le droit (sans obligation) d’acheter (call) ou de vendre (put) un actif sous-jacent à un prix fixé (prix d’exercice), à une date future, contre paiement d’une prime. 
L’acheteur (position longue) paie la prime ; le vendeur (position courte) la reçoit.L'\textbf{option européenne}  (exercice possible uniquement à l’échéance) et   
 \textbf{option américaine} (exercice possible à tout moment jusqu’à l’échéance).

Les options cotées sur actions sont appelées \textit{stock options}.

\end{f}
\hrule


\begin{f}[Les stratégies simples]

\ %

%\textbf{La position longue sur l'option d'achat}

Avec $T$ l'échéance, $K$ le prix d'exercice, $S$ ou $S_T$ le sous-jacent à l'échéance, le retour (\engl{payoff}) est de $\max (0, S_T-K)=( S_T-K)^{+}$.
En notant $C$ la prime , le profit réalisé est de $\max (0, S_T-K)-C$, avec un profit si  ($S_T<V_{PM}=K + C$)  ($PM$ pour \textbf{point mort}).

		\begin{tikzpicture}[yscale=.75]
\def\riskfreeBS{0.05}
\def\xminBS{5}
\def\xmaxBS{15}
\def\PxExerciceBS{10}
\def\sigmaBS{0.2}
\def\TBS{0.75}
\def\PremiumBS{{BSCall(10,{\PxExerciceBS},{\riskfreeBS},{\TBS},{\sigmaBS})}}
\begin{axis}[ 	extra tick style={tick style=BleuProfondIRA},
clip=false,
axis on top,
axis lines=middle, axis line style={BleuProfondIRA,thick,->},
scale only axis, xmin={\xminBS},xmax={\xmaxBS},enlarge x limits=0.05,
enlarge y limits=0.08,
color=BleuProfondIRA,
%		ylabel near ticks,
ylabel={Profit},
x label style={at={(axis cs:\xmaxBS+.1,0)},anchor=north east},
xlabel={underlying (\(T\))},
%		    x label style={at={(axis description cs:0.5,-0.1)},anchor=north},
%		y label style={at={(axis description cs:-0.1,.5)},rotate=90,anchor=south},
ytick=\empty,
xtick=\empty,
extra y ticks ={0},
extra y tick labels={{0}},
extra x ticks ={\PxExerciceBS},
extra x tick labels={{\(E\)}},
extra x tick style={color=BleuProfondIRA,
	tick label style={yshift=7mm}	},
title ={ \textbf{Long Call}},
%			title style={yshift=-5mm},
%	legend style={draw=none,
	%		legend columns=-1,
	%		at={(0.5,1)},
	%		anchor=south,
	%		outer sep=1em,
	%		node font=\small,
	%	},
]
%		
\addplot[name path=A,BleuProfondIRA,thick,domain={{\xminBS}:{\xmaxBS}}, samples=21,dashdotted] {Call(x,\PxExerciceBS,\PremiumBS)} 
node [pos=0.15,yshift=3mm,color=OrangeProfondIRA] {Losses}
node [pos=.8,yshift=-15mm,color=BleuProfondIRA] {Earnings};	
%\draw[BleuProfondIRA,thick] \OV{\PxExerciceBS}{\PremiumBS}{0.4}{\xminBS}{\xmaxBS} ;
%		\addplot[name path=Option,BleuProfondIRA,thick,domain={250:350}, 			 		samples=10,dashdotted,smooth] {BSPut(x,\PxExerciceBS,\riskfreeBS,\TBS,\sigmaBS)} ;	
\path [save path=\xaxis,name path=xaxis]
({\xminBS},0)		-- ({\xmaxBS},0)		;
\addplot [bottom color=OrangeProfondIRA!50, top color=OrangeProfondIRA!10] fill between [
of=A and xaxis,
split,
every segment no 1/.style=
{top color = BleuProfondIRA!50, bottom color=BleuProfondIRA!10}] ;
%\draw[use path=\xaxis, ->,OrangeProfondIRA,thick];
\draw [fill]  ({-\PremiumBS*(-1)+\PxExerciceBS},0) circle (1mm) node [above=5mm] {\(S_{PM}\)};
\draw[BleuProfondIRA, thin] ({\PxExerciceBS},0) -- ({\PxExerciceBS},{-\PremiumBS});
\node at ({\xminBS+0.25*(\xmaxBS-\xminBS)},{\pgfkeysvalueof{/pgfplots/ymin}})
{OUT};
\node at ({\PxExerciceBS},{\pgfkeysvalueof{/pgfplots/ymin}})
{AT};
\node at ({\xminBS+0.8*(\xmaxBS-\xminBS)},{\pgfkeysvalueof{/pgfplots/ymin}})
{IN};
\draw [<->, xshift=-5mm] ({\pgfkeysvalueof{/pgfplots/xmin}},0) -- ({\pgfkeysvalueof{/pgfplots/xmin}},-\PremiumBS) node [pos=0.5, xshift=-2.5mm, rotate=90] {Premium};
\end{axis}
%
\end{tikzpicture}


\medskip

%\textbf{La position courte sur l'option d'achat}


À l'échéance, le retour est de $\min (0,K- S_T)=-\max(0, S_T-K)=-( S_T-K)^{+}$ et le profit réalisé est de $C-\max (0, S_T-K)$.

\medskip

			\begin{tikzpicture}[yscale=.75]
		\def\riskfreeBS{0.05}
		\def\xminBS{5}
		\def\xmaxBS{15}
		\def\PxExerciceBS{10}
		\def\sigmaBS{0.2}
		\def\TBS{0.75}
		\def\PremiumBS{{BSCall(10,{\PxExerciceBS},{\riskfreeBS},{\TBS},{\sigmaBS})}}
\begin{axis}[ 
	clip=false,
	axis on top,
	axis lines=middle, axis line style={BleuProfondIRA,thick,->},
	scale only axis, xmin={\xminBS},xmax={\xmaxBS},enlarge x limits=0.05,
	enlarge y limits=0.125,
	color=BleuProfondIRA,
	%		ylabel near ticks,
	ylabel={Profit},
	x label style={={at={(current axis.right of origin)}}},
	%    x label style={at={(axis description cs:1,-0.1)},anchor=south},
	%		x label style={at={(1,0.5)}},
	xlabel={ss-jacent ($T$)},
	%		    x label style={at={(axis description cs:0.5,-0.1)},anchor=north},
	%		y label style={at={(axis description cs:-0.1,.5)},rotate=90,anchor=south},
	ytick=\empty,
	xtick=\empty,
	extra y ticks ={0},
	extra y tick labels={{0}},
	extra x ticks ={\PxExerciceBS},
	extra x tick labels={{$E$}},
	extra x tick style={color=BleuProfondIRA,
		tick label style={yshift=-0mm}	},
	title ={ \textbf{Short Call}},
	title style={yshift=10mm},
	%	legend style={draw=none,
		%		legend columns=-1,
		%		at={(0.5,1)},
		%		anchor=south,
		%		outer sep=1em,
		%		node font=\small,
		%	},
	]
	%		
\addplot[name path=A,BleuProfondIRA,thick,domain={{\xminBS}:{\xmaxBS}}, samples=21,dashdotted] {-Call(x,\PxExerciceBS,\PremiumBS)} 
node [pos=0.15,yshift=-4mm,color=BleuProfondIRA] {Gains}
node [pos=.8,yshift=10mm,color=OrangeProfondIRA] {Pertes};	
	%\draw[BleuProfondIRA,thick] \OV{\PxExerciceBS}{\PremiumBS}{0.4}{\xminBS}{\xmaxBS} ;
	%		\addplot[name path=Option,BleuProfondIRA,thick,domain={250:350}, 			 		samples=10,dashdotted,smooth] {BSPut(x,\PxExerciceBS,\riskfreeBS,\TBS,\sigmaBS)} ;	
	\path [save path=\xaxis,name path=xaxis]
	({\xminBS},0)		-- ({\xmaxBS},0)		;
	\addplot [bottom color=OrangeProfondIRA!50, top color=OrangeProfondIRA!10] fill between [
	of=A and xaxis,
	split,
	every segment no 0/.style=
	{top color = BleuProfondIRA!50, bottom color=BleuProfondIRA!10}] ;
	%\draw[use path=\xaxis, ->,OrangeProfondIRA,thick];
	\draw [fill]  ({-\PremiumBS*(-1)+\PxExerciceBS},0) circle (1mm) node [above=5mm] {$S_{PM}$};
	\draw[BleuProfondIRA, thin] ({\PxExerciceBS},0) -- ({\PxExerciceBS},{\PremiumBS});
	\node at ({\xminBS+0.25*(\xmaxBS-\xminBS)},{\pgfkeysvalueof{/pgfplots/ymin}+1})
	{OUT};
	\node at ({\PxExerciceBS},{\pgfkeysvalueof{/pgfplots/ymin}+1})
	{AT};
	\node at ({\xminBS+0.8*(\xmaxBS-\xminBS)},{\pgfkeysvalueof{/pgfplots/ymin}+01})
	{IN};
	\draw [<->, xshift=-5mm] ({\pgfkeysvalueof{/pgfplots/xmin}},0) -- ({\pgfkeysvalueof{/pgfplots/xmin}},\PremiumBS) node [pos=0.5, xshift=-2.5mm, rotate=90] {Prime};
\end{axis}
%
	\end{tikzpicture}

\medskip


%\textbf{La position longue sur l'option de vente}

À l'échéance, le retour est de $\max (0,K- S_T)=(K- S_T)^{+}$.
En notant $P$ la prime du put, le profit réalisé est de $\max (0,K- S_T)-P$, positif si$V_{PM}=K -P<S_T$.

\medskip

		\begin{tikzpicture}[yscale=.75]
	\def\riskfreeBS{0.05}
	\def\xminBS{5}
	\def\xmaxBS{15}
	\def\PxExerciceBS{10}
	\def\sigmaBS{0.2}
	\def\TBS{0.75}
	\def\PremiumBS{{BSCall(10,{\PxExerciceBS},{\riskfreeBS},{\TBS},{\sigmaBS})}}
	\begin{axis}[ 
		clip=false,
		axis on top,
		axis lines=middle, axis line style={BleuProfondIRA,thick,->},
		scale only axis, xmin={\xminBS},xmax={\xmaxBS},enlarge x limits=0.05,
		enlarge y limits=0.1,
		color=BleuProfondIRA,
		%		ylabel near ticks,
		ylabel={Profit},
		x label style={={at={(current axis.right of origin)},below=5mm}},
		%    x label style={at={(axis description cs:1,-0.1)},anchor=south},
		%		x label style={at={(1,0.5)}},
		xlabel={underlying ($T$)},
		%		    x label style={at={(axis description cs:0.5,-0.1)},anchor=north},
		%		y label style={at={(axis description cs:-0.1,.5)},rotate=90,anchor=south},
		ytick=\empty,
		xtick=\empty,
		extra y ticks ={0},
		extra y tick labels={{0}},
		extra x ticks ={\PxExerciceBS},
		extra x tick labels={{$E$}},
		extra x tick style={color=BleuProfondIRA,
			tick label style={yshift=7mm}	},
		title ={ \textbf{Long Put}},
%		title style={yshift=0mm},
		%	legend style={draw=none,
			%		legend columns=-1,
			%		at={(0.5,1)},
			%		anchor=south,
			%		outer sep=1em,
			%		node font=\small,
			%	},
		]		%		
		\addplot[name path=A,BleuProfondIRA,thick,domain={{\xminBS}:{\xmaxBS}}, samples=21,dashdotted] {Put(x,\PxExerciceBS,\PremiumBS)} 
		node [pos=0.15,yshift=-15mm,color=BleuProfondIRA] {Earnings}
		node [pos=.75,yshift=4mm,color=OrangeProfondIRA] {Losses};	
		%\draw[BleuProfondIRA,thick] \OV{\PxExerciceBS}{\PremiumBS}{0.4}{\xminBS}{\xmaxBS} ;
		%	\addplot[BleuProfondIRA,thick,dashdotted] {Call(x,50,2)};
		%	\addplot[BleuProfondIRA,thick] {Put(x,50,3)+Call(x,50,2)};
		%\addplot[name path=Option,BleuProfondIRA,thick,domain={250:350}, 			 		samples=10,dashdotted,smooth] {BSPut(x,\PxExerciceBS,\riskfreeBS,\TBS,\sigmaBS)} ;	
		\path [save path=\xaxis,name path=xaxis]
		({\xminBS},0)		-- ({\xmaxBS},0)		;
		\addplot [bottom color=OrangeProfondIRA!50, top color=OrangeProfondIRA!10] fill between [
		of=A and xaxis,
		split,
		every segment no 0/.style=
		{top color = BleuProfondIRA!50, bottom color=BleuProfondIRA!10}] ;
		%\draw[use path=\xaxis, ->,OrangeProfondIRA,thick];
		\draw [fill]  ({-\PremiumBS+\PxExerciceBS},0) circle (1mm) node [above=5mm] {$S_{PM}$};
		\draw[BleuProfondIRA, thin] ({\PxExerciceBS},0) -- ({\PxExerciceBS},{-\PremiumBS});
		\node at ({\xminBS+0.25*(\xmaxBS-\xminBS)},{\pgfkeysvalueof{/pgfplots/ymin}})
		{IN};
		\node at ({\PxExerciceBS},{\pgfkeysvalueof{/pgfplots/ymin}})
		{AT};
		\node at ({\xminBS+0.8*(\xmaxBS-\xminBS)},{\pgfkeysvalueof{/pgfplots/ymin}})
		{OUT};
		\draw [<->, xshift=-5mm] ({\pgfkeysvalueof{/pgfplots/xmin}},0) -- ({\pgfkeysvalueof{/pgfplots/xmin}},-\PremiumBS) node [pos=0.5, xshift=-2.5mm, rotate=90] {Premium};
	\end{axis}
	%
\end{tikzpicture}

    	     

%\textbf{La position courte sur l'option de vente}

\medskip

		\begin{tikzpicture}[yscale=.75]
	\def\riskfreeBS{0.05}
	\def\xminBS{5}
	\def\xmaxBS{15}
	\def\PxExerciceBS{10}
	\def\sigmaBS{0.2}
	\def\TBS{0.75}
	\def\PremiumBS{{BSCall(10,{\PxExerciceBS},{\riskfreeBS},{\TBS},{\sigmaBS})}}
	\begin{axis}[ 	extra tick style={tick style=BleuProfondIRA},
		clip=false,
		axis on top,
		axis lines=middle, axis line style={BleuProfondIRA,thick,->},
		scale only axis, xmin={\xminBS},xmax={\xmaxBS},enlarge x limits=0.05,
		enlarge y limits=0.125,
		color=BleuProfondIRA,
		%		ylabel near ticks,
		ylabel={Profit},
x label style={at={(axis cs:\xmaxBS+.1,0)},anchor=north east},
		xlabel={ss-jacent ($T$)},
		%		    x label style={at={(axis description cs:0.5,-0.1)},anchor=north},
		%		y label style={at={(axis description cs:-0.1,.5)},rotate=90,anchor=south},
		ytick=\empty,
		xtick=\empty,
		extra y ticks ={0},
		extra y tick labels={{0}},
		extra x ticks ={\PxExerciceBS},
		extra x tick labels={{$E$}},
		extra x tick style={color=BleuProfondIRA,
			tick label style={yshift=0mm}	},
		title ={ \textbf{Short Put}},
		title style={yshift=10mm},
		%	legend style={draw=none,
			%		legend columns=-1,
			%		at={(0.5,1)},
			%		anchor=south,
			%		outer sep=1em,
			%		node font=\small,
			%	},
		]		%		
		\addplot[name path=A,BleuProfondIRA,thick,domain={{\xminBS}:{\xmaxBS}}, samples=21,dashdotted] {-Put(x,\PxExerciceBS,\PremiumBS)} 
		node [pos=0.15,yshift=10mm,color=OrangeProfondIRA] {Perte}
		node [pos=.75,yshift=-4mm,color=BleuProfondIRA] {Gains};	
		%\draw[BleuProfondIRA,thick] \OV{\PxExerciceBS}{\PremiumBS}{0.4}{\xminBS}{\xmaxBS} ;
		%	\addplot[BleuProfondIRA,thick,dashdotted] {Call(x,50,2)};
		%	\addplot[BleuProfondIRA,thick] {Put(x,50,3)+Call(x,50,2)};
		%\addplot[name path=Option,BleuProfondIRA,thick,domain={250:350}, 			 		samples=10,dashdotted,smooth] {BSPut(x,\PxExerciceBS,\riskfreeBS,\TBS,\sigmaBS)} ;	
		\path [save path=\xaxis,name path=xaxis]
		({\xminBS},0)		-- ({\xmaxBS},0)		;
		\addplot [bottom color=OrangeProfondIRA!50, top color=OrangeProfondIRA!10] fill between [
		of=A and xaxis,
		split,
		every segment no 1/.style=
		{top color = BleuProfondIRA!50, bottom color=BleuProfondIRA!10}] ;
		%\draw[use path=\xaxis, ->,OrangeProfondIRA,thick];
		\draw [fill]  ({-\PremiumBS+\PxExerciceBS},0) circle (1mm) node [above=5mm] {$S_{PM}$};
		\draw[BleuProfondIRA, thin] ({\PxExerciceBS},0) -- ({\PxExerciceBS},{\PremiumBS});
		\node at ({\xminBS+0.25*(\xmaxBS-\xminBS)},{\pgfkeysvalueof{/pgfplots/ymin}+1})
		{IN};
		\node at ({\PxExerciceBS},{\pgfkeysvalueof{/pgfplots/ymin}+1})
		{AT};
		\node at ({\xminBS+0.8*(\xmaxBS-\xminBS)},{\pgfkeysvalueof{/pgfplots/ymin}+1})
		{OUT};
		\draw [<->, xshift=-5mm] ({\pgfkeysvalueof{/pgfplots/xmin}},0) -- ({\pgfkeysvalueof{/pgfplots/xmin}},\PremiumBS) node [pos=0.5, xshift=-2.5mm, rotate=90] {Prime};
	\end{axis}
	%
\end{tikzpicture}
    	     

À l'échéance, le retour est de $\min (0, S_T-K)=-(K- S_T)^+$.


\end{f}
\hrule

\begin{f}[Les stratégies d'écart]
\textbf{Stratégie d'écart} : utilise deux options ou plus du même type (deux options d'achat ou deux options de vente).  
Si les prix d'exercice varient, c'est un \textbf{écart vertical}.  
Si les échéances changent, c'est un \textbf{écart horizontal}.

Une stratégie d’écart vertical (\textit{spread trading strategy}) implique une position longue et une position courte sur des options d’achat portant sur le même sous-jacent, de même échéance mais avec des prix d’exercice différents.  
On distingue : \textbf{écart vertical haussier} (\textit{Bull spread}) et \textbf{écart vertical baissier} (\textit{Bear spread}).

\textbf{Écart vertical haussier} : anticipant une hausse modérée du sous-jacent, l’investisseur prend une position longue sur $C_1$ et courte sur $C_2$ sous la contrainte $E_1 < E_2$.  
Résultat net à l’échéance :

	\begin{tikzpicture}[scale=.52]
		\def\riskfreeBS{0.05}
		\def\xminBS{5}
		\def\xmaxBS{15}
		\def\PxExerciceBSa{9}
		\def\PxExerciceBSb{12}
		\def\sigmaBS{0.2}
		\def\TBS{0.75}
		\def\PremiumBSa{{BSCall(11,{\PxExerciceBSa},{\riskfreeBS},{\TBS},{\sigmaBS})}}
		\def\PremiumBSb{{BSCall(11,{\PxExerciceBSb},{\riskfreeBS},{\TBS},{\sigmaBS})}}
		\begin{axis}[ 
			width=0.8\textwidth,
			height=0.5\textwidth, 
			extra tick style={tick style=BleuProfondIRA},
			clip=false,
			axis on top,
			axis lines=middle, axis line style={BleuProfondIRA,thick,->},
			scale only axis, xmin={\xminBS},xmax={\xmaxBS},enlarge x limits=0.05,
			enlarge y limits=0.125,
			color=BleuProfondIRA,
			%		ylabel near ticks,
			ylabel={Profit},
			x label style={={at={(current axis.right of origin)}}},
			%    x label style={at={(axis description cs:1,-0.1)},anchor=south},
			%		x label style={at={(1,0.5)}},
			xlabel={ss-jacent ($T$)},
			%		    x label style={at={(axis description cs:0.5,-0.1)},anchor=north},
			%		y label style={at={(axis description cs:-0.1,.5)},rotate=90,anchor=south},
			ytick=\empty,
			xtick=\empty,
			extra y ticks ={0},
			extra y tick labels={{0}},
			extra x ticks ={{\PxExerciceBSa},{\PxExerciceBSb}},
			extra x tick labels={{$E_1\ \ \ \ \ $},{$E_2$}},
			extra x tick style={color=BleuProfondIRA,
				tick label style={yshift=-0mm}	},
			]
			%		
			\addplot[name path=A,BleuProfondIRA,thin,domain={{\xminBS}:{\xmaxBS-1.75}}, samples=21,dashed] {Call(x,\PxExerciceBSa,\PremiumBSa)} 
			node [pos=0.15, above] {\small Long $C_1$};	
			\addplot[name path=B,BleuProfondIRA,thin,domain={{\xminBS}:{\xmaxBS}}, samples=21,dashed] {-Call(x,\PxExerciceBSb,\PremiumBSb)} 
			node [pos=0.15, above] {\small  Short $C_2$};	
			\addplot[name path=EVH,OrangeProfondIRA,thick,domain={{\xminBS}:{\xmaxBS}}, samples=41] {Call(x,\PxExerciceBSa,\PremiumBSa)-Call(x,\PxExerciceBSb,\PremiumBSb)} node [pos=0.15, above] {\'Ecart vertical haussier};	
			%\draw[BleuProfondIRA,thick] \OV{\PxExerciceBSa}{\PremiumBSa}{0.4}{\xminBS}{\xmaxBS} ;
			%		\addplot[name path=Option,BleuProfondIRA,thick,domain={250:350}, 			 		samples=10,dashdotted,smooth] {BSPut(x,\PxExerciceBSa,\riskfreeBS,\TBS,\sigmaBS)} ;	
			
			\draw[BleuProfondIRA, thin, dashed] ({\PxExerciceBSa},0) -- ({\PxExerciceBSa},{-\PremiumBSa})	;
			\draw[BleuProfondIRA, thin, dashed] ({\PxExerciceBSb},0) -- ({\PxExerciceBSb},{\PremiumBSb})	;
		\end{axis}
		%
	\end{tikzpicture}



\textbf{Écart vertical baissier} : anticipant une baisse modérée du sous-jacent, l’investisseur vend l’option la plus chère et achète la moins chère.


	\begin{tikzpicture}[scale=.52]
		\def\riskfreeBS{0.05}
		\def\xminBS{5}
		\def\xmaxBS{15}
		\def\PxExerciceBSa{9}
		\def\PxExerciceBSb{12}
		\def\sigmaBS{0.2}
		\def\TBS{0.75}
		\def\PremiumBSa{{BSCall(11,{\PxExerciceBSa},{\riskfreeBS},{\TBS},{\sigmaBS})}}
		\def\PremiumBSb{{BSCall(11,{\PxExerciceBSb},{\riskfreeBS},{\TBS},{\sigmaBS})}}
		\begin{axis}[ 
			width=0.8\textwidth,
			height=0.5\textwidth, 
			extra tick style={tick style=BleuProfondIRA},
			clip=false,
			axis on top,
			axis lines=middle, axis line style={BleuProfondIRA,thick,->},
			scale only axis, xmin={\xminBS},xmax={\xmaxBS},enlarge x limits=0.05,
			enlarge y limits=0.125,
			color=BleuProfondIRA,
			%		ylabel near ticks,
			ylabel={Profit},
			x label style={={at={(current axis.right of origin)}}},
			%    x label style={at={(axis description cs:1,-0.1)},anchor=south},
			%		x label style={at={(1,0.5)}},
			xlabel={ss-jacent ($T$)},
			%		    x label style={at={(axis description cs:0.5,-0.1)},anchor=north},
			%		y label style={at={(axis description cs:-0.1,.5)},rotate=90,anchor=south},
			ytick=\empty,
			xtick=\empty,
			extra y ticks ={0},
			extra y tick labels={{0}},
			extra x ticks ={{\PxExerciceBSa},{\PxExerciceBSb}},
			extra x tick labels={{$E_1\ \ \ \ \ $},{$E_2$}},
			extra x tick style={color=BleuProfondIRA,
				tick label style={yshift=-0mm}	},
			]
			%		
			\addplot[name path=A,BleuProfondIRA,thin,domain={{\xminBS}:{\xmaxBS-1.75}}, samples=21,dashed] {-Call(x,\PxExerciceBSa,\PremiumBSa)} 
			node [pos=0.15, above] {\small Short $C_1$};	
			\addplot[name path=B,BleuProfondIRA,thin,domain={{\xminBS}:{\xmaxBS}}, samples=21,dashed] {Call(x,\PxExerciceBSb,\PremiumBSb)} 
			node [pos=0.15, above] {\small  Long $C_2$};		
			\addplot[name path=EVH,OrangeProfondIRA,thick,domain={{\xminBS}:{\xmaxBS}}, samples=41] {-Call(x,\PxExerciceBSa,\PremiumBSa)+Call(x,\PxExerciceBSb,\PremiumBSb)} node [pos=0.15, above] {\'Ecart vertical baissier};	
			%\draw[BleuProfondIRA,thick] \OV{\PxExerciceBSa}{\PremiumBSa}{0.4}{\xminBS}{\xmaxBS} ;
			%		\addplot[name path=Option,BleuProfondIRA,thick,domain={250:350}, 			 		samples=10,dashdotted,smooth] {BSPut(x,\PxExerciceBSa,\riskfreeBS,\TBS,\sigmaBS)} ;	
			
			\draw[BleuProfondIRA, thin, dashed] ({\PxExerciceBSa},0) -- ({\PxExerciceBSa},{\PremiumBSa})	;
			\draw[BleuProfondIRA, thin, dashed] ({\PxExerciceBSb},0) -- ({\PxExerciceBSb},{-\PremiumBSb})	;
		\end{axis}
		%
	\end{tikzpicture}

% Code TikZ conservé ici (éventuellement inséré)

\textbf{Écart papillon} (\textit{butterfly spread}) : anticipe une faible variation du sous-jacent.  
C’est la combinaison d’un écart vertical haussier et d’un écart vertical baissier.  
Stratégie adaptée si de grandes variations sont jugées peu probables.  
Investissement initial faible.

	\begin{tikzpicture}[scale=.52]
		\def\riskfreeBS{0.05}
		\def\xminBS{5}
		\def\xmaxBS{15}
		\def\PxExerciceBSa{8}
		\def\PxExerciceBSb{10}
		\def\PxExerciceBSc{12}
		\def\sigmaBS{0.2}
		\def\TBS{0.75}
		\def\PremiumBSa{BSCall(11,{\PxExerciceBSa},{\riskfreeBS},{\TBS},{\sigmaBS})}
		\def\PremiumBSb{BSCall(11,{\PxExerciceBSb},{\riskfreeBS},{\TBS},{\sigmaBS})}
		\def\PremiumBSc{BSCall(11,{\PxExerciceBSc},{\riskfreeBS},{\TBS},{\sigmaBS})}
		\begin{axis}[
			width=0.8\textwidth,
			height=0.5\textwidth, 
			extra tick style={tick style=BleuProfondIRA},
			clip=false,
			axis on top,
			axis lines=middle, axis line style={BleuProfondIRA,thick,->},
			scale only axis, xmin={\xminBS},xmax={\xmaxBS},enlarge x limits=0.05,
			enlarge y limits=0.125,
			color=BleuProfondIRA,
			%		ylabel near ticks,
			ylabel={Profit},
			x label style={={at={(current axis.right of origin)}}},
			%    x label style={at={(axis description cs:1,-0.1)},anchor=south},
			%		x label style={at={(1,0.5)}},
			xlabel={ss-jacent ($T$)},
			%		    x label style={at={(axis description cs:0.5,-0.1)},anchor=north},
			%		y label style={at={(axis description cs:-0.1,.5)},rotate=90,anchor=south},
			ytick=\empty,
			xtick=\empty,
			extra y ticks ={0},
			extra y tick labels={{0}},
			extra x ticks ={{\PxExerciceBSa},{\PxExerciceBSb},{\PxExerciceBSc}},
			extra x tick labels={{$E_1\ \ \ \ \ $},{$E_2$},{$E_3$}},
			extra x tick style={color=BleuProfondIRA,
				tick label style={yshift=-0mm}	},
			]
			%		
			\addplot[name path=A,BleuProfondIRA,thin,domain={{\xminBS}:{\xmaxBS}}, samples=21,dashed] {Call(x,\PxExerciceBSa,\PremiumBSa)} 
			node [pos=0.15, above] {\small Long $C_1$};	
			\addplot[name path=B,BleuProfondIRA,thin,domain={{\xminBS}:{\xmaxBS-1.75}}, samples=21,dashed] {-2*Call(x,\PxExerciceBSb,\PremiumBSb)} 
			node [pos=0.15, above] {\small Short $2\ C_2$};	
			\addplot[name path=C,BleuProfondIRA,thin,domain={{\xminBS}:{\xmaxBS}}, samples=21,dashed] {Call(x,\PxExerciceBSc,\PremiumBSb)} 
			node [pos=0.15, above] {\small Long $C_3$};	
			\addplot[name path=EP,OrangeProfondIRA,thick,domain={{\xminBS}:{\xmaxBS}}, samples=41] {
				Call(x,\PxExerciceBSa,\PremiumBSa) - 2*Call(x,\PxExerciceBSb,\PremiumBSb)+
				Call(x,\PxExerciceBSc,\PremiumBSc)}
			node [pos=0.5, below=30pt] { \'Ecart papillon};	
			\draw[BleuProfondIRA, thin, dashed] ({\PxExerciceBSa},0) -- ({\PxExerciceBSa},{-\PremiumBSa})	;
			\draw[BleuProfondIRA, thin, dashed] ({\PxExerciceBSb},0) -- ({\PxExerciceBSb},{2*\PremiumBSb})	;
			\draw[BleuProfondIRA, thin, dashed] ({\PxExerciceBSc},0) -- ({\PxExerciceBSc},{-\PremiumBSc})	;
		\end{axis}
		%
	\end{tikzpicture}
% Code TikZ conservé ici (éventuellement inséré)

\end{f}
\hrule

\begin{f}[Les stratégies combinées]
	
	Une \textbf{stratégie combinée} utilise à la fois des options d'achat et de vente. On distingue notamment les \textbf{stellages} et les \textbf{strangles}.
	
	Un \textbf{stellage} (straddle) combine l'achat d'une option d'achat et d'une option de vente de même échéance et de même prix d'exercice. Cette stratégie parie sur une forte variation du prix, à la hausse ou à la baisse. La perte maximale survient si le prix à l’échéance est égal au prix d’exercice.
	
	Un \textbf{strangle} est l’achat d’un call et d’un put de même échéance mais à prix d’exercice différents. Il suppose une très forte variation de la valeur du sous-jacent.
	
		\begin{tikzpicture}[scale=.52]
			\def\xminBS{200}
			\def\xmaxBS{275}
			\def\PxExerciceBSa{230}
			\def\PxExerciceBSb{245}
			\def\PremiumBSa{20.69}
			\def\PremiumBSb{23.79}
			\begin{axis}[ 
				width=0.8\textwidth,
				height=0.5\textwidth, 
				extra tick style={tick style=BleuProfondIRA},
				clip=false,
				axis on top,
				axis lines=middle, axis line style={BleuProfondIRA,thick,->},
				scale only axis, xmin={\xminBS},xmax={\xmaxBS},enlarge x limits=0.05,
				enlarge y limits=0.125,
				color=BleuProfondIRA,
				ylabel={Profit},
				x label style={={at={(current axis.right of origin)}}},
				xlabel={ss-jacent ($T$)},
				ytick=\empty,
				extra y ticks ={0},
				extra y tick labels={{0}},
				extra x ticks ={{\PxExerciceBSa},{\PxExerciceBSb}},
				extra x tick labels={{$E_1\ \ \ \ \ $},{$E_2$}},
				extra x tick style={color=BleuProfondIRA,
					tick label style={yshift=-10mm}	},
				]
				\addplot[name path=A,BleuProfondIRA,thin,domain={{\xminBS}:{\xmaxBS-1.75}}, samples=21,dashed] {Call(x,\PxExerciceBSa,\PremiumBSa)} 
				node [pos=0.15, below] {\small Long $C_1$};	
				\addplot[name path=B,BleuProfondIRA,thin,domain={{\xminBS}:{\xmaxBS}}, samples=21,dashed] {Put(x,\PxExerciceBSb,\PremiumBSb)} 
				node [pos=0.85, above] {\small  Long $P_2$};		
				\addplot[name path=EVH,OrangeProfondIRA,thick,domain={{\xminBS}:{\xmaxBS}}, samples=21] {Call(x,\PxExerciceBSa,\PremiumBSa)+Put(x,\PxExerciceBSb,\PremiumBSb)} node [pos=0.5, above] {\small Strangle};	
				\draw[BleuProfondIRA, thin, dashed] ({\PxExerciceBSa},0) -- ({\PxExerciceBSa},{-\PremiumBSa})	;
				\draw[BleuProfondIRA, thin, dashed] ({\PxExerciceBSb},0) -- ({\PxExerciceBSb},{-\PremiumBSb})	;
			\end{axis}
		\end{tikzpicture}


		\begin{tikzpicture}[scale=.52]
			\def\xminBS{200}
			\def\xmaxBS{275}
			\def\PxExerciceBSa{230}
			\def\PxExerciceBSb{245}
			\def\PremiumBSa{15.19}
			\def\PremiumBSb{14.29}
			\begin{axis}[ 
				width=0.8\textwidth,
				height=0.5\textwidth, 
				extra tick style={tick style=BleuProfondIRA},
				clip=false,
				axis on top,
				axis lines=middle, axis line style={BleuProfondIRA,thick,->},
				scale only axis, xmin={\xminBS},xmax={\xmaxBS},enlarge x limits=0.05,
				enlarge y limits=0.125,
				color=BleuProfondIRA,
				ylabel={Profit},
				x label style={={at={(current axis.right of origin)}}},
				xlabel={ss-jacent ($T$)},
				ytick=\empty,
				extra y ticks ={0},
				extra y tick labels={{0}},
				extra x ticks ={{\PxExerciceBSa},{\PxExerciceBSb}},
				extra x tick labels={{$E_1\ \ \ \ \ $},{$E_2$}},
				extra x tick style={color=BleuProfondIRA,
					tick label style={yshift=-10mm}	},
				]
				\addplot[name path=A,BleuProfondIRA,thin,domain={{\xminBS}:{\xmaxBS}}, samples=21,dashed] {Put(x,\PxExerciceBSa,\PremiumBSa)} 
				node [pos=0.85, below] {\small Long $P_1$};	
				\addplot[name path=B,BleuProfondIRA,thin,domain={{\xminBS}:{\xmaxBS}}, samples=21,dashed] {Call(x,\PxExerciceBSb,\PremiumBSb)} 
				node [pos=0.15, above] {\small  Long $C_2$};		
				\addplot[name path=EVH,OrangeProfondIRA,thick,domain={{\xminBS}:{\xmaxBS}}, samples=21] {Put(x,\PxExerciceBSa,\PremiumBSa)+Call(x,\PxExerciceBSb,\PremiumBSb)} node [pos=0.5, above] {\small Strangle};	
				\draw[BleuProfondIRA, thin, dashed] ({\PxExerciceBSa},0) -- ({\PxExerciceBSa},{-\PremiumBSa})	;
				\draw[BleuProfondIRA, thin, dashed] ({\PxExerciceBSb},0) -- ({\PxExerciceBSb},{-\PremiumBSb})	;
			\end{axis}
		\end{tikzpicture}
	
	
	
\end{f}
\hrule

\begin{f}[Absence d'opportunité d'arbitrage]
Aucun profit sans risque n’est possible par exploitation des écarts de prix. 

\end{f}

\begin{f}[La relation de parité]

L'AOA implique la relation suivante entre le Call et le Put :

$$S_t- C_t + P_t = K e^{-i_{f}.\tau}$$	
\end{f}
\hrule

\begin{f}[Le modèle de Cox-Ross-Rubinstein]
	
Il repose sur un processus en temps discret avec deux évolutions possibles du prix à chaque période : une hausse (facteur $u$) ou une baisse (facteur $d$), avec $u > 1 + i_{f}$ et $d < 1 + i_{f}$. Le prix à $t = 1$ est alors \( S_{1}^{u} = S_{0} u \) ou \( S_{1}^{d} = S_{0} d \), selon une probabilité $q$ ou $1-q$.

\begin{tikzpicture}
	[sibling distance=5em,
	every node/.style = {shape=rectangle, rounded corners, fill=OrangeProfondIRA!20,
		align=center,  draw=OrangeProfondIRA, text=BleuProfondIRA } ,grow=right,
	edge from parent/.style={draw=OrangeProfondIRA, thick}]
	\node (A){$S_0$}
	child {node  (B) {$S_d$}
		child {node {$S_{dd}$}}
		child} 
	child {node (C) {$S_u$} 
		child {node  {$S_{du}$}}
		child {node {$S_{uu}$}}    
	};
	\draw [draw=none] 
	($ (A.east) + (0,0.2) $) -- node[draw=none, fill=none, above left, BleuProfondIRA] {$q$} ($ (C.west) + (0,-0.2) $);
	\draw [draw=none]  
	($ (A.east) + (0,-0.2) $) -- node[draw=none, fill=none,below left, BleuProfondIRA] {$1 - q$} ($ (B.west) + (0,0.2) $);
\end{tikzpicture}
\quad
\begin{tikzpicture}
	[sibling distance=5em,
	every node/.style = {shape=rectangle, rounded corners, fill=OrangeProfondIRA!20,
		align=center,  draw=OrangeProfondIRA, text=BleuProfondIRA } ,grow=right,
	edge from parent/.style={draw=OrangeProfondIRA, thick}]
	\node {$C_0$}
	child {node  {$C_d$}
		child {node {$C_{dd}=(S_{dd}-K)^{+}$}}
		child}
	child {node {$C_u$} 
		child {node  {$C_{du}=(S_{du}-K)^{+}$}}
		child {node  {$C_{uu}=(S_{uu}-K)^{+}$}}  
	};
\end{tikzpicture}

Ce modèle s’étend à $n$ périodes avec $n+1$ prix possibles pour $S_T$. À l’échéance, la valeur d’une option d’achat est donnée par \( C_{1}^{u} = (S_{1}^{u} - K)^+ \) et \( C_{1}^{d} = (S_{1}^{d} - K)^+ \).

\textbf{Absence d'opportunité d'arbitrage} implique
\[
d < 1 + i_{f} < u
\]
et une probabilité risque neutre 
\[q = \frac{(1 + i_{f}) - d}{u - d}\]

\textbf{Prix du call} (avec \( S_{1}^{d} < K < S_{1}^{u} \)) :
\[
C_{0} = \frac{1}{1+i_f} \left[ q C_{1}^{u} + (1 - q) C_{1}^{d} \right]
\]

On peut aussi construire un portefeuille de réplication composé de $\Delta$ actions et $B$ obligations, tel que :
\[
\begin{cases}
	\Delta = \frac{S_{1}^{u} - K}{S_{1}^{u} - S_{1}^{d}}, \\
	B = \frac{-S_{1}^{d}}{1+i_f} \cdot \Delta
\end{cases}
\quad \Rightarrow \quad \Pi_0 = \Delta S_0 + B
\]

\textbf{Prix du put} :
\[
P_{0} = \frac{1}{1+i_f} \left[ q P_{1}^{u} + (1 - q) P_{1}^{d} \right]
\]

\textbf{Détermination de $q$, $u$, $d$} :  
En calibrant le modèle pour retrouver les premiers moments du rendement sous la probabilité risque neutre (espérance $i_f$, variance $\sigma^2 \delta t$), on obtient :
\[
e^{i_{f} \delta t} = q u + (1-q) d, \qquad q u^2 + (1-q) d^2 - [q u + (1-q) d]^2 = \sigma^2 \delta t
\]

Avec la contrainte $u = \frac{1}{d}$, on arrive à :
\[
\begin{array}{l}
	q = \frac{e^{-i_f \delta_t} - d}{u - d} \\
	u = e^{\sigma \sqrt{\delta t}} \\
	d = e^{-\sigma \sqrt{\delta t}}
\end{array}
\]

\end{f}
\hrule

\begin{f}[Le modèle de Black \& Scholes]
Hypothèses du modèle 
\begin{itemize}
	\item Le taux sans risque $R$ est constant. On définit $i_f = \ln(1+R)$, ce qui implique $(1+R)^t = e^{i_f t}$.
	\item Le prix de l'action $S_t$ suit un mouvement brownien géométrique :
	$$
	dS_t = \mu S_t dt + \sigma S_t dW_t 
	$$
	$$
	 S_t = S_0 \exp\left(\sigma W_t + \left( \mu - \frac{1}{2}\sigma^2 \right)t \right)
	$$
	\item Pas de dividende pendant la durée de vie de l’option.
	\item L’option est \og{}européenne\fg{} (exercée uniquement à l’échéance).
	\item Marché sans friction : pas d’impôts ni de coûts de transaction.
	\item La vente à découvert est autorisée.
\end{itemize}

L’équation de Black-Scholes-Merton pour évaluer un contrat dérivé $f$ est :
$$
\frac{\partial f}{\partial t} + i_f S \frac{\partial f}{\partial S} + \frac{1}{2}\sigma^2 S^2 \frac{\partial^2 f}{\partial S^2} = i_f f
$$

À l’échéance, le prix d’une option d’achat est $C(S,T) = \max(0, S_T - K)$, et celui d’une option de vente est $P(S,T) = \max(0, K - S_T)$.


\begin{center}
	\begin{tabular}{|c|c|c|}
		\hline
		Déterminants & \textbf{call}&\textbf{put}\\
		\hline
		Cours du sous-jacent	      & +&	-\\
		Prix d'exercice	              & -&	+\\
		La maturité	 (ou le temps)    & + (-)&	+ (-)\\
		Volatilité	              & +&	+\\
		Taux d'intérêt à court terme  & +&	-\\
		Versement de dividende	      & -&	+\\
		\hline
	\end{tabular}
\end{center}

Les solutions analytiques sont :
\begin{align*}
	C_t &= S_t \Phi(d_1) - Ke^{-i_f \tau} \Phi(d_2) \\
	P_t &= Ke^{-i_f \tau} \Phi(-d_2) - S_t \Phi(-d_1)
\end{align*}
où :
\begin{align*}
	d_1 &= \frac{\ln(S_t/K) + (i_f + \frac{1}{2}\sigma^2)\tau}{\sigma \sqrt{\tau}}, \quad
	d_2 = d_1 - \sigma \sqrt{\tau}
\end{align*}

%La sensibilité peut être mesurée par cinq paramètres (lettres grecques) :

\begin{itemize}
	\item \textbf{Delta} $\Delta$ : variation du prix de l’option selon le sous-jacent.
	\item \textbf{Gamma} $\Gamma$ : sensibilité du delta.
	\item \textbf{Thêta} $\Theta$ : sensibilité au temps.
	\item \textbf{Véga} $\mathcal{V}$ : sensibilité à la volatilité.
	\item \textbf{Rho} $\rho$ : sensibilité au taux d’intérêt.
\end{itemize}


Le \textbf{Delta} mesure l’impact d’une variation du sous-jacent :

\begin{align*}
	\Delta_C &= \frac{\partial C}{\partial S} = \Phi(d_1), \quad \Delta \in (0,1) \\
	\Delta_P &= \frac{\partial P}{\partial S} = \Phi(d_1) - 1, \quad \Delta \in (-1,0)
\end{align*}


Le Delta global d’un portefeuille $\Pi$ avec des poids $\omega_i$ est :
$$
\frac{\partial \Pi}{\partial S_t} = \sum_{i=1}^{n} \omega_i \Delta_i
$$



% Graphique TikZ conservé tel quel :

\begin{center}
\begin{tikzpicture}[scale=.52]
\def\riskfreeBS{0.05}
\def\xminBS{7.5}
\def\xmaxBS{12.5}
\def\PxExerciceBS{10}
\def\sigmaBS{0.3}
\def\TBS{0.4}
\def\PremiumBS{BSCall(\PxExerciceBS*exp(-\riskfreeBS*\TBS),\PxExerciceBS,\riskfreeBS,\TBS,\sigmaBS)}
\def\PxExerciceAct{\PxExerciceBS*exp(-(\riskfreeBS+\sigmaBS*\sigmaBS/2)*\TBS)}
\begin{axis}[
	width=0.8\textwidth,
	height=0.5\textwidth, 
	extra tick style={tick style=BleuProfondIRA},
	clip=false,
	axis on top,
	axis lines=middle, axis line style={BleuProfondIRA,thick,->},
	scale only axis, xmin={\xminBS},xmax={\xmaxBS},enlarge x limits=0.05,
	enlarge y limits=0.1,
	color=BleuProfondIRA,
	ylabel={$\Delta$},
	x label style={at={(axis cs:\xmaxBS+.1,0)},anchor=north east},
	xlabel={ss-jacent ($T$)},
	ytick=\empty,
	xtick=\empty,
	extra y ticks ={-.5,0,0.5},
	extra y tick labels={{$-\frac{1}{2}$},{0},{$\frac{1}{2}$}},
	extra x ticks ={\PxExerciceAct,\PxExerciceBS},
	extra x tick labels={{\color{BleuProfondIRA}$K'$\ \ \ \ \ \ \ \ },{\color{BleuProfondIRA}\ \ $K$}},
	extra x tick style={color=BleuProfondIRA,
		tick label style={yshift=-0mm}	},
	title ={ \textbf{Delta de l'option}},
	title style={yshift=-10mm}
	]
	\addplot[name path=optionT,OrangeProfondIRA,thin,domain={{\xminBS}:{\xmaxBS}}, samples=21]
	{normcdf(-ddd(x,\PxExerciceBS,\riskfreeBS,\TBS,\sigmaBS),0,1)} node [above] {Call};
	\addplot[name path=optionT,OrangeProfondIRA,thin,domain={{\xminBS}:{\xmaxBS}}, samples=21]
	{normcdf(-ddd(x,\PxExerciceBS,\riskfreeBS,\TBS,\sigmaBS),0,1)-1} node [above] {Put};
	\draw[dashed,OrangeProfondIRA] 
	(axis cs:{\pgfkeysvalueof{/pgfplots/xmin}},-0.5) --
	(axis cs:{\PxExerciceAct},{-.5}) --
	(axis cs:{\PxExerciceAct},{0.5}) --
	(axis cs:{\pgfkeysvalueof{/pgfplots/xmin}},0.5);
\end{axis}
\end{tikzpicture}
\end{center}

\end{f}
\hrule


\begin{f}[La courbe de taux]
La \textbf{courbe de taux} ou la courbe des rendements ou $r_f(\tau)$, livre une représentation graphique des taux d'intérêt sans risques en fonction de l'échéance (ou maturité).
Elle est aussi appelé courbe des taux \textbf{zéro coupon} qui font référence à un type d'obligation sans risque et sans coupons (une dette composée uniquement de deux flux de sens opposés, un en $t_0$ et l'autre en $T$).
Cette courbe offre aussi un aperçu des attentes du marché en matière de taux d'intérêt futurs (taux \engl{forward}).
\end{f}
\hrule


\begin{f}[Les modèles de  Nelson-Siegel et Svensson]


Les fonctions de \textbf{Nelson-Siegel}  prennent la forme

{\small\begin{align*}
y( m ) =& \beta _0 + \beta _1\frac{{\left[ {1 - \exp \left( { - m/\tau} \right)} \right]}}{m/\tau} + \\
		&\beta _2 {\left(\frac{{\left[ {1 - \exp \left( { - m/\tau} \right)} \right]}}{m/\tau} - \exp \left( { - m/\tau}\right)\right)}
\label{MTNSeq}
\end{align*}}
%
où $y\left( m \right)$ et $m$ sont comme ci-dessus, et $\beta _0$, $\beta_1$, $\beta_2$ et $\tau$ sont des paramètres:


\begin{itemize}

\item   $\beta_0$ est interprété comme le niveau à long terme des taux d'intérêt (le coefficient est 1, c'est une constante qui ne décroît pas),

\item   $\beta_1$ est le composant à court terme, en notant que :
\begin{equation*}
	\lim_{m \rightarrow 0} \frac{{\left[ {1 - \exp \left( { - m/\tau} \right)} \right]}}{m/\tau}=1
\end{equation*}
Il résulte que le taux \engl{overnigth} tel que €str\index{Taux d'intérêts! Estr} sera égal à  $\beta_0+\beta_1$ dans ce modèle.
\item   $\beta_2$ est le composant à moyen terme (il commence à 0, augmente, puis décroît vers zéro, autrement dit en forme de cloche),
\item   $\tau$ est le facteur d'échelle sur la maturité, il détermine où le terme pondéré par $\beta_2$ atteint son maximum. 
\end{itemize}

Svensson (1995) ajoute un second terme en forme de cloche, il s'agit du modèle Nelson–Siegel–Svensson. Le terme supplémentaire est :
%
\begin{equation*}
+\beta _3 {\left(\frac{{\left[ {1 - \exp \left( { - m/\tau_2} \right)} \right]}}{m/\tau_2} - \exp \left( { - m/\tau_2}\right)\right)}
\label{MTSveq}
\end{equation*}
et l'interprétation est la même que pour $\beta_2$ et $\tau$ ci-dessus, il permet deux points d'inflexion à la courbe de taux.

\newcommand{\traintunnel}{	        
\draw[thick, OrangeProfondIRA] svg "M 55.448002 56.380001L 40 39L 28 39L 12.552 56.380001M 12 34C 11.729672 21.575853 21.576109 11.281852 34 11C 46.423893 11.281852 56.270329 21.575853 56.000004 34L 56 55C 56 56.104568 55.104568 57 54 57L 14 57C 12.895431 57 12 56.104568 12 55ZM 28 39L 28 34C 28 30.132 30.302 27 34 27C 37.697998 27 40 30.132 40 34L 40 39M 34 51L 34 57M 34 43L 34 45";
}
\newcommand{\archibuilding}{
\draw[OrangeProfondIRA,yscale=-1] svg "M 12.296 28.886L 12.296 54.453999C 12.451618 55.715763 13.58469 56.623463 14.85 56.500004L 53.150002 56.5C 54.415314 56.623463 55.548386 55.715763 55.704002 54.454002L 55.703999 28.886M 12.296 28.886L 34 11.5L 55.703999 28.886M 34 46.296001L 42.212002 46.296001L 42.214001 50.386002L 49.32 50.386002L 49.32 56.5M 12.296 40.285999L 34 40.285999M 34 35.186001L 55.368 35.186001M 34 11.5L 34 56.236M 12.296 32.106003L 34 32.106003M 19.456001 32.106003L 19.456001 40.285999M 26.84 32.106003L 26.84 40.285999";	        
}
\newcommand{\familialcar}{	%
\draw[thick, OrangeProfondIRA] svg "M 21 45L 21 48C 21 48.552284 20.552284 49 20 49L 16 49C 15.447716 49 15 48.552284 15 48L 15 45M 53 45L 53 48C 53 48.552284 52.552284 49 52 49L 48 49C 47.447716 49 47 48.552284 47 48L 47 45M 54 45C 54.552284 45 55 44.552284 55 44L 55 37.414001C 54.999943 37.149296 54.894939 36.895409 54.708 36.708L 49 31L 19 31L 13.292 36.708C 13.105062 36.895409 13.000056 37.149296 13 37.414001L 13 44C 13 44.552284 13.447716 45 14 45ZM 49 31L 45.228001 19.684C 45.092045 19.275806 44.710239 19.000328 44.279999 19L 23.720001 19C 23.289761 19.000328 22.907955 19.275806 22.771999 19.684L 19 31M 19 31L 14 31C 13.447716 31 13 30.552284 13 30L 13 28C 13 27.447716 13.447716 27 14 27L 20.334 27M 47.666 27L 54 27C 54.552284 27 55 27.447716 55 28L 55 30C 55 30.552284 54.552284 31 54 31L 49 31M 13.092 37L 20 37C 20.552284 37 21 37.447716 21 38L 21 40C 21 40.552284 20.552284 41 20 41L 13 41M 55 41L 48 41C 47.447716 41 47 40.552284 47 40L 47 38C 47 37.447716 47.447716 37 48 37L 54.908001 37";
}

\newcommand{\familialTV}{	        
\draw[thick, OrangeProfondIRA] svg "M 12 17.5L 56 17.5C 56 17.5 57 17.5 57 18.5L 57 43.5C 57 43.5 57 44.5 56 44.5L 12 44.5C 12 44.5 11 44.5 11 43.5L 11 18.5C 11 18.5 11 17.5 12 17.5M 34 44.5L 34 50.5M 24 50.5L 44 50.5";
}

\newcommand{\TresorerieMngt}{	
\draw[OrangeProfondIRA] svg "M 11.008 31C 11.008 32.104568 15.485153 33 21.007999 33C 26.530848 33 31.007999 32.104568 31.007999 31C 31.007999 29.89543 26.530848 29 21.007999 29C 15.485153 29 11.008 29.89543 11.008 31ZM 31 31L 31 37C 31 38.106003 26.524 39 21 39C 15.476 39 11 38.106003 11 37L 11 31M 31 37L 31 43C 31 44.105999 26.524 45 21 45C 15.476 45 11 44.105999 11 43L 11 37M 31 43L 31 49C 31 50.105999 26.524 51 21 51C 15.476 51 11 50.105999 11 49L 11 43M 31 49L 31 55C 31 56.105999 26.524 57 21 57C 15.476 57 11 56.105999 11 55L 11 49M 11 25L 11 13C 11 11.895431 11.895431 11 13 11L 55 11C 56.104568 11 57 11.895431 57 13L 57 37C 57 38.104568 56.104568 39 55 39L 36 39M 28 25C 28.000584 21.94886 30.290909 19.384022 33.32254 19.039516C 36.354168 18.695011 39.161583 20.680555 39.846756 23.65377C 40.531929 26.626984 38.876644 29.640953 36 30.658001M 20 19.5C 20.276142 19.5 20.5 19.723858 20.5 20C 20.5 20.276142 20.276142 20.5 20 20.5C 19.723858 20.5 19.5 20.276142 19.5 20C 19.5 19.723858 19.723858 19.5 20 19.5M 48 29.5C 48.276142 29.5 48.5 29.723858 48.5 30C 48.5 30.276142 48.276142 30.5 48 30.5C 47.723858 30.5 47.5 30.276142 47.5 30C 47.5 29.723858 47.723858 29.5 48 29.5M 15 25L 15 16C 15 15.447716 15.447716 15 16 15L 52 15C 52.552284 15 53 15.447716 53 16L 53 34C 53 34.552284 52.552284 35 52 35L 36 35";	
}

Ces fonctions de Nelson-Siegel et de Svensson, présentent l'avantage de bien se comporter à long terme, et d'être facile à paramétrer. 
Elles sont représentées sur la figure où les pictogrammes \begin{tikzpicture}[xscale=0.2, yscale=-0.2]
\TresorerieMngt\end{tikzpicture} \begin{tikzpicture}[xscale=0.2, yscale=-0.2] \familialTV\end{tikzpicture} \begin{tikzpicture}[xscale=0.2, yscale=-0.2] \familialcar\end{tikzpicture} \begin{tikzpicture}[xscale=0.2, yscale=0.2] \archibuilding\end{tikzpicture} \begin{tikzpicture}[xscale=0.2, yscale=-0.2] \traintunnel
\end{tikzpicture} représentent les différentes maturités usuelles pour ce type de biens ou investissements.
Elles permettent de modéliser pratiquement une large forme de courbe de taux. 
Une fois ajustée, l'utilisateur peut alors évaluer des actifs ou définir diverses mesures de sensibilité.


\begin{center}
\begin{tikzpicture}[scale=0.55]
\def\MTbetaa{0.03}
\def\MTbetab{-0.02}
\def\MTbetac{0.01}
\def\MTbetad{-0.005}  % Svensson
\def\MTtaua{4.5}
\def\MTtaub{11}  % Svensson
\begin{axis}[
	width=0.8\textwidth,
	height=0.5\textwidth, 
	xlabel={Maturité (années)},ylabel={Taux (\%)},
	xmin=0, xmax=30,
	ymin=0, ymax=100*(\MTbetaa+.005),
	enlarge y limits=0.125,
	thick,
	axis x line=bottom,
	axis y line=left,
	yticklabel=\pgfmathprintnumber{\tick}\% 
	]
	\Large
	% Courbe de Nelson-Siegel
	\addplot[OrangeProfondIRA, thick, domain=0.01:27, samples=27] 
	{100*(\MTbetaa + \MTbetab * ((1 - exp(-x/\MTtaua)) / (x/\MTtaua)) + \MTbetac * (((1 - exp(-x/\MTtaua)) / (x/\MTtaua)) - exp(-x/\MTtaua)))}
	node  [pos=0.005] (M) {}
	node  [pos=0.10] (N) {}
	node  [pos=0.30] (O) {}
	node  [pos=0.75] (P) {}
	node  [pos=1] (Q) {};
	\addplot[dashed, OrangeProfondIRA, thick, domain=0.01:30, samples=27] 
	{100*(\MTbetaa + \MTbetab * ((1 - exp(-x/\MTtaua)) / (x/\MTtaua)) + \MTbetac * (((1 - exp(-x/\MTtaua)) / (x/\MTtaua)) - exp(-x/\MTtaua)) + \MTbetad * (((1 - exp(-x/\MTtaub)) / (x/\MTtaub)) - exp(-x/\MTtaub)))};
	% Affichage des paramètres
	\node[anchor=south east, text=BleuProfondIRA] at (rel axis cs:1,0.15) {\small 
		$\beta_0 = \MTbetaa$\quad
		$\beta_1 = \MTbetab$\quad
		$\beta_2 = \MTbetac$\quad
		$\tau_1 = \MTtaua$\quad			
	};	\node[anchor=south east, text=BleuProfondIRA] at (rel axis cs:1,0.05) {\small 
		$\beta_3 = \MTbetad$\quad
		$\tau_2 = \MTtaub$			
	};
	\node[xscale=0.2, yscale=-0.2, above=25pt] at (M) {\TresorerieMngt};
	\node[xscale=0.2, yscale=-0.2, above=25pt] at (N) {\familialTV};
	\node[xscale=0.2, yscale=-0.2, above=20pt] at (O) {\familialcar};
	\node[xscale=0.2, yscale=-0.2, above=20pt] at (P) {\archibuilding};
	\node[xscale=0.2, yscale=-0.2, above=20pt] at (Q) {\traintunnel};
\end{axis}
\end{tikzpicture}
\end{center}
\end{f}
\hrule


\begin{f}[Modèle de Vasicek]
	
	Sous une probabilité risque-neutre $\mathbb{Q}$, le taux court $(r_t)$ suit un processus d’Ornstein–Uhlenbeck à coefficients constants :
	\[
	dr_t = \kappa(\theta - r_t)\, dt + \sigma\, dW_t, \quad r_0 \in \mathbb{R}
	\]
	où :
	\begin{itemize}[nosep]
		\item $\kappa > 0$ est la vitesse de retour à la moyenne,
		\item $\theta$ est le niveau moyen de long terme,
		\item $\sigma > 0$ est la volatilité,
		\item $W_t$ est un brownien standard sous $\mathbb{Q}$.
	\end{itemize}
	
La solution de l’EDS (application du lemme d'Itô à $Y_{t}=r(t) e^{\kappa t}$):
	\[
	r_t = r_s e^{-\kappa(t-s)} + \theta(1 - e^{-\kappa(t-s)}) + \sigma \int_s^t e^{-\kappa(t-u)} dW_u
	\]
	
	\textbf{Ainsi} :
	\[
	\begin{aligned}
		\mathbb{E}_\mathbb{Q}[r_t \mid \mathcal{F}_s] &= r_s e^{-\kappa(t-s)} + \theta(1 - e^{-\kappa(t-s)}) \\
		\operatorname{Var}_\mathbb{Q}[r_t \mid \mathcal{F}_s] &= \frac{\sigma^2}{2\kappa} \left(1 - e^{-2\kappa(t-s)}\right)
	\end{aligned}
	\]
Le processus $(r_t)$ est gaussien, les taux négatifs sont possibles.
	
\end{f}

\begin{f}[Prix d'une obligation zéro-coupon (Vasicek)]

Le prix à l’instant $t$ d’une obligation zéro-coupon de maturité $T$ est donné par :
\[
ZC(t, T) = A(t, T) \, e^{-B(t, T)\, r_t}
\]
où :
\[
\begin{aligned}
	B(t, T) &= \frac{1 - e^{-\kappa(T - t)}}{\kappa} \\
	A(t, T) &= \exp \left[ \left(\theta - \frac{\sigma^2}{2\kappa^2}\right) (B(t, T) - (T - t)) - \frac{\sigma^2}{4\kappa} B(t, T)^2 \right]
\end{aligned}
\]

Cette formulation est possible grâce au fait que $\int_t^T r_s ds$ soit une variable gaussienne conditionnellement à $\mathcal{F}_t$.

\[
ZC(t, T) = \mathbb{E}_\mathbb{Q} \left[ \exp\left( -\int_t^T r_s\, ds \right) \Big| \mathcal{F}_t \right]
\]

\end{f}
\hrule

\begin{f}[Modèle Cox–Ingersoll–Ross (CIR)]
	
Sous la mesure risque-neutre $\mathbb{Q}$, le taux court $(r_t)$ suit la dynamique :
\[
dr_t = \kappa(\theta - r_t)\,dt + \sigma \sqrt{r_t}\, dW_t, \quad r_0 \geq 0
\]
avec :
\begin{itemize}[nosep]
	\item $\kappa > 0$ : vitesse de retour à la moyenne,
	\item $\theta > 0$ : niveau de long terme,
	\item $\sigma > 0$ : volatilité,
	\item $W_t$ : mouvement brownien sous $\mathbb{Q}$.
\end{itemize}

\textbf{Ainsi} :
\begin{itemize}
	\item La racine carrée $\sqrt{r_t}$ garantit $r_t \geq 0$ si $2\kappa\theta \geq \sigma^2$ (condition de Feller).
	\item Le processus $(r_t)$ est un processus de diffusion non gaussien mais à trajectoires continues.
	\item Le taux est \textbf{mean-reverting} autour de $\theta$.
\end{itemize}

Ainsi, le processus $(r_t)$ est une diffusion à lois conditionnelles explicites (sous $\mathbb{Q}$) :

Pour $s < t$, la variable $r_t$ suit une loi de type $\chi^2$ non centrale :
\[
r_t \mid \mathcal{F}_s \sim c \cdot \chi^2_{d}(\lambda)
\]
avec :
\begin{itemize}[nosep]
	\item $\displaystyle c = \frac{\sigma^2 (1 - e^{-\kappa (t - s)})}{4\kappa}$
	\item $\displaystyle d = \frac{4\kappa\theta}{\sigma^2}$ : degrés de liberté
	\item $\displaystyle \lambda = \frac{4\kappa e^{-\kappa (t - s)} r_s}{\sigma^2 (1 - e^{-\kappa (t - s)})}$
\end{itemize}

et
\[
\begin{aligned}
	\mathbb{E}_\mathbb{Q}[r_t \mid \mathcal{F}_s] =& r_s e^{-\kappa(t-s)} + \theta (1 - e^{-\kappa(t-s)}) \\
	\operatorname{Var}_\mathbb{Q}[r_t \mid \mathcal{F}_s] =& \frac{\sigma^2 r_s e^{-\kappa(t-s)} (1 - e^{-\kappa(t-s)})}{\kappa} \\
			&+ \frac{\theta \sigma^2}{2\kappa} (1 - e^{-\kappa(t-s)})^2
\end{aligned}
\]

\end{f}
\begin{f}[Prix d’une obligation zéro-coupon (CIR)]
Dans le modèle CIR, le prix d’une obligation zéro-coupon à l’instant $t$ de maturité $T$ s’écrit :
\[
ZC(t, T) = A(t, T) \cdot e^{-B(t, T)\, r_t}
\]
avec :
\[
\begin{aligned}
	B(t, T) &= \frac{2 (e^{\gamma (T - t)} - 1)}{(\gamma + \kappa)(e^{\gamma (T - t)} - 1) + 2\gamma} \\
	A(t, T) &= \left[ \frac{2\gamma e^{\frac{(\kappa + \gamma)}{2}(T - t)}}{(\gamma + \kappa)(e^{\gamma (T - t)} - 1) + 2\gamma} \right]^{\frac{2\kappa\theta}{\sigma^2}}
\end{aligned}
\]
où :
\[
\gamma = \sqrt{\kappa^2 + 2\sigma^2}
\]

\end{f}
\hrule

\begin{f}[Swaption, modèle de  Black]

Une \textbf{swaption} est une option sur un swap de taux. Elle donne le droit (et non l'obligation) de contracter un swap à une date future $T$.

\begin{itemize}[nosep]
	\item \textbf{Payer swaption} : droit de \emph{payer le taux fixe} et \emph{recevoir le taux variable}.
	\item \textbf{Receiver swaption} : droit de \emph{recevoir le taux fixe} et \emph{payer le taux variable}.
\end{itemize}

\textbf{Notations} :
\begin{itemize}[nosep]
	\item $T$ : date d’exercice de la swaption
	\item $K$ : taux fixe (strike)
	\item $S(T)$ : taux swap à la date $T$
	\item $A(T)$ : valeur actuelle des flux fixes futurs.
	\item $\sigma$ : volatilité du taux swap
\end{itemize}

Le modèle de Black (1976) est une adaptation du modèle Black–Scholes pour les produits à taux d'intérêt. Ici, le taux swap $S(T)$ joue le rôle de l’actif sous-jacent, avec un payoff de type option européenne.

\textbf{Formule de Black pour une payer swaption} :
\[
\text{SW}_{\text{payer}} = A(T) \left[ S_0 N(d_1) - K N(d_2) \right]
\]
où :
\[
\begin{aligned}
	d_1 &= \frac{\ln(S_0 / K) + \frac{1}{2} \sigma^2 T}{\sigma \sqrt{T}} \\
	d_2 &= d_1 - \sigma \sqrt{T}
\end{aligned}
\]
et $N(\cdot)$ est la fonction de répartition de la loi normale standard.

\textbf{Formule pour une receiver swaption} :
\[
\text{SW}_{\text{receiver}} = A(T) \left[ K N(-d_2) - S_0 N(-d_1) \right]
\]

\end{f}

