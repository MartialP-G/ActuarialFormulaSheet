% !TeX root = ActuarialFormSheet_MBFA-en.tex
% !TeX spellcheck = en_GB
\def\scaleBS{.95}

\begin{f}[Market Functioning]
	
The \textbf{Exchange} – a place of exchange – enables, in fact, the physical meeting between capital demanders and suppliers.
The main listings concern \textbf{equities}, \textbf{bonds} (Fixed Income), and \textbf{commodities}.
Listed are \textbf{securities} such as stocks or bonds, \textbf{funds} (Exchange Traded Funds that replicate equity indices, ETC or ETN that replicate more specific indices or commodities, SICAV or FCP, subscription bonds, warrant), \textbf{futures contracts}, \textbf{options}, \textbf{swaps}, and \textbf{structured products}.

The \textbf{Securities and Exchange Commission} (\href{https://www.sec.gov/}{SEC}) oversees :
\begin{itemize}
	\item the protection of invested savings; 
	\item the information of investors; 
	\item the proper functioning of the markets.
\end{itemize}

\textbf{\href{https://www.euronext.com/fr}{Euronext}} (including
\href{https://www.euronext.com/en/markets/amsterdam}{Amsterdam}, 
\href{https://www.euronext.com/en/markets/brussels}{Brussels}, 
\href{https://www.euronext.com/en/markets/lisbon}{Lisbon}, 
and \href{https://www.euronext.com/en/markets/paris}{Paris}) is the main stock exchange in France.
%
Its competitors include \textbf{\href{https://www.deutsche-boerse.com/dbg-en/}{Deutsche Börse}} 
(which includes \href{https://www.eurex.com/}{Eurex}, 
\href{https://www.eex.com/en/}{EEX}) in Europe,
%
or \textbf{\href{https://www.ice.com/}{ICE}} (which includes
\href{https://www.nyse.com/index}{NYSE (2012)}, 
\href{https://www.ice.com/about/history}{NYBOT (2005)}, 
\href{https://www.ice.com/futures-europe}{IPE (2001), LIFFE})
%
and \textbf{\href{https://www.cmegroup.com/}{CME Group}} (including
\href{https://www.cmegroup.com/company/cbot.html}{CBOT}, 
\href{https://www.cmegroup.com/company/nymex.html}{NYMEX}, 
\href{https://www.cmegroup.com/company/comex.html}{COMEX}) in the United States.

The \textbf{over-the-counter market} (\textbf{OTC}) represents a major share of volumes traded outside organized markets.
Since the Pittsburgh G20 (2009), certain standardized OTC derivatives must be cleared through a central entity.
These \textbf{CCPs} (Central Counterparties) thus play the role of \textbf{clearinghouses}: they replace the bilateral contract with two contracts between each party and the CCP.

\begin{tikzpicture}[scale=.47]
	\node[businessman,minimum size=1.75cm,monogramtext=MPG,tie=OrangeProfondIRA, shirt=BleuProfondIRA,skin=white,hair=OrangeProfondIRA] at (-8,0) (MPG){Short position};
	\node[businessman,mirrored,minimum size=1.75cm,monogramtext=SPG,shirt=FushiaIRA, tie=black,skin=FushiaIRA, hair=black] at (8,0) (SPG){Long position};
	%  	
	\begin{scope}[yscale=-0.03, xscale=0.03, shift={(-150,-200)}]
		%  		
		%Usine
		\draw [OrangeMoyenIRA!40,fill] (65,132) rectangle (195,225)  node [pos=0.5,text width=1.8cm, align=center] {\color{black}\bfseries Clearing\-house} ;
		%Straight Lines [id:da13081779728777443] 
		%	\draw (65,272) -- (474,272) -- (474,320);
		%Shape: Polygon [id:ds025531371089408395] 
		\draw   [GrisLogoIRA] (195,272) -- (65,272) -- (65,129) -- (85,129) -- (85,20) -- (117,20)  -- (117,52)-- (175,52) -- (175,129) -- (195,129) -- cycle ;
		%Shape: Grid [id:dp09099790819019526] 
		%	\draw  [draw opacity=0] (117,129) -- (195,129) -- (195,169) -- (117,169) -- cycle ; 
		\draw  [GrisLogoIRA] (95,100) -- (95,60)(105,100) -- (105,60)(115,100) -- (115,60)(125,100) -- (125,60)(135,100) -- (135,60)(145,100) -- (145,60)(155,100) -- (155,60)(165,100) -- (165,60) ; 
		\draw  [GrisLogoIRA]  (85,100) -- (175,100)(85,90) -- (175,90)(85,80) -- (175,80)(85,70) -- (175,70)(85,60) -- (175,60) ; 
		%
		\draw  (65,175) node (a) {} ;
		\draw  (195,175) node (b) {} ;
		%Shape: Rectangle [id:dp30583722611654374] 
	\end{scope}
	% 	
	\draw[->,>=latex, thick] (MPG) to [bend left]  node[pos=0.4, above]{Short} (a);
	\draw[<-,>=latex, thick,OrangeProfondIRA] (MPG) to [bend right]  node[pos=0.4, below=.2]{Payment} (a);
	\draw[->,>=latex, thick,FushiaIRA] (SPG) to [bend left] node[pos=0.4, below=.2]{No payment}  (b);
	\draw[->,>=latex, thick] (SPG) to [bend right] node[pos=0.4, above]{Long} (b);
	%  	
\end{tikzpicture}
\medskip

\end{f}
\hrule

\begin{f}[The Money Market]
Short-term interest-bearing securities, traded on money markets, are generally at \textbf{discounted interest}. 
Nominal rates are then annual and calculations use \textbf{proportional rates} to adjust for durations less than one year.
These securities are quoted or valued according to the discount principle and with a Euro-30/360 calendar convention.

In the American market, public debt securities are called :
Treasury Bills (T-bills) : ZC < 1 year, Treasury Notes (T-notes): ZC < 10 years,
Treasury Bonds (T-bonds) : coupon bonds with maturity > 10 years.


They are mainly :
\begin{itemize}
	\item \textbf{BTF (fixed-rate Treasury bills, France) :} issued at 13, 26, 52 weeks, discounted rate, weekly auction, nominal 1~€, settlement at T+2.
	
	\item \textbf{Treasury bills > 1 year :} same rules as bonds (see next section).
	
	\item \textbf{Certificates of deposit (CDN) :} securities issued by banks at fixed/discounted rate (short term) or variable/post-discounted rate (long term), also called BMTN.
	
	\item \textbf{Eurodollars :} USD deposits outside the USA, formerly indexed on LIBOR, now declining.
	
	\item \textbf{Commercial paper :} unsecured short-term securities issued by large companies to finance their cash flow.
\end{itemize}

\textbf{Price calculations of a fixed-rate Treasury bill with discounted interest}

In the case of a discounted interest security according to the Euro-30/360 convention, the discount \(D\) is expressed as :
\[
D=F \cdot d \cdot \frac{k}{360}
\]
where \(F\) denotes the nominal value, \(d\) the annual discount rate used to value the discounted security, and \(k\) the maturity in days.

If the discount rate \(d\) is known, then the price \(P\) is expressed as :
\[
P=F-D=F\left(1-d \cdot \frac{k}{360}\right)
\]
Similarly, if the price \(P\) is known, then the discount rate \(d\) is derived as :
\[
d=\frac{F-P}{F} \cdot \frac{360}{k}
\]


The main Futures Contracts : Federal Funds Futures (US),
Three-Month SOFR Futures (US),
ESTR Futures (EU),
SONIA Futures (UK),
Euribor Futures (EU).

\end{f}
\hrule



\begin{f}[Bond Market]

\textbf{Bonds} are long-term debt securities in which the issuer (central or local government, bank, borrowing company) promises the bondholder (the lender) to pay interest (\textbf{coupons}) periodically and to repay the \textbf{nominal value} (or face value, or principal) at maturity.
As mentioned in the previous section, \textbf{Treasury bills with a maturity greater than one year} will be treated as bonds with maturities under 5 years because their functioning is similar.

%Les \textbf{coupons} : sont payés régulièrement à la fin des périodes de coupon (annuelles ou semestrielles) jusqu'à la date d'échéance.

\textbf{Zero-coupon bonds} : pay only the nominal value at maturity. With \(E\) the issue price and \(R\) its redemption value :
	
\begin{center}
\begin{tikzpicture}[scale=0.85]
    % Draw the x-axis and y-axis.
    \def\w{11}
    \def\n{9}
    \draw[ line width=1, color=OrangeProfondIRA, arrows={-Stealth[length=4, inset=0]}] (0,0) -- (0,-0.4909) node[above left] (A) {\(E\)};	
    \foreach \y in  {0,...,3} {
        \draw (\y,0) -- (\y,-0.1);
        \ifthenelse{\y>0 }{	\node[below] at (\y,-0.1) {\tiny \( \scriptstyle \y\)};
        }{
            \node[above] at (\y,-0.1) {\tiny 0};}
    }
    \foreach \y in  {-3,...,1} {
        \draw (\y+\n,0) -- (\y+\n,-0.1);
        \ifthenelse{\y<0 }{\node[below] at (\y+\n,-0.1) {\tiny \(\scriptstyle n \y\)};}{}
        \ifthenelse{\y=0 }{\node[below] at (\y+\n,-0.1) {\tiny \(\scriptstyle n \)};
            \draw[ line width=1, color=OrangeProfondIRA, arrows={-Stealth[length=4, inset=0]}] (\y+\n,0) -- (\y+\n,1) node[below right] (B) {\(R\)};}{}
        \ifthenelse{\y>0}{	\node[below] at (\y+\n,-0.1) {\tiny \(\scriptstyle n+\y\)};}{}
    }
    \draw[ line width=1] (-.25,0) -- (4,0);
    \draw[ line width=1, dashed] (4,0) -- (5,0);
    \draw[arrows={-Stealth[length=4, inset=0]}, line width=1] (5,0) -- (\w,0);
\end{tikzpicture}
\end{center}

\textbf{Coupon bonds} :
\textbf{Fixed-rate bonds} have a coupon rate that remains constant until maturity. Assuming a \emph{bullet} repayment, with \(E\) the issue price, \(c\) the coupons, and \(R\) the redemption value, it can be illustrated as follows :
\begin{center}
\begin{tikzpicture}[scale=0.85]
    % Draw the x-axis and y-axis.
    \def\w{11}
    \def\n{9}
    \draw[ line width=1, color=OrangeProfondIRA, arrows={-Stealth[length=4, inset=0]}] (0,0) -- (0,-1.8) node[above left] (A) {\(E\)};	
    \foreach \y in  {0,...,3} {
        \draw (\y,0) -- (\y,-0.1);
        \ifthenelse{\y>0 }{	\node[below] at (\y,-0.1) {\tiny \( \scriptstyle \y\)};
        \draw[ line width=1, color=OrangeProfondIRA, arrows={-Stealth[length=4, inset=0]}] (\y,0) -- (\y,.25) node [above] {\(c\)};
        }{
            \node[above] at (\y,-0.1) {\tiny 0};}
        }
    \foreach \y in  {-3,...,1} {
        \draw (\y+\n,0) -- (\y+\n,-0.1);
        \ifthenelse{\y<0 }{\node[below] at (\y+\n,-0.1) {\tiny \(\scriptstyle n \y\)};
        \draw[ line width=1, color=OrangeProfondIRA, arrows={-Stealth[length=4, inset=0]}] (\y+\n,0) -- (\y+\n,.25) node [above] {\(c\)};}{}
        \ifthenelse{\y=0 }{\node[below] at (\y+\n,-0.1) {\tiny \(\scriptstyle n \)};
        \draw[ line width=1, color=OrangeProfondIRA, arrows={-Stealth[length=4, inset=0]}] (\y+\n,0) -- (\y+\n,2) node[below right] (B) {\(R\)};
        \draw[ line width=1, color=OrangeProfondIRA, arrows={-Stealth[length=4, inset=0]}] (\y+\n,2) -- (\y+\n,2.25) node[right] (B) {\(c\)};}{}
        \ifthenelse{\y>0}{	\node[below] at (\y+\n,-0.1) {\tiny \(\scriptstyle n+\y\)};}{}
    }
    \draw[ line width=1] (-.25,0) -- (4,0);
    \draw[ line width=1, dashed] (4,0) -- (5,0);
    \draw[arrows={-Stealth[length=4, inset=0]}, line width=1] (5,0) -- (\w,0);
\end{tikzpicture}
\end{center}

 \textbf{Indexed bonds} (inflation-linked bonds) have coupons and sometimes also the nominal value indexed to inflation or another economic indicator, such as Treasury Assimilable Bonds
		indexed to inflation (OATi). The values of \(c\) vary.
        
Bonds with \textbf{floating rate}, \textbf{variable rate}, or \textbf{resettable rate} : have a coupon rate linked to a reference interest rate (for example, the euro short-
		term rate (\href{https://www.cmegroup.com/markets/interest-rates/stirs/euro-short-term-rate.quotes.html}{€STR})).

\textbf{Perpetual bonds} have no maturity date; the principal is never repaid.
 
A distinction is often made between government bonds (Treasury bonds) and
 corporate bonds issued by private companies.



A bond is mainly defined by a \textbf{nominal value} \(F\) (Face Value), the \textbf{nominal rate} \(i\), its duration or \textbf{maturity} \(n\).
In the default case, the bondholder lends the amount \(E=F\) at issuance at time 0, receives each year a coupon \(c = i \times F\), and at \(n\), the principal or capital \(R=F\) is returned.
When \(E=F\), the issue is said to be at par, and when \(R=F\), the redemption is said to be at par.


The price of a bond is determined by the present value of the expected future cash flows (coupons and principal repayment) discounted at the market yield rate \(r\).
 
The price calculation of bonds simply relies on the present value formula :
\[
VP = \sum_{k=1}^{n} \frac{c}{(1 + r)^k} + \frac{F}{(1 + r)^n}
 \]
where :
\begin{itemize}
	\item \(PV\) : price or present value\index{Present Value} of the bond,
	\item \(r\) : market interest rate for the relevant maturity.
\end{itemize}


For bonds with periodic coupons, the coupon is divided by the number of periods (\(m\)) per year and the formula becomes :
\[ 
PV = \sum_{k=1}^{mn} \frac{c/m}{(1 + r^{(m)})^k} + \frac{R}{(1 + r^{(m)})^{mn}}
 \]
where \(c/m\) represents the periodic coupon payment and \(r^{(m)}\) the periodic interest rate.

The bond yield is the value \(r^{(m)}\), the equivalent rate of \(r\) over \(m\) periods in the year, which equates the present value \(PV_r\) with the current or market price of this bond.

\textbf{The quotation of a bond} is given as a percentage. Thus, a quotation of 97.9 on Euronext indicates a quoted value of \(97.9 / 100 \times F\). 
It is quoted excluding \textbf{accrued coupons}, the portion of the next coupon to which the seller is entitled if the bond is sold before the payment of that coupon.
\end{f}
\hrule

\begin{f}[Duration \& Convexity]

The Macaulay duration :
\[ 		
D = \sum_{t=1}^{n} t \cdot w_t, \quad \text{où} \quad w_t = \frac{PV(C_t)}{P}.
 \]	
If the payment frequency is \(k\) per year, the duration expressed in years is obtained by dividing by \(k\).
The modified duration \(D^*\) :
\[ 	
D^* = \frac{D}{1 + i}.
 \]
Which allows approximating the portfolio change \(\Delta P\) in case of interest rate changes \(\Delta_i\)
\[ 
\Delta P \approx -P\ D^* \Delta_i 
 \]
Similarly, the convexity
\[ 	
C = \frac{1}{P(i)} \times \frac{d^2 P(i)}{di^2},
 \]
which allows refining the approximation of \(\Delta P\)
\[ 	
P(i + \Delta_i) \approx P(i) \left( 1 -D^*\Delta_i + \frac{1}{2} C (\Delta_i)^2 \right).
 \]
\end{f}
\hrule



\begin{f}[CAPM]
  Capital Asset Pricing Model :

\[
E(r_i) = r_f + \beta_i (E(r_m) - r_f)
\]

\begin{itemize}
    \item \( E(r_i) \) is the expected return of asset \( i \),
    \item \( r_f \) is the risk-free rate,
    \item \( E(r_m) \) is the expected market return,
    \item \( \beta_i \) is the sensitivity coefficient of asset \( i \) with respect to market variations.
\end{itemize}

The coefficient \( \beta_i \) measures the volatility of asset \( i \) relative to the overall market.

\end{f}
\hrule


\begin{f}[Derivatives Market]

A \textbf{derivative contract} (or contingent asset) is a financial instrument whose value depends on an underlying asset or variable. Options are part of derivative contracts.


An \textbf{option} is a contract that gives the right (without obligation) to buy (call) or sell (put) an underlying asset at a fixed price (strike price) at a future date, in exchange for the payment of a premium.
The buyer (long position) pays the premium; the seller (short position) receives it. \textbf{European option} (exercise possible only at maturity) and  
 \textbf{American option} (exercise possible at any time until maturity).

Options listed on stocks are called \textit{stock options}.

\end{f}
\hrule


\begin{f}[Simple Strategies]

\ %

%\textbf{La position longue sur l'option d'achat}

With \(T\) the maturity, \(K\) the strike price, \(S\) or \(S_T\) the underlying at maturity, the payoff is \(\max (0, S_T-K)=( S_T-K)^{+}\).
Letting \(C\) be the premium, the profit realized is \(\max (0, S_T-K)-C\), with a profit if (\(S_T<V_{PM}=K + C\)) (\(PM\) stands for \textbf{break-even point}).

		\begin{tikzpicture}[yscale=.75]
\def\riskfreeBS{0.05}
\def\xminBS{5}
\def\xmaxBS{15}
\def\PxExerciceBS{10}
\def\sigmaBS{0.2}
\def\TBS{0.75}
\def\PremiumBS{{BSCall(10,{\PxExerciceBS},{\riskfreeBS},{\TBS},{\sigmaBS})}}
\begin{axis}[ 	extra tick style={tick style=BleuProfondIRA},
clip=false,
axis on top,
axis lines=middle, axis line style={BleuProfondIRA,thick,->},
scale only axis, xmin={\xminBS},xmax={\xmaxBS},enlarge x limits=0.05,
enlarge y limits=0.08,
color=BleuProfondIRA,
%		ylabel near ticks,
ylabel={Profit},
x label style={at={(axis cs:\xmaxBS+.1,0)},anchor=north east},
xlabel={underlying (\(T\))},
%		    x label style={at={(axis description cs:0.5,-0.1)},anchor=north},
%		y label style={at={(axis description cs:-0.1,.5)},rotate=90,anchor=south},
ytick=\empty,
xtick=\empty,
extra y ticks ={0},
extra y tick labels={{0}},
extra x ticks ={\PxExerciceBS},
extra x tick labels={{\(E\)}},
extra x tick style={color=BleuProfondIRA,
	tick label style={yshift=7mm}	},
title ={ \textbf{Long Call}},
%			title style={yshift=-5mm},
%	legend style={draw=none,
	%		legend columns=-1,
	%		at={(0.5,1)},
	%		anchor=south,
	%		outer sep=1em,
	%		node font=\small,
	%	},
]
%		
\addplot[name path=A,BleuProfondIRA,thick,domain={{\xminBS}:{\xmaxBS}}, samples=21,dashdotted] {Call(x,\PxExerciceBS,\PremiumBS)} 
node [pos=0.15,yshift=3mm,color=OrangeProfondIRA] {Losses}
node [pos=.8,yshift=-15mm,color=BleuProfondIRA] {Earnings};	
%\draw[BleuProfondIRA,thick] \OV{\PxExerciceBS}{\PremiumBS}{0.4}{\xminBS}{\xmaxBS} ;
%		\addplot[name path=Option,BleuProfondIRA,thick,domain={250:350}, 			 		samples=10,dashdotted,smooth] {BSPut(x,\PxExerciceBS,\riskfreeBS,\TBS,\sigmaBS)} ;	
\path [save path=\xaxis,name path=xaxis]
({\xminBS},0)		-- ({\xmaxBS},0)		;
\addplot [bottom color=OrangeProfondIRA!50, top color=OrangeProfondIRA!10] fill between [
of=A and xaxis,
split,
every segment no 1/.style=
{top color = BleuProfondIRA!50, bottom color=BleuProfondIRA!10}] ;
%\draw[use path=\xaxis, ->,OrangeProfondIRA,thick];
\draw [fill]  ({-\PremiumBS*(-1)+\PxExerciceBS},0) circle (1mm) node [above=5mm] {\(S_{PM}\)};
\draw[BleuProfondIRA, thin] ({\PxExerciceBS},0) -- ({\PxExerciceBS},{-\PremiumBS});
\node at ({\xminBS+0.25*(\xmaxBS-\xminBS)},{\pgfkeysvalueof{/pgfplots/ymin}})
{OUT};
\node at ({\PxExerciceBS},{\pgfkeysvalueof{/pgfplots/ymin}})
{AT};
\node at ({\xminBS+0.8*(\xmaxBS-\xminBS)},{\pgfkeysvalueof{/pgfplots/ymin}})
{IN};
\draw [<->, xshift=-5mm] ({\pgfkeysvalueof{/pgfplots/xmin}},0) -- ({\pgfkeysvalueof{/pgfplots/xmin}},-\PremiumBS) node [pos=0.5, xshift=-2.5mm, rotate=90] {Premium};
\end{axis}
%
\end{tikzpicture}


\medskip

%\textbf{La position courte sur l'option d'achat}


At maturity, the payoff is \(\min (0,K- S_T)=-\max(0, S_T-K)=-( S_T-K)^{+}\) and the profit realized is \(C-\max (0, S_T-K)\).

\medskip

			\begin{tikzpicture}[yscale=.75]
		\def\riskfreeBS{0.05}
		\def\xminBS{5}
		\def\xmaxBS{15}
		\def\PxExerciceBS{10}
		\def\sigmaBS{0.2}
		\def\TBS{0.75}
		\def\PremiumBS{{BSCall(10,{\PxExerciceBS},{\riskfreeBS},{\TBS},{\sigmaBS})}}
\begin{axis}[ 
	clip=false,
	axis on top,
	axis lines=middle, axis line style={BleuProfondIRA,thick,->},
	scale only axis, xmin={\xminBS},xmax={\xmaxBS},enlarge x limits=0.05,
	enlarge y limits=0.125,
	color=BleuProfondIRA,
	%		ylabel near ticks,
	ylabel={Profit},
	x label style={={at={(current axis.right of origin)}}},
	%    x label style={at={(axis description cs:1,-0.1)},anchor=south},
	%		x label style={at={(1,0.5)}},
	xlabel={ss-jacent ($T$)},
	%		    x label style={at={(axis description cs:0.5,-0.1)},anchor=north},
	%		y label style={at={(axis description cs:-0.1,.5)},rotate=90,anchor=south},
	ytick=\empty,
	xtick=\empty,
	extra y ticks ={0},
	extra y tick labels={{0}},
	extra x ticks ={\PxExerciceBS},
	extra x tick labels={{$E$}},
	extra x tick style={color=BleuProfondIRA,
		tick label style={yshift=-0mm}	},
	title ={ \textbf{Short Call}},
	title style={yshift=10mm},
	%	legend style={draw=none,
		%		legend columns=-1,
		%		at={(0.5,1)},
		%		anchor=south,
		%		outer sep=1em,
		%		node font=\small,
		%	},
	]
	%		
\addplot[name path=A,BleuProfondIRA,thick,domain={{\xminBS}:{\xmaxBS}}, samples=21,dashdotted] {-Call(x,\PxExerciceBS,\PremiumBS)} 
node [pos=0.15,yshift=-4mm,color=BleuProfondIRA] {Gains}
node [pos=.8,yshift=10mm,color=OrangeProfondIRA] {Pertes};	
	%\draw[BleuProfondIRA,thick] \OV{\PxExerciceBS}{\PremiumBS}{0.4}{\xminBS}{\xmaxBS} ;
	%		\addplot[name path=Option,BleuProfondIRA,thick,domain={250:350}, 			 		samples=10,dashdotted,smooth] {BSPut(x,\PxExerciceBS,\riskfreeBS,\TBS,\sigmaBS)} ;	
	\path [save path=\xaxis,name path=xaxis]
	({\xminBS},0)		-- ({\xmaxBS},0)		;
	\addplot [bottom color=OrangeProfondIRA!50, top color=OrangeProfondIRA!10] fill between [
	of=A and xaxis,
	split,
	every segment no 0/.style=
	{top color = BleuProfondIRA!50, bottom color=BleuProfondIRA!10}] ;
	%\draw[use path=\xaxis, ->,OrangeProfondIRA,thick];
	\draw [fill]  ({-\PremiumBS*(-1)+\PxExerciceBS},0) circle (1mm) node [above=5mm] {$S_{PM}$};
	\draw[BleuProfondIRA, thin] ({\PxExerciceBS},0) -- ({\PxExerciceBS},{\PremiumBS});
	\node at ({\xminBS+0.25*(\xmaxBS-\xminBS)},{\pgfkeysvalueof{/pgfplots/ymin}+1})
	{OUT};
	\node at ({\PxExerciceBS},{\pgfkeysvalueof{/pgfplots/ymin}+1})
	{AT};
	\node at ({\xminBS+0.8*(\xmaxBS-\xminBS)},{\pgfkeysvalueof{/pgfplots/ymin}+01})
	{IN};
	\draw [<->, xshift=-5mm] ({\pgfkeysvalueof{/pgfplots/xmin}},0) -- ({\pgfkeysvalueof{/pgfplots/xmin}},\PremiumBS) node [pos=0.5, xshift=-2.5mm, rotate=90] {Prime};
\end{axis}
%
	\end{tikzpicture}

\medskip


%\textbf{La position longue sur l'option de vente}

At maturity, the payoff is \(\max (0,K- S_T)=(K- S_T)^{+}\).
Letting \(P\) be the put premium, the profit realized is \(\max (0,K- S_T)-P\), positive if \(V_{PM}=K -P<S_T\).

\medskip

		\begin{tikzpicture}[yscale=.75]
	\def\riskfreeBS{0.05}
	\def\xminBS{5}
	\def\xmaxBS{15}
	\def\PxExerciceBS{10}
	\def\sigmaBS{0.2}
	\def\TBS{0.75}
	\def\PremiumBS{{BSCall(10,{\PxExerciceBS},{\riskfreeBS},{\TBS},{\sigmaBS})}}
	\begin{axis}[ 
		clip=false,
		axis on top,
		axis lines=middle, axis line style={BleuProfondIRA,thick,->},
		scale only axis, xmin={\xminBS},xmax={\xmaxBS},enlarge x limits=0.05,
		enlarge y limits=0.1,
		color=BleuProfondIRA,
		%		ylabel near ticks,
		ylabel={Profit},
		x label style={={at={(current axis.right of origin)},below=5mm}},
		%    x label style={at={(axis description cs:1,-0.1)},anchor=south},
		%		x label style={at={(1,0.5)}},
		xlabel={underlying ($T$)},
		%		    x label style={at={(axis description cs:0.5,-0.1)},anchor=north},
		%		y label style={at={(axis description cs:-0.1,.5)},rotate=90,anchor=south},
		ytick=\empty,
		xtick=\empty,
		extra y ticks ={0},
		extra y tick labels={{0}},
		extra x ticks ={\PxExerciceBS},
		extra x tick labels={{$E$}},
		extra x tick style={color=BleuProfondIRA,
			tick label style={yshift=7mm}	},
		title ={ \textbf{Long Put}},
%		title style={yshift=0mm},
		%	legend style={draw=none,
			%		legend columns=-1,
			%		at={(0.5,1)},
			%		anchor=south,
			%		outer sep=1em,
			%		node font=\small,
			%	},
		]		%		
		\addplot[name path=A,BleuProfondIRA,thick,domain={{\xminBS}:{\xmaxBS}}, samples=21,dashdotted] {Put(x,\PxExerciceBS,\PremiumBS)} 
		node [pos=0.15,yshift=-15mm,color=BleuProfondIRA] {Earnings}
		node [pos=.75,yshift=4mm,color=OrangeProfondIRA] {Losses};	
		%\draw[BleuProfondIRA,thick] \OV{\PxExerciceBS}{\PremiumBS}{0.4}{\xminBS}{\xmaxBS} ;
		%	\addplot[BleuProfondIRA,thick,dashdotted] {Call(x,50,2)};
		%	\addplot[BleuProfondIRA,thick] {Put(x,50,3)+Call(x,50,2)};
		%\addplot[name path=Option,BleuProfondIRA,thick,domain={250:350}, 			 		samples=10,dashdotted,smooth] {BSPut(x,\PxExerciceBS,\riskfreeBS,\TBS,\sigmaBS)} ;	
		\path [save path=\xaxis,name path=xaxis]
		({\xminBS},0)		-- ({\xmaxBS},0)		;
		\addplot [bottom color=OrangeProfondIRA!50, top color=OrangeProfondIRA!10] fill between [
		of=A and xaxis,
		split,
		every segment no 0/.style=
		{top color = BleuProfondIRA!50, bottom color=BleuProfondIRA!10}] ;
		%\draw[use path=\xaxis, ->,OrangeProfondIRA,thick];
		\draw [fill]  ({-\PremiumBS+\PxExerciceBS},0) circle (1mm) node [above=5mm] {$S_{PM}$};
		\draw[BleuProfondIRA, thin] ({\PxExerciceBS},0) -- ({\PxExerciceBS},{-\PremiumBS});
		\node at ({\xminBS+0.25*(\xmaxBS-\xminBS)},{\pgfkeysvalueof{/pgfplots/ymin}})
		{IN};
		\node at ({\PxExerciceBS},{\pgfkeysvalueof{/pgfplots/ymin}})
		{AT};
		\node at ({\xminBS+0.8*(\xmaxBS-\xminBS)},{\pgfkeysvalueof{/pgfplots/ymin}})
		{OUT};
		\draw [<->, xshift=-5mm] ({\pgfkeysvalueof{/pgfplots/xmin}},0) -- ({\pgfkeysvalueof{/pgfplots/xmin}},-\PremiumBS) node [pos=0.5, xshift=-2.5mm, rotate=90] {Premium};
	\end{axis}
	%
\end{tikzpicture}

    	     

%\textbf{La position courte sur l'option de vente}

\medskip

		\begin{tikzpicture}[yscale=.75]
	\def\riskfreeBS{0.05}
	\def\xminBS{5}
	\def\xmaxBS{15}
	\def\PxExerciceBS{10}
	\def\sigmaBS{0.2}
	\def\TBS{0.75}
	\def\PremiumBS{{BSCall(10,{\PxExerciceBS},{\riskfreeBS},{\TBS},{\sigmaBS})}}
	\begin{axis}[ 	extra tick style={tick style=BleuProfondIRA},
		clip=false,
		axis on top,
		axis lines=middle, axis line style={BleuProfondIRA,thick,->},
		scale only axis, xmin={\xminBS},xmax={\xmaxBS},enlarge x limits=0.05,
		enlarge y limits=0.125,
		color=BleuProfondIRA,
		%		ylabel near ticks,
		ylabel={Profit},
x label style={at={(axis cs:\xmaxBS+.1,0)},anchor=north east},
		xlabel={ss-jacent ($T$)},
		%		    x label style={at={(axis description cs:0.5,-0.1)},anchor=north},
		%		y label style={at={(axis description cs:-0.1,.5)},rotate=90,anchor=south},
		ytick=\empty,
		xtick=\empty,
		extra y ticks ={0},
		extra y tick labels={{0}},
		extra x ticks ={\PxExerciceBS},
		extra x tick labels={{$E$}},
		extra x tick style={color=BleuProfondIRA,
			tick label style={yshift=0mm}	},
		title ={ \textbf{Short Put}},
		title style={yshift=10mm},
		%	legend style={draw=none,
			%		legend columns=-1,
			%		at={(0.5,1)},
			%		anchor=south,
			%		outer sep=1em,
			%		node font=\small,
			%	},
		]		%		
		\addplot[name path=A,BleuProfondIRA,thick,domain={{\xminBS}:{\xmaxBS}}, samples=21,dashdotted] {-Put(x,\PxExerciceBS,\PremiumBS)} 
		node [pos=0.15,yshift=10mm,color=OrangeProfondIRA] {Perte}
		node [pos=.75,yshift=-4mm,color=BleuProfondIRA] {Gains};	
		%\draw[BleuProfondIRA,thick] \OV{\PxExerciceBS}{\PremiumBS}{0.4}{\xminBS}{\xmaxBS} ;
		%	\addplot[BleuProfondIRA,thick,dashdotted] {Call(x,50,2)};
		%	\addplot[BleuProfondIRA,thick] {Put(x,50,3)+Call(x,50,2)};
		%\addplot[name path=Option,BleuProfondIRA,thick,domain={250:350}, 			 		samples=10,dashdotted,smooth] {BSPut(x,\PxExerciceBS,\riskfreeBS,\TBS,\sigmaBS)} ;	
		\path [save path=\xaxis,name path=xaxis]
		({\xminBS},0)		-- ({\xmaxBS},0)		;
		\addplot [bottom color=OrangeProfondIRA!50, top color=OrangeProfondIRA!10] fill between [
		of=A and xaxis,
		split,
		every segment no 1/.style=
		{top color = BleuProfondIRA!50, bottom color=BleuProfondIRA!10}] ;
		%\draw[use path=\xaxis, ->,OrangeProfondIRA,thick];
		\draw [fill]  ({-\PremiumBS+\PxExerciceBS},0) circle (1mm) node [above=5mm] {$S_{PM}$};
		\draw[BleuProfondIRA, thin] ({\PxExerciceBS},0) -- ({\PxExerciceBS},{\PremiumBS});
		\node at ({\xminBS+0.25*(\xmaxBS-\xminBS)},{\pgfkeysvalueof{/pgfplots/ymin}+1})
		{IN};
		\node at ({\PxExerciceBS},{\pgfkeysvalueof{/pgfplots/ymin}+1})
		{AT};
		\node at ({\xminBS+0.8*(\xmaxBS-\xminBS)},{\pgfkeysvalueof{/pgfplots/ymin}+1})
		{OUT};
		\draw [<->, xshift=-5mm] ({\pgfkeysvalueof{/pgfplots/xmin}},0) -- ({\pgfkeysvalueof{/pgfplots/xmin}},\PremiumBS) node [pos=0.5, xshift=-2.5mm, rotate=90] {Prime};
	\end{axis}
	%
\end{tikzpicture}
    	     

At maturity, the payoff is \(\min (0, S_T-K)=-(K- S_T)^+\).


\end{f}
\hrule

\begin{f}[Spread Strategies]
\textbf{Spread strategy} : uses two or more options of the same type (two call options or two put options).  
If the strike prices vary, it is a \textbf{vertical spread}.  
If the maturities change, it is a \textbf{horizontal spread}.

A vertical spread strategy involves a long position and a short position on call options on the same underlying asset, with the same maturity but different strike prices.  
We distinguish : \textbf{bull vertical spread} and \textbf{bear vertical spread}.

\textbf{Bull vertical spread} : anticipating a moderate rise in the underlying asset, the investor takes a long position on \(C_1\) and a short position on \(C_2\) under the condition \(E_1 < E_2\).  
Net result at maturity :

	\begin{tikzpicture}[scale=.52]
		\def\riskfreeBS{0.05}
		\def\xminBS{5}
		\def\xmaxBS{15}
		\def\PxExerciceBSa{9}
		\def\PxExerciceBSb{12}
		\def\sigmaBS{0.2}
		\def\TBS{0.75}
		\def\PremiumBSa{{BSCall(11,{\PxExerciceBSa},{\riskfreeBS},{\TBS},{\sigmaBS})}}
		\def\PremiumBSb{{BSCall(11,{\PxExerciceBSb},{\riskfreeBS},{\TBS},{\sigmaBS})}}
		\begin{axis}[ 
			width=0.8\textwidth,
			height=0.5\textwidth, 
			extra tick style={tick style=BleuProfondIRA},
			clip=false,
			axis on top,
			axis lines=middle, axis line style={BleuProfondIRA,thick,->},
			scale only axis, xmin={\xminBS},xmax={\xmaxBS},enlarge x limits=0.05,
			enlarge y limits=0.125,
			color=BleuProfondIRA,
			%		ylabel near ticks,
			ylabel={Profit},
			x label style={={at={(current axis.right of origin)}}},
			%    x label style={at={(axis description cs:1,-0.1)},anchor=south},
			%		x label style={at={(1,0.5)}},
			xlabel={underlying (\(T\))},
			%		    x label style={at={(axis description cs:0.5,-0.1)},anchor=north},
			%		y label style={at={(axis description cs:-0.1,.5)},rotate=90,anchor=south},
			ytick=\empty,
			xtick=\empty,
			extra y ticks ={0},
			extra y tick labels={{0}},
			extra x ticks ={{\PxExerciceBSa},{\PxExerciceBSb}},
			extra x tick labels={{\(E_1\ \ \ \ \ \)},{\(E_2\)}},
			extra x tick style={color=BleuProfondIRA,
				tick label style={yshift=-0mm}	},
			]
			%		
			\addplot[name path=A,BleuProfondIRA,thin,domain={{\xminBS}:{\xmaxBS-1.75}}, samples=21,dashed] {Call(x,\PxExerciceBSa,\PremiumBSa)} 
			node [pos=0.15, above] {\small Long \(C_1\)};	
			\addplot[name path=B,BleuProfondIRA,thin,domain={{\xminBS}:{\xmaxBS}}, samples=21,dashed] {-Call(x,\PxExerciceBSb,\PremiumBSb)} 
			node [pos=0.15, above] {\small  Short \(C_2\)};	
			\addplot[name path=EVH,OrangeProfondIRA,thick,domain={{\xminBS}:{\xmaxBS}}, samples=41] {Call(x,\PxExerciceBSa,\PremiumBSa)-Call(x,\PxExerciceBSb,\PremiumBSb)} node [pos=0.15, above] {Bull vertical spread};	
			%\draw[BleuProfondIRA,thick] \OV{\PxExerciceBSa}{\PremiumBSa}{0.4}{\xminBS}{\xmaxBS} ;
			%		\addplot[name path=Option,BleuProfondIRA,thick,domain={250:350}, 			 		samples=10,dashdotted,smooth] {BSPut(x,\PxExerciceBSa,\riskfreeBS,\TBS,\sigmaBS)} ;	
			
			\draw[BleuProfondIRA, thin, dashed] ({\PxExerciceBSa},0) -- ({\PxExerciceBSa},{-\PremiumBSa})	;
			\draw[BleuProfondIRA, thin, dashed] ({\PxExerciceBSb},0) -- ({\PxExerciceBSb},{\PremiumBSb})	;
		\end{axis}
		%
	\end{tikzpicture}



\textbf{Bear vertical spread} : anticipating a moderate decline in the underlying asset, the investor sells the more expensive option and buys the cheaper one.


	\begin{tikzpicture}[scale=.52]
		\def\riskfreeBS{0.05}
		\def\xminBS{5}
		\def\xmaxBS{15}
		\def\PxExerciceBSa{9}
		\def\PxExerciceBSb{12}
		\def\sigmaBS{0.2}
		\def\TBS{0.75}
		\def\PremiumBSa{{BSCall(11,{\PxExerciceBSa},{\riskfreeBS},{\TBS},{\sigmaBS})}}
		\def\PremiumBSb{{BSCall(11,{\PxExerciceBSb},{\riskfreeBS},{\TBS},{\sigmaBS})}}
		\begin{axis}[ 
			width=0.8\textwidth,
			height=0.5\textwidth, 
			extra tick style={tick style=BleuProfondIRA},
			clip=false,
			axis on top,
			axis lines=middle, axis line style={BleuProfondIRA,thick,->},
			scale only axis, xmin={\xminBS},xmax={\xmaxBS},enlarge x limits=0.05,
			enlarge y limits=0.125,
			color=BleuProfondIRA,
			%		ylabel near ticks,
			ylabel={Profit},
			x label style={={at={(current axis.right of origin)}}},
			%    x label style={at={(axis description cs:1,-0.1)},anchor=south},
			%		x label style={at={(1,0.5)}},
			xlabel={underlying (\(T\))},
			%		    x label style={at={(axis description cs:0.5,-0.1)},anchor=north},
			%		y label style={at={(axis description cs:-0.1,.5)},rotate=90,anchor=south},
			ytick=\empty,
			xtick=\empty,
			extra y ticks ={0},
			extra y tick labels={{0}},
			extra x ticks ={{\PxExerciceBSa},{\PxExerciceBSb}},
			extra x tick labels={{\(E_1\ \ \ \ \ \)},{\(E_2\)}},
			extra x tick style={color=BleuProfondIRA,
				tick label style={yshift=-0mm}	},
			]
			%		
			\addplot[name path=A,BleuProfondIRA,thin,domain={{\xminBS}:{\xmaxBS-1.75}}, samples=21,dashed] {-Call(x,\PxExerciceBSa,\PremiumBSa)} 
			node [pos=0.15, above] {\small Short \(C_1\)};	
			\addplot[name path=B,BleuProfondIRA,thin,domain={{\xminBS}:{\xmaxBS}}, samples=21,dashed] {Call(x,\PxExerciceBSb,\PremiumBSb)} 
			node [pos=0.15, above] {\small  Long \(C_2\)};		
			\addplot[name path=EVH,OrangeProfondIRA,thick,domain={{\xminBS}:{\xmaxBS}}, samples=41] {-Call(x,\PxExerciceBSa,\PremiumBSa)+Call(x,\PxExerciceBSb,\PremiumBSb)} node [pos=0.15, above] {Bear vertical spread};	
			%\draw[BleuProfondIRA,thick] \OV{\PxExerciceBSa}{\PremiumBSa}{0.4}{\xminBS}{\xmaxBS} ;
			%		\addplot[name path=Option,BleuProfondIRA,thick,domain={250:350}, 			 		samples=10,dashdotted,smooth] {BSPut(x,\PxExerciceBSa,\riskfreeBS,\TBS,\sigmaBS)} ;	
			
			\draw[BleuProfondIRA, thin, dashed] ({\PxExerciceBSa},0) -- ({\PxExerciceBSa},{\PremiumBSa})	;
			\draw[BleuProfondIRA, thin, dashed] ({\PxExerciceBSb},0) -- ({\PxExerciceBSb},{-\PremiumBSb})	;
		\end{axis}
		%
	\end{tikzpicture}

% Code TikZ conservé ici (éventuellement inséré)

\textbf{Butterfly spread} : anticipates a small movement in the underlying asset.  
It is a combination of a bull vertical spread and a bear vertical spread.  
This strategy is suitable when large movements are considered unlikely.  
Requires a low initial investment.

	\begin{tikzpicture}[scale=.52]
		\def\riskfreeBS{0.05}
		\def\xminBS{5}
		\def\xmaxBS{15}
		\def\PxExerciceBSa{8}
		\def\PxExerciceBSb{10}
		\def\PxExerciceBSc{12}
		\def\sigmaBS{0.2}
		\def\TBS{0.75}
		\def\PremiumBSa{BSCall(11,{\PxExerciceBSa},{\riskfreeBS},{\TBS},{\sigmaBS})}
		\def\PremiumBSb{BSCall(11,{\PxExerciceBSb},{\riskfreeBS},{\TBS},{\sigmaBS})}
		\def\PremiumBSc{BSCall(11,{\PxExerciceBSc},{\riskfreeBS},{\TBS},{\sigmaBS})}
		\begin{axis}[
			width=0.8\textwidth,
			height=0.5\textwidth, 
			extra tick style={tick style=BleuProfondIRA},
			clip=false,
			axis on top,
			axis lines=middle, axis line style={BleuProfondIRA,thick,->},
			scale only axis, xmin={\xminBS},xmax={\xmaxBS},enlarge x limits=0.05,
			enlarge y limits=0.125,
			color=BleuProfondIRA,
			%		ylabel near ticks,
			ylabel={Profit},
			x label style={={at={(current axis.right of origin)}}},
			%    x label style={at={(axis description cs:1,-0.1)},anchor=south},
			%		x label style={at={(1,0.5)}},
			xlabel={underlying (\(T\))},
			%		    x label style={at={(axis description cs:0.5,-0.1)},anchor=north},
			%		y label style={at={(axis description cs:-0.1,.5)},rotate=90,anchor=south},
			ytick=\empty,
			xtick=\empty,
			extra y ticks ={0},
			extra y tick labels={{0}},
			extra x ticks ={{\PxExerciceBSa},{\PxExerciceBSb},{\PxExerciceBSc}},
			extra x tick labels={{\(E_1\ \ \ \ \ \)},{\(E_2\)},{\(E_3\)}},
			extra x tick style={color=BleuProfondIRA,
				tick label style={yshift=-0mm}	},
			]
			%		
			\addplot[name path=A,BleuProfondIRA,thin,domain={{\xminBS}:{\xmaxBS}}, samples=21,dashed] {Call(x,\PxExerciceBSa,\PremiumBSa)} 
			node [pos=0.15, above] {\small Long \(C_1\)};	
			\addplot[name path=B,BleuProfondIRA,thin,domain={{\xminBS}:{\xmaxBS-1.75}}, samples=21,dashed] {-2*Call(x,\PxExerciceBSb,\PremiumBSb)} 
			node [pos=0.15, above] {\small Short \(2\ C_2\)};	
			\addplot[name path=C,BleuProfondIRA,thin,domain={{\xminBS}:{\xmaxBS}}, samples=21,dashed] {Call(x,\PxExerciceBSc,\PremiumBSb)} 
			node [pos=0.15, above] {\small Long \(C_3\)};	
			\addplot[name path=EP,OrangeProfondIRA,thick,domain={{\xminBS}:{\xmaxBS}}, samples=41] {
				Call(x,\PxExerciceBSa,\PremiumBSa) - 2*Call(x,\PxExerciceBSb,\PremiumBSb)+
				Call(x,\PxExerciceBSc,\PremiumBSc)}
			node [pos=0.5, below=30pt] {Butterfly spread};	
			\draw[BleuProfondIRA, thin, dashed] ({\PxExerciceBSa},0) -- ({\PxExerciceBSa},{-\PremiumBSa})	;
			\draw[BleuProfondIRA, thin, dashed] ({\PxExerciceBSb},0) -- ({\PxExerciceBSb},{2*\PremiumBSb})	;
			\draw[BleuProfondIRA, thin, dashed] ({\PxExerciceBSc},0) -- ({\PxExerciceBSc},{-\PremiumBSc})	;
		\end{axis}
		%
	\end{tikzpicture}
% Code TikZ conservé ici (éventuellement inséré)

\end{f}
\hrule

\begin{f}[Combined strategies]
	
	A \textbf{combined strategy} uses both call and put options. Notably, we distinguish between \textbf{straddles} and \textbf{strangles}.
	
	A \textbf{straddle} combines the purchase of a call option and a put option with the same expiration date and strike price. This strategy bets on a large price movement, either upward or downward. The maximum loss occurs if the price at expiration is equal to the strike price.
	
	A \textbf{strangle} is the purchase of a call and a put with the same expiration date but different strike prices. It assumes a very large movement in the value of the underlying asset.
	
		\begin{tikzpicture}[scale=.52]
			\def\xminBS{200}
			\def\xmaxBS{275}
			\def\PxExerciceBSa{230}
			\def\PxExerciceBSb{245}
			\def\PremiumBSa{20.69}
			\def\PremiumBSb{23.79}
			\begin{axis}[ 
				width=0.8\textwidth,
				height=0.5\textwidth, 
				extra tick style={tick style=BleuProfondIRA},
				clip=false,
				axis on top,
				axis lines=middle, axis line style={BleuProfondIRA,thick,->},
				scale only axis, xmin={\xminBS},xmax={\xmaxBS},enlarge x limits=0.05,
				enlarge y limits=0.125,
				color=BleuProfondIRA,
				ylabel={Profit},
				x label style={={at={(current axis.right of origin)}}},
				xlabel={underlying (\(T\))},
				ytick=\empty,
				extra y ticks ={0},
				extra y tick labels={{0}},
				extra x ticks ={{\PxExerciceBSa},{\PxExerciceBSb}},
				extra x tick labels={{\(E_1\ \ \ \ \ \)},{\(E_2\)}},
				extra x tick style={color=BleuProfondIRA,
					tick label style={yshift=-10mm}	},
				]
				\addplot[name path=A,BleuProfondIRA,thin,domain={{\xminBS}:{\xmaxBS-1.75}}, samples=21,dashed] {Call(x,\PxExerciceBSa,\PremiumBSa)} 
				node [pos=0.15, below] {\small Long \(C_1\)};	
				\addplot[name path=B,BleuProfondIRA,thin,domain={{\xminBS}:{\xmaxBS}}, samples=21,dashed] {Put(x,\PxExerciceBSb,\PremiumBSb)} 
				node [pos=0.85, above] {\small  Long \(P_2\)};		
				\addplot[name path=EVH,OrangeProfondIRA,thick,domain={{\xminBS}:{\xmaxBS}}, samples=21] {Call(x,\PxExerciceBSa,\PremiumBSa)+Put(x,\PxExerciceBSb,\PremiumBSb)} node [pos=0.5, above] {\small Strangle};	
				\draw[BleuProfondIRA, thin, dashed] ({\PxExerciceBSa},0) -- ({\PxExerciceBSa},{-\PremiumBSa})	;
				\draw[BleuProfondIRA, thin, dashed] ({\PxExerciceBSb},0) -- ({\PxExerciceBSb},{-\PremiumBSb})	;
			\end{axis}
		\end{tikzpicture}


		\begin{tikzpicture}[scale=.52]
			\def\xminBS{200}
			\def\xmaxBS{275}
			\def\PxExerciceBSa{230}
			\def\PxExerciceBSb{245}
			\def\PremiumBSa{15.19}
			\def\PremiumBSb{14.29}
			\begin{axis}[ 
				width=0.8\textwidth,
				height=0.5\textwidth, 
				extra tick style={tick style=BleuProfondIRA},
				clip=false,
				axis on top,
				axis lines=middle, axis line style={BleuProfondIRA,thick,->},
				scale only axis, xmin={\xminBS},xmax={\xmaxBS},enlarge x limits=0.05,
				enlarge y limits=0.125,
				color=BleuProfondIRA,
				ylabel={Profit},
				x label style={={at={(current axis.right of origin)}}},
				xlabel={underlying (\(T\))},
				ytick=\empty,
				extra y ticks ={0},
				extra y tick labels={{0}},
				extra x ticks ={{\PxExerciceBSa},{\PxExerciceBSb}},
				extra x tick labels={{\(E_1\ \ \ \ \ \)},{\(E_2\)}},
				extra x tick style={color=BleuProfondIRA,
					tick label style={yshift=-10mm}	},
				]
				\addplot[name path=A,BleuProfondIRA,thin,domain={{\xminBS}:{\xmaxBS}}, samples=21,dashed] {Put(x,\PxExerciceBSa,\PremiumBSa)} 
				node [pos=0.85, below] {\small Long \(P_1\)};	
				\addplot[name path=B,BleuProfondIRA,thin,domain={{\xminBS}:{\xmaxBS}}, samples=21,dashed] {Call(x,\PxExerciceBSb,\PremiumBSb)} 
				node [pos=0.15, above] {\small  Long \(C_2\)};		
				\addplot[name path=EVH,OrangeProfondIRA,thick,domain={{\xminBS}:{\xmaxBS}}, samples=21] {Put(x,\PxExerciceBSa,\PremiumBSa)+Call(x,\PxExerciceBSb,\PremiumBSb)} node [pos=0.5, above] {\small Strangle};	
				\draw[BleuProfondIRA, thin, dashed] ({\PxExerciceBSa},0) -- ({\PxExerciceBSa},{-\PremiumBSa})	;
				\draw[BleuProfondIRA, thin, dashed] ({\PxExerciceBSb},0) -- ({\PxExerciceBSb},{-\PremiumBSb})	;
			\end{axis}
		\end{tikzpicture}
	
	
	
\end{f}
\hrule

\begin{f}[Absence of arbitrage opportunity]
It is impossible to realize a risk-free gain from a zero initial investment. Thus, no risk-free profit is possible by exploiting price differences. 

\end{f}

\begin{f}[Parity relation]
	
AAO implies the following relationship between the Call and the Put (stock) :

\[S_t- C_t + P_t = K e^{-i_{f}.\tau}\]	
\end{f}
\hrule

\begin{f}[The Cox-Ross-Rubinstein model]
	
It is based on a discrete-time process with two possible price movements at each period : an increase (factor \(u\)) or a decrease (factor \(d\)), with \(u > 1 + i_{f}\) and \(d < 1 + i_{f}\). The price at \(t = 1\) is then \( S_{1}^{u} = S_{0} u \) or \( S_{1}^{d} = S_{0} d \), according to a probability \(q\) or \(1-q\).

\begin{tikzpicture}
	[sibling distance=5em,
	every node/.style = {shape=rectangle, rounded corners, fill=OrangeProfondIRA!20,
		align=center,  draw=OrangeProfondIRA, text=BleuProfondIRA } ,grow=right,
	edge from parent/.style={draw=OrangeProfondIRA, thick}]
	\node (A){\(S_0\)}
	child {node  (B) {\(S_d\)}
		child {node {\(S_{dd}\)}}
		child} 
	child {node (C) {\(S_u\)} 
		child {node  {\(S_{du}\)}}
		child {node {\(S_{uu}\)}}    
	};
	\draw [draw=none] 
	($ (A.east) + (0,0.2) $) -- node[draw=none, fill=none, above left, BleuProfondIRA] {\(q\)} ($ (C.west) + (0,-0.2) $);
	\draw [draw=none]  
	($ (A.east) + (0,-0.2) $) -- node[draw=none, fill=none,below left, BleuProfondIRA] {\(1 - q\)} ($ (B.west) + (0,0.2) $);
\end{tikzpicture}
\quad
\begin{tikzpicture}
	[sibling distance=5em,
	every node/.style = {shape=rectangle, rounded corners, fill=OrangeProfondIRA!20,
		align=center,  draw=OrangeProfondIRA, text=BleuProfondIRA } ,grow=right,
	edge from parent/.style={draw=OrangeProfondIRA, thick}]
	\node {\(C_0\)}
	child {node  {\(C_d\)}
		child {node {\(C_{dd}=(S_{dd}-K)^{+}\)}}
		child}
	child {node {\(C_u\)} 
		child {node  {\(C_{du}=(S_{du}-K)^{+}\)}}
		child {node  {\(C_{uu}=(S_{uu}-K)^{+}\)}}  
	};
\end{tikzpicture}

This model extends to \(n\) periods with \(n+1\) possible prices for \(S_T\). At expiration, the value of a call option is given by \( C_{1}^{u} = (S_{1}^{u} - K)^+ \) and \( C_{1}^{d} = (S_{1}^{d} - K)^+ \).

\textbf{Absence of arbitrage opportunity} implies
\[
d < 1 + i_{f} < u
\]
and a risk-neutral probability
\[q = \frac{(1 + i_{f}) - d}{u - d}\]

\textbf{Call price} (with \( S_{1}^{d} < K < S_{1}^{u} \)) :
\[
C_{0} = \frac{1}{1+i_f} \left[ q C_{1}^{u} + (1 - q) C_{1}^{d} \right]
\]

We can also construct a replication portfolio composed of \(\Delta\) shares and \(B\) bonds, such that :
\[
\begin{cases}
	\Delta = \frac{S_{1}^{u} - K}{S_{1}^{u} - S_{1}^{d}}, \\
	B = \frac{-S_{1}^{d}}{1+i_f} \cdot \Delta
\end{cases}
\quad \Rightarrow \quad \Pi_0 = \Delta S_0 + B
\]

\textbf{Put price} :
\[
P_{0} = \frac{1}{1+i_f} \left[ q P_{1}^{u} + (1 - q) P_{1}^{d} \right]
\]

\textbf{Determination of \(q\), \(u\), \(d\)} :  
By calibrating the model to match the first moments of the return under the risk-neutral probability (expected value \(i_f\), variance \(\sigma^2 \delta t\)), we obtain :
\[
e^{i_{f} \delta t} = q u + (1-q) d, \qquad q u^2 + (1-q) d^2 - [q u + (1-q) d]^2 = \sigma^2 \delta t
\]

With the constraint \(u = \frac{1}{d}\), we obtain :
\[
\begin{array}{l}
	q = \frac{e^{-i_f \delta_t} - d}{u - d} \\
	u = e^{\sigma \sqrt{\delta t}} \\
	d = e^{-\sigma \sqrt{\delta t}}
\end{array}
\]

\end{f}
\hrule

\begin{f}[The Black \& Scholes Model]
Assumptions of the model
\begin{itemize}
	\item The risk-free rate \(R\) is constant. We define \(i_f = \ln(1+R)\), which implies \((1+R)^t = e^{i_f t}\).
	\item The stock price \(S_t\) follows a geometric Brownian motion :
	\[
	dS_t = \mu S_t dt + \sigma S_t dW_t 
	\]
	\[
	 S_t = S_0 \exp\left(\sigma W_t + \left( \mu - \frac{1}{2}\sigma^2 \right)t \right)
	\]
	\item No dividend during the option's lifetime.  
	\item The option is "European" (exercised only at maturity).  
	\item Frictionless market : no taxes or transaction costs.  
	\item Short selling is allowed. 
\end{itemize}

The Black-Scholes-Merton equation for valuing a derivative contract \(f\) is :
\[
\frac{\partial f}{\partial t} + i_f S \frac{\partial f}{\partial S} + \frac{1}{2}\sigma^2 S^2 \frac{\partial^2 f}{\partial S^2} = i_f f
\]

At maturity, the price of a call option is \(C(S,T) = \max(0, S_T - K)\), and that of a put option is \(P(S,T) = \max(0, K - S_T)\).


\begin{center}
	\begin{tabular}{|c|c|c|}
		\hline
		Determinants & \textbf{call}&\textbf{put}\\
		\hline
		Underlying price	      & +&	-\\
		Strike price	              & -&	+\\
		Maturity (or time)    & + (-)&	+ (-)\\
		Volatility	              & +&	+\\
		Short-term interest rates  & +&	-\\
		Dividend payment	      & -&	+\\
		\hline
	\end{tabular}
\end{center}

The analytical solutions are :
\begin{align*}
	C_t &= S_t \Phi(d_1) - Ke^{-i_f \tau} \Phi(d_2) \\
	P_t &= Ke^{-i_f \tau} \Phi(-d_2) - S_t \Phi(-d_1)
\end{align*}
or :
\begin{align*}
	d_1 &= \frac{\ln(S_t/K) + (i_f + \frac{1}{2}\sigma^2)\tau}{\sigma \sqrt{\tau}}, \quad
	d_2 = d_1 - \sigma \sqrt{\tau}
\end{align*}

%La sensibilité peut être mesurée par cinq paramètres (lettres grecques) :

\begin{itemize}
	\item \textbf{Delta} \(\Delta\) : variation in the option price depending on the underlying.
	\item \textbf{Gamma} \(\Gamma\) : delta sensitivity.
	\item \textbf{Thêta} \(\Theta\) : sensitivity to time.
	\item \textbf{Véga} \(\mathcal{V}\) : sensitivity to volatility.
	\item \textbf{Rho} \(\rho\) : interest rate sensitivity.
\end{itemize}


The \textbf{Delta} measures the impact of a change in the underlying asset :

\begin{align*}
	\Delta_C &= \frac{\partial C}{\partial S} = \Phi(d_1), \quad \Delta \in (0,1) \\
	\Delta_P &= \frac{\partial P}{\partial S} = \Phi(d_1) - 1, \quad \Delta \in (-1,0)
\end{align*}


The global Delta of a portfolio \(\Pi\) with weights \(\omega_i\) is :
\[
\frac{\partial \Pi}{\partial S_t} = \sum_{i=1}^{n} \omega_i \Delta_i
\]



% Graphique TikZ conservé tel quel :

\begin{center}
\begin{tikzpicture}[scale=.52]
\def\riskfreeBS{0.05}
\def\xminBS{7.5}
\def\xmaxBS{12.5}
\def\PxExerciceBS{10}
\def\sigmaBS{0.3}
\def\TBS{0.4}
\def\PremiumBS{BSCall(\PxExerciceBS*exp(-\riskfreeBS*\TBS),\PxExerciceBS,\riskfreeBS,\TBS,\sigmaBS)}
\def\PxExerciceAct{\PxExerciceBS*exp(-(\riskfreeBS+\sigmaBS*\sigmaBS/2)*\TBS)}
\begin{axis}[
	width=0.8\textwidth,
	height=0.5\textwidth, 
	extra tick style={tick style=BleuProfondIRA},
	clip=false,
	axis on top,
	axis lines=middle, axis line style={BleuProfondIRA,thick,->},
	scale only axis, xmin={\xminBS},xmax={\xmaxBS},enlarge x limits=0.05,
	enlarge y limits=0.1,
	color=BleuProfondIRA,
	ylabel={\(\Delta\)},
	x label style={at={(axis cs:\xmaxBS+.1,0)},anchor=north east},
	xlabel={underlying (\(T\))},
	ytick=\empty,
	xtick=\empty,
	extra y ticks ={-.5,0,0.5},
	extra y tick labels={{\(-\frac{1}{2}\)},{0},{\(\frac{1}{2}\)}},
	extra x ticks ={\PxExerciceAct,\PxExerciceBS},
	extra x tick labels={{\color{BleuProfondIRA}\(K'\)\ \ \ \ \ \ \ \ },{\color{BleuProfondIRA}\ \ \(K\)}},
	extra x tick style={color=BleuProfondIRA,
		tick label style={yshift=-0mm}	},
	title ={ \textbf{Delta of the option}},
	title style={yshift=-10mm}
	]
	\addplot[name path=optionT,OrangeProfondIRA,thin,domain={{\xminBS}:{\xmaxBS}}, samples=21]
	{normcdf(-ddd(x,\PxExerciceBS,\riskfreeBS,\TBS,\sigmaBS),0,1)} node [above] {Call};
	\addplot[name path=optionT,OrangeProfondIRA,thin,domain={{\xminBS}:{\xmaxBS}}, samples=21]
	{normcdf(-ddd(x,\PxExerciceBS,\riskfreeBS,\TBS,\sigmaBS),0,1)-1} node [above] {Put};
	\draw[dashed,OrangeProfondIRA] 
	(axis cs:{\pgfkeysvalueof{/pgfplots/xmin}},-0.5) --
	(axis cs:{\PxExerciceAct},{-.5}) --
	(axis cs:{\PxExerciceAct},{0.5}) --
	(axis cs:{\pgfkeysvalueof{/pgfplots/xmin}},0.5);
\end{axis}
\end{tikzpicture}
\end{center}

\end{f}
\hrule


\begin{f}[The Yield Curve]
The \textbf{yield curve}, or the curve of returns, or \(r_f(\tau)\), provides a graphical representation of risk-free interest rates as a function of maturity (or term).
It is also called the \textbf{zero-coupon} yield curve, referring to a type of risk-free bond with no coupons (a debt composed only of two opposite cash flows, one at \(t_0\) and the other at \(T\)).
This curve also provides insight into market expectations regarding future interest rates (\engl{forward} rates).
\end{f}
\hrule


\begin{f}[The Nelson-Siegel and Svensson models]


The \textbf{Nelson-Siegel} functions take the form

{\small\begin{align*}
y( m ) =& \beta _0 + \beta _1\frac{{\left[ {1 - \exp \left( { - m/\tau} \right)} \right]}}{m/\tau} + \\
		&\beta _2 {\left(\frac{{\left[ {1 - \exp \left( { - m/\tau} \right)} \right]}}{m/\tau} - \exp \left( { - m/\tau}\right)\right)}
\label{MTNSeq}
\end{align*}}
%
where \(y\left( m \right)\) and \(m\) are as above, and \(\beta_0\), \(\beta_1\), \(\beta_2\), and \(\tau\) are parameters :


\begin{itemize}

\item   \(\beta_0\) is interpreted as the long-term level of interest rates (the coefficient is 1, it is a constant that does not decrease),

\item   \(\beta_1\) is the short-term component, noting that :
\begin{equation*}
	\lim_{m \rightarrow 0} \frac{{\left[ {1 - \exp \left( { - m/\tau} \right)} \right]}}{m/\tau}=1
\end{equation*}
It follows that the overnight rate such as €str\index{Taux d'intérêts! Estr} will equal \(\beta_0 + \beta_1\) in this model.
\item \(\beta_2\) is the medium-term component (it starts at 0, increases, then decreases back toward zero — i.e., bell-shaped),
\item \(\tau\) is the scale factor on maturity; it determines where the term weighted by \(\beta_2\) reaches its maximum.
\end{itemize}

Svensson (1995) adds a second bell-shaped term; this is the Nelson–Siegel–Svensson model. The additional term is :
%
\begin{equation*}
+\beta _3 {\left(\frac{{\left[ {1 - \exp \left( { - m/\tau_2} \right)} \right]}}{m/\tau_2} - \exp \left( { - m/\tau_2}\right)\right)}
\label{MTSveq}
\end{equation*}
and the interpretation is the same as for \(\beta_2\) and \(\tau\) above; it allows for two inflection points on the yield curve.

\newcommand{\traintunnel}{	        
\draw[thick, OrangeProfondIRA] svg "M 55.448002 56.380001L 40 39L 28 39L 12.552 56.380001M 12 34C 11.729672 21.575853 21.576109 11.281852 34 11C 46.423893 11.281852 56.270329 21.575853 56.000004 34L 56 55C 56 56.104568 55.104568 57 54 57L 14 57C 12.895431 57 12 56.104568 12 55ZM 28 39L 28 34C 28 30.132 30.302 27 34 27C 37.697998 27 40 30.132 40 34L 40 39M 34 51L 34 57M 34 43L 34 45";
}
\newcommand{\archibuilding}{
\draw[OrangeProfondIRA,yscale=-1] svg "M 12.296 28.886L 12.296 54.453999C 12.451618 55.715763 13.58469 56.623463 14.85 56.500004L 53.150002 56.5C 54.415314 56.623463 55.548386 55.715763 55.704002 54.454002L 55.703999 28.886M 12.296 28.886L 34 11.5L 55.703999 28.886M 34 46.296001L 42.212002 46.296001L 42.214001 50.386002L 49.32 50.386002L 49.32 56.5M 12.296 40.285999L 34 40.285999M 34 35.186001L 55.368 35.186001M 34 11.5L 34 56.236M 12.296 32.106003L 34 32.106003M 19.456001 32.106003L 19.456001 40.285999M 26.84 32.106003L 26.84 40.285999";	        
}
\newcommand{\familialcar}{	%
\draw[thick, OrangeProfondIRA] svg "M 21 45L 21 48C 21 48.552284 20.552284 49 20 49L 16 49C 15.447716 49 15 48.552284 15 48L 15 45M 53 45L 53 48C 53 48.552284 52.552284 49 52 49L 48 49C 47.447716 49 47 48.552284 47 48L 47 45M 54 45C 54.552284 45 55 44.552284 55 44L 55 37.414001C 54.999943 37.149296 54.894939 36.895409 54.708 36.708L 49 31L 19 31L 13.292 36.708C 13.105062 36.895409 13.000056 37.149296 13 37.414001L 13 44C 13 44.552284 13.447716 45 14 45ZM 49 31L 45.228001 19.684C 45.092045 19.275806 44.710239 19.000328 44.279999 19L 23.720001 19C 23.289761 19.000328 22.907955 19.275806 22.771999 19.684L 19 31M 19 31L 14 31C 13.447716 31 13 30.552284 13 30L 13 28C 13 27.447716 13.447716 27 14 27L 20.334 27M 47.666 27L 54 27C 54.552284 27 55 27.447716 55 28L 55 30C 55 30.552284 54.552284 31 54 31L 49 31M 13.092 37L 20 37C 20.552284 37 21 37.447716 21 38L 21 40C 21 40.552284 20.552284 41 20 41L 13 41M 55 41L 48 41C 47.447716 41 47 40.552284 47 40L 47 38C 47 37.447716 47.447716 37 48 37L 54.908001 37";
}

\newcommand{\familialTV}{	        
\draw[thick, OrangeProfondIRA] svg "M 12 17.5L 56 17.5C 56 17.5 57 17.5 57 18.5L 57 43.5C 57 43.5 57 44.5 56 44.5L 12 44.5C 12 44.5 11 44.5 11 43.5L 11 18.5C 11 18.5 11 17.5 12 17.5M 34 44.5L 34 50.5M 24 50.5L 44 50.5";
}

\newcommand{\TresorerieMngt}{	
\draw[OrangeProfondIRA] svg "M 11.008 31C 11.008 32.104568 15.485153 33 21.007999 33C 26.530848 33 31.007999 32.104568 31.007999 31C 31.007999 29.89543 26.530848 29 21.007999 29C 15.485153 29 11.008 29.89543 11.008 31ZM 31 31L 31 37C 31 38.106003 26.524 39 21 39C 15.476 39 11 38.106003 11 37L 11 31M 31 37L 31 43C 31 44.105999 26.524 45 21 45C 15.476 45 11 44.105999 11 43L 11 37M 31 43L 31 49C 31 50.105999 26.524 51 21 51C 15.476 51 11 50.105999 11 49L 11 43M 31 49L 31 55C 31 56.105999 26.524 57 21 57C 15.476 57 11 56.105999 11 55L 11 49M 11 25L 11 13C 11 11.895431 11.895431 11 13 11L 55 11C 56.104568 11 57 11.895431 57 13L 57 37C 57 38.104568 56.104568 39 55 39L 36 39M 28 25C 28.000584 21.94886 30.290909 19.384022 33.32254 19.039516C 36.354168 18.695011 39.161583 20.680555 39.846756 23.65377C 40.531929 26.626984 38.876644 29.640953 36 30.658001M 20 19.5C 20.276142 19.5 20.5 19.723858 20.5 20C 20.5 20.276142 20.276142 20.5 20 20.5C 19.723858 20.5 19.5 20.276142 19.5 20C 19.5 19.723858 19.723858 19.5 20 19.5M 48 29.5C 48.276142 29.5 48.5 29.723858 48.5 30C 48.5 30.276142 48.276142 30.5 48 30.5C 47.723858 30.5 47.5 30.276142 47.5 30C 47.5 29.723858 47.723858 29.5 48 29.5M 15 25L 15 16C 15 15.447716 15.447716 15 16 15L 52 15C 52.552284 15 53 15.447716 53 16L 53 34C 53 34.552284 52.552284 35 52 35L 36 35";	
}

These Nelson-Siegel and Svensson functions have the advantage of behaving well in the long term and being easy to parameterize.  
They are illustrated in the figure where the pictograms \begin{tikzpicture}[xscale=0.2, yscale=-0.2]
\TresorerieMngt\end{tikzpicture} \begin{tikzpicture}[xscale=0.2, yscale=-0.2] \familialTV\end{tikzpicture} \begin{tikzpicture}[xscale=0.2, yscale=-0.2] \familialcar\end{tikzpicture} \begin{tikzpicture}[xscale=0.2, yscale=0.2] \archibuilding\end{tikzpicture} \begin{tikzpicture}[xscale=0.2, yscale=-0.2] \traintunnel
\end{tikzpicture} represent the different usual maturities for this type of property or investment.
They allow for the modeling of a broad yield curve.
Once adjusted, the user can then evaluate assets or define various sensitivity measures.


\begin{center}
\begin{tikzpicture}[scale=0.55]
\def\MTbetaa{0.03}
\def\MTbetab{-0.02}
\def\MTbetac{0.01}
\def\MTbetad{-0.005}  % Svensson
\def\MTtaua{4.5}
\def\MTtaub{11}  % Svensson
\begin{axis}[
	width=0.8\textwidth,
	height=0.5\textwidth, 
	xlabel={Maturity (years)},ylabel={Rate (\%)},
	xmin=0, xmax=30,
	ymin=0, ymax=100*(\MTbetaa+.005),
	enlarge y limits=0.125,
	thick,
	axis x line=bottom,
	axis y line=left,
	yticklabel=\pgfmathprintnumber{\tick}\% 
	]
	\Large
	% Courbe de Nelson-Siegel
	\addplot[OrangeProfondIRA, thick, domain=0.01:27, samples=27] 
	{100*(\MTbetaa + \MTbetab * ((1 - exp(-x/\MTtaua)) / (x/\MTtaua)) + \MTbetac * (((1 - exp(-x/\MTtaua)) / (x/\MTtaua)) - exp(-x/\MTtaua)))}
	node  [pos=0.005] (M) {}
	node  [pos=0.10] (N) {}
	node  [pos=0.30] (O) {}
	node  [pos=0.75] (P) {}
	node  [pos=1] (Q) {};
	\addplot[dashed, OrangeProfondIRA, thick, domain=0.01:30, samples=27] 
	{100*(\MTbetaa + \MTbetab * ((1 - exp(-x/\MTtaua)) / (x/\MTtaua)) + \MTbetac * (((1 - exp(-x/\MTtaua)) / (x/\MTtaua)) - exp(-x/\MTtaua)) + \MTbetad * (((1 - exp(-x/\MTtaub)) / (x/\MTtaub)) - exp(-x/\MTtaub)))};
	% Affichage des paramètres
	\node[anchor=south east, text=BleuProfondIRA] at (rel axis cs:1,0.15) { 
	\(\beta_0 = \MTbetaa\)\quad
	\(\beta_1 = \MTbetab\)\quad
	\(\beta_2 = \MTbetac\)\quad
	\(\tau_1 = \MTtaua\)\quad			
};	\node[anchor=south east, text=BleuProfondIRA] at (rel axis cs:1,0.05) { 
	\(\beta_3 = \MTbetad\)\quad
	\(\tau_2 = \MTtaub\)			
};
\node[xscale=0.3, yscale=-0.3, above=35pt] at (M) {\TresorerieMngt};
\node[xscale=0.3, yscale=-0.3, above=35pt] at (N) {\familialTV};
\node[xscale=0.3, yscale=-0.3, above=30pt] at (O) {\familialcar};
\node[xscale=0.3, yscale=-0.3, above=30pt] at (P) {\archibuilding};
\node[xscale=0.3, yscale=-0.3, above=30pt] at (Q) {\traintunnel};
\end{axis}
\end{tikzpicture}
\end{center}
\end{f}
\hrule


\begin{f}[Vasicek model]
	
	Under a risk-neutral probability \(\mathbb{Q}\), the short rate \((r_t)\) follows an Ornstein–Uhlenbeck process with constant coefficients :
	\[
	dr_t = \kappa(\theta - r_t)\, dt + \sigma\, dW_t, \quad r_0 \in \mathbb{R}
	\]
	or :
	\begin{itemize}[nosep]
		\item \(\kappa > 0\) is the speed of mean reversion,
		\item \(\theta\) is the long-term mean level,
		\item \(\sigma > 0\) is the volatility,
		\item \(W_t\) is a standard Brownian motion under \(\mathbb{Q}\).
	\end{itemize}
	
The EDS solution (application of Itô’s lemma to \(Y_{t}=r(t) e^{\kappa t}\)) :
	\[
	r_t = r_s e^{-\kappa(t-s)} + \theta(1 - e^{-\kappa(t-s)}) + \sigma \int_s^t e^{-\kappa(t-u)} dW_u
	\]
	
	\textbf{So} :
	\[
	\begin{aligned}
		\mathbb{E}_\mathbb{Q}[r_t \mid \mathcal{F}_s] &= r_s e^{-\kappa(t-s)} + \theta(1 - e^{-\kappa(t-s)}) \\
		\operatorname{Var}_\mathbb{Q}[r_t \mid \mathcal{F}_s] &= \frac{\sigma^2}{2\kappa} \left(1 - e^{-2\kappa(t-s)}\right)
	\end{aligned}
	\]
The process \((r_t)\) is Gaussian; negative rates are possible.
	
\end{f}

\begin{f}[Price of a zero-coupon bond (Vasicek)]

The price at time \(t\) of a zero-coupon bond maturing at \(T\) is given by :
\[
ZC(t, T) = A(t, T) \, e^{-B(t, T)\, r_t}
\]
où :
\[
\begin{aligned}
	B(t, T) &= \frac{1 - e^{-\kappa(T - t)}}{\kappa} \\
	A(t, T) &= \exp \left[ \left(\theta - \frac{\sigma^2}{2\kappa^2}\right) (B(t, T) - (T - t)) - \frac{\sigma^2}{4\kappa} B(t, T)^2 \right]
\end{aligned}
\]

This formulation is possible due to the fact that \(\int_t^T r_s ds\) is a Gaussian random variable conditional on \(\mathcal{F}_t\).

\[
ZC(t, T) = \mathbb{E}_\mathbb{Q} \left[ \exp\left( -\int_t^T r_s\, ds \right) \Big| \mathcal{F}_t \right]
\]

\end{f}
\hrule

\begin{f}[Cox–Ingersoll–Ross (CIR) model]
	
Under the risk-neutral measure \(\mathbb{Q}\), the short rate \((r_t)\) follows the dynamics :
\[
dr_t = \kappa(\theta - r_t)\,dt + \sigma \sqrt{r_t}\, dW_t, \quad r_0 \geq 0
\]
with :
\begin{itemize}[nosep]
	\item \(\kappa > 0\) : mean reversion speed,
	\item \(\theta > 0\) : long-term level,
	\item \(\sigma > 0\) : volatility,
	\item \(W_t\) : Brownian motion under \(\mathbb{Q}\).
\end{itemize}

\textbf{So} :
\begin{itemize}
	\item The square root \(\sqrt{r_t}\) guarantees \(r_t \geq 0\) if \(2\kappa\theta \geq \sigma^2\) (Feller condition).
	\item The process \((r_t)\) is a non-Gaussian diffusion process but with continuous trajectories.
	\item The rate is \textbf{mean-reverting} around \(\theta\).
\end{itemize}

Thus, the process \((r_t)\) is a diffusion with explicit conditional distributions (under \(\mathbb{Q}\)) :

For \(s < t\), the variable \(r_t\) follows a non-central \(\chi^2\) distribution :
\[
r_t \mid \mathcal{F}_s \sim c \cdot \chi^2_{d}(\lambda)
\]
with :
\begin{itemize}[nosep]
	\item \(\displaystyle c = \frac{\sigma^2 (1 - e^{-\kappa (t - s)})}{4\kappa}\)
	\item \(\displaystyle d = \frac{4\kappa\theta}{\sigma^2}\) : degrees of freedom
	\item \(\displaystyle \lambda = \frac{4\kappa e^{-\kappa (t - s)} r_s}{\sigma^2 (1 - e^{-\kappa (t - s)})}\)
\end{itemize}

and
\[
\begin{aligned}
	\mathbb{E}_\mathbb{Q}[r_t \mid \mathcal{F}_s] =& r_s e^{-\kappa(t-s)} + \theta (1 - e^{-\kappa(t-s)}) \\
	\operatorname{Var}_\mathbb{Q}[r_t \mid \mathcal{F}_s] =& \frac{\sigma^2 r_s e^{-\kappa(t-s)} (1 - e^{-\kappa(t-s)})}{\kappa} \\
			&+ \frac{\theta \sigma^2}{2\kappa} (1 - e^{-\kappa(t-s)})^2
\end{aligned}
\]

\end{f}
\begin{f}[Price of a zero-coupon bond (CIR)]
In the CIR model, the price of a zero-coupon bond at time \(t\) with maturity \(T\) is given by :
\[
ZC(t, T) = A(t, T) \cdot e^{-B(t, T)\, r_t}
\]
with :
\[
\begin{aligned}
	B(t, T) &= \frac{2 (e^{\gamma (T - t)} - 1)}{(\gamma + \kappa)(e^{\gamma (T - t)} - 1) + 2\gamma} \\
	A(t, T) &= \left[ \frac{2\gamma e^{\frac{(\kappa + \gamma)}{2}(T - t)}}{(\gamma + \kappa)(e^{\gamma (T - t)} - 1) + 2\gamma} \right]^{\frac{2\kappa\theta}{\sigma^2}}
\end{aligned}
\]
or :
\[
\gamma = \sqrt{\kappa^2 + 2\sigma^2}
\]

\end{f}
\hrule

\begin{f}[Swaption, Black model]

A \textbf{swaption} is an option on an interest rate swap. It gives the right (but not the obligation) to enter into a swap at a future date \(T\).

\begin{itemize}[nosep]
	\item \textbf{Payer swaption}: right to \emph{pay the fixed rate} and \emph{receive the floating rate}.
	\item \textbf{Receiver swaption}: right to \emph{receive the fixed rate} and \emph{pay the floating rate}.
\end{itemize}

\textbf{Notation} :
\begin{itemize}[nosep]
	\item \(T\) : swaption exercise date
	\item \(K\) : fixed rate (strike)
	\item \(S(T)\) : swap rate on the date \(T\)
	\item \(A(T)\) : present value of future fixed flows.
	\item \(\sigma\) : swap rate volatility
\end{itemize}

The Black (1976) model is an adaptation of the Black–Scholes model for interest rate products. Here, the swap rate \(S(T)\) plays the role of the underlying asset, with a European option-type payoff.

\textbf{Black's formula for a payer swaption} :
\[
\text{SW}_{\text{payer}} = A(T) \left[ S_0 N(d_1) - K N(d_2) \right]
\]
or :
\[
\begin{aligned}
	d_1 &= \frac{\ln(S_0 / K) + \frac{1}{2} \sigma^2 T}{\sigma \sqrt{T}} \\
	d_2 &= d_1 - \sigma \sqrt{T}
\end{aligned}
\]
and \(N(\cdot)\) is the cumulative distribution function of the standard normal distribution.

\textbf{Formula for a receiver swaption} :
\[
\text{SW}_{\text{receiver}} = A(T) \left[ K N(-d_2) - S_0 N(-d_1) \right]
\]

\end{f}

