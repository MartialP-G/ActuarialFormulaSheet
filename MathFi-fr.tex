% !TeX root = ActuarialFormSheet_MBFA-fr.tex
% !TeX spellcheck = fr_FR

\begin{f}[Capitalisation Actualisation]

\begin{tikzpicture}[scale=0.85]
% Draw the x-axis and y-axis.
\def\w{11}
\def\n{9}
\draw[ line width=1,dotted, color=OrangeProfondIRA, arrows={-Stealth[length=4, inset=0]}] (0,0) -- (0,0.4909) node (A) {};	
\foreach \y in  {0,...,3} {
	\draw (\y,0) -- (\y,-0.1);
	\ifthenelse{\y>0 }{	\node[below] at (\y,-0.1) {\tiny $ \scriptstyle \y$};
		}{
		\node[below] at (\y,-0.1) {\tiny 0};}
}
\foreach \y in  {-3,...,1} {
	\draw (\y+\n,0) -- (\y+\n,-0.1);
	\ifthenelse{\y<0 }{\node[below] at (\y+\n,-0.1) {\tiny $\scriptstyle n \y$};}{}
	\ifthenelse{\y=0 }{\node[below] at (\y+\n,-0.1) {\tiny $\scriptstyle n $};
		\draw[ line width=1, color=OrangeProfondIRA, arrows={-Stealth[length=4, inset=0]}] (\y+\n,0) -- (\y+\n,1) node (B) {};}{}
	\ifthenelse{\y>0}{	\node[below] at (\y+\n,-0.1) {\tiny $\scriptstyle n+\y$};}{}
}
\draw[ line width=1] (-.25,0) -- (4,0);
\draw[ line width=1, dashed] (4,0) -- (5,0);
\draw[arrows={-Stealth[length=4, inset=0]}, line width=1] (5,0) -- (\w,0);
\draw[arrows={-Stealth[length=4, inset=0]},color=OrangeProfondIRA] (B)  to [bend right]  node [pos=0.5, below=10pt] {Actualisation}(A) ;
\end{tikzpicture}

\begin{tikzpicture}[scale=0.85]
% Draw the x-axis and y-axis.
\def\w{11}
\def\n{9}
\draw[ line width=1, color=OrangeProfondIRA, arrows={-Stealth[length=4, inset=0]}] (0,0) -- (0,1) node (A) {};	
\foreach \y in  {0,...,3} {
	\draw (\y,0) -- (\y,-0.1);
	\ifthenelse{\y>0 }{	\node[below] at (\y,-0.1) {\tiny $ \scriptstyle \y$};
	}{
		\node[below] at (\y,-0.1) {\tiny 0};}
}
\foreach \y in  {-3,...,1} {
	\draw (\y+\n,0) -- (\y+\n,-0.1);
	\ifthenelse{\y<0 }{\node[below] at (\y+\n,-0.1) {\tiny $\scriptstyle n \y$};}{}
	\ifthenelse{\y=0 }{\node[below] at (\y+\n,-0.1) {\tiny $\scriptstyle n $};
		\draw[ line width=1,dotted, color=OrangeProfondIRA, arrows={-Stealth[length=4, inset=0]}] (\y+\n,0) -- (\y+\n,2.0368) node (B) {};}{}
	\ifthenelse{\y>0}{	\node[below] at (\y+\n,-0.1) {\tiny $\scriptstyle n+\y$};}{}
}
\draw[ line width=1] (-.25,0) -- (4,0);
\draw[ line width=1, dashed] (4,0) -- (5,0);
\draw[arrows={-Stealth[length=4, inset=0]}, line width=1] (5,0) -- (\w,0);
\draw[arrows={-Stealth[length=4, inset=0]},color=OrangeProfondIRA] (A)  to [bend left]  node [pos=0.55, below=10pt] {Capitalisation}(B) ;
\end{tikzpicture}

\end{f}
\hrule

\begin{f}[Les Intérêts]

\emph{Escompte ou taux précompté $d$}
     $$d=i/(1+i)$$
\emph{Intérêt simple $i$}
$$
I_t=P i t=P i \frac{k}{365}
$$
\emph{Intérêt composés $i$}
$$
V_n=P(1+i)^n=P\left(1+\frac{p}{100}\right)^n
$$
\emph{Intérêt continu $r$}
$$V_t=V_0\ e^{rt}$$

\emph{Taux effectif $i_e$}
$$
i_e=\left( 1+\frac{i}{m}\right) ^{m}-1
$$
où $i$ est le taux nominal et $m$ le nombre de périodes sur un an.

\emph{Taux équivalent $i^{(m)}$}
$$
i^{(m)}=m(1+i)^{1 / m}-1
$$

\emph{Taux nominal $i$ et taux périodique}

Le taux \textbf{nominal} ou taux \textbf{facial} permet de calculer les intérêts dus sur un an.
Le taux \textbf{périodique} correspond au taux nominal divisé par le nombre de périodes sur un an $i/m$.
Si le taux périodique est hebdomadaire, le taux nominal sera divisé par 52.
\end{f}
\hrule

\begin{f}[Valeur actuelle et valeur future]

La valeur actuelle (VA)  ou valeur présente (VP)  représente le capital qui doit être investi aujourd'hui à un taux d'intérêt composé annuel $i$ pour obtenir des flux de trésorerie futurs ($F_k$) aux moments $t_k$:
\begin{equation}
	VP = \sum_{k=1}^{n} F_k \times \frac{1}{(1+i)^{t_k}}
\label{ValeurActuelle}
\end{equation}
Lorsque les $F_k$ sont constants
\begin{equation}
	VP = K  \frac{1 - (1+i)^{-n}} {i}
\label{ValeurActuelleFluxCt}
\end{equation}

La valeur future (VF) représente la valeur du capital en $T$ qui, avec un taux d'intérêt composé annuel $i$,  capitalise les flux de trésorerie futurs ($F_k$) aux moments $t_k$.
\begin{equation}
	VF=V_n = \sum_{k=1}^{n} F_k \times (1+i)^{n-t_k}
\end{equation}
Plus généralement $VF= (1+i)^{n}VP$.
\end{f}
\hrule

\begin{f}[Annuités]
\input{AnnuiteTikz-fr}
\end{f}
\hrule

\begin{f}[L'emprunt (Indivis)]
\ \newline

La principale propriété de l'emprunt est de considéré séparément les intérêts du remboursement (ou de l'amortissement).

Par un remboursement constant ou par annuité constante: la somme de l'amortissement et de l'intérêt à chaque période est constante. 

\begin{center}
\begin{tikzpicture}[scale=0.85]
% Draw the x-axis and y-axis.
\def\w{11}
\def\n{9}
\def\duree{23}
\def\a{16.713332269}
\def\v{1.031415^(-1)} 
\draw[ line width=1, color=BleuProfondIRA, arrows={-Stealth[length=4, inset=0]}] (0,0) -- (0,-2);
\foreach \y in  {0,...,3} {
	\draw (\y,0) -- (\y,-0.1);
	\ifthenelse{\y>0 }{	\node[below] at (\y,-0.1) {\tiny $ \scriptstyle \y$};
	\pgfmathpow{\v}{\duree-\y} 	
	\let\Interet\pgfmathresult
		\draw[ line width=1, color=BleuProfondIRA, arrows={-Stealth[length=4, inset=0]}] (\y,0) -- (\y,\Interet);
		\draw[ line width=1, color=OrangeProfondIRA, arrows={-Stealth[length=4, inset=0]}] (\y,\Interet) -- (\y,1);}{
		\node[below] at (\y,-0.1) {\tiny 0};}
}
\foreach \y in  {-3,...,1} {
	\pgfmathpow{\v}{-\y+1} 	
	\let\Interet\pgfmathresult
	\draw (\y+\n,0) -- (\y+\n,-0.1);
	\ifthenelse{\y<0 }{\node[below] at (\y+\n,-0.1) {\tiny $\scriptstyle n \y$};}{}
	\ifthenelse{\y=0 }{\node[below] at (\y+\n,-0.1) {\tiny $\scriptstyle n $};}{}
	\ifthenelse{\y>0}{	\node[below] at (\y+\n,-0.1) {\tiny $\scriptstyle n+\y$};}{}
	\ifthenelse{\y<1 }{	
		\draw[ line width=1, color=BleuProfondIRA, arrows={-Stealth[length=4, inset=0]}] (\y+\n,0) -- (\y+\n,\Interet);
		\draw[ line width=1, color=OrangeProfondIRA, arrows={-Stealth[length=4, inset=0]}] (\y+\n,\Interet) -- (\y+\n,1);}
}
\draw[ line width=1] (-.25,0) -- (4,0);
\draw[ line width=1, dashed] (4,0) -- (5,0);
\draw[arrows={-Stealth[length=4, inset=0]}, line width=1] (5,0) -- (\w,0);
\end{tikzpicture}
\end{center}
	
    
    Par un amortissement constant.
\begin{center}
\begin{tikzpicture}[scale=0.85]
% Draw the x-axis and y-axis.
\def\w{11}
\def\n{9}
\def\duree{23}
\def\i{0.031415}
\draw[ line width=1, color=BleuProfondIRA, arrows={-Stealth[length=4, inset=0]}] (0,0) -- (0,-2);
\foreach \y in  {0,...,3} {
\draw (\y,0) -- (\y,-0.1);
\ifthenelse{\y>0 }{	\node[below] at (\y,-0.1) {\tiny $ \scriptstyle \y$};
	\pgfmathparse{(\duree-\y)*\i} 	
	\let\Interet\pgfmathresult
	\draw[ line width=1, color=BleuProfondIRA, arrows={-Stealth[length=4, inset=0]}] (\y,0) -- (\y,1);
	\draw[ line width=1, color=OrangeProfondIRA, arrows={-Stealth[length=4, inset=0]}] (\y,1) -- (\y,1+\Interet);}{
	\node[below] at (\y,-0.1) {\tiny 0};}
}
\foreach \y in  {-3,...,1} {
	\pgfmathparse{(-\y+1)*\i} 	
\let\Interet\pgfmathresult
\draw (\y+\n,0) -- (\y+\n,-0.1);
\ifthenelse{\y<0 }{\node[below] at (\y+\n,-0.1) {\tiny $\scriptstyle n \y$};}{}
\ifthenelse{\y=0 }{\node[below] at (\y+\n,-0.1) {\tiny $\scriptstyle n $};}{}
\ifthenelse{\y>0}{	\node[below] at (\y+\n,-0.1) {\tiny $\scriptstyle n+\y$};}{}
\ifthenelse{\y<1 }{	
	\draw[ line width=1, color=BleuProfondIRA, arrows={-Stealth[length=4, inset=0]}] (\y+\n,0) -- (\y+\n,1);
\draw[ line width=1, color=OrangeProfondIRA, arrows={-Stealth[length=4, inset=0]}] (\y+\n,1) -- (\y+\n,1+\Interet);}{}
}
\draw[ line width=1] (-.25,0) -- (4,0);
\draw[ line width=1, dashed] (4,0) -- (5,0);
\draw[arrows={-Stealth[length=4, inset=0]}, line width=1] (5,0) -- (\w,0);
\end{tikzpicture}
\end{center}

    
Par un remboursement in fine, où l'intérêt est constant. Seuls les intérêts sont versés périodiquement jusqu'au terme, moment où le remboursement total est effectué.
	
\begin{center}
\begin{tikzpicture}[scale=0.85]
% Draw the x-axis and y-axis.
\def\w{11}
\def\n{9}
\def\duree{23}
\def\i{3*0.031415}
\draw[ line width=1, color=BleuProfondIRA, arrows={-Stealth[length=4, inset=0]}] (0,0) -- (0,-2);
\foreach \y in  {0,...,3} {
	\draw (\y,0) -- (\y,-0.1);
	\ifthenelse{\y>0 }{	\node[below] at (\y,-0.1) {\tiny $ \scriptstyle \y$};
%				\draw[ line width=1, color=BleuProfondIRA, arrows={-Stealth[length=4, inset=0]}] (\y,0) -- (\y,1);
		\draw[ line width=1, color=OrangeProfondIRA, arrows={-Stealth[length=4, inset=0]}] (\y,0) -- (\y,2*\i);}{
		\node[below] at (\y,-0.1) {\tiny 0};}
}
\foreach \y in  {-3,...,1} {
	\draw (\y+\n,0) -- (\y+\n,-0.1);
	\ifthenelse{\y<0 }{\node[below] at (\y+\n,-0.1) {\tiny $\scriptstyle n \y$};
		\draw[ line width=1, color=OrangeProfondIRA, arrows={-Stealth[length=4, inset=0]}] (\y+\n,0) -- (\y+\n,2*\i);}{}
	\ifthenelse{\y=0 }{
		\draw[ line width=1, color=BleuProfondIRA, arrows={-Stealth[length=4, inset=0]}] (\y+\n,0) -- (\y+\n,2);
		\draw[ line width=1, color=OrangeProfondIRA, arrows={-Stealth[length=4, inset=0]}] (\y+\n,2) -- (\y+\n,2+2*\i);
		\node[below] at (\y+\n,-0.1) {\tiny $\scriptstyle n $};}{}
	\ifthenelse{\y>0}{	\node[below] at (\y+\n,-0.1) {\tiny $\scriptstyle n+\y$};}{}
}
\draw[ line width=1] (-.25,0) -- (4,0);
\draw[ line width=1, dashed] (4,0) -- (5,0);
\draw[arrows={-Stealth[length=4, inset=0]}, line width=1] (5,0) -- (\w,0);
\end{tikzpicture}
\end{center}

\end{f}
\hrule \

\begin{f} [Tableau d'amortissement de l'emprunt]
\ \newline

	\footnotesize
\renewcommand{\arraystretch}{2}
\begin{tabular}{|m{10ex}|m{19ex}|m{19ex}|m{17ex}|}
\rowcolor{BleuProfondIRA!40}   	\hline &\textbf{In fine}			&   	\textbf{Amortissements  constants} &  		\textbf{Annuités  constantes} \\
	\hline Capital restant dû $S_k$ & $T_k=S_0$, $T_n=0$& $S_{0}\left(1-\frac{k}{n}\right)$ & $S_{0} \frac{1-v^{n-k}}{1-v^{n}}$\\
	\hline Intérêts $U_k$ 		& $i\times S_0$ & $S_{0}\left(1-\frac{k-1}{n}\right) i$ & $K\left(1-v^{n-k+1}\right)$ \\
	\hline Amortis\-sements $T_k$ &	$T_k=O$, $T_n=S_0$ & $\frac{S_{0}}{n}$ & $K v^{n-k+1}$ \\
	\hline Annuité $K_k$ & 	$K_k=i S_0$, $K_n=(1+i)S_0$ 	 & $\frac{S_{0}}{n}(1-(n-k+1) i) $ & $K=S_{0} \frac{i}{1-v^{n}} $ \\
%	\hline Coût de l'emprunt & $1 \times K \times N$ & $1 \times K \times \frac{N+1}{2}$ & $K\left(\frac{N i}{1-(1+i)^{-N}}-1\right)$ \\
	\hline
\end{tabular}

\end{f}
